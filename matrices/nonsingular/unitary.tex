\documentclass{ximera}

% These macros are automatically generated from the "macros"
% XML element.  Make permanent edits there.
%
% History
%   2004/01/01  Initiated for FCLA, evolved from there
%   2006/09/17  Converted  _, ^  to \sb, \sp for TeX4ht
%   2014/02/01  Updated for MathBook XML projects
%               Obsolete in FCLA: \codeindent, \computerfont, \define
%               Change: MathJax wants \lt, so replaced by \lteval
%   2014/02/22  New: \orderof, \reals, \per
%   2015/08/16  Incorporated into MathBook XML version of FCLA
%
%%%%%%%%%%%%%%%%%%%%%
%
%     Conveniences
%
%%%%%%%%%%%%%%%%%%%%%
%
%  Order of (asymptotically limit of fraction is 1)
%  Usage: \orderof{some function}
%
\newcommand{\orderof}[1]{\sim #1}
%
%  Integers
%  Usage:  \Z
\newcommand{\Z}{\mathbb{Z}}
%
%  Real numbers, as set of scalars
%  Usage:  \reals
\newcommand{\reals}{\mathbb{R}}
%
%  n-space over real field
%  Usage: \complex{integer-dimension}
\newcommand{\real}[1]{\mathbb{R}^{#1}}
%
%  Complex numbers, as set of scalars
%  Usage:  \complexes
\newcommand{\complexes}{\mathbb{C}}
%
%  n-space over complex field
%  Usage: \complex{integer-dimension}
\newcommand{\complex}[1]{\mathbb{C}^{#1}}
%
%  Complex conjugation (scalar, vector, matrix)
%  Usage: \conjugate{object}
\newcommand{\conjugate}[1]{\overline{#1}}
%
%  Complex number modulus
%  Usage: \modulus{a+bi}
%  Presumes math mode
\newcommand{\modulus}[1]{\left\lvert#1\right\rvert}
%
%  Zero vector
%  Usage: \zerovector
\newcommand{\zerovector}{\vect{0}}
%
%  Zero matrix
%  Usage: \zeromatrix, use a subscript when size is important
\newcommand{\zeromatrix}{\mathcal{O}}
%
%  Inner product (brackets, not quadratic form)
%  Usage: \innerproduct{a-vector}{a-vector}
\newcommand{\innerproduct}[2]{\left\langle#1,\,#2\right\rangle}
%
%  Norm of a vector
%  Usage: \norm{a-vector}
\newcommand{\norm}[1]{\left\lVert#1\right\rVert}
%
%  Dimension
%  Usage: \dimension{vector-space-letter}
\newcommand{\dimension}[1]{\dim\left(#1\right)}
%
%  Nullity
%  Usage: \nullity{matrix-or-lintrans-letter}
\newcommand{\nullity}[1]{n\left(#1\right)}
%
%  Rank
%  Usage: \rank{matrix-or-lintrans-letter}
\newcommand{\rank}[1]{r\left(#1\right)}
%
%  Direct sum
%  Usage: \ds between a couple of subspaces
%
\newcommand{\ds}{\oplus}
%
%  Determinant of a matrix (functional)
%  Usage: \detname{A}
\newcommand{\detname}[1]{\det\left(#1\right)}
%
%  Determinant of a matrix (vertical bars)
%  Usage: \detbars{A}
\newcommand{\detbars}[1]{\left\lvert#1\right\rvert}
%
%  Trace of a Matrix
%  Usage: \trace{matrix name}
\newcommand{\trace}[1]{t\left(#1\right)}
%
%  Square Root of a Matrix
%  Usage: \sr{a-matrix}
\newcommand{\sr}[1]{#1^{1/2}}
%
%%%%%%%%%%%%%%%%%%%%%
%
%     Subspace Constructions
%
%%%%%%%%%%%%%%%%%%%%%
%
%  Span of a set of vectors
%  \span and \sp are used by TeX for other things
%  Usage: \spn{set-of-vectors}
\newcommand{\spn}[1]{\left\langle#1\right\rangle}
%
%  Null space of a matrix
%  Usage:  \nsp{A}
\newcommand{\nsp}[1]{\mathcal{N}\!\left(#1\right)}
%
%  Column space of a matrix
%  Usage:  \csp{A}
\newcommand{\csp}[1]{\mathcal{C}\!\left(#1\right)}
%
%  Row space of a matrix
%  Usage:  \rsp{A}
\newcommand{\rsp}[1]{\mathcal{R}\!\left(#1\right)}
%
%  Left null space of a matrix
%  Usage:  \lns{A}
\newcommand{\lns}[1]{\mathcal{L}\!\left(#1\right)}
%
%  Orthogonal complement of a vector space
%  Avoiding TeX's \perp
%  Usage:  \per{A}
\newcommand{\per}[1]{#1^\perp}
%
%%%%%%%%%%%%%%%%%%%%%
%
%     Systems of Equations
%
%%%%%%%%%%%%%%%%%%%%%
%
%  In-line form of an augmented matrix for a system of equations
%  Usage: \augmented{coefficient-matrix}{constant-vector}
\newcommand{\augmented}[2]{\left\lbrack\left.#1\,\right\rvert\,#2\right\rbrack}
%
%  Notation for a linear system before introducing matrix multiplication
%  Usage: \linearsystem{coefficient-matrix}{constant-vector}
\newcommand{\linearsystem}[2]{\mathcal{LS}\!\left(#1,\,#2\right)}
%
%  Notation for a homogenous system before introducing matrix multiplication
%  Usage: \homosystem{coefficient-matrix}
\newcommand{\homosystem}[1]{\linearsystem{#1}{\zerovector}}
%
%%%%%%%%%%%%%%%%%%%%%
%
%     Row Operations, Echelon Form
%
%%%%%%%%%%%%%%%%%%%%%
%
% Row operations on matrices
%
% Three commands to shorten up descriptions of gaussian elimination
%
% Usage: \rowopswap{row-i}{row-j}
% Usage: \rowopmult{scalar}{row-i}
% Usage: \rowopadd{scalar}{row-multiplied}{row-added-to}
\newcommand{\rowopswap}[2]{R_{#1}\leftrightarrow R_{#2}}
\newcommand{\rowopmult}[2]{#1R_{#2}}
\newcommand{\rowopadd}[3]{#1R_{#2}+R_{#3}}
%
% Mark leading 1's in echelon form with fbox
% Usage: \leading{a-1-usually}
\newcommand{\leading}[1]{\fbox{#1}}
%
%  Row-reduce arrow
%  Usage:  \rref inbetween a matrix and its reduced row-echelon form
\newcommand{\rref}{\xrightarrow{\text{RREF}}}
%
%  Elementary Matrices
%  Usage: \elemswap{subscript}{subscript}
%  Usage: \elemmult{scalar}{subscript}
%  Usage: \elemadd{scalar}{subscript-mult}{subscript-target}
%
\newcommand{\elemswap}[2]{E_{#1,#2}}
\newcommand{\elemmult}[2]{E_{#2}\left(#1\right)}
\newcommand{\elemadd}[3]{E_{#2,#3}\left(#1\right)}
%
%%%%%%%%%%%%%%%%%%%%%
%
%     2-D Constructions (Lists, Vectors, Matrices)
%
%%%%%%%%%%%%%%%%%%%%%
%
%  A list of scalars of generic length
%  Usage:  \scalarlist{scalar letter}{terminal subscript}
\newcommand{\scalarlist}[2]{{#1}_{1},\,{#1}_{2},\,{#1}_{3},\,\ldots,\,{#1}_{#2}}
%
%  Vector styling, bold (or use wiggles, arrows, whatever)
%  Subscripts go outside this construction
%  Usage: \vect{a symbol to use as a vector}
%  Have to already be in math mode
%
\newcommand{\vect}[1]{\mathbf{#1}}
%
%  A column vector
%  Usage: \colvector{list-delimited-by-\\}
%
\newcommand{\colvector}[1]{\begin{bmatrix}#1\end{bmatrix}}
%
%  A generic vector with components
%  Usage: \vectorcomponents{component-letter}{final-subscript}
\newcommand{\vectorcomponents}[2]{\colvector{#1_{1}\\#1_{2}\\#1_{3}\\\vdots\\#1_{#2}}}
%
%  A list of vectors of generic length
%  Usage:  \vectorlist{vector letter}{terminal subscript}
\newcommand{\vectorlist}[2]{\vect{#1}_{1},\,\vect{#1}_{2},\,\vect{#1}_{3},\,\ldots,\,\vect{#1}_{#2}}
%
%  Vector entries, entry i of vector v
%  (vector-expession still needs \vect, etc.)
%  Usage:  \vectorentry{vector-expression}{single-subscript}
\newcommand{\vectorentry}[2]{\left\lbrack#1\right\rbrack_{#2}}
%
%  Matrix entries, entry i,j of matrix A
%  Usage:  \matrixentry{matrix-expression}{paired-subscripts}
%
\newcommand{\matrixentry}[2]{\left\lbrack#1\right\rbrack_{#2}}
%
%  A generic linear combination
%  Usage:  \lincombo{scalar letter}{vector letter}{terminal subscript}
\newcommand{\lincombo}[3]{#1_{1}\vect{#2}_{1}+#1_{2}\vect{#2}_{2}+#1_{3}\vect{#2}_{3}+\cdots +#1_{#3}\vect{#2}_{#3}}
%
%  Matrix, column by column, as vectors
%  Usage:  \matrixcolumns{matrix letter}{terminal subscript}
\newcommand{\matrixcolumns}[2]{\left\lbrack\vect{#1}_{1}|\vect{#1}_{2}|\vect{#1}_{3}|\ldots|\vect{#1}_{#2}\right\rbrack}
%
%%%%%%%%%%%%%%%%%%%%%
%
%     Special Matrices
%
%%%%%%%%%%%%%%%%%%%%%
%
%  Transpose of a matrix
%  Usage:  \transpose{A}
\newcommand{\transpose}[1]{#1^{t}}
%
%  Inverse of a matrix
%  Usage:  \inverse{A}
\newcommand{\inverse}[1]{#1^{-1}}
%
%  Submatrix (for minors, determinants)
%  Usage: \submatrix{matrix-name}{delete-row}{delete-col}
\newcommand{\submatrix}[3]{#1\left(#2|#3\right)}
%
%  Adjoint of a matrix (twice)
%  This macro is a convenience to call \transpose and \conjugate properly
%  It shouldn't need to be modified (or mathematical meanings will change)
%  Usage:  \adj{A}
\newcommand{\adj}[1]{\transpose{\left(\conjugate{#1}\right)}}
%
%  This macro controls the symbol used for the adjoint
%  It can be edited to taste
%  Usage:  \adjoint{A}
\newcommand{\adjoint}[1]{#1^\ast}
%
%%%%%%%%%%%%%%%%%%%%%
%
%     Sets
%
%%%%%%%%%%%%%%%%%%%%%
%
%  A convenience for simple sets
%  Usage:  \set{list of element}
\newcommand{\set}[1]{\left\{#1\right\}}
%
%  Sets with vertical bar, "such that", sized for objects, not condition
%  Usage:  \setparts{objects}{condition}
%
%%\newcommand{\setparts}[2]{\left\{ #1\mid#2\right\}}
%%\newcommand{\setparts}[2]{\left\{\left. #1\right\rvert#2\right\}}
\newcommand{\setparts}[2]{\left\lbrace#1\,\middle|\,#2\right\rbrace}
%
%  Set Cardinality
%  Usage:  \card{a-set-letter}
\newcommand{\card}[1]{\left\lvert#1\right\rvert}
%
%  Set Union
%  Use \cup
%
%  Set Intersection
%  Use \cap
%
%  Set Complement
%  Usage:  \setcomplement{a-set-letter}
\newcommand{\setcomplement}[1]{\overline{#1}}
%
%%%%%%%%%%%%%%%%%%%%%
%
%     Eigenvalues and Eigenspaces
%
%%%%%%%%%%%%%%%%%%%%%
%
%  Characteristic polynomial
%  Usage: \charpoly{matrix-letter}{variable-letter}
\newcommand{\charpoly}[2]{p_{#1}\left(#2\right)}
%
%  Eigenspace
%  Usage: \eigenspace{matrix-letter}{eigenvalue-letter}
\newcommand{\eigenspace}[2]{\mathcal{E}_{#1}\left(#2\right)}
%
%  2013/10/03 Including ampersands is problematic here, 
%  think about fixes later
%  2014/02/22 Limited testing, seems &amp; is fine for HTML and LaTeX
%  2016-07-20 only employed in Archetypes, MBX has gather/align override
%  Eigensystem (presumes wrapped in an mrow within md)
%  Usage: \eigensystem{matrixletter}{eigenvalue}{list of basis vectors}
\newcommand{\eigensystem}[3]{\lambda&amp;=#2&amp;\eigenspace{#1}{#2}&amp;=\spn{\set{#3}}} 
%
%  Generalized Eigenspace
%  Usage: \geneigenspace{lin-trans-letter}{eigenvalue-letter}
\newcommand{\geneigenspace}[2]{\mathcal{G}_{#1}\left(#2\right)}
%
%  Algebraic multiplicty
%  Usage: \algmult{matrix-letter}{eigenvalue-letter}
\newcommand{\algmult}[2]{\alpha_{#1}\left(#2\right)}
%
%  Geometric multiplicty
%  Usage: \geomult{matrix-letter}{eigenvalue-letter}
\newcommand{\geomult}[2]{\gamma_{#1}\left(#2\right)}
%
%  Index (of eigenvalue)
%  Usage: \indx{matrix-letter}{eigenvalue-letter}
\newcommand{\indx}[2]{\iota_{#1}\left(#2\right)}
%
%%%%%%%%%%%%%%%%%%%%%
%
%     Linear Transformations
%
%%%%%%%%%%%%%%%%%%%%%
%
%  MathJax defines \lt to ease XML confusion
%
%  Linear transformation definition
%  Usage: \ltdefn{name-letter}{domain}{range}
\newcommand{\ltdefn}[3]{#1\colon #2\rightarrow#3}
%
%  Linear transformation evaluation
%  Usage: \lteval{name-letter}{input}
%  Replaces old \lt desired by MathJax
\newcommand{\lteval}[2]{#1\left(#2\right)}
%
% Linear transformation inverse
%  Usage: \ltinverse{name-letter}
\newcommand{\ltinverse}[1]{#1^{-1}}
%
%  Linear transformation restriction
%  Usage: \restrict{name-letter}{subspace-letter}
\newcommand{\restrict}[2]{{#1}|_{#2}}
%
%  Linear transformation preimage
%  Usage: \preimage{name-letter}{codomain-element}
\newcommand{\preimage}[2]{#1^{-1}\left(#2\right)}
%
%  Range of a linear transformation
%  TeX uses \range for something else
%  Usage:  \rng{T}
\newcommand{\rng}[1]{\mathcal{R}\!\left(#1\right)}
%
%  Kernel of a linear transformation
%  TeX uses \ker to do something different
%  Usage:  \krn{T}
\newcommand{\krn}[1]{\mathcal{K}\!\left(#1\right)}
%
%  Linear transformation composition
%  Usage: \compose{function-name}{function-name}
\newcommand{\compose}[2]{{#1}\circ{#2}}
%
%  Vector space of linear transformations
%  Usage: \vslt{domains}{codomains}
%  Presumes math mode
\newcommand{\vslt}[2]{\mathcal{LT}\left(#1,\,#2\right)}
%
%%%%%%%%%%%%%%%%%%%%%
%
%     Vector and Matrix Representations
%
%%%%%%%%%%%%%%%%%%%%%
%
%  Isomorphism symbol
%  Usage: \isomorphic
\newcommand{\isomorphic}{\cong}
%
%  Similarity
%  Usage: \similar{inner-matrix}{outer-invertible-matrix}
%  Rearranging this will not "fix" all desired changes throughout
%
\newcommand{\similar}[2]{\inverse{#2}#1#2}
%
%  Vector representation function name
%  Usage: \vectrepname{basis-letter}
\newcommand{\vectrepname}[1]{\rho_{#1}}
%
%  Vector representation output
%  Usage: \vectrep{basis-letter}{input}
\newcommand{\vectrep}[2]{\lteval{\vectrepname{#1}}{#2}}
%
%  Vector representation inverse function name
%  (Added later, not used consistently in FCLA)
%  Usage: \vectrepinvname{basis-letter}
\newcommand{\vectrepinvname}[1]{\ltinverse{\vectrepname{#1}}}
%
%  Vector representation inverse output
%  Usage: \vectrepinv{basis-letter}{input}
\newcommand{\vectrepinv}[2]{\lteval{\ltinverse{\vectrepname{#1}}}{#2}}
%
%  Matrix representation
%  Usage: \matrixrep{transformation-letter}{domain-basis-letter}{codomain-basis-letter}
\newcommand{\matrixrep}[3]{M^{#1}_{#2,#3}}
%
%  Matrix representation column-by-colum
%  2016-07-20 only employed once?
%  Usage: \matrixrepcolumns{transformation-letter}{codomain-basis-letter}{codomain-basis-vector-letter}{final-index}
\newcommand{\matrixrepcolumns}[4]{\left\lbrack \left.\vectrep{#2}{\lteval{#1}{\vect{#3}_{1}}}\right|\left.\vectrep{#2}{\lteval{#1}{\vect{#3}_{2}}}\right|\left.\vectrep{#2}{\lteval{#1}{\vect{#3}_{3}}}\right|\ldots\left|\vectrep{#2}{\lteval{#1}{\vect{#3}_{#4}}}\right.\right\rbrack}
%
%  Change of basis matrix
%  Usage: \cbm{domain-basis-letter}{codomain-basis-letter}
\newcommand{\cbm}[2]{C_{#1,#2}}
%
%%%%%%%%%%%%%%%%%%%%%
%
%     Canonical Forms
%
%%%%%%%%%%%%%%%%%%%%%
%
%  Jordan blocks
%  Usage: \jordan{size}{diagonal-element}
\newcommand{\jordan}[2]{J_{#1}\left(#2\right)}
%
%%%%%%%%%%%%%%%%%%%%%
%
%     Hadamard Matrices
%     Contributed by Elizabeth Million
%
%%%%%%%%%%%%%%%%%%%%%
%
%  Hadamard Product
%  Usage: \hadamard{a-matrix}{a-matrix}
\newcommand{\hadamard}[2]{#1\circ #2}
%
%  Hadamard identity matrix
%  Usage: \hadamardidentity{paired-subscripts-size-of-matrix}
\newcommand{\hadamardidentity}[1]{J_{#1}}
%
%  Hadamard inverse matrix
%  Usage: \hadamardinverse{matrix-expression}
\newcommand{\hadamardinverse}[1]{\widehat{#1}}


\title{Unitary Matrices}

\begin{document}
\begin{abstract}
  The inverse of a square matrix, and solutions to linear systems with square coefficient matrices, are intimately connected.
\end{abstract}
\maketitle

Recall that the adjoint of a matrix is
$\adjoint{A}=\transpose{\left(\conjugate{A}\right)}$.

\begin{definition}[Unitary Matrices]
  Suppose that $U$ is a square matrix of size $n$ such that
  $\adjoint{U}U=I_n$.  Then we say $U$ is \dfn{unitary}.
\end{definition}

This condition may seem rather far-fetched at first glance.  Would
there be \textit{any} matrix that behaved this way?  Well, yes, here
is one.

\begin{example}[Unitary matrix of size 3]
  Consider
  \[
    U=
    \begin{bmatrix}
      \frac{1 + i }{{\sqrt{5}}} &
      \frac{3 + 2\,i }{{\sqrt{55}}} &
      \frac{2+2i}{\sqrt{22}} \\
      \frac{1 - i }{{\sqrt{5}}} &
      \frac{2 + 2\,i }{{\sqrt{55}}} &
      \frac{-3 + i }{{\sqrt{22}}} \\
      \frac{i }{{\sqrt{5}}} &
      \frac{3 - 5\,i }{{\sqrt{55}}} &
      -\frac{2}{\sqrt{22}}
    \end{bmatrix}
  \]
  The computations get a bit tiresome, but if you work your way through the computation of $\adjoint{U}U$, you \textit{will} arrive at the $3\times 3$ identity matrix $I_3$.
\end{example}


Unitary matrices do not have to look quite so gruesome.  Here is a larger one that is a bit more pleasing.

\begin{example}[Unitary permutation matrix]
  The matrix
  \[
    P=
    \begin{bmatrix}
      0&1&0&0&0\\
      0&0&0&1&0\\
      1&0&0&0&0\\
      0&0&0&0&1\\
      0&0&1&0&0
    \end{bmatrix}
  \]
  is unitary as can be easily checked.  Notice that it is just a rearrangement of the columns of the $5\times 5$ identity matrix, $I_5$ (\ref{definition:IM}).

  An interesting exercise is to build another $5\times 5$ unitary
  matrix, $R$, using a different rearrangement of the columns of
  $I_5$.  Then form the product $PR$.  This will be another unitary
  matrix.  If you were to build all
  $5!=5\times 4\times 3\times 2\times 1=120$ matrices of this type you
  would have a set that remains closed under matrix multiplication.
  It is an example of another algebraic structure known as a
  \dfn{group} since together the set and the one operation (matrix
  multiplication here) is closed, associative, has an identity
  ($I_5$), and inverses (\ref{theorem:UMI}).  Notice though that the
  operation in this group is not commutative!
\end{example}

If a matrix $A$ has only real number entries (we say it is a \dfn{real
  matrix}) then the defining property of being unitary simplifies to
$\transpose{A}A=I_n$.  In this case we (and everybody else!) call the
matrix \dfn{orthogonal}, so you may often encounter this term in your
other reading when the complex numbers are not under consideration.

Unitary matrices have easily computed inverses.  They also have
columns that form orthonormal sets.  Here are the theorems that show
us that unitary matrices are not as strange as they might initially
appear.

\begin{theorem}[Unitary Matrices are Invertible]
  \label{theorem:UMI}
  Suppose that $U$ is a unitary matrix of size $n$.  Then $U$ is
  nonsingular, and $\inverse{U}=\adjoint{U}$.

  \begin{proof}
    By \ref{definition:UM}, we know that $\adjoint{U}U=I_n$.  The
    matrix $I_n$ is nonsingular (since it row-reduces easily to $I_n$,
    \ref{theorem:NMRRI}).  So by \ref{theorem:NPNT}, $U$ and
    $\adjoint{U}$ are both nonsingular matrices.

    The equation $\adjoint{U}U=I_n$ gets us halfway to an inverse of
    $U$, and \ref{theorem:OSIS} tells us that then $U\adjoint{U}=I_n$
    also.  So $U$ and $\adjoint{U}$ are inverses of each other
    (\ref{definition:MI}).

\end{proof}
\end{theorem}

\begin{theorem}[Columns of Unitary Matrices are Orthonormal Sets]
\label{theorem:CUMOS}

Suppose that $S=\set{\vectorlist{A}{n}}$ is the set of columns of a
square matrix $A$ of size $n$.  Then $A$ is a unitary matrix if and
only if $S$ is an orthonormal set.

\begin{proof}
  The proof revolves around recognizing that a typical entry of the product $\adjoint{A}A$ is an inner product of columns of $A$.  Here are the details to support this claim.
  \begin{align*}
    \matrixentry{\adjoint{A}A}{ij}
    &=\sum_{k=1}^{n}\matrixentry{\adjoint{A}}{ik}\matrixentry{A}{kj}
    &&\ref{theorem:EMP}\\
    &=\sum_{k=1}^{n}\matrixentry{\transpose{\conjugate{A}}}{ik}\matrixentry{A}{kj}
    &&\ref{theorem:EMP}\\
    &=\sum_{k=1}^{n}\matrixentry{\,\conjugate{A}\,}{ki}\matrixentry{A}{kj}
    &&\ref{definition:TM}\\
    &=\sum_{k=1}^{n}\conjugate{\matrixentry{A}{ki}}\matrixentry{A}{kj}
    &&\ref{definition:CCM}\\
    &=\sum_{k=1}^{n}\conjugate{\vectorentry{\vect{A}_i}{k}}\vectorentry{\vect{A}_j}{k}\\
    &=\innerproduct{\vect{A}_i}{\vect{A}_j}
    &&\ref{definition:IP}
  \end{align*}
  
  We now employ this equality in a chain of equivalences,
  \begin{align*}
    &\text{$S=\set{\vectorlist{A}{n}}$ is an orthonormal set}\\
    &\iff \innerproduct{\vect{A}_i}{\vect{A}_j}=
      \begin{cases}
        0 &\text{if $i\neq j$}\\
        1 & \text{if $i=j$}
      \end{cases}&&\ref{definition:ONS}\\
    &\iff \matrixentry{\adjoint{A}A}{ij}=
      \begin{cases}
        0 &\text{if $i\neq j$}\\
        1 & \text{if $i=j$}
      \end{cases}\\
    &\iff \matrixentry{\adjoint{A}A}{ij}=\matrixentry{I_n}{ij},\ 1\leq i\leq n,\ 1\leq j\leq n
    &&\ref{definition:IM}\\
    &\iff \adjoint{A}A=I_n
    &&\ref{definition:ME}\\
    &\iff \text{$A$ is a unitary matrix}
    &&\ref{definition:UM}
  \end{align*}

\end{proof}
\end{theorem}


\begin{question}
  Consider
  \[
    A=\begin{bmatrix}
      \frac{1}{\sqrt{22}}\left(4+2i\right) & \frac{1}{\sqrt{374}}\left(5+3i\right) \\
      \frac{1}{\sqrt{22}}\left(-1-i\right) & \frac{1}{\sqrt{374}}\left(12+14i\right) \\
    \end{bmatrix}
  \]
  Is the matrix $A$ unitary?

  \begin{multipleChoice}
    \choice{Yes.}
    \choice[correct]{No.}
  \end{multipleChoice}

  \begin{feedback}[correct]
    Consider the first column
    \[
      \vec{v} = \begin{bmatrix}\frac{1}{\sqrt{22}}\left(4+2i\right) \\
        \frac{1}{\sqrt{22}}\left(-1-i\right) 
    \end{bmatrix}.
    \]
    But the inner product of $\vec{v}$ with itself is not one, since it is
    \[
      \frac{(4+2i)(4-2i)}{22} + \frac{(-1-i)(-1+i)}{22},
    \]
    which is BADBAD
  \end{feedback}
\end{question}

\begin{example}[Orthonormal set from matrix columns]

  The matrix
  \[
    U=
    \begin{bmatrix}
      \frac{1 + i }{{\sqrt{5}}} &
      \frac{3 + 2\,i }{{\sqrt{55}}} &
      \frac{2+2i}{\sqrt{22}} \\
      \frac{1 - i }{{\sqrt{5}}} &
      \frac{2 + 2\,i }{{\sqrt{55}}} &
      \frac{-3 + i }{{\sqrt{22}}} \\
      \frac{i }{{\sqrt{5}}} &
      \frac{3 - 5\,i }{{\sqrt{55}}} &
      -\frac{2}{\sqrt{22}}
    \end{bmatrix}
  \]
  is a unitary matrix.  By \ref{theorem:CUMOS}, its columns
  \[
    \set{
      \colvector{
        \frac{1 + i }{{\sqrt{5}}}\\
        \frac{1 - i }{{\sqrt{5}}}\\
        \frac{i }{{\sqrt{5}}}
      },\,
      \colvector{
        \frac{3 + 2\,i }{{\sqrt{55}}}\\
        \frac{2 + 2\,i }{{\sqrt{55}}}\\
        \frac{3 - 5\,i }{{\sqrt{55}}}
      },\,
      \colvector{
        \frac{2+2i}{\sqrt{22}}\\
        \frac{-3 + i }{{\sqrt{22}}}\\
        -\frac{2}{\sqrt{22}}
      }
    }
  \]
  form an orthonormal set.  You might find checking the six inner
  products of pairs of these vectors easier than doing the matrix
  product $\adjoint{U}U$.  Or, because the inner product is
  anti-commutative (\ref{theorem:IPAC}) you only need check three
  inner products.
\end{example}

When using vectors and matrices that only have real number entries,
orthogonal matrices are those matrices with inverses that equal their
transpose.  Similarly, the inner product is the familiar dot product.
Keep this special case in mind as you read the next theorem.

\begin{theorem}[Unitary Matrices Preserve Inner Products]
  \label{theorem:UMPIP}

  Suppose that $U$ is a unitary matrix of size $n$ and $\vect{u}$ and $\vect{v}$ are two vectors from $\complex{n}$.  Then
  \begin{align*}
    \innerproduct{U\vect{u}}{U\vect{v}}&=\innerproduct{\vect{u}}{\vect{v}}
    &
    &\text{and}
    &
      \norm{U\vect{v}}&=\norm{\vect{v}}
  \end{align*}

  \begin{proof}
    \begin{align*}
      \innerproduct{U\vect{u}}{U\vect{v}}
      &=\transpose{\left(\conjugate{U\vect{u}}\right)}U\vect{v}
      &&\ref{theorem:MMIP}\\
      &=\transpose{\left(\conjugate{U}\conjugate{\vect{u}}\right)}U\vect{v}
      &&\ref{theorem:MMCC}\\
      &=\transpose{\conjugate{\vect{u}}}\transpose{\conjugate{U}}U\vect{v}
      &&\ref{theorem:MMT}\\
      &=\transpose{\conjugate{\vect{u}}}\adjoint{U}U\vect{v}
      &&\ref{definition:A}\\
      &=\transpose{\conjugate{\vect{u}}}I_n\vect{v}
      &&\ref{definition:UM}\\
      &=\transpose{\conjugate{\vect{u}}}\vect{v}
      &&\ref{theorem:MMIM}\\
      &=\innerproduct{\vect{u}}{\vect{v}}
      &&\ref{theorem:MMIP}
    \end{align*}
    
    The second conclusion is just a specialization of the first conclusion.
    \begin{align*}
      \norm{U\vect{v}}
      &=\sqrt{\norm{U\vect{v}}^2}\\
      &=\sqrt{\innerproduct{U\vect{v}}{U\vect{v}}}&&\ref{theorem:IPN}\\
      &=\sqrt{\innerproduct{\vect{v}}{\vect{v}}}\\
      &=\sqrt{\norm{\vect{v}}^2}&&\ref{theorem:IPN}\\
      &=\norm{\vect{v}}
    \end{align*}
    
  \end{proof}
\end{theorem}

Aside from the inherent interest in this theorem, it makes a bigger
statement about unitary matrices.  When we view vectors geometrically
as directions or forces, then the norm equates to a notion of length.
If we transform a vector by multiplication with a unitary matrix, then
the length (norm) of that vector stays the same.  If we consider
column vectors with two or three slots containing only real numbers,
then the inner product of two such vectors is just the dot product,
and this quantity can be used to compute the angle between two
vectors.  When two vectors are multiplied (transformed) by the same
unitary matrix, their dot product is unchanged and their individual
lengths are unchanged.  This results in the angle between the two
vectors remaining unchanged.


A ``unitary transformation'' (matrix-vector products with unitary
matrices) thus preserve geometrical relationships among vectors
representing directions, forces, or other physical quantities.  In the
case of a two-slot vector with real entries, this is simply a
rotation.  These sorts of computations are exceedingly important in
computer graphics such as games and real-time simulations, especially
when increased realism is achieved by performing many such
computations quickly.  We will see unitary matrices again in
subsequent sections (especially \ref{theorem:OD}) and in each
instance, consider the interpretation of the unitary matrix as a sort
of geometry-preserving transformation.  Some authors use the term
\dfn{isometry} to highlight this behavior.  We will speak loosely of a
unitary matrix as being a sort of generalized rotation.

\end{document}
