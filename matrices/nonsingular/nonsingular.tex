\documentclass{ximera}

% These macros are automatically generated from the "macros"
% XML element.  Make permanent edits there.
%
% History
%   2004/01/01  Initiated for FCLA, evolved from there
%   2006/09/17  Converted  _, ^  to \sb, \sp for TeX4ht
%   2014/02/01  Updated for MathBook XML projects
%               Obsolete in FCLA: \codeindent, \computerfont, \define
%               Change: MathJax wants \lt, so replaced by \lteval
%   2014/02/22  New: \orderof, \reals, \per
%   2015/08/16  Incorporated into MathBook XML version of FCLA
%
%%%%%%%%%%%%%%%%%%%%%
%
%     Conveniences
%
%%%%%%%%%%%%%%%%%%%%%
%
%  Order of (asymptotically limit of fraction is 1)
%  Usage: \orderof{some function}
%
\newcommand{\orderof}[1]{\sim #1}
%
%  Integers
%  Usage:  \Z
\newcommand{\Z}{\mathbb{Z}}
%
%  Real numbers, as set of scalars
%  Usage:  \reals
\newcommand{\reals}{\mathbb{R}}
%
%  n-space over real field
%  Usage: \complex{integer-dimension}
\newcommand{\real}[1]{\mathbb{R}^{#1}}
%
%  Complex numbers, as set of scalars
%  Usage:  \complexes
\newcommand{\complexes}{\mathbb{C}}
%
%  n-space over complex field
%  Usage: \complex{integer-dimension}
\newcommand{\complex}[1]{\mathbb{C}^{#1}}
%
%  Complex conjugation (scalar, vector, matrix)
%  Usage: \conjugate{object}
\newcommand{\conjugate}[1]{\overline{#1}}
%
%  Complex number modulus
%  Usage: \modulus{a+bi}
%  Presumes math mode
\newcommand{\modulus}[1]{\left\lvert#1\right\rvert}
%
%  Zero vector
%  Usage: \zerovector
\newcommand{\zerovector}{\vect{0}}
%
%  Zero matrix
%  Usage: \zeromatrix, use a subscript when size is important
\newcommand{\zeromatrix}{\mathcal{O}}
%
%  Inner product (brackets, not quadratic form)
%  Usage: \innerproduct{a-vector}{a-vector}
\newcommand{\innerproduct}[2]{\left\langle#1,\,#2\right\rangle}
%
%  Norm of a vector
%  Usage: \norm{a-vector}
\newcommand{\norm}[1]{\left\lVert#1\right\rVert}
%
%  Dimension
%  Usage: \dimension{vector-space-letter}
\newcommand{\dimension}[1]{\dim\left(#1\right)}
%
%  Nullity
%  Usage: \nullity{matrix-or-lintrans-letter}
\newcommand{\nullity}[1]{n\left(#1\right)}
%
%  Rank
%  Usage: \rank{matrix-or-lintrans-letter}
\newcommand{\rank}[1]{r\left(#1\right)}
%
%  Direct sum
%  Usage: \ds between a couple of subspaces
%
\newcommand{\ds}{\oplus}
%
%  Determinant of a matrix (functional)
%  Usage: \detname{A}
\newcommand{\detname}[1]{\det\left(#1\right)}
%
%  Determinant of a matrix (vertical bars)
%  Usage: \detbars{A}
\newcommand{\detbars}[1]{\left\lvert#1\right\rvert}
%
%  Trace of a Matrix
%  Usage: \trace{matrix name}
\newcommand{\trace}[1]{t\left(#1\right)}
%
%  Square Root of a Matrix
%  Usage: \sr{a-matrix}
\newcommand{\sr}[1]{#1^{1/2}}
%
%%%%%%%%%%%%%%%%%%%%%
%
%     Subspace Constructions
%
%%%%%%%%%%%%%%%%%%%%%
%
%  Span of a set of vectors
%  \span and \sp are used by TeX for other things
%  Usage: \spn{set-of-vectors}
\newcommand{\spn}[1]{\left\langle#1\right\rangle}
%
%  Null space of a matrix
%  Usage:  \nsp{A}
\newcommand{\nsp}[1]{\mathcal{N}\!\left(#1\right)}
%
%  Column space of a matrix
%  Usage:  \csp{A}
\newcommand{\csp}[1]{\mathcal{C}\!\left(#1\right)}
%
%  Row space of a matrix
%  Usage:  \rsp{A}
\newcommand{\rsp}[1]{\mathcal{R}\!\left(#1\right)}
%
%  Left null space of a matrix
%  Usage:  \lns{A}
\newcommand{\lns}[1]{\mathcal{L}\!\left(#1\right)}
%
%  Orthogonal complement of a vector space
%  Avoiding TeX's \perp
%  Usage:  \per{A}
\newcommand{\per}[1]{#1^\perp}
%
%%%%%%%%%%%%%%%%%%%%%
%
%     Systems of Equations
%
%%%%%%%%%%%%%%%%%%%%%
%
%  In-line form of an augmented matrix for a system of equations
%  Usage: \augmented{coefficient-matrix}{constant-vector}
\newcommand{\augmented}[2]{\left\lbrack\left.#1\,\right\rvert\,#2\right\rbrack}
%
%  Notation for a linear system before introducing matrix multiplication
%  Usage: \linearsystem{coefficient-matrix}{constant-vector}
\newcommand{\linearsystem}[2]{\mathcal{LS}\!\left(#1,\,#2\right)}
%
%  Notation for a homogenous system before introducing matrix multiplication
%  Usage: \homosystem{coefficient-matrix}
\newcommand{\homosystem}[1]{\linearsystem{#1}{\zerovector}}
%
%%%%%%%%%%%%%%%%%%%%%
%
%     Row Operations, Echelon Form
%
%%%%%%%%%%%%%%%%%%%%%
%
% Row operations on matrices
%
% Three commands to shorten up descriptions of gaussian elimination
%
% Usage: \rowopswap{row-i}{row-j}
% Usage: \rowopmult{scalar}{row-i}
% Usage: \rowopadd{scalar}{row-multiplied}{row-added-to}
\newcommand{\rowopswap}[2]{R_{#1}\leftrightarrow R_{#2}}
\newcommand{\rowopmult}[2]{#1R_{#2}}
\newcommand{\rowopadd}[3]{#1R_{#2}+R_{#3}}
%
% Mark leading 1's in echelon form with fbox
% Usage: \leading{a-1-usually}
\newcommand{\leading}[1]{\fbox{#1}}
%
%  Row-reduce arrow
%  Usage:  \rref inbetween a matrix and its reduced row-echelon form
\newcommand{\rref}{\xrightarrow{\text{RREF}}}
%
%  Elementary Matrices
%  Usage: \elemswap{subscript}{subscript}
%  Usage: \elemmult{scalar}{subscript}
%  Usage: \elemadd{scalar}{subscript-mult}{subscript-target}
%
\newcommand{\elemswap}[2]{E_{#1,#2}}
\newcommand{\elemmult}[2]{E_{#2}\left(#1\right)}
\newcommand{\elemadd}[3]{E_{#2,#3}\left(#1\right)}
%
%%%%%%%%%%%%%%%%%%%%%
%
%     2-D Constructions (Lists, Vectors, Matrices)
%
%%%%%%%%%%%%%%%%%%%%%
%
%  A list of scalars of generic length
%  Usage:  \scalarlist{scalar letter}{terminal subscript}
\newcommand{\scalarlist}[2]{{#1}_{1},\,{#1}_{2},\,{#1}_{3},\,\ldots,\,{#1}_{#2}}
%
%  Vector styling, bold (or use wiggles, arrows, whatever)
%  Subscripts go outside this construction
%  Usage: \vect{a symbol to use as a vector}
%  Have to already be in math mode
%
\newcommand{\vect}[1]{\mathbf{#1}}
%
%  A column vector
%  Usage: \colvector{list-delimited-by-\\}
%
\newcommand{\colvector}[1]{\begin{bmatrix}#1\end{bmatrix}}
%
%  A generic vector with components
%  Usage: \vectorcomponents{component-letter}{final-subscript}
\newcommand{\vectorcomponents}[2]{\colvector{#1_{1}\\#1_{2}\\#1_{3}\\\vdots\\#1_{#2}}}
%
%  A list of vectors of generic length
%  Usage:  \vectorlist{vector letter}{terminal subscript}
\newcommand{\vectorlist}[2]{\vect{#1}_{1},\,\vect{#1}_{2},\,\vect{#1}_{3},\,\ldots,\,\vect{#1}_{#2}}
%
%  Vector entries, entry i of vector v
%  (vector-expession still needs \vect, etc.)
%  Usage:  \vectorentry{vector-expression}{single-subscript}
\newcommand{\vectorentry}[2]{\left\lbrack#1\right\rbrack_{#2}}
%
%  Matrix entries, entry i,j of matrix A
%  Usage:  \matrixentry{matrix-expression}{paired-subscripts}
%
\newcommand{\matrixentry}[2]{\left\lbrack#1\right\rbrack_{#2}}
%
%  A generic linear combination
%  Usage:  \lincombo{scalar letter}{vector letter}{terminal subscript}
\newcommand{\lincombo}[3]{#1_{1}\vect{#2}_{1}+#1_{2}\vect{#2}_{2}+#1_{3}\vect{#2}_{3}+\cdots +#1_{#3}\vect{#2}_{#3}}
%
%  Matrix, column by column, as vectors
%  Usage:  \matrixcolumns{matrix letter}{terminal subscript}
\newcommand{\matrixcolumns}[2]{\left\lbrack\vect{#1}_{1}|\vect{#1}_{2}|\vect{#1}_{3}|\ldots|\vect{#1}_{#2}\right\rbrack}
%
%%%%%%%%%%%%%%%%%%%%%
%
%     Special Matrices
%
%%%%%%%%%%%%%%%%%%%%%
%
%  Transpose of a matrix
%  Usage:  \transpose{A}
\newcommand{\transpose}[1]{#1^{t}}
%
%  Inverse of a matrix
%  Usage:  \inverse{A}
\newcommand{\inverse}[1]{#1^{-1}}
%
%  Submatrix (for minors, determinants)
%  Usage: \submatrix{matrix-name}{delete-row}{delete-col}
\newcommand{\submatrix}[3]{#1\left(#2|#3\right)}
%
%  Adjoint of a matrix (twice)
%  This macro is a convenience to call \transpose and \conjugate properly
%  It shouldn't need to be modified (or mathematical meanings will change)
%  Usage:  \adj{A}
\newcommand{\adj}[1]{\transpose{\left(\conjugate{#1}\right)}}
%
%  This macro controls the symbol used for the adjoint
%  It can be edited to taste
%  Usage:  \adjoint{A}
\newcommand{\adjoint}[1]{#1^\ast}
%
%%%%%%%%%%%%%%%%%%%%%
%
%     Sets
%
%%%%%%%%%%%%%%%%%%%%%
%
%  A convenience for simple sets
%  Usage:  \set{list of element}
\newcommand{\set}[1]{\left\{#1\right\}}
%
%  Sets with vertical bar, "such that", sized for objects, not condition
%  Usage:  \setparts{objects}{condition}
%
%%\newcommand{\setparts}[2]{\left\{ #1\mid#2\right\}}
%%\newcommand{\setparts}[2]{\left\{\left. #1\right\rvert#2\right\}}
\newcommand{\setparts}[2]{\left\lbrace#1\,\middle|\,#2\right\rbrace}
%
%  Set Cardinality
%  Usage:  \card{a-set-letter}
\newcommand{\card}[1]{\left\lvert#1\right\rvert}
%
%  Set Union
%  Use \cup
%
%  Set Intersection
%  Use \cap
%
%  Set Complement
%  Usage:  \setcomplement{a-set-letter}
\newcommand{\setcomplement}[1]{\overline{#1}}
%
%%%%%%%%%%%%%%%%%%%%%
%
%     Eigenvalues and Eigenspaces
%
%%%%%%%%%%%%%%%%%%%%%
%
%  Characteristic polynomial
%  Usage: \charpoly{matrix-letter}{variable-letter}
\newcommand{\charpoly}[2]{p_{#1}\left(#2\right)}
%
%  Eigenspace
%  Usage: \eigenspace{matrix-letter}{eigenvalue-letter}
\newcommand{\eigenspace}[2]{\mathcal{E}_{#1}\left(#2\right)}
%
%  2013/10/03 Including ampersands is problematic here, 
%  think about fixes later
%  2014/02/22 Limited testing, seems &amp; is fine for HTML and LaTeX
%  2016-07-20 only employed in Archetypes, MBX has gather/align override
%  Eigensystem (presumes wrapped in an mrow within md)
%  Usage: \eigensystem{matrixletter}{eigenvalue}{list of basis vectors}
\newcommand{\eigensystem}[3]{\lambda&amp;=#2&amp;\eigenspace{#1}{#2}&amp;=\spn{\set{#3}}} 
%
%  Generalized Eigenspace
%  Usage: \geneigenspace{lin-trans-letter}{eigenvalue-letter}
\newcommand{\geneigenspace}[2]{\mathcal{G}_{#1}\left(#2\right)}
%
%  Algebraic multiplicty
%  Usage: \algmult{matrix-letter}{eigenvalue-letter}
\newcommand{\algmult}[2]{\alpha_{#1}\left(#2\right)}
%
%  Geometric multiplicty
%  Usage: \geomult{matrix-letter}{eigenvalue-letter}
\newcommand{\geomult}[2]{\gamma_{#1}\left(#2\right)}
%
%  Index (of eigenvalue)
%  Usage: \indx{matrix-letter}{eigenvalue-letter}
\newcommand{\indx}[2]{\iota_{#1}\left(#2\right)}
%
%%%%%%%%%%%%%%%%%%%%%
%
%     Linear Transformations
%
%%%%%%%%%%%%%%%%%%%%%
%
%  MathJax defines \lt to ease XML confusion
%
%  Linear transformation definition
%  Usage: \ltdefn{name-letter}{domain}{range}
\newcommand{\ltdefn}[3]{#1\colon #2\rightarrow#3}
%
%  Linear transformation evaluation
%  Usage: \lteval{name-letter}{input}
%  Replaces old \lt desired by MathJax
\newcommand{\lteval}[2]{#1\left(#2\right)}
%
% Linear transformation inverse
%  Usage: \ltinverse{name-letter}
\newcommand{\ltinverse}[1]{#1^{-1}}
%
%  Linear transformation restriction
%  Usage: \restrict{name-letter}{subspace-letter}
\newcommand{\restrict}[2]{{#1}|_{#2}}
%
%  Linear transformation preimage
%  Usage: \preimage{name-letter}{codomain-element}
\newcommand{\preimage}[2]{#1^{-1}\left(#2\right)}
%
%  Range of a linear transformation
%  TeX uses \range for something else
%  Usage:  \rng{T}
\newcommand{\rng}[1]{\mathcal{R}\!\left(#1\right)}
%
%  Kernel of a linear transformation
%  TeX uses \ker to do something different
%  Usage:  \krn{T}
\newcommand{\krn}[1]{\mathcal{K}\!\left(#1\right)}
%
%  Linear transformation composition
%  Usage: \compose{function-name}{function-name}
\newcommand{\compose}[2]{{#1}\circ{#2}}
%
%  Vector space of linear transformations
%  Usage: \vslt{domains}{codomains}
%  Presumes math mode
\newcommand{\vslt}[2]{\mathcal{LT}\left(#1,\,#2\right)}
%
%%%%%%%%%%%%%%%%%%%%%
%
%     Vector and Matrix Representations
%
%%%%%%%%%%%%%%%%%%%%%
%
%  Isomorphism symbol
%  Usage: \isomorphic
\newcommand{\isomorphic}{\cong}
%
%  Similarity
%  Usage: \similar{inner-matrix}{outer-invertible-matrix}
%  Rearranging this will not "fix" all desired changes throughout
%
\newcommand{\similar}[2]{\inverse{#2}#1#2}
%
%  Vector representation function name
%  Usage: \vectrepname{basis-letter}
\newcommand{\vectrepname}[1]{\rho_{#1}}
%
%  Vector representation output
%  Usage: \vectrep{basis-letter}{input}
\newcommand{\vectrep}[2]{\lteval{\vectrepname{#1}}{#2}}
%
%  Vector representation inverse function name
%  (Added later, not used consistently in FCLA)
%  Usage: \vectrepinvname{basis-letter}
\newcommand{\vectrepinvname}[1]{\ltinverse{\vectrepname{#1}}}
%
%  Vector representation inverse output
%  Usage: \vectrepinv{basis-letter}{input}
\newcommand{\vectrepinv}[2]{\lteval{\ltinverse{\vectrepname{#1}}}{#2}}
%
%  Matrix representation
%  Usage: \matrixrep{transformation-letter}{domain-basis-letter}{codomain-basis-letter}
\newcommand{\matrixrep}[3]{M^{#1}_{#2,#3}}
%
%  Matrix representation column-by-colum
%  2016-07-20 only employed once?
%  Usage: \matrixrepcolumns{transformation-letter}{codomain-basis-letter}{codomain-basis-vector-letter}{final-index}
\newcommand{\matrixrepcolumns}[4]{\left\lbrack \left.\vectrep{#2}{\lteval{#1}{\vect{#3}_{1}}}\right|\left.\vectrep{#2}{\lteval{#1}{\vect{#3}_{2}}}\right|\left.\vectrep{#2}{\lteval{#1}{\vect{#3}_{3}}}\right|\ldots\left|\vectrep{#2}{\lteval{#1}{\vect{#3}_{#4}}}\right.\right\rbrack}
%
%  Change of basis matrix
%  Usage: \cbm{domain-basis-letter}{codomain-basis-letter}
\newcommand{\cbm}[2]{C_{#1,#2}}
%
%%%%%%%%%%%%%%%%%%%%%
%
%     Canonical Forms
%
%%%%%%%%%%%%%%%%%%%%%
%
%  Jordan blocks
%  Usage: \jordan{size}{diagonal-element}
\newcommand{\jordan}[2]{J_{#1}\left(#2\right)}
%
%%%%%%%%%%%%%%%%%%%%%
%
%     Hadamard Matrices
%     Contributed by Elizabeth Million
%
%%%%%%%%%%%%%%%%%%%%%
%
%  Hadamard Product
%  Usage: \hadamard{a-matrix}{a-matrix}
\newcommand{\hadamard}[2]{#1\circ #2}
%
%  Hadamard identity matrix
%  Usage: \hadamardidentity{paired-subscripts-size-of-matrix}
\newcommand{\hadamardidentity}[1]{J_{#1}}
%
%  Hadamard inverse matrix
%  Usage: \hadamardinverse{matrix-expression}
\newcommand{\hadamardinverse}[1]{\widehat{#1}}


\title{Nonsingular Matrices are Invertible}

\begin{document}
\begin{abstract}
  Although we have defined them differently, a matrix being
  nonsingular and a matrix being invertible amount to the same thing.
\end{abstract}
\maketitle

If a square matrix $A$ is nonsingular, then there is a matrix $B$ so
that $AB=I_n$.  In other words, $B$ is halfway to being an inverse of
$A$.  We will see in this section that $B$ automatically fulfills the
second condition ($BA=I_n$).  We have seen examples in which the
coefficient matrix had no inverse; not coincidentally, in those cases
the coefficient matrix was singular.  We will make all these
connections precise now.  Not many examples or definitions in this
section, just theorems.

We need a couple of technical results for starters.  Some books would
call these minor, but essential, results ``lemmas.''  We'll just call
them theorems.

The first of these technical results is interesting in that the
hypothesis says something about a product of two square matrices and
the conclusion then says the same thing about each individual matrix
in the product.  This result has an analogy in the algebra of complex
numbers: suppose $\alpha,\,\beta\in\complexes$, then
$\alpha\beta\neq 0$ if and only if $\alpha\neq 0$ and $\beta\neq 0$.
We can view this result as suggesting that the term ``nonsingular''
for matrices is like the term ``nonzero'' for scalars.  Consider too
that we know singular matrices, as coefficient matrices for systems of
equations, will sometimes lead to systems with no solutions, or
systems with infinitely many solutions.  What do linear equations with
zero look like?  Consider $0x=5$, which has no solution, and $0x=0$,
which has infinitely many solutions.  In the algebra of scalars, zero
is exceptional (meaning different, not better), and in the algebra of
matrices, singular matrices are also the exception.  While there is
only one zero scalar, and there are infinitely many singular matrices,
we will see that singular matrices are a distinct minority.

\begin{theorem}[Nonsingular Product has Nonsingular Terms]
  \label{theorem:NPNT}
  
  Suppose that $A$ and $B$ are square matrices of size $n$.  The
  product $AB$ is nonsingular if and only if $A$ and $B$ are both
  nonsingular.

  \begin{proof}
    ($\Rightarrow$) For this portion of the proof we will form the
    logically-equivalent contrapositive and prove that statement using
    two cases.  ``$AB$ is nonsingular implies $A$ and $B$ are both
    nonsingular'' becomes ``$A$ \wordChoice{\choice[correct]{or}\choice{and}} $B$ is singular implies $AB$ is
    \wordChoice{\choice[correct]{singular}\choice{nonsingular}}.''

    \textbf{Case 1.}  Suppose $B$ is singular.  Then there is a
    nonzero vector $\vect{z}$ that is a solution to $\homosystem{B}$.
    So
    \begin{align*}
      (AB)\vect{z}
      &=A(B\vect{z})&&\ref{theorem:MMA}\\
      &=A\zerovector&&\ref{theorem:SLEMM}\\
      &=\zerovector&&\ref{theorem:MMZM}\\
    \end{align*}

    With \ref{theorem:SLEMM} we can translate this vector equality to
    the statement that $\vect{z}$ is a nonzero solution to
    $\homosystem{AB}$.  Thus $AB$ is singular (\ref{definition:NM}),
    as desired.

    \textbf{Case 2.}  Suppose $A$ is singular, and $B$ is not
    singular.  In other words, with Case 1 complete, we can be more
    precise about this remaining case and assume that $B$ is
    nonsingular.  Because $A$ is singular, there is a nonzero vector
    $\vect{y}$ that is a solution to $\homosystem{A}$.  Now consider
    the linear system $\linearsystem{B}{\vect{y}}$.  Since $B$ is
    nonsingular, the system has a unique solution
    (\ref{theorem:NMUS}), which we will denote as $\vect{w}$.  We
    first claim $\vect{w}$ is not the zero vector either.  Assuming
    the opposite, suppose that $\vect{w}=\zerovector$
    (\ref{technique:CD}).  Then
    \begin{align*}
      \vect{y}
      &=B\vect{w}&&\ref{theorem:SLEMM}\\
      &=B\zerovector&&\text{Hypothesis}\\
      &=\zerovector&&\ref{theorem:MMZM}
    \end{align*}
    contrary to $\vect{y}$ being nonzero.  So $\vect{w}\neq\zerovector$.  The pieces are in place, so here we go,
    \begin{align*}
      (AB)\vect{w}
      &=A(B\vect{w})&&\ref{theorem:MMA}\\
      &=A\vect{y}&&\ref{theorem:SLEMM}\\
      &=\zerovector&&\ref{theorem:SLEMM}\\
    \end{align*}

    With \ref{theorem:SLEMM} we can translate this vector equality to
    the statement that $\vect{w}$ is a nonzero solution to
    $\homosystem{AB}$.  Thus $AB$ is singular (\ref{definition:NM}),
    as desired.  And this conclusion holds for both cases.

    ($\Leftarrow$) Now assume that both $A$ and $B$ are nonsingular.
    Suppose that $\vect{x}\in\complex{n}$ is a solution to
    $\homosystem{AB}$.  Then
    \begin{align*}
      \zerovector
      &=\left(AB\right)\vect{x}&&\ref{theorem:SLEMM}\\
      &=A\left(B\vect{x}\right)&&\ref{theorem:MMA}
    \end{align*}

    By \ref{theorem:SLEMM}, $B\vect{x}$ is a solution to
    $\homosystem{A}$, and by the definition of a nonsingular matrix
    (\ref{definition:NM}), we conclude that $B\vect{x}=\zerovector$.
    Now, by an entirely similar argument, the nonsingularity of $B$
    forces us to conclude that $\vect{x}=\zerovector$.  So the only
    solution to $\homosystem{AB}$ is the zero vector and we conclude
    that $AB$ is nonsingular by \ref{definition:NM}.
  \end{proof}
\end{theorem}

This is a powerful result in the ``forward'' direction, because it
allows us to begin with a hypothesis that something complicated (the
matrix product $AB$) has the property of being nonsingular, and we can
then conclude that the simpler constituents ($A$ and $B$ individually)
then also have the property of being nonsingular.  If we had thought
that the matrix product was an artificial construction, results like
this would make us begin to think twice.

The contrapositive of this entire result is equally interesting.  It
says that $A$ or $B$ (or both) is a singular matrix if and only if the
product $AB$ is singular.

\begin{theorem}[One-Sided Inverse is Sufficient]
  \label{theorem:OSIS}

  Suppose $A$ and $B$ are  square matrices of size $n$ such that $AB=I_n$.  Then $BA=I_n$.

  \begin{proof}
    The matrix $I_n$ is nonsingular (since it row-reduces easily to
    $I_n$, \ref{theorem:NMRRI}).  So $A$ and $B$ are nonsingular by
    \ref{theorem:NPNT}, so in particular $B$ is nonsingular.  We can
    therefore apply \ref{theorem:CINM} to assert the existence of a
    matrix $C$ so that $BC=I_n$.  This application of
    \ref{theorem:CINM} could be a bit confusing, mostly because of the
    names of the matrices involved.  $B$ is nonsingular, so there must
    be a ``right-inverse'' for $B$, and we are calling it $C$.

    Now
    \begin{align*}
      BA
      &=(BA)I_n&&\ref{theorem:MMIM}\\
      &=(BA)(BC)&&\ref{theorem:CINM}\\
      &=B(AB)C&&\ref{theorem:MMA}\\
      &=BI_nC&&\text{Hypothesis}\\
      &=BC&&\ref{theorem:MMIM}\\
      &=I_n&&\ref{theorem:CINM}
    \end{align*}
    which is the desired conclusion.
  \end{proof}
\end{theorem}

So \ref{theorem:OSIS} tells us that if $A$ is nonsingular, then the
matrix $B$ guaranteed by \ref{theorem:CINM} will be both a
``right-inverse'' and a ``left-inverse'' for $A$, so $A$ is invertible
and $\inverse{A}=B$.

So if you have a nonsingular matrix, $A$, you can use the procedure
described in \ref{theorem:CINM} to find an inverse for $A$.  If $A$ is
singular, then the procedure in \ref{theorem:CINM} will fail as the
first $n$ columns of $M$ will not row-reduce to the identity matrix.
However, we can say a bit more.  When $A$ is singular, then $A$ does
not have an inverse (which is very different from saying that the
procedure in \ref{theorem:CINM} fails to find an inverse).  This may
feel like we are splitting hairs, but it is important that we do not
make unfounded assumptions.  These observations motivate the next
theorem.

\begin{theorem}[Nonsingularity is Invertibility]
  \label{theorem:NI}

  Suppose that $A$ is a square matrix.  Then $A$ is nonsingular if and only if $A$ is invertible.
  
  \begin{proof}
    ($\Leftarrow$) Since $A$ is invertible, we can write
    $I_n=A\inverse{A}$ (\ref{definition:MI}).  Notice that $I_n$ is
    nonsingular (\ref{theorem:NMRRI}) so \ref{theorem:NPNT} implies
    that $A$ (and $\inverse{A}$) is nonsingular.

    ($\Rightarrow$) Suppose now that $A$ is nonsingular.  By
    \ref{theorem:CINM} we find $B$ so that $AB=I_n$.  Then
    \ref{theorem:OSIS} tells us that $BA=I_n$.  So $B$ is $A$'s
    inverse, and by construction, $A$ is invertible.

  \end{proof}
\end{theorem}

So for a square matrix, the properties of having an inverse and of
having a trivial null space are one and the same.  Cannot have one
without the other.

\begin{theorem}[Nonsingular Matrix Equivalences, Round 3]
  Suppose that $A$ is a square matrix of size $n$.  The following are equivalent.
  \begin{enumerate}\item $A$ is nonsingular.
  \item $A$ row-reduces to the identity matrix.
  \item The null space of $A$ contains only the zero vector, $\nsp{A}=\set{\zerovector}$.
  \item The linear system $\linearsystem{A}{\vect{b}}$ has a unique solution for every possible choice of $\vect{b}$.
  \item The columns of $A$ are a linearly independent set.
  \item $A$ is invertible.
  \end{enumerate}
  
  \begin{proof}
    We can update our list of equivalences for nonsingular matrices
    with the equivalent condition from \ref{theorem:NI}.
  \end{proof}
\end{theorem}

In the case that $A$ is a nonsingular coefficient matrix of a system
of equations, the inverse allows us to very quickly compute the unique
solution, for any vector of constants.

\begin{theorem}[Solution with Nonsingular Coefficient Matrix]
  \label{theorem:SNCM}

  Suppose that $A$ is nonsingular.  Then the unique solution to
  $\linearsystem{A}{\vect{b}}$ is $\inverse{A}\vect{b}$.

  \begin{proof}
    By \ref{theorem:NMUS} we know already that
    $\linearsystem{A}{\vect{b}}$ has \wordChoice{\choice{infinitely
        many solutions}\choice[correct]{a unique solution}} for every
  choice of $\vect{b}$.  We need to show that the expression stated is
  indeed a solution (\textit{the} solution).  That is easy, just
  ``plug it in'' to the vector equation representation of the system
  (\ref{theorem:SLEMM}),
    \begin{align*}
      A\left(\inverse{A}\vect{b}\right)
      &=\left(A\inverse{A}\right)\vect{b}&&\ref{theorem:MMA}\\
      &=I_n\vect{b}&&\ref{definition:MI}\\
      &=\vect{b}&&\ref{theorem:MMIM}
    \end{align*}

    Since $A\vect{x}=\vect{b}$ is true when we substitute
    $\inverse{A}\vect{b}$ for $\vect{x}$, $\inverse{A}\vect{b}$ is a
    (the!) solution to $\linearsystem{A}{\vect{b}}$.
  \end{proof}
\end{theorem}

\end{document}
