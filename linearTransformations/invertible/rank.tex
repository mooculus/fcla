\documentclass{ximera}

% These macros are automatically generated from the "macros"
% XML element.  Make permanent edits there.
%
% History
%   2004/01/01  Initiated for FCLA, evolved from there
%   2006/09/17  Converted  _, ^  to \sb, \sp for TeX4ht
%   2014/02/01  Updated for MathBook XML projects
%               Obsolete in FCLA: \codeindent, \computerfont, \define
%               Change: MathJax wants \lt, so replaced by \lteval
%   2014/02/22  New: \orderof, \reals, \per
%   2015/08/16  Incorporated into MathBook XML version of FCLA
%
%%%%%%%%%%%%%%%%%%%%%
%
%     Conveniences
%
%%%%%%%%%%%%%%%%%%%%%
%
%  Order of (asymptotically limit of fraction is 1)
%  Usage: \orderof{some function}
%
\newcommand{\orderof}[1]{\sim #1}
%
%  Integers
%  Usage:  \Z
\newcommand{\Z}{\mathbb{Z}}
%
%  Real numbers, as set of scalars
%  Usage:  \reals
\newcommand{\reals}{\mathbb{R}}
%
%  n-space over real field
%  Usage: \complex{integer-dimension}
\newcommand{\real}[1]{\mathbb{R}^{#1}}
%
%  Complex numbers, as set of scalars
%  Usage:  \complexes
\newcommand{\complexes}{\mathbb{C}}
%
%  n-space over complex field
%  Usage: \complex{integer-dimension}
\newcommand{\complex}[1]{\mathbb{C}^{#1}}
%
%  Complex conjugation (scalar, vector, matrix)
%  Usage: \conjugate{object}
\newcommand{\conjugate}[1]{\overline{#1}}
%
%  Complex number modulus
%  Usage: \modulus{a+bi}
%  Presumes math mode
\newcommand{\modulus}[1]{\left\lvert#1\right\rvert}
%
%  Zero vector
%  Usage: \zerovector
\newcommand{\zerovector}{\vect{0}}
%
%  Zero matrix
%  Usage: \zeromatrix, use a subscript when size is important
\newcommand{\zeromatrix}{\mathcal{O}}
%
%  Inner product (brackets, not quadratic form)
%  Usage: \innerproduct{a-vector}{a-vector}
\newcommand{\innerproduct}[2]{\left\langle#1,\,#2\right\rangle}
%
%  Norm of a vector
%  Usage: \norm{a-vector}
\newcommand{\norm}[1]{\left\lVert#1\right\rVert}
%
%  Dimension
%  Usage: \dimension{vector-space-letter}
\newcommand{\dimension}[1]{\dim\left(#1\right)}
%
%  Nullity
%  Usage: \nullity{matrix-or-lintrans-letter}
\newcommand{\nullity}[1]{n\left(#1\right)}
%
%  Rank
%  Usage: \rank{matrix-or-lintrans-letter}
\newcommand{\rank}[1]{r\left(#1\right)}
%
%  Direct sum
%  Usage: \ds between a couple of subspaces
%
\newcommand{\ds}{\oplus}
%
%  Determinant of a matrix (functional)
%  Usage: \detname{A}
\newcommand{\detname}[1]{\det\left(#1\right)}
%
%  Determinant of a matrix (vertical bars)
%  Usage: \detbars{A}
\newcommand{\detbars}[1]{\left\lvert#1\right\rvert}
%
%  Trace of a Matrix
%  Usage: \trace{matrix name}
\newcommand{\trace}[1]{t\left(#1\right)}
%
%  Square Root of a Matrix
%  Usage: \sr{a-matrix}
\newcommand{\sr}[1]{#1^{1/2}}
%
%%%%%%%%%%%%%%%%%%%%%
%
%     Subspace Constructions
%
%%%%%%%%%%%%%%%%%%%%%
%
%  Span of a set of vectors
%  \span and \sp are used by TeX for other things
%  Usage: \spn{set-of-vectors}
\newcommand{\spn}[1]{\left\langle#1\right\rangle}
%
%  Null space of a matrix
%  Usage:  \nsp{A}
\newcommand{\nsp}[1]{\mathcal{N}\!\left(#1\right)}
%
%  Column space of a matrix
%  Usage:  \csp{A}
\newcommand{\csp}[1]{\mathcal{C}\!\left(#1\right)}
%
%  Row space of a matrix
%  Usage:  \rsp{A}
\newcommand{\rsp}[1]{\mathcal{R}\!\left(#1\right)}
%
%  Left null space of a matrix
%  Usage:  \lns{A}
\newcommand{\lns}[1]{\mathcal{L}\!\left(#1\right)}
%
%  Orthogonal complement of a vector space
%  Avoiding TeX's \perp
%  Usage:  \per{A}
\newcommand{\per}[1]{#1^\perp}
%
%%%%%%%%%%%%%%%%%%%%%
%
%     Systems of Equations
%
%%%%%%%%%%%%%%%%%%%%%
%
%  In-line form of an augmented matrix for a system of equations
%  Usage: \augmented{coefficient-matrix}{constant-vector}
\newcommand{\augmented}[2]{\left\lbrack\left.#1\,\right\rvert\,#2\right\rbrack}
%
%  Notation for a linear system before introducing matrix multiplication
%  Usage: \linearsystem{coefficient-matrix}{constant-vector}
\newcommand{\linearsystem}[2]{\mathcal{LS}\!\left(#1,\,#2\right)}
%
%  Notation for a homogenous system before introducing matrix multiplication
%  Usage: \homosystem{coefficient-matrix}
\newcommand{\homosystem}[1]{\linearsystem{#1}{\zerovector}}
%
%%%%%%%%%%%%%%%%%%%%%
%
%     Row Operations, Echelon Form
%
%%%%%%%%%%%%%%%%%%%%%
%
% Row operations on matrices
%
% Three commands to shorten up descriptions of gaussian elimination
%
% Usage: \rowopswap{row-i}{row-j}
% Usage: \rowopmult{scalar}{row-i}
% Usage: \rowopadd{scalar}{row-multiplied}{row-added-to}
\newcommand{\rowopswap}[2]{R_{#1}\leftrightarrow R_{#2}}
\newcommand{\rowopmult}[2]{#1R_{#2}}
\newcommand{\rowopadd}[3]{#1R_{#2}+R_{#3}}
%
% Mark leading 1's in echelon form with fbox
% Usage: \leading{a-1-usually}
\newcommand{\leading}[1]{\fbox{#1}}
%
%  Row-reduce arrow
%  Usage:  \rref inbetween a matrix and its reduced row-echelon form
\newcommand{\rref}{\xrightarrow{\text{RREF}}}
%
%  Elementary Matrices
%  Usage: \elemswap{subscript}{subscript}
%  Usage: \elemmult{scalar}{subscript}
%  Usage: \elemadd{scalar}{subscript-mult}{subscript-target}
%
\newcommand{\elemswap}[2]{E_{#1,#2}}
\newcommand{\elemmult}[2]{E_{#2}\left(#1\right)}
\newcommand{\elemadd}[3]{E_{#2,#3}\left(#1\right)}
%
%%%%%%%%%%%%%%%%%%%%%
%
%     2-D Constructions (Lists, Vectors, Matrices)
%
%%%%%%%%%%%%%%%%%%%%%
%
%  A list of scalars of generic length
%  Usage:  \scalarlist{scalar letter}{terminal subscript}
\newcommand{\scalarlist}[2]{{#1}_{1},\,{#1}_{2},\,{#1}_{3},\,\ldots,\,{#1}_{#2}}
%
%  Vector styling, bold (or use wiggles, arrows, whatever)
%  Subscripts go outside this construction
%  Usage: \vect{a symbol to use as a vector}
%  Have to already be in math mode
%
\newcommand{\vect}[1]{\mathbf{#1}}
%
%  A column vector
%  Usage: \colvector{list-delimited-by-\\}
%
\newcommand{\colvector}[1]{\begin{bmatrix}#1\end{bmatrix}}
%
%  A generic vector with components
%  Usage: \vectorcomponents{component-letter}{final-subscript}
\newcommand{\vectorcomponents}[2]{\colvector{#1_{1}\\#1_{2}\\#1_{3}\\\vdots\\#1_{#2}}}
%
%  A list of vectors of generic length
%  Usage:  \vectorlist{vector letter}{terminal subscript}
\newcommand{\vectorlist}[2]{\vect{#1}_{1},\,\vect{#1}_{2},\,\vect{#1}_{3},\,\ldots,\,\vect{#1}_{#2}}
%
%  Vector entries, entry i of vector v
%  (vector-expession still needs \vect, etc.)
%  Usage:  \vectorentry{vector-expression}{single-subscript}
\newcommand{\vectorentry}[2]{\left\lbrack#1\right\rbrack_{#2}}
%
%  Matrix entries, entry i,j of matrix A
%  Usage:  \matrixentry{matrix-expression}{paired-subscripts}
%
\newcommand{\matrixentry}[2]{\left\lbrack#1\right\rbrack_{#2}}
%
%  A generic linear combination
%  Usage:  \lincombo{scalar letter}{vector letter}{terminal subscript}
\newcommand{\lincombo}[3]{#1_{1}\vect{#2}_{1}+#1_{2}\vect{#2}_{2}+#1_{3}\vect{#2}_{3}+\cdots +#1_{#3}\vect{#2}_{#3}}
%
%  Matrix, column by column, as vectors
%  Usage:  \matrixcolumns{matrix letter}{terminal subscript}
\newcommand{\matrixcolumns}[2]{\left\lbrack\vect{#1}_{1}|\vect{#1}_{2}|\vect{#1}_{3}|\ldots|\vect{#1}_{#2}\right\rbrack}
%
%%%%%%%%%%%%%%%%%%%%%
%
%     Special Matrices
%
%%%%%%%%%%%%%%%%%%%%%
%
%  Transpose of a matrix
%  Usage:  \transpose{A}
\newcommand{\transpose}[1]{#1^{t}}
%
%  Inverse of a matrix
%  Usage:  \inverse{A}
\newcommand{\inverse}[1]{#1^{-1}}
%
%  Submatrix (for minors, determinants)
%  Usage: \submatrix{matrix-name}{delete-row}{delete-col}
\newcommand{\submatrix}[3]{#1\left(#2|#3\right)}
%
%  Adjoint of a matrix (twice)
%  This macro is a convenience to call \transpose and \conjugate properly
%  It shouldn't need to be modified (or mathematical meanings will change)
%  Usage:  \adj{A}
\newcommand{\adj}[1]{\transpose{\left(\conjugate{#1}\right)}}
%
%  This macro controls the symbol used for the adjoint
%  It can be edited to taste
%  Usage:  \adjoint{A}
\newcommand{\adjoint}[1]{#1^\ast}
%
%%%%%%%%%%%%%%%%%%%%%
%
%     Sets
%
%%%%%%%%%%%%%%%%%%%%%
%
%  A convenience for simple sets
%  Usage:  \set{list of element}
\newcommand{\set}[1]{\left\{#1\right\}}
%
%  Sets with vertical bar, "such that", sized for objects, not condition
%  Usage:  \setparts{objects}{condition}
%
%%\newcommand{\setparts}[2]{\left\{ #1\mid#2\right\}}
%%\newcommand{\setparts}[2]{\left\{\left. #1\right\rvert#2\right\}}
\newcommand{\setparts}[2]{\left\lbrace#1\,\middle|\,#2\right\rbrace}
%
%  Set Cardinality
%  Usage:  \card{a-set-letter}
\newcommand{\card}[1]{\left\lvert#1\right\rvert}
%
%  Set Union
%  Use \cup
%
%  Set Intersection
%  Use \cap
%
%  Set Complement
%  Usage:  \setcomplement{a-set-letter}
\newcommand{\setcomplement}[1]{\overline{#1}}
%
%%%%%%%%%%%%%%%%%%%%%
%
%     Eigenvalues and Eigenspaces
%
%%%%%%%%%%%%%%%%%%%%%
%
%  Characteristic polynomial
%  Usage: \charpoly{matrix-letter}{variable-letter}
\newcommand{\charpoly}[2]{p_{#1}\left(#2\right)}
%
%  Eigenspace
%  Usage: \eigenspace{matrix-letter}{eigenvalue-letter}
\newcommand{\eigenspace}[2]{\mathcal{E}_{#1}\left(#2\right)}
%
%  2013/10/03 Including ampersands is problematic here, 
%  think about fixes later
%  2014/02/22 Limited testing, seems &amp; is fine for HTML and LaTeX
%  2016-07-20 only employed in Archetypes, MBX has gather/align override
%  Eigensystem (presumes wrapped in an mrow within md)
%  Usage: \eigensystem{matrixletter}{eigenvalue}{list of basis vectors}
\newcommand{\eigensystem}[3]{\lambda&amp;=#2&amp;\eigenspace{#1}{#2}&amp;=\spn{\set{#3}}} 
%
%  Generalized Eigenspace
%  Usage: \geneigenspace{lin-trans-letter}{eigenvalue-letter}
\newcommand{\geneigenspace}[2]{\mathcal{G}_{#1}\left(#2\right)}
%
%  Algebraic multiplicty
%  Usage: \algmult{matrix-letter}{eigenvalue-letter}
\newcommand{\algmult}[2]{\alpha_{#1}\left(#2\right)}
%
%  Geometric multiplicty
%  Usage: \geomult{matrix-letter}{eigenvalue-letter}
\newcommand{\geomult}[2]{\gamma_{#1}\left(#2\right)}
%
%  Index (of eigenvalue)
%  Usage: \indx{matrix-letter}{eigenvalue-letter}
\newcommand{\indx}[2]{\iota_{#1}\left(#2\right)}
%
%%%%%%%%%%%%%%%%%%%%%
%
%     Linear Transformations
%
%%%%%%%%%%%%%%%%%%%%%
%
%  MathJax defines \lt to ease XML confusion
%
%  Linear transformation definition
%  Usage: \ltdefn{name-letter}{domain}{range}
\newcommand{\ltdefn}[3]{#1\colon #2\rightarrow#3}
%
%  Linear transformation evaluation
%  Usage: \lteval{name-letter}{input}
%  Replaces old \lt desired by MathJax
\newcommand{\lteval}[2]{#1\left(#2\right)}
%
% Linear transformation inverse
%  Usage: \ltinverse{name-letter}
\newcommand{\ltinverse}[1]{#1^{-1}}
%
%  Linear transformation restriction
%  Usage: \restrict{name-letter}{subspace-letter}
\newcommand{\restrict}[2]{{#1}|_{#2}}
%
%  Linear transformation preimage
%  Usage: \preimage{name-letter}{codomain-element}
\newcommand{\preimage}[2]{#1^{-1}\left(#2\right)}
%
%  Range of a linear transformation
%  TeX uses \range for something else
%  Usage:  \rng{T}
\newcommand{\rng}[1]{\mathcal{R}\!\left(#1\right)}
%
%  Kernel of a linear transformation
%  TeX uses \ker to do something different
%  Usage:  \krn{T}
\newcommand{\krn}[1]{\mathcal{K}\!\left(#1\right)}
%
%  Linear transformation composition
%  Usage: \compose{function-name}{function-name}
\newcommand{\compose}[2]{{#1}\circ{#2}}
%
%  Vector space of linear transformations
%  Usage: \vslt{domains}{codomains}
%  Presumes math mode
\newcommand{\vslt}[2]{\mathcal{LT}\left(#1,\,#2\right)}
%
%%%%%%%%%%%%%%%%%%%%%
%
%     Vector and Matrix Representations
%
%%%%%%%%%%%%%%%%%%%%%
%
%  Isomorphism symbol
%  Usage: \isomorphic
\newcommand{\isomorphic}{\cong}
%
%  Similarity
%  Usage: \similar{inner-matrix}{outer-invertible-matrix}
%  Rearranging this will not "fix" all desired changes throughout
%
\newcommand{\similar}[2]{\inverse{#2}#1#2}
%
%  Vector representation function name
%  Usage: \vectrepname{basis-letter}
\newcommand{\vectrepname}[1]{\rho_{#1}}
%
%  Vector representation output
%  Usage: \vectrep{basis-letter}{input}
\newcommand{\vectrep}[2]{\lteval{\vectrepname{#1}}{#2}}
%
%  Vector representation inverse function name
%  (Added later, not used consistently in FCLA)
%  Usage: \vectrepinvname{basis-letter}
\newcommand{\vectrepinvname}[1]{\ltinverse{\vectrepname{#1}}}
%
%  Vector representation inverse output
%  Usage: \vectrepinv{basis-letter}{input}
\newcommand{\vectrepinv}[2]{\lteval{\ltinverse{\vectrepname{#1}}}{#2}}
%
%  Matrix representation
%  Usage: \matrixrep{transformation-letter}{domain-basis-letter}{codomain-basis-letter}
\newcommand{\matrixrep}[3]{M^{#1}_{#2,#3}}
%
%  Matrix representation column-by-colum
%  2016-07-20 only employed once?
%  Usage: \matrixrepcolumns{transformation-letter}{codomain-basis-letter}{codomain-basis-vector-letter}{final-index}
\newcommand{\matrixrepcolumns}[4]{\left\lbrack \left.\vectrep{#2}{\lteval{#1}{\vect{#3}_{1}}}\right|\left.\vectrep{#2}{\lteval{#1}{\vect{#3}_{2}}}\right|\left.\vectrep{#2}{\lteval{#1}{\vect{#3}_{3}}}\right|\ldots\left|\vectrep{#2}{\lteval{#1}{\vect{#3}_{#4}}}\right.\right\rbrack}
%
%  Change of basis matrix
%  Usage: \cbm{domain-basis-letter}{codomain-basis-letter}
\newcommand{\cbm}[2]{C_{#1,#2}}
%
%%%%%%%%%%%%%%%%%%%%%
%
%     Canonical Forms
%
%%%%%%%%%%%%%%%%%%%%%
%
%  Jordan blocks
%  Usage: \jordan{size}{diagonal-element}
\newcommand{\jordan}[2]{J_{#1}\left(#2\right)}
%
%%%%%%%%%%%%%%%%%%%%%
%
%     Hadamard Matrices
%     Contributed by Elizabeth Million
%
%%%%%%%%%%%%%%%%%%%%%
%
%  Hadamard Product
%  Usage: \hadamard{a-matrix}{a-matrix}
\newcommand{\hadamard}[2]{#1\circ #2}
%
%  Hadamard identity matrix
%  Usage: \hadamardidentity{paired-subscripts-size-of-matrix}
\newcommand{\hadamardidentity}[1]{J_{#1}}
%
%  Hadamard inverse matrix
%  Usage: \hadamardinverse{matrix-expression}
\newcommand{\hadamardinverse}[1]{\widehat{#1}}


\title{Rank and Nullity of a Linear Transformation}

\begin{document}
\begin{abstract}
  The rank and nullity of a linear transformation are related.
\end{abstract}
\maketitle


Just as a matrix has a rank and a nullity, so too do linear transformations.  And just like the rank and nullity of a matrix are related (they sum to the number of columns, \ref{theorem:RPNC}) the rank and nullity of a linear transformation are related.  Here are the definitions and theorems.

\begin{definition}[Rank Of a Linear Transformation]

Suppose that $\ltdefn{T}{U}{V}$ is a linear transformation.  Then the \dfn{rank} of $T$, $\rank{T}$, is the dimension of the range of $T$,
\[
\rank{T}=\dimension{\rng{T}}
\]

\end{definition}

\begin{definition}[Nullity Of a Linear Transformation]

Suppose that $\ltdefn{T}{U}{V}$ is a linear transformation.  Then the \dfn{nullity} of $T$, $\nullity{T}$, is the dimension of the kernel of $T$,
\[
\nullity{T}=\dimension{\krn{T}}
\]
\end{definition}

Here are two quick theorems.

\begin{theorem}[Rank Of a Surjective Linear Transformation]
\label{theorem:ROSLT}


Suppose that $\ltdefn{T}{U}{V}$ is a linear transformation.  Then the rank of $T$ is the dimension of $V$, $\rank{T}=\dimension{V}$, if and only if $T$ is surjective.





\begin{proof}
By \ref{theorem:RSLT}, $T$ is surjective if and only if $\rng{T}=V$.  Applying \ref{definition:ROLT}, $\rng{T}=V$ if and only if $\rank{T}=\dimension{\rng{T}}=\dimension{V}$.



\end{proof}
\end{theorem}

\begin{theorem}[Nullity Of an Injective Linear Transformation]
\label{theorem:NOILT}


Suppose that $\ltdefn{T}{U}{V}$ is a linear transformation.  Then the nullity of $T$ is zero, $\nullity{T}=0$, if and only if $T$ is injective.





\begin{proof}
By \ref{theorem:KILT}, $T$ is injective if and only if $\krn{T}=\set{\zerovector}$.  Applying \ref{definition:NOLT}, $\krn{T}=\set{\zerovector}$ if and only if $\nullity{T}=0$.



\end{proof}
\end{theorem}

Just as injectivity and surjectivity come together in invertible linear transformations, there is a clear relationship between rank and nullity of a linear transformation.  If one is big, the other is small.



\begin{theorem}[Rank Plus Nullity is Domain Dimension]
\label{theorem:RPNDD}


Suppose that $\ltdefn{T}{U}{V}$ is a linear transformation.  Then
\[
\rank{T}+\nullity{T}=\dimension{U}
\]






\begin{proof}
Let $r=\rank{T}$ and $s=\nullity{T}$.  Suppose that $R=\set{\vectorlist{v}{r}}\subseteq V$ is a basis of the range of $T$, $\rng{T}$, and $S=\set{\vectorlist{u}{s}}\subseteq U$ is a basis of the kernel of $T$, $\krn{T}$.  Note that $R$ and $S$ are possibly empty, which means that some of the sums in this proof are ``empty'' and are equal to the zero vector.



Because the elements of $R$ are all in the range of $T$, each must have a nonempty pre-image by \ref{theorem:RPI}.  Choose vectors $\vect{w}_i\in U$, $1\leq i\leq r$ such that $\vect{w}_i\in\preimage{T}{\vect{v}_i}$.  So $\lteval{T}{\vect{w}_i}=\vect{v}_i$, $1\leq i\leq r$.  Consider the set
\[
B=\set{\vectorlist{u}{s},\,\vectorlist{w}{r}}
\]
We claim that $B$ is a basis for $U$.



To establish linear independence for $B$, begin with a relation of linear dependence on $B$.  So suppose there are scalars $\scalarlist{a}{s}$ and $\scalarlist{b}{r}$
\[
\zerovector=\lincombo{a}{u}{s}+\lincombo{b}{w}{r}
\]




Then
\begin{align*}
\zerovector&=\lteval{T}{\zerovector}&&\ref{theorem:LTTZZ}\\
&=T\left(\lincombo{a}{u}{s}+\right.\\
&\quad\quad\quad\quad\left.\lincombo{b}{w}{r}\right)&&\ref{definition:LI}\\
&=a_1\lteval{T}{\vect{u}_1}+a_2\lteval{T}{\vect{u}_2}+a_3\lteval{T}{\vect{u}_3}+\cdots+a_s\lteval{T}{\vect{u}_s}+\\
&\quad\quad b_1\lteval{T}{\vect{w}_1}+b_2\lteval{T}{\vect{w}_2}+b_3\lteval{T}{\vect{w}_3}+\cdots+b_r\lteval{T}{\vect{w}_r}
&&\ref{theorem:LTLC}\\
&=a_1\zerovector+a_2\zerovector+a_3\zerovector+\cdots+a_s\zerovector+\\
&\quad\quad b_1\lteval{T}{\vect{w}_1}+b_2\lteval{T}{\vect{w}_2}+b_3\lteval{T}{\vect{w}_3}+\cdots+b_r\lteval{T}{\vect{w}_r}
&&\ref{definition:KLT}\\
&=\zerovector+\zerovector+\zerovector+\cdots+\zerovector+\\
&\quad\quad b_1\lteval{T}{\vect{w}_1}+b_2\lteval{T}{\vect{w}_2}+b_3\lteval{T}{\vect{w}_3}+\cdots+b_r\lteval{T}{\vect{w}_r}
&&\ref{theorem:ZVSM}\\
&=b_1\lteval{T}{\vect{w}_1}+b_2\lteval{T}{\vect{w}_2}+b_3\lteval{T}{\vect{w}_3}+\cdots+b_r\lteval{T}{\vect{w}_r}&&\ref{property:Z}\\
&=b_1\vect{v}_1+b_2\vect{v}_2+b_3\vect{v}_3+\cdots+b_r\vect{v}_r&&\ref{definition:PI}
\end{align*}




This is a relation of linear dependence on $R$ (\ref{definition:RLD}), and since $R$ is a linearly independent set (\ref{definition:LI}), we see that $b_1=b_2=b_3=\ldots=b_r=0$.  Then the original relation of linear dependence on $B$ becomes
\begin{align*}
\zerovector&=\lincombo{a}{u}{s}+0\vect{w}_1+0\vect{w}_2+\ldots+0\vect{w}_r\\
&=\lincombo{a}{u}{s}+\zerovector+\zerovector+\ldots+\zerovector&&\ref{theorem:ZSSM}\\
&=\lincombo{a}{u}{s}&&\ref{property:Z}
\end{align*}




But this is again a relation of linear independence (\ref{definition:RLD}), now on the set $S$.  Since $S$ is linearly independent (\ref{definition:LI}), we have $a_1=a_2=a_3=\ldots=a_r=0$.  Since we now know that all the scalars in the relation of linear dependence on $B$ must be zero, we have established the linear independence of $S$ through \ref{definition:LI}.



To now establish that $B$ spans $U$, choose an arbitrary vector $\vect{u}\in U$.  Then $\lteval{T}{\vect{u}}\in R(T)$, so there are scalars $\scalarlist{c}{r}$ such that
\[
\lteval{T}{\vect{u}}=\lincombo{c}{v}{r}
\]




Use the scalars $\scalarlist{c}{r}$ to define a vector $\vect{y}\in U$,
\[
\vect{y}=\lincombo{c}{w}{r}
\]




Then
\begin{align*}
\lteval{T}{\vect{u}-\vect{y}}&=\lteval{T}{\vect{u}}-\lteval{T}{\vect{y}}&&\ref{theorem:LTLC}\\
&=\lteval{T}{\vect{u}}-\lteval{T}{\lincombo{c}{w}{r}}&&\text{Substitution}\\
&=\lteval{T}{\vect{u}}-\left(c_1\lteval{T}{\vect{w}_1}+c_2\lteval{T}{\vect{w}_2}+\cdots+c_r\lteval{T}{\vect{w}_r}\right)&&\ref{theorem:LTLC}\\
&=\lteval{T}{\vect{u}}-\left(\lincombo{c}{v}{r}\right)&&\vect{w}_i\in\preimage{T}{\vect{v}_i}\\
&=\lteval{T}{\vect{u}}-\lteval{T}{\vect{u}}&&\text{Substitution}\\
&=\zerovector&&\ref{property:AI}
\end{align*}




So the vector $\vect{u}-\vect{y}$ is sent to the zero vector by $T$ and hence is an element of the kernel of $T$.  As such it can be written as a linear combination of the basis vectors for $\krn{T}$, the elements of the set $S$.  So there are scalars $\scalarlist{d}{s}$ such that
\[
\vect{u}-\vect{y}=\lincombo{d}{u}{s}
\]




Then
\begin{align*}
\vect{u}&=\left(\vect{u}-\vect{y}\right)+\vect{y}\\
&=\lincombo{d}{u}{s}+\lincombo{c}{w}{r}
\end{align*}




This says that for any vector, $\vect{u}$, from $U$, there exist scalars ($\scalarlist{d}{s}$, $\scalarlist{c}{r}$) that form $\vect{u}$ as a linear combination of the vectors in the set $B$.  In other words, $B$ spans $U$ (\ref{definition:SS}).



So $B$ is a basis (\ref{definition:B}) of $U$ with $s+r$ vectors, and thus
\[
\dimension{U}=s+r=\nullity{T}+\rank{T}
\]
as desired.



\end{proof}
\end{theorem}

\ref{theorem:RPNC} said that the rank and nullity of a matrix sum to the number of columns of the matrix.  This result is now an easy consequence of \ref{theorem:RPNDD} when we consider the linear transformation $\ltdefn{T}{\complex{n}}{\complex{m}}$ defined with the $m\times n$ matrix $A$ by $\lteval{T}{\vect{x}}=A\vect{x}$.  The range and kernel of $T$ are identical to the column space and null space of the matrix $A$ (<acroref type="exercise" acro="ILT.T20" />, <acroref type="exercise" acro="SLT.T20" />), so the rank and nullity of the matrix $A$ are identical to the rank and nullity of the linear transformation $T$.  The dimension of the domain of $T$ is the dimension of $\complex{n}$, exactly the number of columns for the matrix $A$.



This theorem can be especially useful in determining basic properties of linear transformations.  For example, suppose that $\ltdefn{T}{\complex{6}}{\complex{6}}$ is a linear transformation and you are able to quickly establish that the kernel is trivial.  Then $\nullity{T}=0$.  First this means that $T$ is injective by \ref{theorem:NOILT}.  Also, \ref{theorem:RPNDD} becomes
\[
6=\dimension{\complex{6}}=\rank{T}+\nullity{T}=\rank{T}+0=\rank{T}
\]
So the rank of $T$ is equal to the dimension of the codomain, and by \ref{theorem:ROSLT} we know $T$ is surjective.  Finally, we know $T$ is invertible by \ref{theorem:ILTIS}.  So from the determination that the kernel is trivial, and consideration of various dimensions, the theorems of this section allow us to conclude the existence of an inverse linear transformation for $T$.
Similarly, \ref{theorem:RPNDD} can be used to provide alternative proofs for \ref{theorem:ILTD}, \ref{theorem:SLTD} and \ref{theorem:IVSED}.  It would be an interesting exercise to construct these proofs.



It would be instructive to study the archetypes that are linear transformations and see how many of their properties can be deduced just from considering only the dimensions of the domain and codomain.  Then add in just knowledge of either the nullity or rank, and see how much more you can learn about the linear transformation.  The table preceding all of the archetypes (<miscref type="archetype" text="Archetypes" />) could be a good place to start this analysis.

\end{document}
