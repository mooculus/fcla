\documentclass{ximera}

% These macros are automatically generated from the "macros"
% XML element.  Make permanent edits there.
%
% History
%   2004/01/01  Initiated for FCLA, evolved from there
%   2006/09/17  Converted  _, ^  to \sb, \sp for TeX4ht
%   2014/02/01  Updated for MathBook XML projects
%               Obsolete in FCLA: \codeindent, \computerfont, \define
%               Change: MathJax wants \lt, so replaced by \lteval
%   2014/02/22  New: \orderof, \reals, \per
%   2015/08/16  Incorporated into MathBook XML version of FCLA
%
%%%%%%%%%%%%%%%%%%%%%
%
%     Conveniences
%
%%%%%%%%%%%%%%%%%%%%%
%
%  Order of (asymptotically limit of fraction is 1)
%  Usage: \orderof{some function}
%
\newcommand{\orderof}[1]{\sim #1}
%
%  Integers
%  Usage:  \Z
\newcommand{\Z}{\mathbb{Z}}
%
%  Real numbers, as set of scalars
%  Usage:  \reals
\newcommand{\reals}{\mathbb{R}}
%
%  n-space over real field
%  Usage: \complex{integer-dimension}
\newcommand{\real}[1]{\mathbb{R}^{#1}}
%
%  Complex numbers, as set of scalars
%  Usage:  \complexes
\newcommand{\complexes}{\mathbb{C}}
%
%  n-space over complex field
%  Usage: \complex{integer-dimension}
\newcommand{\complex}[1]{\mathbb{C}^{#1}}
%
%  Complex conjugation (scalar, vector, matrix)
%  Usage: \conjugate{object}
\newcommand{\conjugate}[1]{\overline{#1}}
%
%  Complex number modulus
%  Usage: \modulus{a+bi}
%  Presumes math mode
\newcommand{\modulus}[1]{\left\lvert#1\right\rvert}
%
%  Zero vector
%  Usage: \zerovector
\newcommand{\zerovector}{\vect{0}}
%
%  Zero matrix
%  Usage: \zeromatrix, use a subscript when size is important
\newcommand{\zeromatrix}{\mathcal{O}}
%
%  Inner product (brackets, not quadratic form)
%  Usage: \innerproduct{a-vector}{a-vector}
\newcommand{\innerproduct}[2]{\left\langle#1,\,#2\right\rangle}
%
%  Norm of a vector
%  Usage: \norm{a-vector}
\newcommand{\norm}[1]{\left\lVert#1\right\rVert}
%
%  Dimension
%  Usage: \dimension{vector-space-letter}
\newcommand{\dimension}[1]{\dim\left(#1\right)}
%
%  Nullity
%  Usage: \nullity{matrix-or-lintrans-letter}
\newcommand{\nullity}[1]{n\left(#1\right)}
%
%  Rank
%  Usage: \rank{matrix-or-lintrans-letter}
\newcommand{\rank}[1]{r\left(#1\right)}
%
%  Direct sum
%  Usage: \ds between a couple of subspaces
%
\newcommand{\ds}{\oplus}
%
%  Determinant of a matrix (functional)
%  Usage: \detname{A}
\newcommand{\detname}[1]{\det\left(#1\right)}
%
%  Determinant of a matrix (vertical bars)
%  Usage: \detbars{A}
\newcommand{\detbars}[1]{\left\lvert#1\right\rvert}
%
%  Trace of a Matrix
%  Usage: \trace{matrix name}
\newcommand{\trace}[1]{t\left(#1\right)}
%
%  Square Root of a Matrix
%  Usage: \sr{a-matrix}
\newcommand{\sr}[1]{#1^{1/2}}
%
%%%%%%%%%%%%%%%%%%%%%
%
%     Subspace Constructions
%
%%%%%%%%%%%%%%%%%%%%%
%
%  Span of a set of vectors
%  \span and \sp are used by TeX for other things
%  Usage: \spn{set-of-vectors}
\newcommand{\spn}[1]{\left\langle#1\right\rangle}
%
%  Null space of a matrix
%  Usage:  \nsp{A}
\newcommand{\nsp}[1]{\mathcal{N}\!\left(#1\right)}
%
%  Column space of a matrix
%  Usage:  \csp{A}
\newcommand{\csp}[1]{\mathcal{C}\!\left(#1\right)}
%
%  Row space of a matrix
%  Usage:  \rsp{A}
\newcommand{\rsp}[1]{\mathcal{R}\!\left(#1\right)}
%
%  Left null space of a matrix
%  Usage:  \lns{A}
\newcommand{\lns}[1]{\mathcal{L}\!\left(#1\right)}
%
%  Orthogonal complement of a vector space
%  Avoiding TeX's \perp
%  Usage:  \per{A}
\newcommand{\per}[1]{#1^\perp}
%
%%%%%%%%%%%%%%%%%%%%%
%
%     Systems of Equations
%
%%%%%%%%%%%%%%%%%%%%%
%
%  In-line form of an augmented matrix for a system of equations
%  Usage: \augmented{coefficient-matrix}{constant-vector}
\newcommand{\augmented}[2]{\left\lbrack\left.#1\,\right\rvert\,#2\right\rbrack}
%
%  Notation for a linear system before introducing matrix multiplication
%  Usage: \linearsystem{coefficient-matrix}{constant-vector}
\newcommand{\linearsystem}[2]{\mathcal{LS}\!\left(#1,\,#2\right)}
%
%  Notation for a homogenous system before introducing matrix multiplication
%  Usage: \homosystem{coefficient-matrix}
\newcommand{\homosystem}[1]{\linearsystem{#1}{\zerovector}}
%
%%%%%%%%%%%%%%%%%%%%%
%
%     Row Operations, Echelon Form
%
%%%%%%%%%%%%%%%%%%%%%
%
% Row operations on matrices
%
% Three commands to shorten up descriptions of gaussian elimination
%
% Usage: \rowopswap{row-i}{row-j}
% Usage: \rowopmult{scalar}{row-i}
% Usage: \rowopadd{scalar}{row-multiplied}{row-added-to}
\newcommand{\rowopswap}[2]{R_{#1}\leftrightarrow R_{#2}}
\newcommand{\rowopmult}[2]{#1R_{#2}}
\newcommand{\rowopadd}[3]{#1R_{#2}+R_{#3}}
%
% Mark leading 1's in echelon form with fbox
% Usage: \leading{a-1-usually}
\newcommand{\leading}[1]{\fbox{#1}}
%
%  Row-reduce arrow
%  Usage:  \rref inbetween a matrix and its reduced row-echelon form
\newcommand{\rref}{\xrightarrow{\text{RREF}}}
%
%  Elementary Matrices
%  Usage: \elemswap{subscript}{subscript}
%  Usage: \elemmult{scalar}{subscript}
%  Usage: \elemadd{scalar}{subscript-mult}{subscript-target}
%
\newcommand{\elemswap}[2]{E_{#1,#2}}
\newcommand{\elemmult}[2]{E_{#2}\left(#1\right)}
\newcommand{\elemadd}[3]{E_{#2,#3}\left(#1\right)}
%
%%%%%%%%%%%%%%%%%%%%%
%
%     2-D Constructions (Lists, Vectors, Matrices)
%
%%%%%%%%%%%%%%%%%%%%%
%
%  A list of scalars of generic length
%  Usage:  \scalarlist{scalar letter}{terminal subscript}
\newcommand{\scalarlist}[2]{{#1}_{1},\,{#1}_{2},\,{#1}_{3},\,\ldots,\,{#1}_{#2}}
%
%  Vector styling, bold (or use wiggles, arrows, whatever)
%  Subscripts go outside this construction
%  Usage: \vect{a symbol to use as a vector}
%  Have to already be in math mode
%
\newcommand{\vect}[1]{\mathbf{#1}}
%
%  A column vector
%  Usage: \colvector{list-delimited-by-\\}
%
\newcommand{\colvector}[1]{\begin{bmatrix}#1\end{bmatrix}}
%
%  A generic vector with components
%  Usage: \vectorcomponents{component-letter}{final-subscript}
\newcommand{\vectorcomponents}[2]{\colvector{#1_{1}\\#1_{2}\\#1_{3}\\\vdots\\#1_{#2}}}
%
%  A list of vectors of generic length
%  Usage:  \vectorlist{vector letter}{terminal subscript}
\newcommand{\vectorlist}[2]{\vect{#1}_{1},\,\vect{#1}_{2},\,\vect{#1}_{3},\,\ldots,\,\vect{#1}_{#2}}
%
%  Vector entries, entry i of vector v
%  (vector-expession still needs \vect, etc.)
%  Usage:  \vectorentry{vector-expression}{single-subscript}
\newcommand{\vectorentry}[2]{\left\lbrack#1\right\rbrack_{#2}}
%
%  Matrix entries, entry i,j of matrix A
%  Usage:  \matrixentry{matrix-expression}{paired-subscripts}
%
\newcommand{\matrixentry}[2]{\left\lbrack#1\right\rbrack_{#2}}
%
%  A generic linear combination
%  Usage:  \lincombo{scalar letter}{vector letter}{terminal subscript}
\newcommand{\lincombo}[3]{#1_{1}\vect{#2}_{1}+#1_{2}\vect{#2}_{2}+#1_{3}\vect{#2}_{3}+\cdots +#1_{#3}\vect{#2}_{#3}}
%
%  Matrix, column by column, as vectors
%  Usage:  \matrixcolumns{matrix letter}{terminal subscript}
\newcommand{\matrixcolumns}[2]{\left\lbrack\vect{#1}_{1}|\vect{#1}_{2}|\vect{#1}_{3}|\ldots|\vect{#1}_{#2}\right\rbrack}
%
%%%%%%%%%%%%%%%%%%%%%
%
%     Special Matrices
%
%%%%%%%%%%%%%%%%%%%%%
%
%  Transpose of a matrix
%  Usage:  \transpose{A}
\newcommand{\transpose}[1]{#1^{t}}
%
%  Inverse of a matrix
%  Usage:  \inverse{A}
\newcommand{\inverse}[1]{#1^{-1}}
%
%  Submatrix (for minors, determinants)
%  Usage: \submatrix{matrix-name}{delete-row}{delete-col}
\newcommand{\submatrix}[3]{#1\left(#2|#3\right)}
%
%  Adjoint of a matrix (twice)
%  This macro is a convenience to call \transpose and \conjugate properly
%  It shouldn't need to be modified (or mathematical meanings will change)
%  Usage:  \adj{A}
\newcommand{\adj}[1]{\transpose{\left(\conjugate{#1}\right)}}
%
%  This macro controls the symbol used for the adjoint
%  It can be edited to taste
%  Usage:  \adjoint{A}
\newcommand{\adjoint}[1]{#1^\ast}
%
%%%%%%%%%%%%%%%%%%%%%
%
%     Sets
%
%%%%%%%%%%%%%%%%%%%%%
%
%  A convenience for simple sets
%  Usage:  \set{list of element}
\newcommand{\set}[1]{\left\{#1\right\}}
%
%  Sets with vertical bar, "such that", sized for objects, not condition
%  Usage:  \setparts{objects}{condition}
%
%%\newcommand{\setparts}[2]{\left\{ #1\mid#2\right\}}
%%\newcommand{\setparts}[2]{\left\{\left. #1\right\rvert#2\right\}}
\newcommand{\setparts}[2]{\left\lbrace#1\,\middle|\,#2\right\rbrace}
%
%  Set Cardinality
%  Usage:  \card{a-set-letter}
\newcommand{\card}[1]{\left\lvert#1\right\rvert}
%
%  Set Union
%  Use \cup
%
%  Set Intersection
%  Use \cap
%
%  Set Complement
%  Usage:  \setcomplement{a-set-letter}
\newcommand{\setcomplement}[1]{\overline{#1}}
%
%%%%%%%%%%%%%%%%%%%%%
%
%     Eigenvalues and Eigenspaces
%
%%%%%%%%%%%%%%%%%%%%%
%
%  Characteristic polynomial
%  Usage: \charpoly{matrix-letter}{variable-letter}
\newcommand{\charpoly}[2]{p_{#1}\left(#2\right)}
%
%  Eigenspace
%  Usage: \eigenspace{matrix-letter}{eigenvalue-letter}
\newcommand{\eigenspace}[2]{\mathcal{E}_{#1}\left(#2\right)}
%
%  2013/10/03 Including ampersands is problematic here, 
%  think about fixes later
%  2014/02/22 Limited testing, seems &amp; is fine for HTML and LaTeX
%  2016-07-20 only employed in Archetypes, MBX has gather/align override
%  Eigensystem (presumes wrapped in an mrow within md)
%  Usage: \eigensystem{matrixletter}{eigenvalue}{list of basis vectors}
\newcommand{\eigensystem}[3]{\lambda&amp;=#2&amp;\eigenspace{#1}{#2}&amp;=\spn{\set{#3}}} 
%
%  Generalized Eigenspace
%  Usage: \geneigenspace{lin-trans-letter}{eigenvalue-letter}
\newcommand{\geneigenspace}[2]{\mathcal{G}_{#1}\left(#2\right)}
%
%  Algebraic multiplicty
%  Usage: \algmult{matrix-letter}{eigenvalue-letter}
\newcommand{\algmult}[2]{\alpha_{#1}\left(#2\right)}
%
%  Geometric multiplicty
%  Usage: \geomult{matrix-letter}{eigenvalue-letter}
\newcommand{\geomult}[2]{\gamma_{#1}\left(#2\right)}
%
%  Index (of eigenvalue)
%  Usage: \indx{matrix-letter}{eigenvalue-letter}
\newcommand{\indx}[2]{\iota_{#1}\left(#2\right)}
%
%%%%%%%%%%%%%%%%%%%%%
%
%     Linear Transformations
%
%%%%%%%%%%%%%%%%%%%%%
%
%  MathJax defines \lt to ease XML confusion
%
%  Linear transformation definition
%  Usage: \ltdefn{name-letter}{domain}{range}
\newcommand{\ltdefn}[3]{#1\colon #2\rightarrow#3}
%
%  Linear transformation evaluation
%  Usage: \lteval{name-letter}{input}
%  Replaces old \lt desired by MathJax
\newcommand{\lteval}[2]{#1\left(#2\right)}
%
% Linear transformation inverse
%  Usage: \ltinverse{name-letter}
\newcommand{\ltinverse}[1]{#1^{-1}}
%
%  Linear transformation restriction
%  Usage: \restrict{name-letter}{subspace-letter}
\newcommand{\restrict}[2]{{#1}|_{#2}}
%
%  Linear transformation preimage
%  Usage: \preimage{name-letter}{codomain-element}
\newcommand{\preimage}[2]{#1^{-1}\left(#2\right)}
%
%  Range of a linear transformation
%  TeX uses \range for something else
%  Usage:  \rng{T}
\newcommand{\rng}[1]{\mathcal{R}\!\left(#1\right)}
%
%  Kernel of a linear transformation
%  TeX uses \ker to do something different
%  Usage:  \krn{T}
\newcommand{\krn}[1]{\mathcal{K}\!\left(#1\right)}
%
%  Linear transformation composition
%  Usage: \compose{function-name}{function-name}
\newcommand{\compose}[2]{{#1}\circ{#2}}
%
%  Vector space of linear transformations
%  Usage: \vslt{domains}{codomains}
%  Presumes math mode
\newcommand{\vslt}[2]{\mathcal{LT}\left(#1,\,#2\right)}
%
%%%%%%%%%%%%%%%%%%%%%
%
%     Vector and Matrix Representations
%
%%%%%%%%%%%%%%%%%%%%%
%
%  Isomorphism symbol
%  Usage: \isomorphic
\newcommand{\isomorphic}{\cong}
%
%  Similarity
%  Usage: \similar{inner-matrix}{outer-invertible-matrix}
%  Rearranging this will not "fix" all desired changes throughout
%
\newcommand{\similar}[2]{\inverse{#2}#1#2}
%
%  Vector representation function name
%  Usage: \vectrepname{basis-letter}
\newcommand{\vectrepname}[1]{\rho_{#1}}
%
%  Vector representation output
%  Usage: \vectrep{basis-letter}{input}
\newcommand{\vectrep}[2]{\lteval{\vectrepname{#1}}{#2}}
%
%  Vector representation inverse function name
%  (Added later, not used consistently in FCLA)
%  Usage: \vectrepinvname{basis-letter}
\newcommand{\vectrepinvname}[1]{\ltinverse{\vectrepname{#1}}}
%
%  Vector representation inverse output
%  Usage: \vectrepinv{basis-letter}{input}
\newcommand{\vectrepinv}[2]{\lteval{\ltinverse{\vectrepname{#1}}}{#2}}
%
%  Matrix representation
%  Usage: \matrixrep{transformation-letter}{domain-basis-letter}{codomain-basis-letter}
\newcommand{\matrixrep}[3]{M^{#1}_{#2,#3}}
%
%  Matrix representation column-by-colum
%  2016-07-20 only employed once?
%  Usage: \matrixrepcolumns{transformation-letter}{codomain-basis-letter}{codomain-basis-vector-letter}{final-index}
\newcommand{\matrixrepcolumns}[4]{\left\lbrack \left.\vectrep{#2}{\lteval{#1}{\vect{#3}_{1}}}\right|\left.\vectrep{#2}{\lteval{#1}{\vect{#3}_{2}}}\right|\left.\vectrep{#2}{\lteval{#1}{\vect{#3}_{3}}}\right|\ldots\left|\vectrep{#2}{\lteval{#1}{\vect{#3}_{#4}}}\right.\right\rbrack}
%
%  Change of basis matrix
%  Usage: \cbm{domain-basis-letter}{codomain-basis-letter}
\newcommand{\cbm}[2]{C_{#1,#2}}
%
%%%%%%%%%%%%%%%%%%%%%
%
%     Canonical Forms
%
%%%%%%%%%%%%%%%%%%%%%
%
%  Jordan blocks
%  Usage: \jordan{size}{diagonal-element}
\newcommand{\jordan}[2]{J_{#1}\left(#2\right)}
%
%%%%%%%%%%%%%%%%%%%%%
%
%     Hadamard Matrices
%     Contributed by Elizabeth Million
%
%%%%%%%%%%%%%%%%%%%%%
%
%  Hadamard Product
%  Usage: \hadamard{a-matrix}{a-matrix}
\newcommand{\hadamard}[2]{#1\circ #2}
%
%  Hadamard identity matrix
%  Usage: \hadamardidentity{paired-subscripts-size-of-matrix}
\newcommand{\hadamardidentity}[1]{J_{#1}}
%
%  Hadamard inverse matrix
%  Usage: \hadamardinverse{matrix-expression}
\newcommand{\hadamardinverse}[1]{\widehat{#1}}


\title{Linear Transformations and Linear Combinations}

\begin{document}
\begin{abstract}
The interaction between linear transformations and linear combinations lies at the heart of many of the important theorems of linear algebra.
\end{abstract}
\maketitle

This theorem says that we can ``push'' linear transformations ``down into'' linear combinations, or ``pull'' linear transformations ``up out'' of linear combinations.  We will have opportunities to both push and pull.

\begin{theorem}[Linear Transformations and Linear Combinations]
\label{theorem:LTLC}

Suppose that $\ltdefn{T}{U}{V}$ is a linear transformation, $\vectorlist{u}{t}$ are vectors from $U$ and $\scalarlist{a}{t}$ are scalars from $\complexes$.  Then
\[
\lteval{T}{\lincombo{a}{u}{t}}
=
a_1\lteval{T}{\vect{u}_1}+
a_2\lteval{T}{\vect{u}_2}+
a_3\lteval{T}{\vect{u}_3}+\cdots+
a_t\lteval{T}{\vect{u}_t}
\]

\begin{proof}

\begin{align*}
&\lteval{T}{\lincombo{a}{u}{t}}\\
&\quad\quad=
\lteval{T}{a_1\vect{u}_1}+
\lteval{T}{a_2\vect{u}_2}+
\lteval{T}{a_3\vect{u}_3}+\cdots+
\lteval{T}{a_t\vect{u}_t}&&\ref{definition:LT}\\
&\quad\quad=
a_1\lteval{T}{\vect{u}_1}+
a_2\lteval{T}{\vect{u}_2}+
a_3\lteval{T}{\vect{u}_3}+\cdots+
a_t\lteval{T}{\vect{u}_t}&&\ref{definition:LT}
\end{align*}

\end{proof}
\end{theorem}

Some authors, especially in more advanced texts, take the conclusion of \ref{theorem:LTLC} as the defining condition of a linear transformation.  This has the appeal of being a single condition, rather than the two-part condition of \ref{definition:LT}.

Our next theorem says, informally, that it is enough to know how a linear transformation behaves for inputs from any basis of the domain, and \textit{all} the other outputs are described by a linear combination of these few values.  Again, the statement of the theorem, and its proof, are not remarkable, but the insight that goes along with it is very fundamental.

\begin{theorem}[Linear Transformation Defined on a Basis]
\label{theorem:LTDB}

Suppose $U$ is a vector space with basis $B=\set{\vectorlist{u}{n}}$ and the vector space $V$ contains the vectors $\vectorlist{v}{n}$ (which may not be distinct).   Then there is a unique linear transformation, $\ltdefn{T}{U}{V}$, such that $\lteval{T}{\vect{u}_i}=\vect{v}_i$, $1\leq i\leq n$.


\begin{proof}
To prove the existence of $T$, we construct a function and show that it is a linear transformation (\ref{technique:C}).  Suppose $\vect{w}\in U$ is an arbitrary element of the domain.  Then by \ref{theorem:VRRB} there are unique scalars $\scalarlist{a}{n}$ such that
\[
\vect{w}=\lincombo{a}{u}{n}
\]

Then \textit{define} the function $T$ by
\[
\lteval{T}{\vect{w}}=\lincombo{a}{v}{n}
\]

It should be clear that $T$ behaves as required for $n$ inputs from $B$.  Since the scalars provided by \ref{theorem:VRRB} are unique, there is no ambiguity in this definition, and $T$ qualifies as a function with domain $U$ and codomain $V$ (i.e.,  $T$ is well-defined).  But is $T$ a linear transformation as well?

Let $\vect{x}\in U$ be a second element of the domain, and suppose the scalars provided by \ref{theorem:VRRB} (relative to $B$) are $\scalarlist{b}{n}$.  Then
\begin{align*}
\lteval{T}{\vect{w}+\vect{x}}&=
\lteval{T}{
a_1\vect{u}_1+
\cdots+
a_n\vect{u}_n+
b_1\vect{u}_1+
\cdots+
b_n\vect{u}_n
}\\
&=
\lteval{T}{
\left(a_1+b_1\right)\vect{u}_1+
\cdots+
\left(a_n+b_n\right)\vect{u}_n
}
&&\ref{definition:VS}\\
&=
\left(a_1+b_1\right)\vect{v}_1+
\cdots+
\left(a_n+b_n\right)\vect{v}_n
&&\text{Definition of $T$}\\
&=
a_1\vect{v}_1+
\cdots+
a_n\vect{v}_n+
b_1\vect{v}_1+
\cdots+
b_n\vect{v}_n
&&\ref{definition:VS}\\
&=\lteval{T}{\vect{w}}+\lteval{T}{\vect{x}}
\end{align*}

Let $\alpha\in\complexes$ be any scalar.  Then
\begin{align*}
\lteval{T}{\alpha\vect{w}}&=
\lteval{T}{\alpha\left(\lincombo{a}{u}{n}\right)}\\
&=
\lteval{T}{\lincombo{\alpha a}{u}{n}}
&&\ref{definition:VS}\\
&=\lincombo{\alpha a}{v}{n}
&&\text{Definition of $T$}\\
&=\alpha\left(\lincombo{a}{v}{n}\right)
&&\ref{definition:VS}\\
&=\alpha\lteval{T}{\vect{w}}
\end{align*}

So by \ref{definition:LT}, $T$ is a linear transformation.

Is $T$ unique (among all linear transformations that take the $\vect{u}_i$ to the $\vect{v}_i$)?  Applying \ref{technique:U}, we posit the existence of a second linear transformation, $\ltdefn{S}{U}{V}$ such that $\lteval{S}{\vect{u}_i}=\vect{v}_i$, $1\leq i\leq n$.  Again, let $\vect{w}\in U$ represent an arbitrary element of $U$ and let $\scalarlist{a}{n}$ be the scalars provided by \ref{theorem:VRRB} (relative to $B$).  We have,
\begin{align*}
\lteval{T}{\vect{w}}&=
\lteval{T}{\lincombo{a}{u}{n}}
&&\ref{theorem:VRRB}\\
&=
a_1\lteval{T}{\vect{u}_1}+
a_2\lteval{T}{\vect{u}_2}+
a_3\lteval{T}{\vect{u}_3}+
\cdots+
a_n\lteval{T}{\vect{u}_n}
&&\ref{theorem:LTLC}\\
&=
a_1\vect{v}_1+
a_2\vect{v}_2+
a_3\vect{v}_3+
\cdots+
a_n\vect{v}_n
&&\text{Definition of $T$}\\
&=
a_1\lteval{S}{\vect{u}_1}+
a_2\lteval{S}{\vect{u}_2}+
a_3\lteval{S}{\vect{u}_3}+
\cdots+
a_n\lteval{S}{\vect{u}_n}
&&\text{Definition of $S$}\\
&=
\lteval{S}{\lincombo{a}{u}{n}}
&&\ref{theorem:LTLC}\\
&=
\lteval{S}{\vect{w}}
&&\ref{theorem:VRRB}
\end{align*}


So the output of $T$ and $S$ agree on every input, which means they are equal as functions, $T=S$.  So $T$ is unique.

\end{proof}
\end{theorem}

You might recall facts from analytic geometry, such as ``any two
points determine a line'' and ``any three non-collinear points
determine a parabola.''  \ref{theorem:LTDB} has much of the same feel.
By specifying the $n$ outputs for inputs from a basis, an entire
linear transformation is determined.  The analogy is not perfect, but
the style of these facts are not very dissimilar from
\ref{theorem:LTDB}.

Notice that the statement of \ref{theorem:LTDB} asserts the
\textit{existence} of a linear transformation with certain properties,
while the proof shows us exactly how to define the desired linear
transformation. The next two examples show how to compute values of
linear transformations that we create this way.

\begin{example}
[Linear transformation defined on a basis]

Consider the linear transformation $\ltdefn{T}{\complex{3}}{\complex{2}}$ that is required to have the following three values,
\begin{align*}
\lteval{T}{\colvector{1\\0\\0}}=\colvector{2\\1}&&
\lteval{T}{\colvector{0\\1\\0}}=\colvector{-1\\4}&&
\lteval{T}{\colvector{0\\0\\1}}=\colvector{6\\0}
\end{align*}

Because
\[
B=\set{
\colvector{1\\0\\0},\,
\colvector{0\\1\\0},\,
\colvector{0\\0\\1}
}
\]
is a basis for $\complex{3}$ (\ref{theorem:SUVB}), \ref{theorem:LTDB} says there is a unique linear transformation $T$ that behaves this way.

How do we compute other values of $T$?  Consider the input
\[
\vect{w}=\colvector{2\\-3\\1}=(2)\colvector{1\\0\\0}+(-3)\colvector{0\\1\\0}+(1)\colvector{0\\0\\1}
\]

Then
\[
\lteval{T}{\vect{w}}=(2)\colvector{\answer{2}\\\answer{1}}+ (-3)\colvector{-1\\4}+ (1)\colvector{6\\0}=\colvector{13\\-10}
\]

Doing it again,
\[
\vect{x}=\colvector{5\\2\\-3}=(5)\colvector{1\\0\\0}+(2)\colvector{0\\1\\0}+(-3)\colvector{0\\0\\1}
\]
so
\[
\lteval{T}{\vect{x}}=(\answer{5})\colvector{2\\1}+ (2)\colvector{-1\\4}+ (-3)\colvector{6\\0}=\colvector{-10\\13}
\]


Any other value of $T$ could be computed in a similar manner.  So
rather than being given a \textit{formula} for the outputs of $T$, the
\textit{requirement} that $T$ behave in a certain way for the inputs
chosen from a basis of the domain, is as sufficient as a formula for
computing any value of the function.  You might notice some parallels
between this example and \ref{example:MOLT} or \ref{theorem:MLTCV}.

\end{example}

\begin{example}[Linear transformation defined on a basis]

Consider the linear transformation $\ltdefn{R}{\complex{3}}{\complex{2}}$ with the three values,
\begin{align*}
\lteval{R}{\colvector{1\\2\\1}}=\colvector{5\\-1}&&
\lteval{R}{\colvector{-1\\5\\1}}=\colvector{0\\4}&&
\lteval{R}{\colvector{3\\1\\4}}=\colvector{2\\3}
\end{align*}

You can check that
\[
D=\set{
\colvector{1\\2\\1},\,
\colvector{-1\\5\\1},\,
\colvector{3\\1\\4}
}
\]
is a basis for $\complex{3}$ (make the vectors the columns of a square matrix and check that the matrix is nonsingular,  \ref{theorem:CNMB}).  By \ref{theorem:LTDB} we know there is a unique linear transformation $R$ with the three specified outputs.  However, we have to work just a bit harder to take an input vector and express it as a linear combination of the vectors in $D$.

For example, consider,
\[
\vect{y}=\colvector{8\\-3\\5}
\]

Then we must first write $\vect{y}$ as a linear combination of the vectors in $D$ and solve for the unknown scalars, to arrive at
\[
\vect{y}=\colvector{8\\-3\\5}= (3)\colvector{1\\2\\1}+ (-2)\colvector{-1\\5\\1}+ (1)\colvector{3\\1\\4}
\]

Then the proof of \ref{theorem:LTDB} gives us
\[
\lteval{R}{\vect{y}}=(\answer{3})\colvector{5\\-1}+ (-2)\colvector{0\\4}+ (1)\colvector{2\\3}= \colvector{17\\-8}
\]

Any other value of $R$ could be computed in a similar manner.

\end{example}

Here is a third example of a linear transformation defined by its action on a basis, only with more abstract vector spaces involved.



\begin{example}[Linear transformation defined on a basis]

The set $W=\set{p(x)\in P_3\mid p(1)=0, p(3)=0}\subseteq P_3$ is a subspace of the vector space of polynomials $P_3$.  This subspace has $C=\set{3-4x+x^2,\,12-13x+x^3}$ as a basis (check this!).  Suppose we consider the linear transformation $\ltdefn{S}{P_3}{M_{22}}$ with values
\begin{align*}
\lteval{S}{3-4x+x^2}=\begin{bmatrix}1&-3\\2&0\end{bmatrix}&&
\lteval{S}{12-13x+x^3}=\begin{bmatrix}0&1\\1&0\end{bmatrix}
\end{align*}




By \ref{theorem:LTDB} we know there is a unique linear transformation with these two values.  To illustrate a sample computation of $S$, consider $q(x)=9-6x-5x^2+2x^3$.  Verify that $q(x)$ is an element of $W$ (does it have roots at $x=1$ and $x=3$?), then find the scalars needed to write it as a linear combination of the basis vectors in $C$.  Because
\[
q(x)=9-6x-5x^2+2x^3=(-5)(3-4x+x^2)+(2)(12-13x+x^3)
\]




The proof of \ref{theorem:LTDB} gives us
\[
\lteval{S}{q}=(\answer{-5})\begin{bmatrix}1&-3\\2&0\end{bmatrix}
+
(2)\begin{bmatrix}0&1\\1&0\end{bmatrix}
=
\begin{bmatrix}-5&17\\-8&0\end{bmatrix}
\]


And all the other outputs of $S$ could be computed in the same manner.  Every output of $S$ will have a zero in the second row, second column.  Can you see why this is so?


\end{example}

Informally, we can describe \ref{theorem:LTDB} by saying ``it is enough to know what a linear transformation does to a basis (of the domain).''

\end{document}
%%% Local Variables:
%%% mode: latex
%%% TeX-master: t
%%% End:
