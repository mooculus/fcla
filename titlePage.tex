\documentclass{ximera}
\title{Linear Algebra}

\begin{document}
\begin{abstract}
By using linear algebra, seeming different ``linear'' things in algebra, geometry, and calculus—things like systems of certain equations, rigid motions in geometry, certain differential equations—can be placed in a common framework of vectors, matrices, and linear transformations.  Viewing different things as somehow analogous provides not only insight, but also a common toolkit of surprisingly powerful algorithms.
\end{abstract}
\maketitle

\begin{center}
\Huge This is Linear Algebra
\end{center}

Authored by Jim Fowler (for the modified version for Ximera) and Rob Beezer (for the original textbook).  Published by the MOOCulus team at Ohio State University.  Copyright 2004--2015 Robert A. Beezer; Copyright 2016 Jim Fowler.


Permission is granted to copy, distribute and/or modify this document
under the terms of the GNU Free Documentation License, Version 1.3
or any later version published by the Free Software Foundation;
with no Invariant Sections, no Front-Cover Texts, and no Back-Cover Texts.
A copy of the license is included in the section entitled ``GNU
Free Documentation License.''

\section{History}   
  If
   there is no section Entitled ``History'' in the Document, create one
   stating the title, year, authors, and publisher of the Document as
   given on its Title Page, then add an item describing the Modified
   Version as stated in the previous sentence.
   
\item[J.]
   Preserve the network location, if any, given in the Document for
   public access to a Transparent copy of the Document, and likewise
   the network locations given in the Document for previous versions
   it was based on.  These may be placed in the ``History'' section.
   You may omit a network location for a work that was published at
   least four years before the Document itself, or if the original
   publisher of the version it refers to gives permission.

\section{Dedications}

To my wife, Pat.

\section{Acknowledgements}

Many people have helped to make this book, and its freedoms, possible.

First, the time to create, edit and distribute the book has been provided implicitly and explicitly by the University of Puget Sound. A sabbatical leave Spring 2004, a course release in Spring 2007 and a Lantz Senior Fellowship for the 2010-11 academic year are three obvious examples of explicit support. The course release was provided by support from the Lind-VanEnkevort Fund. The university has also provided clerical support, computer hardware, network servers and bandwidth. Thanks to Dean Kris Bartanen and the chairs of the Mathematics and Computer Science Department, Professors Martin Jackson, Sigrun Bodine and Bryan Smith, for their support, encouragement and flexibility.

My colleagues in the Mathematics and Computer Science Department have graciously taught our introductory linear algebra course using earlier versions and have provided valuable suggestions that have improved the book immeasurably. Thanks to Professor Martin Jackson (v0.30), Professor David Scott (v0.70), Professor Bryan Smith (v0.70, 0.80, v1.00, v2.00, v2.20), Professor Manley Perkel (v2.10), and Professor Cynthia Gibson (v2.20).

University of Puget Sound librarians Lori Ricigliano, Elizabeth Knight and Jeanne Kimura provided valuable advice on production, and interesting conversations about copyrights.

Many aspects of the book have been influenced by insightful questions and creative suggestions from the students who have labored through the book in our courses. For example, the flashcards with theorems and definitions are a direct result of a student suggestion. I will single out a handful of students at the University of Puget Sound who have been especially adept at finding and reporting mathematically significant typographical errors: Jake Linenthal, Christie Su, Kim Le, Sarah McQuate, Andy Zimmer, Travis Osborne, Andrew Tapay, Mark Shoemaker, Tasha Underhill, Tim Zitzer, Elizabeth Million, Steve Canfield, Jinshil Yi, Cliff Berger, Preston Van Buren, Duncan Bennett, Dan Messenger, Caden Robinson, Glenna Toomey, Tyler Ueltschi, Kyle Whitcomb, Anna Dovzhik, Chris Spalding and Jenna Fontaine. All the students of the Fall 2012 Math 290 sections were very helpful and patient through the major changes required in making Version 3.00.

I have tried to be as original as possible in the organization and presentation of this beautiful subject. However, I have been influenced by many years of teaching from another excellent textbook, Introduction to Linear Algebra by L.W. Johnson, R.D. Reiss and J.T. Arnold. When I have needed inspiration for the correct approach to particularly important proofs, I have learned to eventually consult two other textbooks. Sheldon Axler's Linear Algebra Done Right is a highly original exposition, while Ben Noble's Applied Linear Algebra frequently strikes just the right note between rigor and intuition. Noble's excellent book is highly recommended, even though its publication dates to 1969.

Conversion to various electronic formats have greatly depended on assistance from: Eitan Gurari, author of the powerful LaTeX translator, tex4ht; Davide Cervone, author of jsMath and MathJax; and Carl Witty, who advised and tested the Sony Reader format. Thanks to these individuals for their critical assistance.

Incorporation of Sage code is made possible by the entire community of Sage developers and users, who create and refine the mathematical routines, the user interfaces and applications in educational settings. Technical and logistical aspects of incorporating Sage code in open textbooks was supported by a grant from the United States National Science Foundation (DUE-1022574), which has been administered by the American Institute of Mathematics, and in particular, David Farmer. The support and assistance of my fellow Principal Investigators, Jason Grout, Tom Judson, Kiran Kedlaya, Sandra Laursen, Susan Lynds, and William Stein is especially appreciated.

David Farmer and Sally Koutsoliotas are responsible for the vision and initial experiments which lead to the knowl-enabled web version, as part of the Version 3 project.

General support and encouragement of free and affordable textbooks, in addition to specific promotion of this text, was provided by Nicole Allen, Textbook Advocate at Student Public Interest Research Groups. Nicole was an early consumer of this material, back when it looked more like lecture notes than a textbook.

Finally, in every respect, the production and distribution of this book has been accomplished with open source software. The range of individuals and projects is far too great to pretend to list them all. This project is an attempt to pay it forward.
  
\end{document}

%%% Local Variables:
%%% mode: latex
%%% TeX-master: t
%%% End:
