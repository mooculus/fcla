\documentclass{ximera}

% These macros are automatically generated from the "macros"
% XML element.  Make permanent edits there.
%
% History
%   2004/01/01  Initiated for FCLA, evolved from there
%   2006/09/17  Converted  _, ^  to \sb, \sp for TeX4ht
%   2014/02/01  Updated for MathBook XML projects
%               Obsolete in FCLA: \codeindent, \computerfont, \define
%               Change: MathJax wants \lt, so replaced by \lteval
%   2014/02/22  New: \orderof, \reals, \per
%   2015/08/16  Incorporated into MathBook XML version of FCLA
%
%%%%%%%%%%%%%%%%%%%%%
%
%     Conveniences
%
%%%%%%%%%%%%%%%%%%%%%
%
%  Order of (asymptotically limit of fraction is 1)
%  Usage: \orderof{some function}
%
\newcommand{\orderof}[1]{\sim #1}
%
%  Integers
%  Usage:  \Z
\newcommand{\Z}{\mathbb{Z}}
%
%  Real numbers, as set of scalars
%  Usage:  \reals
\newcommand{\reals}{\mathbb{R}}
%
%  n-space over real field
%  Usage: \complex{integer-dimension}
\newcommand{\real}[1]{\mathbb{R}^{#1}}
%
%  Complex numbers, as set of scalars
%  Usage:  \complexes
\newcommand{\complexes}{\mathbb{C}}
%
%  n-space over complex field
%  Usage: \complex{integer-dimension}
\newcommand{\complex}[1]{\mathbb{C}^{#1}}
%
%  Complex conjugation (scalar, vector, matrix)
%  Usage: \conjugate{object}
\newcommand{\conjugate}[1]{\overline{#1}}
%
%  Complex number modulus
%  Usage: \modulus{a+bi}
%  Presumes math mode
\newcommand{\modulus}[1]{\left\lvert#1\right\rvert}
%
%  Zero vector
%  Usage: \zerovector
\newcommand{\zerovector}{\vect{0}}
%
%  Zero matrix
%  Usage: \zeromatrix, use a subscript when size is important
\newcommand{\zeromatrix}{\mathcal{O}}
%
%  Inner product (brackets, not quadratic form)
%  Usage: \innerproduct{a-vector}{a-vector}
\newcommand{\innerproduct}[2]{\left\langle#1,\,#2\right\rangle}
%
%  Norm of a vector
%  Usage: \norm{a-vector}
\newcommand{\norm}[1]{\left\lVert#1\right\rVert}
%
%  Dimension
%  Usage: \dimension{vector-space-letter}
\newcommand{\dimension}[1]{\dim\left(#1\right)}
%
%  Nullity
%  Usage: \nullity{matrix-or-lintrans-letter}
\newcommand{\nullity}[1]{n\left(#1\right)}
%
%  Rank
%  Usage: \rank{matrix-or-lintrans-letter}
\newcommand{\rank}[1]{r\left(#1\right)}
%
%  Direct sum
%  Usage: \ds between a couple of subspaces
%
\newcommand{\ds}{\oplus}
%
%  Determinant of a matrix (functional)
%  Usage: \detname{A}
\newcommand{\detname}[1]{\det\left(#1\right)}
%
%  Determinant of a matrix (vertical bars)
%  Usage: \detbars{A}
\newcommand{\detbars}[1]{\left\lvert#1\right\rvert}
%
%  Trace of a Matrix
%  Usage: \trace{matrix name}
\newcommand{\trace}[1]{t\left(#1\right)}
%
%  Square Root of a Matrix
%  Usage: \sr{a-matrix}
\newcommand{\sr}[1]{#1^{1/2}}
%
%%%%%%%%%%%%%%%%%%%%%
%
%     Subspace Constructions
%
%%%%%%%%%%%%%%%%%%%%%
%
%  Span of a set of vectors
%  \span and \sp are used by TeX for other things
%  Usage: \spn{set-of-vectors}
\newcommand{\spn}[1]{\left\langle#1\right\rangle}
%
%  Null space of a matrix
%  Usage:  \nsp{A}
\newcommand{\nsp}[1]{\mathcal{N}\!\left(#1\right)}
%
%  Column space of a matrix
%  Usage:  \csp{A}
\newcommand{\csp}[1]{\mathcal{C}\!\left(#1\right)}
%
%  Row space of a matrix
%  Usage:  \rsp{A}
\newcommand{\rsp}[1]{\mathcal{R}\!\left(#1\right)}
%
%  Left null space of a matrix
%  Usage:  \lns{A}
\newcommand{\lns}[1]{\mathcal{L}\!\left(#1\right)}
%
%  Orthogonal complement of a vector space
%  Avoiding TeX's \perp
%  Usage:  \per{A}
\newcommand{\per}[1]{#1^\perp}
%
%%%%%%%%%%%%%%%%%%%%%
%
%     Systems of Equations
%
%%%%%%%%%%%%%%%%%%%%%
%
%  In-line form of an augmented matrix for a system of equations
%  Usage: \augmented{coefficient-matrix}{constant-vector}
\newcommand{\augmented}[2]{\left\lbrack\left.#1\,\right\rvert\,#2\right\rbrack}
%
%  Notation for a linear system before introducing matrix multiplication
%  Usage: \linearsystem{coefficient-matrix}{constant-vector}
\newcommand{\linearsystem}[2]{\mathcal{LS}\!\left(#1,\,#2\right)}
%
%  Notation for a homogenous system before introducing matrix multiplication
%  Usage: \homosystem{coefficient-matrix}
\newcommand{\homosystem}[1]{\linearsystem{#1}{\zerovector}}
%
%%%%%%%%%%%%%%%%%%%%%
%
%     Row Operations, Echelon Form
%
%%%%%%%%%%%%%%%%%%%%%
%
% Row operations on matrices
%
% Three commands to shorten up descriptions of gaussian elimination
%
% Usage: \rowopswap{row-i}{row-j}
% Usage: \rowopmult{scalar}{row-i}
% Usage: \rowopadd{scalar}{row-multiplied}{row-added-to}
\newcommand{\rowopswap}[2]{R_{#1}\leftrightarrow R_{#2}}
\newcommand{\rowopmult}[2]{#1R_{#2}}
\newcommand{\rowopadd}[3]{#1R_{#2}+R_{#3}}
%
% Mark leading 1's in echelon form with fbox
% Usage: \leading{a-1-usually}
\newcommand{\leading}[1]{\fbox{#1}}
%
%  Row-reduce arrow
%  Usage:  \rref inbetween a matrix and its reduced row-echelon form
\newcommand{\rref}{\xrightarrow{\text{RREF}}}
%
%  Elementary Matrices
%  Usage: \elemswap{subscript}{subscript}
%  Usage: \elemmult{scalar}{subscript}
%  Usage: \elemadd{scalar}{subscript-mult}{subscript-target}
%
\newcommand{\elemswap}[2]{E_{#1,#2}}
\newcommand{\elemmult}[2]{E_{#2}\left(#1\right)}
\newcommand{\elemadd}[3]{E_{#2,#3}\left(#1\right)}
%
%%%%%%%%%%%%%%%%%%%%%
%
%     2-D Constructions (Lists, Vectors, Matrices)
%
%%%%%%%%%%%%%%%%%%%%%
%
%  A list of scalars of generic length
%  Usage:  \scalarlist{scalar letter}{terminal subscript}
\newcommand{\scalarlist}[2]{{#1}_{1},\,{#1}_{2},\,{#1}_{3},\,\ldots,\,{#1}_{#2}}
%
%  Vector styling, bold (or use wiggles, arrows, whatever)
%  Subscripts go outside this construction
%  Usage: \vect{a symbol to use as a vector}
%  Have to already be in math mode
%
\newcommand{\vect}[1]{\mathbf{#1}}
%
%  A column vector
%  Usage: \colvector{list-delimited-by-\\}
%
\newcommand{\colvector}[1]{\begin{bmatrix}#1\end{bmatrix}}
%
%  A generic vector with components
%  Usage: \vectorcomponents{component-letter}{final-subscript}
\newcommand{\vectorcomponents}[2]{\colvector{#1_{1}\\#1_{2}\\#1_{3}\\\vdots\\#1_{#2}}}
%
%  A list of vectors of generic length
%  Usage:  \vectorlist{vector letter}{terminal subscript}
\newcommand{\vectorlist}[2]{\vect{#1}_{1},\,\vect{#1}_{2},\,\vect{#1}_{3},\,\ldots,\,\vect{#1}_{#2}}
%
%  Vector entries, entry i of vector v
%  (vector-expession still needs \vect, etc.)
%  Usage:  \vectorentry{vector-expression}{single-subscript}
\newcommand{\vectorentry}[2]{\left\lbrack#1\right\rbrack_{#2}}
%
%  Matrix entries, entry i,j of matrix A
%  Usage:  \matrixentry{matrix-expression}{paired-subscripts}
%
\newcommand{\matrixentry}[2]{\left\lbrack#1\right\rbrack_{#2}}
%
%  A generic linear combination
%  Usage:  \lincombo{scalar letter}{vector letter}{terminal subscript}
\newcommand{\lincombo}[3]{#1_{1}\vect{#2}_{1}+#1_{2}\vect{#2}_{2}+#1_{3}\vect{#2}_{3}+\cdots +#1_{#3}\vect{#2}_{#3}}
%
%  Matrix, column by column, as vectors
%  Usage:  \matrixcolumns{matrix letter}{terminal subscript}
\newcommand{\matrixcolumns}[2]{\left\lbrack\vect{#1}_{1}|\vect{#1}_{2}|\vect{#1}_{3}|\ldots|\vect{#1}_{#2}\right\rbrack}
%
%%%%%%%%%%%%%%%%%%%%%
%
%     Special Matrices
%
%%%%%%%%%%%%%%%%%%%%%
%
%  Transpose of a matrix
%  Usage:  \transpose{A}
\newcommand{\transpose}[1]{#1^{t}}
%
%  Inverse of a matrix
%  Usage:  \inverse{A}
\newcommand{\inverse}[1]{#1^{-1}}
%
%  Submatrix (for minors, determinants)
%  Usage: \submatrix{matrix-name}{delete-row}{delete-col}
\newcommand{\submatrix}[3]{#1\left(#2|#3\right)}
%
%  Adjoint of a matrix (twice)
%  This macro is a convenience to call \transpose and \conjugate properly
%  It shouldn't need to be modified (or mathematical meanings will change)
%  Usage:  \adj{A}
\newcommand{\adj}[1]{\transpose{\left(\conjugate{#1}\right)}}
%
%  This macro controls the symbol used for the adjoint
%  It can be edited to taste
%  Usage:  \adjoint{A}
\newcommand{\adjoint}[1]{#1^\ast}
%
%%%%%%%%%%%%%%%%%%%%%
%
%     Sets
%
%%%%%%%%%%%%%%%%%%%%%
%
%  A convenience for simple sets
%  Usage:  \set{list of element}
\newcommand{\set}[1]{\left\{#1\right\}}
%
%  Sets with vertical bar, "such that", sized for objects, not condition
%  Usage:  \setparts{objects}{condition}
%
%%\newcommand{\setparts}[2]{\left\{ #1\mid#2\right\}}
%%\newcommand{\setparts}[2]{\left\{\left. #1\right\rvert#2\right\}}
\newcommand{\setparts}[2]{\left\lbrace#1\,\middle|\,#2\right\rbrace}
%
%  Set Cardinality
%  Usage:  \card{a-set-letter}
\newcommand{\card}[1]{\left\lvert#1\right\rvert}
%
%  Set Union
%  Use \cup
%
%  Set Intersection
%  Use \cap
%
%  Set Complement
%  Usage:  \setcomplement{a-set-letter}
\newcommand{\setcomplement}[1]{\overline{#1}}
%
%%%%%%%%%%%%%%%%%%%%%
%
%     Eigenvalues and Eigenspaces
%
%%%%%%%%%%%%%%%%%%%%%
%
%  Characteristic polynomial
%  Usage: \charpoly{matrix-letter}{variable-letter}
\newcommand{\charpoly}[2]{p_{#1}\left(#2\right)}
%
%  Eigenspace
%  Usage: \eigenspace{matrix-letter}{eigenvalue-letter}
\newcommand{\eigenspace}[2]{\mathcal{E}_{#1}\left(#2\right)}
%
%  2013/10/03 Including ampersands is problematic here, 
%  think about fixes later
%  2014/02/22 Limited testing, seems &amp; is fine for HTML and LaTeX
%  2016-07-20 only employed in Archetypes, MBX has gather/align override
%  Eigensystem (presumes wrapped in an mrow within md)
%  Usage: \eigensystem{matrixletter}{eigenvalue}{list of basis vectors}
\newcommand{\eigensystem}[3]{\lambda&amp;=#2&amp;\eigenspace{#1}{#2}&amp;=\spn{\set{#3}}} 
%
%  Generalized Eigenspace
%  Usage: \geneigenspace{lin-trans-letter}{eigenvalue-letter}
\newcommand{\geneigenspace}[2]{\mathcal{G}_{#1}\left(#2\right)}
%
%  Algebraic multiplicty
%  Usage: \algmult{matrix-letter}{eigenvalue-letter}
\newcommand{\algmult}[2]{\alpha_{#1}\left(#2\right)}
%
%  Geometric multiplicty
%  Usage: \geomult{matrix-letter}{eigenvalue-letter}
\newcommand{\geomult}[2]{\gamma_{#1}\left(#2\right)}
%
%  Index (of eigenvalue)
%  Usage: \indx{matrix-letter}{eigenvalue-letter}
\newcommand{\indx}[2]{\iota_{#1}\left(#2\right)}
%
%%%%%%%%%%%%%%%%%%%%%
%
%     Linear Transformations
%
%%%%%%%%%%%%%%%%%%%%%
%
%  MathJax defines \lt to ease XML confusion
%
%  Linear transformation definition
%  Usage: \ltdefn{name-letter}{domain}{range}
\newcommand{\ltdefn}[3]{#1\colon #2\rightarrow#3}
%
%  Linear transformation evaluation
%  Usage: \lteval{name-letter}{input}
%  Replaces old \lt desired by MathJax
\newcommand{\lteval}[2]{#1\left(#2\right)}
%
% Linear transformation inverse
%  Usage: \ltinverse{name-letter}
\newcommand{\ltinverse}[1]{#1^{-1}}
%
%  Linear transformation restriction
%  Usage: \restrict{name-letter}{subspace-letter}
\newcommand{\restrict}[2]{{#1}|_{#2}}
%
%  Linear transformation preimage
%  Usage: \preimage{name-letter}{codomain-element}
\newcommand{\preimage}[2]{#1^{-1}\left(#2\right)}
%
%  Range of a linear transformation
%  TeX uses \range for something else
%  Usage:  \rng{T}
\newcommand{\rng}[1]{\mathcal{R}\!\left(#1\right)}
%
%  Kernel of a linear transformation
%  TeX uses \ker to do something different
%  Usage:  \krn{T}
\newcommand{\krn}[1]{\mathcal{K}\!\left(#1\right)}
%
%  Linear transformation composition
%  Usage: \compose{function-name}{function-name}
\newcommand{\compose}[2]{{#1}\circ{#2}}
%
%  Vector space of linear transformations
%  Usage: \vslt{domains}{codomains}
%  Presumes math mode
\newcommand{\vslt}[2]{\mathcal{LT}\left(#1,\,#2\right)}
%
%%%%%%%%%%%%%%%%%%%%%
%
%     Vector and Matrix Representations
%
%%%%%%%%%%%%%%%%%%%%%
%
%  Isomorphism symbol
%  Usage: \isomorphic
\newcommand{\isomorphic}{\cong}
%
%  Similarity
%  Usage: \similar{inner-matrix}{outer-invertible-matrix}
%  Rearranging this will not "fix" all desired changes throughout
%
\newcommand{\similar}[2]{\inverse{#2}#1#2}
%
%  Vector representation function name
%  Usage: \vectrepname{basis-letter}
\newcommand{\vectrepname}[1]{\rho_{#1}}
%
%  Vector representation output
%  Usage: \vectrep{basis-letter}{input}
\newcommand{\vectrep}[2]{\lteval{\vectrepname{#1}}{#2}}
%
%  Vector representation inverse function name
%  (Added later, not used consistently in FCLA)
%  Usage: \vectrepinvname{basis-letter}
\newcommand{\vectrepinvname}[1]{\ltinverse{\vectrepname{#1}}}
%
%  Vector representation inverse output
%  Usage: \vectrepinv{basis-letter}{input}
\newcommand{\vectrepinv}[2]{\lteval{\ltinverse{\vectrepname{#1}}}{#2}}
%
%  Matrix representation
%  Usage: \matrixrep{transformation-letter}{domain-basis-letter}{codomain-basis-letter}
\newcommand{\matrixrep}[3]{M^{#1}_{#2,#3}}
%
%  Matrix representation column-by-colum
%  2016-07-20 only employed once?
%  Usage: \matrixrepcolumns{transformation-letter}{codomain-basis-letter}{codomain-basis-vector-letter}{final-index}
\newcommand{\matrixrepcolumns}[4]{\left\lbrack \left.\vectrep{#2}{\lteval{#1}{\vect{#3}_{1}}}\right|\left.\vectrep{#2}{\lteval{#1}{\vect{#3}_{2}}}\right|\left.\vectrep{#2}{\lteval{#1}{\vect{#3}_{3}}}\right|\ldots\left|\vectrep{#2}{\lteval{#1}{\vect{#3}_{#4}}}\right.\right\rbrack}
%
%  Change of basis matrix
%  Usage: \cbm{domain-basis-letter}{codomain-basis-letter}
\newcommand{\cbm}[2]{C_{#1,#2}}
%
%%%%%%%%%%%%%%%%%%%%%
%
%     Canonical Forms
%
%%%%%%%%%%%%%%%%%%%%%
%
%  Jordan blocks
%  Usage: \jordan{size}{diagonal-element}
\newcommand{\jordan}[2]{J_{#1}\left(#2\right)}
%
%%%%%%%%%%%%%%%%%%%%%
%
%     Hadamard Matrices
%     Contributed by Elizabeth Million
%
%%%%%%%%%%%%%%%%%%%%%
%
%  Hadamard Product
%  Usage: \hadamard{a-matrix}{a-matrix}
\newcommand{\hadamard}[2]{#1\circ #2}
%
%  Hadamard identity matrix
%  Usage: \hadamardidentity{paired-subscripts-size-of-matrix}
\newcommand{\hadamardidentity}[1]{J_{#1}}
%
%  Hadamard inverse matrix
%  Usage: \hadamardinverse{matrix-expression}
\newcommand{\hadamardinverse}[1]{\widehat{#1}}


\title{Testing subspaces}

\begin{document}
\begin{abstract}
  To check that a subset is a subspace, six of the properties were
  easy to prove, and we can lean on some of the properties of the
  larger vector space to make the other four easier.
\end{abstract}
\maketitle

Here is a theorem that will make it easier to test if a subset is a vector space.  It's a shortcut if there ever was one!

\begin{theorem}[Testing Subsets for Subspaces]
  \label{theorem:TSS}

  Suppose that $V$ is a vector space and $W$ is a subset of $V$,
  $W\subseteq V$.  Endow $W$ with the same operations as $V$.  Then $W$
  is a subspace if and only if three conditions are met
  \begin{enumerate}
  \item $W$ is nonempty, $W\neq\emptyset$.
  \item If $\vect{x}\in W$ and $\vect{y}\in W$, then $\vect{x}+\vect{y}\in W$.
  \item If $\alpha\in\complexes$ and $\vect{x}\in W$, then $\alpha\vect{x}\in W$.
  \end{enumerate}
  
  \begin{proof}
    ($\Rightarrow$) We have the hypothesis that $W$ is a subspace, so
    by \ref{property:Z} we know that $W$ contains a zero vector.  This
    is enough to show that $W\neq\emptyset$.  Also, since $W$ is a
    vector space it satisfies the additive and scalar multiplication
    closure properties (\ref{property:AC}, \ref{property:SC}), and so
    exactly meets the second and third conditions.  If that was easy,
    the other direction might require a bit more work.

    ($\Leftarrow$) We have three properties for our hypothesis, and
    from this we should conclude that $W$ has the ten defining
    properties of a vector space.  The second and third conditions of
    our hypothesis are exactly \ref{property:AC} and
    \ref{property:SC}.  Our hypothesis that $V$ is a vector space
    implies that \ref{property:C}, \ref{property:AA},
    \ref{property:SMA}, \ref{property:DVA}, \ref{property:DSA} and
    \ref{property:O} all hold.  They continue to be true for vectors
    from $W$ since passing to a subset, and keeping the operation the
    same, leaves their statements unchanged.  Eight down, two to go.

    Suppose $\vect{x}\in W$.  Then by the third part of our hypothesis
    (scalar closure), we know that $(-1)\vect{x}\in W$.  By
    \ref{theorem:AISM} $(-1)\vect{x}=\vect{-x}$, so together these
    statements show us that $\vect{-x}\in W$.  $\vect{-x}$ is the
    additive inverse of $\vect{x}$ in $V$, but will continue in this
    role when viewed as an element of the subset $W$.  So every
    element of $W$ has an additive inverse that is an element of $W$
    and \ref{property:AI} is completed.  Just one property left.

    While we have implicitly discussed the zero vector in the previous
    paragraph, we need to be certain that the zero vector (of $V$)
    really lives in $W$.  Since $W$ is nonempty, we can choose some
    vector $\vect{z}\in W$.  Then by the argument in the previous
    paragraph, we know $\vect{-z}\in W$.  Now by \ref{property:AI} for
    $V$ and then by the second part of our hypothesis (additive
    closure) we see that
    \[
      \zerovector=\vect{z}+(\vect{-z})\in W
    \]

    So $W$ contains the zero vector from $V$.  Since this vector
    performs the required duties of a zero vector in $V$, it will
    continue in that role as an element of $W$. This gives us,
    \ref{property:Z}, the final property of the ten required.
    %(<contributorname code="sarahfellez" /> contributed to this    proof.)

\end{proof}
\end{theorem}

So just three conditions, plus being a subset of a known vector space,
gets us all ten properties.  Fabulous!  This theorem can be
paraphrased by saying that a subspace is ``a nonempty subset (of a
vector space) that is closed under vector addition and scalar
multiplication.''

You might want to go back and rework previous examples in light of
this result, perhaps seeing where we can now economize or where the
work done in the example mirrored the proof and where it did not.  We
will press on and apply this theorem in a slightly more abstract
setting.

\begin{example}[A subspace of $P_4$]

  $P_4$ is the vector space of polynomials with degree at most $4$ (\ref{example:VSP}).  Define a subset $W$ as
  \[
    W=\setparts{p(x)}{p\in P_4,\ p(2)=0}
  \]
  so $W$ is the collection of those polynomials (with degree 4 or
  less) whose graphs cross the $x$-axis at $x=2$.  Whenever we
  encounter a new set it is a good idea to gain a better understanding
  of the set by finding a few elements in the set, and a few outside
  it.  For example $x^2-x-2\in W$, while $x^4+x^3-7\not\in W$.

  Is $W$ nonempty?  
  \begin{multipleChoice}
    \choice[correct]{Yes}
    \choice{No}
  \end{multipleChoice}

  \begin{feedback}[correct]
    Indeed $x-2\in W$, so $W \neq \varnothing$.
  \end{feedback}

  Additive closure?  Suppose $p\in W$ and $q\in W$.  Is $p+q\in W$?
  $p$ and $q$ are not totally arbitrary, we know that $p(2)=0$ and
  $q(2)=0$.  Then we can check $p+q$ for membership in $W$,
  \begin{align*}
    (p+q)(2)&=p(2)+q(2)&&\text{Addition in }P_4\\
            &=0+0&&p\in W,\,q\in W\\
            &=0
  \end{align*}
  so we see that $p+q$ qualifies for membership in $W$.
  
  Scalar multiplication closure?  Suppose that $\alpha\in\complexes$ and $p\in W$.  Then we know that $p(2)=0$.  Testing $\alpha p$ for membership,
  \begin{align*}
    (\alpha p)(2)&=\alpha p(2)&&\text{Scalar multiplication in }P_4\\
                 &=\alpha 0&&p\in W\\
                 &=0
  \end{align*}
  so $\alpha p\in W$.
  
  We have shown that $W$ meets the three conditions of
  \ref{theorem:TSS} and so qualifies as a subspace of $P_4$.  Notice
  that by \ref{definition:S} we now know that $W$ is also a vector
  space.  So all the properties of a vector space
  (\ref{definition:VS}) and the theorems of \ref{section:VS} apply in
  full.

\end{example}

Much of the power of \ref{theorem:TSS} is that we can easily establish
new vector spaces if we can locate them as subsets of other vector
spaces, such as the vector spaces presented previously.

It can be as instructive to consider some subsets that are
\textit{not} subspaces.  Since \ref{theorem:TSS} is an equivalence
(see \ref{technique:E}) we can be assured that a subset is not a
subspace if it violates one of the three conditions, and in any
example of interest this will not be the ``nonempty'' condition.
However, since a subspace has to be a vector space in its own right,
we can also search for a violation of any one of the ten defining
properties in \ref{definition:VS} or any inherent property of a vector
space.  Notice also that a violation need only be for a specific
vector or pair of vectors.

\begin{example}[A non-subspace in $\complex{2}$, zero vector]

  Consider the subset $W$ below as a candidate for being a subspace of $\complex{2}$
  \[
    W=\setparts{\colvector{x_1\\x_2}}{3x_1-5x_2=12}
  \]

  The zero vector of $\complex{2}$, $\zerovector=\colvector{0\\0}$
  will need to be the zero vector in $W$ also.  However,
  $\zerovector\not\in W$ since $3(0)-5(0)=0\neq 12$.  So $W$ has no
  zero vector and fails \ref{property:Z} of \ref{definition:VS}.  This
  subspace also fails to be closed under addition and scalar
  multiplication.  Can you find examples of this?
\end{example}

\begin{example}[A non-subspace in $\complex{2}$, additive closure]
  
  Consider the subset $X$ below as a candidate for being a subspace of $\complex{2}$
  \[
    X=\setparts{\colvector{x_1\\x_2}}{x_1x_2=0}
  \]

  You can check that $\zerovector\in X$, so the approach of the last
  example will not get us anywhere.  However, notice that
  $\vect{x}=\colvector{1\\0}\in X$ and
  $\vect{y}=\colvector{0\\1}\in X$.  Yet
  \[
    \vect{x}+\vect{y}=\colvector{1\\0}+\colvector{0\\1}=\colvector{1\\1}\not\in X
  \]

  So $X$ fails the additive closure requirement of either
  \ref{property:AC} or \ref{theorem:TSS}, and is therefore not a
  subspace.
\end{example}

\begin{example}[A non-subspace in $\complex{2}$, scalar multiplication closure]

  Consider the subset $Y$ below as a candidate for being a subspace of $\complex{2}$
  \[
    Y=\setparts{\colvector{x_1\\x_2}}{x_1\in{\mathbb Z},\,x_2\in{\mathbb Z}}
  \]
  The set ${\mathbb Z}$ is the set of integers, so we are only
  allowing ``whole numbers'' as the constituents of our vectors.  Now,
  $\zerovector\in Y$, and additive closure also holds (can you prove
  these claims?).  So we will have to try something different.  Note
  that $\alpha = \frac{1}{2}\in\complexes$ and
  $\colvector{2\\3}\in Y$, but
  \[
    \alpha\vect{x}=\frac{1}{2}\colvector{2\\3}=\colvector{1\\\frac{3}{2}}\not\in Y
  \]
  So $Y$ fails the scalar multiplication closure requirement of either
  \ref{property:SC} or \ref{theorem:TSS}, and is therefore not a
  subspace.
\end{example}

There are two examples of subspaces that are trivial.  Suppose that
$V$ is any vector space.  Then $V$ is a subset of itself and is a
vector space.  By \ref{definition:S}, $V$ qualifies as a subspace of
itself.  The set containing just the zero vector $Z=\set{\zerovector}$
is also a subspace as can be seen by applying \ref{theorem:TSS} or by
simple modifications of the techniques hinted at in \ref{example:VSS}.
Since these subspaces are so obvious (and therefore not too
interesting) we will refer to them as being trivial.

\begin{definition}[Trivial Subspaces]
  Given the vector space $V$, the subspaces $V$ and
  $\set{\zerovector}$ are each called a \dfn{trivial subspace}.
\end{definition}

We can also use \ref{theorem:TSS} to prove more general statements
about subspaces, as illustrated in the next theorem.

\begin{theorem}[Null Space of a Matrix is a Subspace]
  \label{theorem:NSMS}

  Suppose that $A$ is an $m\times n$ matrix.  Then the null space of
  $A$, $\nsp{A}$, is a subspace of $\complex{n}$.
  
  \begin{proof}
    We will examine the three requirements of \ref{theorem:TSS}.
    Recall that \ref{definition:NSM} can be formulated as
    $\nsp{A}=\setparts{\vect{x}\in\complex{n}}{A\vect{x}=\zerovector}$.

    First, $\zerovector\in\nsp{A}$, which can be inferred as a
    consequence of \ref{theorem:HSC}.  Therefore
    \begin{multipleChoice}
      \choice{$\nsp{A}=\emptyset$.}
      \choice[correct]{$\nsp{A}\neq\emptyset$.}
    \end{multipleChoice}

    Second, check additive closure by supposing that
    $\vect{x}\in\nsp{A}$ and $\vect{y}\in\nsp{A}$.  So we know a
    little something about $\vect{x}$ and $\vect{y}$:
    $A\vect{x}=\zerovector$ and $A\vect{y}=\zerovector$, and that is
    all we know.  Question: Is $\vect{x}+\vect{y}\in\nsp{A}$?  Let us
    check.
    \begin{align*}
      A(\vect{x}+\vect{y})&=A\vect{x}+A\vect{y}&&\ref{theorem:MMDAA}\\
                          &=\zerovector+\zerovector&&\vect{x}\in\nsp{A},\ \vect{y}\in\nsp{A}\\
                          &=\zerovector&&\ref{theorem:VSPCV}
    \end{align*}
    So, yes, $\vect{x}+\vect{y}$ qualifies for membership in $\nsp{A}$.

    Third, check scalar multiplication closure by supposing that
    $\alpha\in\complexes$ and $\vect{x}\in\nsp{A}$.  So we know a
    little something about $\vect{x}$, namely $A\vect{x}=\zerovector$,
    and that is all we know.  Question: Is $\alpha\vect{x}\in\nsp{A}$?
    Let us check.
    \begin{align*}
      A(\alpha\vect{x})&=\alpha(A\vect{x})&&\ref{theorem:MMSMM}\\
                       &=\alpha\zerovector&&\vect{x}\in\nsp{A}\\
                       &=\zerovector&&\ref{theorem:ZVSM}
    \end{align*}
    And therefore
    \begin{multipleChoice}
      \choice[correct]{$\alpha\vect{x}$ qualifies for membership in $\nsp{A}$.}
      \choice{$\alpha\vect{x} \not\in \nsp{A}$.}
    \end{multipleChoice}

    Having met the three conditions in \ref{theorem:TSS} we can now
    say that the null space of a matrix is a subspace (and hence a
    vector space in its own right!).
  \end{proof}
\end{theorem}

Here is an example where we can exercise \ref{theorem:NSMS}.

\begin{example}[Recasting a subspace as a null space]

  Consider the subset of $\complex{5}$ defined as
  \[
    W =\setparts{\colvector{x_1\\x_2\\x_3\\x_4\\x_5}}{
      \begin{array}{l}
        3x_1+x_2-5x_3+7x_4+x_5=0,\\
        4x_1+6x_2+3x_3-6x_4-5x_5=0,\\
        -2x_1+4x_2+7x_4+x_5=0
      \end{array}
    }
  \]

  It is possible to show that $W$ is a subspace of $\complex{5}$ by
  checking the three conditions of \ref{theorem:TSS} directly, but it
  will get tedious rather quickly.  Instead, give $W$ a fresh look and
  notice that it is a set of solutions to a homogeneous system of
  equations.  Define the matrix
  \[
    A=\begin{bmatrix}
      3&1&-5&7&1\\
      4&6&3&-6&-5\\
      -2&4&0&7&1
    \end{bmatrix}
  \]
  and then recognize that $W=\nsp{A}$.  By \ref{theorem:NSMS} we can
  immediately see that $W$ is a subspace.  Boom!

\end{example}



\end{document}
