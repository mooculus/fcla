\documentclass{ximera}

% These macros are automatically generated from the "macros"
% XML element.  Make permanent edits there.
%
% History
%   2004/01/01  Initiated for FCLA, evolved from there
%   2006/09/17  Converted  _, ^  to \sb, \sp for TeX4ht
%   2014/02/01  Updated for MathBook XML projects
%               Obsolete in FCLA: \codeindent, \computerfont, \define
%               Change: MathJax wants \lt, so replaced by \lteval
%   2014/02/22  New: \orderof, \reals, \per
%   2015/08/16  Incorporated into MathBook XML version of FCLA
%
%%%%%%%%%%%%%%%%%%%%%
%
%     Conveniences
%
%%%%%%%%%%%%%%%%%%%%%
%
%  Order of (asymptotically limit of fraction is 1)
%  Usage: \orderof{some function}
%
\newcommand{\orderof}[1]{\sim #1}
%
%  Integers
%  Usage:  \Z
\newcommand{\Z}{\mathbb{Z}}
%
%  Real numbers, as set of scalars
%  Usage:  \reals
\newcommand{\reals}{\mathbb{R}}
%
%  n-space over real field
%  Usage: \complex{integer-dimension}
\newcommand{\real}[1]{\mathbb{R}^{#1}}
%
%  Complex numbers, as set of scalars
%  Usage:  \complexes
\newcommand{\complexes}{\mathbb{C}}
%
%  n-space over complex field
%  Usage: \complex{integer-dimension}
\newcommand{\complex}[1]{\mathbb{C}^{#1}}
%
%  Complex conjugation (scalar, vector, matrix)
%  Usage: \conjugate{object}
\newcommand{\conjugate}[1]{\overline{#1}}
%
%  Complex number modulus
%  Usage: \modulus{a+bi}
%  Presumes math mode
\newcommand{\modulus}[1]{\left\lvert#1\right\rvert}
%
%  Zero vector
%  Usage: \zerovector
\newcommand{\zerovector}{\vect{0}}
%
%  Zero matrix
%  Usage: \zeromatrix, use a subscript when size is important
\newcommand{\zeromatrix}{\mathcal{O}}
%
%  Inner product (brackets, not quadratic form)
%  Usage: \innerproduct{a-vector}{a-vector}
\newcommand{\innerproduct}[2]{\left\langle#1,\,#2\right\rangle}
%
%  Norm of a vector
%  Usage: \norm{a-vector}
\newcommand{\norm}[1]{\left\lVert#1\right\rVert}
%
%  Dimension
%  Usage: \dimension{vector-space-letter}
\newcommand{\dimension}[1]{\dim\left(#1\right)}
%
%  Nullity
%  Usage: \nullity{matrix-or-lintrans-letter}
\newcommand{\nullity}[1]{n\left(#1\right)}
%
%  Rank
%  Usage: \rank{matrix-or-lintrans-letter}
\newcommand{\rank}[1]{r\left(#1\right)}
%
%  Direct sum
%  Usage: \ds between a couple of subspaces
%
\newcommand{\ds}{\oplus}
%
%  Determinant of a matrix (functional)
%  Usage: \detname{A}
\newcommand{\detname}[1]{\det\left(#1\right)}
%
%  Determinant of a matrix (vertical bars)
%  Usage: \detbars{A}
\newcommand{\detbars}[1]{\left\lvert#1\right\rvert}
%
%  Trace of a Matrix
%  Usage: \trace{matrix name}
\newcommand{\trace}[1]{t\left(#1\right)}
%
%  Square Root of a Matrix
%  Usage: \sr{a-matrix}
\newcommand{\sr}[1]{#1^{1/2}}
%
%%%%%%%%%%%%%%%%%%%%%
%
%     Subspace Constructions
%
%%%%%%%%%%%%%%%%%%%%%
%
%  Span of a set of vectors
%  \span and \sp are used by TeX for other things
%  Usage: \spn{set-of-vectors}
\newcommand{\spn}[1]{\left\langle#1\right\rangle}
%
%  Null space of a matrix
%  Usage:  \nsp{A}
\newcommand{\nsp}[1]{\mathcal{N}\!\left(#1\right)}
%
%  Column space of a matrix
%  Usage:  \csp{A}
\newcommand{\csp}[1]{\mathcal{C}\!\left(#1\right)}
%
%  Row space of a matrix
%  Usage:  \rsp{A}
\newcommand{\rsp}[1]{\mathcal{R}\!\left(#1\right)}
%
%  Left null space of a matrix
%  Usage:  \lns{A}
\newcommand{\lns}[1]{\mathcal{L}\!\left(#1\right)}
%
%  Orthogonal complement of a vector space
%  Avoiding TeX's \perp
%  Usage:  \per{A}
\newcommand{\per}[1]{#1^\perp}
%
%%%%%%%%%%%%%%%%%%%%%
%
%     Systems of Equations
%
%%%%%%%%%%%%%%%%%%%%%
%
%  In-line form of an augmented matrix for a system of equations
%  Usage: \augmented{coefficient-matrix}{constant-vector}
\newcommand{\augmented}[2]{\left\lbrack\left.#1\,\right\rvert\,#2\right\rbrack}
%
%  Notation for a linear system before introducing matrix multiplication
%  Usage: \linearsystem{coefficient-matrix}{constant-vector}
\newcommand{\linearsystem}[2]{\mathcal{LS}\!\left(#1,\,#2\right)}
%
%  Notation for a homogenous system before introducing matrix multiplication
%  Usage: \homosystem{coefficient-matrix}
\newcommand{\homosystem}[1]{\linearsystem{#1}{\zerovector}}
%
%%%%%%%%%%%%%%%%%%%%%
%
%     Row Operations, Echelon Form
%
%%%%%%%%%%%%%%%%%%%%%
%
% Row operations on matrices
%
% Three commands to shorten up descriptions of gaussian elimination
%
% Usage: \rowopswap{row-i}{row-j}
% Usage: \rowopmult{scalar}{row-i}
% Usage: \rowopadd{scalar}{row-multiplied}{row-added-to}
\newcommand{\rowopswap}[2]{R_{#1}\leftrightarrow R_{#2}}
\newcommand{\rowopmult}[2]{#1R_{#2}}
\newcommand{\rowopadd}[3]{#1R_{#2}+R_{#3}}
%
% Mark leading 1's in echelon form with fbox
% Usage: \leading{a-1-usually}
\newcommand{\leading}[1]{\fbox{#1}}
%
%  Row-reduce arrow
%  Usage:  \rref inbetween a matrix and its reduced row-echelon form
\newcommand{\rref}{\xrightarrow{\text{RREF}}}
%
%  Elementary Matrices
%  Usage: \elemswap{subscript}{subscript}
%  Usage: \elemmult{scalar}{subscript}
%  Usage: \elemadd{scalar}{subscript-mult}{subscript-target}
%
\newcommand{\elemswap}[2]{E_{#1,#2}}
\newcommand{\elemmult}[2]{E_{#2}\left(#1\right)}
\newcommand{\elemadd}[3]{E_{#2,#3}\left(#1\right)}
%
%%%%%%%%%%%%%%%%%%%%%
%
%     2-D Constructions (Lists, Vectors, Matrices)
%
%%%%%%%%%%%%%%%%%%%%%
%
%  A list of scalars of generic length
%  Usage:  \scalarlist{scalar letter}{terminal subscript}
\newcommand{\scalarlist}[2]{{#1}_{1},\,{#1}_{2},\,{#1}_{3},\,\ldots,\,{#1}_{#2}}
%
%  Vector styling, bold (or use wiggles, arrows, whatever)
%  Subscripts go outside this construction
%  Usage: \vect{a symbol to use as a vector}
%  Have to already be in math mode
%
\newcommand{\vect}[1]{\mathbf{#1}}
%
%  A column vector
%  Usage: \colvector{list-delimited-by-\\}
%
\newcommand{\colvector}[1]{\begin{bmatrix}#1\end{bmatrix}}
%
%  A generic vector with components
%  Usage: \vectorcomponents{component-letter}{final-subscript}
\newcommand{\vectorcomponents}[2]{\colvector{#1_{1}\\#1_{2}\\#1_{3}\\\vdots\\#1_{#2}}}
%
%  A list of vectors of generic length
%  Usage:  \vectorlist{vector letter}{terminal subscript}
\newcommand{\vectorlist}[2]{\vect{#1}_{1},\,\vect{#1}_{2},\,\vect{#1}_{3},\,\ldots,\,\vect{#1}_{#2}}
%
%  Vector entries, entry i of vector v
%  (vector-expession still needs \vect, etc.)
%  Usage:  \vectorentry{vector-expression}{single-subscript}
\newcommand{\vectorentry}[2]{\left\lbrack#1\right\rbrack_{#2}}
%
%  Matrix entries, entry i,j of matrix A
%  Usage:  \matrixentry{matrix-expression}{paired-subscripts}
%
\newcommand{\matrixentry}[2]{\left\lbrack#1\right\rbrack_{#2}}
%
%  A generic linear combination
%  Usage:  \lincombo{scalar letter}{vector letter}{terminal subscript}
\newcommand{\lincombo}[3]{#1_{1}\vect{#2}_{1}+#1_{2}\vect{#2}_{2}+#1_{3}\vect{#2}_{3}+\cdots +#1_{#3}\vect{#2}_{#3}}
%
%  Matrix, column by column, as vectors
%  Usage:  \matrixcolumns{matrix letter}{terminal subscript}
\newcommand{\matrixcolumns}[2]{\left\lbrack\vect{#1}_{1}|\vect{#1}_{2}|\vect{#1}_{3}|\ldots|\vect{#1}_{#2}\right\rbrack}
%
%%%%%%%%%%%%%%%%%%%%%
%
%     Special Matrices
%
%%%%%%%%%%%%%%%%%%%%%
%
%  Transpose of a matrix
%  Usage:  \transpose{A}
\newcommand{\transpose}[1]{#1^{t}}
%
%  Inverse of a matrix
%  Usage:  \inverse{A}
\newcommand{\inverse}[1]{#1^{-1}}
%
%  Submatrix (for minors, determinants)
%  Usage: \submatrix{matrix-name}{delete-row}{delete-col}
\newcommand{\submatrix}[3]{#1\left(#2|#3\right)}
%
%  Adjoint of a matrix (twice)
%  This macro is a convenience to call \transpose and \conjugate properly
%  It shouldn't need to be modified (or mathematical meanings will change)
%  Usage:  \adj{A}
\newcommand{\adj}[1]{\transpose{\left(\conjugate{#1}\right)}}
%
%  This macro controls the symbol used for the adjoint
%  It can be edited to taste
%  Usage:  \adjoint{A}
\newcommand{\adjoint}[1]{#1^\ast}
%
%%%%%%%%%%%%%%%%%%%%%
%
%     Sets
%
%%%%%%%%%%%%%%%%%%%%%
%
%  A convenience for simple sets
%  Usage:  \set{list of element}
\newcommand{\set}[1]{\left\{#1\right\}}
%
%  Sets with vertical bar, "such that", sized for objects, not condition
%  Usage:  \setparts{objects}{condition}
%
%%\newcommand{\setparts}[2]{\left\{ #1\mid#2\right\}}
%%\newcommand{\setparts}[2]{\left\{\left. #1\right\rvert#2\right\}}
\newcommand{\setparts}[2]{\left\lbrace#1\,\middle|\,#2\right\rbrace}
%
%  Set Cardinality
%  Usage:  \card{a-set-letter}
\newcommand{\card}[1]{\left\lvert#1\right\rvert}
%
%  Set Union
%  Use \cup
%
%  Set Intersection
%  Use \cap
%
%  Set Complement
%  Usage:  \setcomplement{a-set-letter}
\newcommand{\setcomplement}[1]{\overline{#1}}
%
%%%%%%%%%%%%%%%%%%%%%
%
%     Eigenvalues and Eigenspaces
%
%%%%%%%%%%%%%%%%%%%%%
%
%  Characteristic polynomial
%  Usage: \charpoly{matrix-letter}{variable-letter}
\newcommand{\charpoly}[2]{p_{#1}\left(#2\right)}
%
%  Eigenspace
%  Usage: \eigenspace{matrix-letter}{eigenvalue-letter}
\newcommand{\eigenspace}[2]{\mathcal{E}_{#1}\left(#2\right)}
%
%  2013/10/03 Including ampersands is problematic here, 
%  think about fixes later
%  2014/02/22 Limited testing, seems &amp; is fine for HTML and LaTeX
%  2016-07-20 only employed in Archetypes, MBX has gather/align override
%  Eigensystem (presumes wrapped in an mrow within md)
%  Usage: \eigensystem{matrixletter}{eigenvalue}{list of basis vectors}
\newcommand{\eigensystem}[3]{\lambda&amp;=#2&amp;\eigenspace{#1}{#2}&amp;=\spn{\set{#3}}} 
%
%  Generalized Eigenspace
%  Usage: \geneigenspace{lin-trans-letter}{eigenvalue-letter}
\newcommand{\geneigenspace}[2]{\mathcal{G}_{#1}\left(#2\right)}
%
%  Algebraic multiplicty
%  Usage: \algmult{matrix-letter}{eigenvalue-letter}
\newcommand{\algmult}[2]{\alpha_{#1}\left(#2\right)}
%
%  Geometric multiplicty
%  Usage: \geomult{matrix-letter}{eigenvalue-letter}
\newcommand{\geomult}[2]{\gamma_{#1}\left(#2\right)}
%
%  Index (of eigenvalue)
%  Usage: \indx{matrix-letter}{eigenvalue-letter}
\newcommand{\indx}[2]{\iota_{#1}\left(#2\right)}
%
%%%%%%%%%%%%%%%%%%%%%
%
%     Linear Transformations
%
%%%%%%%%%%%%%%%%%%%%%
%
%  MathJax defines \lt to ease XML confusion
%
%  Linear transformation definition
%  Usage: \ltdefn{name-letter}{domain}{range}
\newcommand{\ltdefn}[3]{#1\colon #2\rightarrow#3}
%
%  Linear transformation evaluation
%  Usage: \lteval{name-letter}{input}
%  Replaces old \lt desired by MathJax
\newcommand{\lteval}[2]{#1\left(#2\right)}
%
% Linear transformation inverse
%  Usage: \ltinverse{name-letter}
\newcommand{\ltinverse}[1]{#1^{-1}}
%
%  Linear transformation restriction
%  Usage: \restrict{name-letter}{subspace-letter}
\newcommand{\restrict}[2]{{#1}|_{#2}}
%
%  Linear transformation preimage
%  Usage: \preimage{name-letter}{codomain-element}
\newcommand{\preimage}[2]{#1^{-1}\left(#2\right)}
%
%  Range of a linear transformation
%  TeX uses \range for something else
%  Usage:  \rng{T}
\newcommand{\rng}[1]{\mathcal{R}\!\left(#1\right)}
%
%  Kernel of a linear transformation
%  TeX uses \ker to do something different
%  Usage:  \krn{T}
\newcommand{\krn}[1]{\mathcal{K}\!\left(#1\right)}
%
%  Linear transformation composition
%  Usage: \compose{function-name}{function-name}
\newcommand{\compose}[2]{{#1}\circ{#2}}
%
%  Vector space of linear transformations
%  Usage: \vslt{domains}{codomains}
%  Presumes math mode
\newcommand{\vslt}[2]{\mathcal{LT}\left(#1,\,#2\right)}
%
%%%%%%%%%%%%%%%%%%%%%
%
%     Vector and Matrix Representations
%
%%%%%%%%%%%%%%%%%%%%%
%
%  Isomorphism symbol
%  Usage: \isomorphic
\newcommand{\isomorphic}{\cong}
%
%  Similarity
%  Usage: \similar{inner-matrix}{outer-invertible-matrix}
%  Rearranging this will not "fix" all desired changes throughout
%
\newcommand{\similar}[2]{\inverse{#2}#1#2}
%
%  Vector representation function name
%  Usage: \vectrepname{basis-letter}
\newcommand{\vectrepname}[1]{\rho_{#1}}
%
%  Vector representation output
%  Usage: \vectrep{basis-letter}{input}
\newcommand{\vectrep}[2]{\lteval{\vectrepname{#1}}{#2}}
%
%  Vector representation inverse function name
%  (Added later, not used consistently in FCLA)
%  Usage: \vectrepinvname{basis-letter}
\newcommand{\vectrepinvname}[1]{\ltinverse{\vectrepname{#1}}}
%
%  Vector representation inverse output
%  Usage: \vectrepinv{basis-letter}{input}
\newcommand{\vectrepinv}[2]{\lteval{\ltinverse{\vectrepname{#1}}}{#2}}
%
%  Matrix representation
%  Usage: \matrixrep{transformation-letter}{domain-basis-letter}{codomain-basis-letter}
\newcommand{\matrixrep}[3]{M^{#1}_{#2,#3}}
%
%  Matrix representation column-by-colum
%  2016-07-20 only employed once?
%  Usage: \matrixrepcolumns{transformation-letter}{codomain-basis-letter}{codomain-basis-vector-letter}{final-index}
\newcommand{\matrixrepcolumns}[4]{\left\lbrack \left.\vectrep{#2}{\lteval{#1}{\vect{#3}_{1}}}\right|\left.\vectrep{#2}{\lteval{#1}{\vect{#3}_{2}}}\right|\left.\vectrep{#2}{\lteval{#1}{\vect{#3}_{3}}}\right|\ldots\left|\vectrep{#2}{\lteval{#1}{\vect{#3}_{#4}}}\right.\right\rbrack}
%
%  Change of basis matrix
%  Usage: \cbm{domain-basis-letter}{codomain-basis-letter}
\newcommand{\cbm}[2]{C_{#1,#2}}
%
%%%%%%%%%%%%%%%%%%%%%
%
%     Canonical Forms
%
%%%%%%%%%%%%%%%%%%%%%
%
%  Jordan blocks
%  Usage: \jordan{size}{diagonal-element}
\newcommand{\jordan}[2]{J_{#1}\left(#2\right)}
%
%%%%%%%%%%%%%%%%%%%%%
%
%     Hadamard Matrices
%     Contributed by Elizabeth Million
%
%%%%%%%%%%%%%%%%%%%%%
%
%  Hadamard Product
%  Usage: \hadamard{a-matrix}{a-matrix}
\newcommand{\hadamard}[2]{#1\circ #2}
%
%  Hadamard identity matrix
%  Usage: \hadamardidentity{paired-subscripts-size-of-matrix}
\newcommand{\hadamardidentity}[1]{J_{#1}}
%
%  Hadamard inverse matrix
%  Usage: \hadamardinverse{matrix-expression}
\newcommand{\hadamardinverse}[1]{\widehat{#1}}


\title{Orthonormal Bases and Coordinates}

\begin{document}
\begin{abstract}
  When an orthogonal set also spans a subspace, then the set is a
  basis for that subspace.
\end{abstract}
\maketitle

We learned about orthogonal sets of vectors in $\real{m}$ back in
\ref{section:O}, and we also learned that orthogonal sets are
automatically linearly independent (\ref{theorem:OSLI}).  When an
orthogonal set also spans a subspace of $\real{m}$, then the set is a
basis.  And when the set is orthonormal, then the set is an incredibly
nice basis.  We will back up this claim with a theorem, but first
consider how you might manufacture such a set.

Suppose that $W$ is a subspace of $\real{m}$ with basis $B$.  Then $B$
spans $W$ and is a linearly independent set of nonzero vectors.  We
can apply the Gram-Schmidt Procedure (\ref{theorem:GSP}) and obtain a
linearly independent set $T$ such that $\spn{T}=\spn{B}=W$ and $T$ is
orthogonal.  In other words, $T$ is a basis for $W$, and is an
orthogonal set.  By scaling each vector of $T$ to norm 1, we can
convert $T$ into an orthonormal set, without destroying the properties
that make it a basis of $W$.  In short, we can convert any basis into
an orthonormal basis.  \ref{example:GSTV}, followed by
\ref{example:ONTV}, illustrates this process.

\begin{question}
  Suppose that $Q$ is a unitary matrix of size $n$.

  \begin{multipleChoice}
    \choice{Then the matrix $Q$ is singular.}
    \choice[correct]{Then the $n$ columns of $Q$ form an orthonormal set.}
  \end{multipleChoice}

  \begin{feedback}[correct]
    Unitary matrices (\ref{definition:UM}) are another good source of
    orthonormal bases (and vice versa).  Then the $n$ columns of $Q$
    form an orthonormal set (\ref{theorem:CUMOS}) that is therefore
    linearly independent (\ref{theorem:OSLI}).  Since $Q$ is
    invertible (\ref{theorem:UMI}), we know $Q$ is nonsingular
    (\ref{theorem:NI}), and then the columns of $Q$ span $\complex{n}$
    (\ref{theorem:CSNM}).  So the columns of a unitary matrix of size
    $n$ are an orthonormal basis for $\complex{n}$.
  \end{feedback}
\end{question}

Why all the fuss about orthonormal bases?  \ref{theorem:VRRB} told us
that any vector in a vector space could be written, uniquely, as a
linear combination of basis vectors.  For an orthonormal basis,
finding the scalars for this linear combination is extremely easy, and
this is the content of the next theorem.  Furthermore, with vectors
written this way (as linear combinations of the elements of an
orthonormal set) certain computations and analysis become much easier.
Here is the promised theorem.

\begin{theorem}[Coordinates and Orthonormal Bases]
  \label{theorem:COB}
  Suppose that $B=\set{\vectorlist{v}{p}}$ is an orthonormal basis of
  the subspace $W$ of $\complex{m}$.  For any $\vect{w}\in W$,
  \[
    \vect{w}=
    \innerproduct{\vect{v}_1}{\vect{w}}\vect{v}_1+
    \innerproduct{\vect{v}_2}{\vect{w}}\vect{v}_2+
    \innerproduct{\vect{v}_3}{\vect{w}}\vect{v}_3+
    \cdots+
    \innerproduct{\vect{v}_p}{\vect{w}}\vect{v}_p
  \]

  \begin{proof}
    Because $B$ is a basis of $W$, \ref{theorem:VRRB} tells us that we
    can write $\vect{w}$ uniquely as a linear combination of the
    vectors in $B$.  So it is not this aspect of the conclusion that
    makes this theorem interesting.  What is interesting is that the
    particular scalars are so easy to compute.  No need to solve big
    systems of equations---just do an inner product of $\vect{w}$ with
    $\vect{v}_i$ to arrive at the coefficient of $\vect{v}_i$ in the
    linear combination.

    So begin the proof by writing $\vect{w}$ as a linear combination
    of the vectors in $B$, using unknown scalars,
    \[
      \vect{w}=\lincombo{a}{v}{p}
    \]
    and compute,
    \begin{align*}
      \innerproduct{\vect{v}_i}{\vect{w}}
      &=\innerproduct{\vect{v}_i}{\sum_{k=1}^{p}a_k\vect{v}_k}
      &&\ref{theorem:VRRB}\\
      &=\sum_{k=1}^{p}\innerproduct{\vect{v}_i}{a_k\vect{v}_k}
      &&\ref{theorem:IPVA}\\
      &=\sum_{k=1}^{p}a_k\innerproduct{\vect{v}_i}{\vect{v}_k}
      &&\ref{theorem:IPSM}\\
      &=a_i\innerproduct{\vect{v}_i}{\vect{v}_i}+
        \sum_{\substack{k=1\\k\neq i}}^{p}a_k\innerproduct{\vect{v}_i}{\vect{v}_k}
      &&\ref{property:C}\\
      &=a_i(1)+\sum_{\substack{k=1\\k\neq i}}^{p}a_k(0)
      &&\ref{definition:ONS}\\
      &=a_i
    \end{align*}
    
    So the (unique) scalars for the linear combination are indeed the
    inner products advertised in the conclusion of the theorem's
    statement.
  \end{proof}
\end{theorem}

\begin{example}[Coordinatization relative to an orthonormal basis, $\complex{4}$]
  The set
  \[
    \set{\vect{x}_1,\,\vect{x}_2,\,\vect{x}_3,\,\vect{x}_4}=
    \set{
      \colvector{1+i\\1\\1-i\\i},\,
      \colvector{1+5i\\6+5i\\-7-i\\1-6i},\,
      \colvector{-7+34i\\-8-23i\\-10+22i\\30+13i},\,
      \colvector{-2-4i\\6+i\\4+3i\\6-i}
    }
  \]
  was proposed, and partially verified, as an orthogonal set in
  \ref{example:AOS}.  Let us scale each vector to norm 1, so as to
  form an orthonormal set in $\complex{4}$.  Then by
  \ref{theorem:OSLI} the set will be linearly independent, and the set
  will be a basis for $\complex{4}$.  So, once scaled to norm 1, the
  adjusted set will be an orthonormal basis of $\complex{4}$.  The
  norms are,
  \begin{align*}
    \norm{\vect{x}_1}=\sqrt{\answer{6}}
    &&
       \norm{\vect{x}_2}=\sqrt{174}
    &&
       \norm{\vect{x}_3}=\sqrt{3451}
    &&
       \norm{\vect{x}_4}=\sqrt{119}
  \end{align*}

  \begin{feedback}[correct]
    Therefore an orthonormal basis is
    \begin{align*}
      B&=
         \set{\vect{v}_1,\,\vect{v}_2,\,\vect{v}_3,\,\vect{v}_4}\\
       &=\set{
         \frac{1}{\sqrt{6}}\colvector{1+i\\1\\1-i\\i},\,
      \frac{1}{\sqrt{174}}\colvector{1+5i\\6+5i\\-7-i\\1-6i},\,
      \frac{1}{\sqrt{3451}}\colvector{-7+34i\\-8-23i\\-10+22i\\30+13i},\,
      \frac{1}{\sqrt{119}}\colvector{-2-4i\\6+i\\4+3i\\6-i}
      }
    \end{align*}

    Now, to illustrate \ref{theorem:COB}, choose any vector from $\complex{4}$, say $\vect{w}=\colvector{2\\-3\\1\\4}$, and compute
    \begin{align*}
      \innerproduct{\vect{v}_1}{\vect{w}}&=\frac{-5i}{\sqrt{6}}&
                                                                 \innerproduct{\vect{v}_2}{\vect{w}}&=\frac{-19+30i}{\sqrt{174}}\\
      \innerproduct{\vect{v}_3}{\vect{w}}&=\frac{120-211i}{\sqrt{3451}}&
                                                                         \innerproduct{\vect{v}_4}{\vect{w}}&=\frac{6+12i}{\sqrt{119}}
    \end{align*}
    
    Then \ref{theorem:COB} guarantees that
    \begin{align*}
      \colvector{2\\-3\\1\\4}&=
                               \frac{-5i}{\sqrt{6}}\left(\frac{1}{\sqrt{6}}\colvector{1+i\\1\\1-i\\i}\right)+
      \frac{-19+30i}{\sqrt{174}}\left(\frac{1}{\sqrt{174}}\colvector{1+5i\\6+5i\\-7-i\\1-6i}\right)\\
                             &\quad\quad+
                               \frac{120-211i}{\sqrt{3451}}\left(\frac{1}{\sqrt{3451}}\colvector{-7+34i\\-8-23i\\-10+22i\\30+13i}\right)+
      \frac{6+12i}{\sqrt{119}}\left(\frac{1}{\sqrt{119}}\colvector{-2-4i\\6+i\\4+3i\\6-i}\right)
    \end{align*}
    as you might want to check (if you have unlimited patience).
  \end{feedback}
\end{example}

A slightly less intimidating example follows, in three dimensions and
with just real numbers.

\begin{example}[Coordinatization relative to an orthonormal basis, $\real{3}$]
  The set
  \[
    \set{\vect{x}_1,\,\vect{x}_2,\,\vect{x}_3}
    =\set{
      \colvector{1\\2\\1},\,
      \colvector{-1\\0\\1},\,
      \colvector{2\\1\\1}
    }
  \]
  is a linearly independent set, which the Gram-Schmidt Process
  (\ref{theorem:GSP}) converts to an orthogonal set, and which can
  then be converted to the orthonormal set,
  \[
    B=
    \set{\vect{v}_1,\,\vect{v}_2,\,\vect{v}_3}
    =\set{
      \frac{1}{\sqrt{6}}\colvector{1\\2\\1},\,
      \frac{1}{\sqrt{2}}\colvector{-1\\0\\1},\,
      \frac{1}{\sqrt{3}}\colvector{1\\-1\\1}
    }
  \]
  which is therefore an orthonormal basis of $\real{3}$.  With three
  vectors in $\real{3}$, all with real number entries, the inner
  product (\ref{definition:IP}) reduces to the usual ``dot product''
  (or scalar product) and the orthogonal pairs of vectors can be
  interpreted as perpendicular pairs of directions.  So the vectors in
  $B$ serve as replacements for our usual 3-D axes, or the usual 3-D
  unit vectors $\vec{i},\vec{j}$ and $\vec{k}$.  We would like to
  decompose arbitrary vectors into ``components'' in the directions of
  each of these basis vectors.  It is \ref{theorem:COB} that tells us
  how to do this.

  Suppose that we choose $\vect{w}=\colvector{2\\-1\\5}$.  Compute
  \begin{align*}
    \innerproduct{\vect{v}_1}{\vect{w}}=\frac{5}{\sqrt{6}}&&
                                                             \innerproduct{\vect{v}_2}{\vect{w}}=\frac{3}{\sqrt{2}}&&
                                                                                                                      \innerproduct{\vect{v}_3}{\vect{w}}=\frac{8}{\sqrt{3}}
  \end{align*}
  then \ref{theorem:COB} guarantees that
  \[
    \colvector{2\\-1\\5}=
    \frac{\answer{5}}{\sqrt{6}}\left(\frac{1}{\sqrt{6}}\colvector{1\\2\\1}\right)+
    \frac{3}{\sqrt{2}}\left(\frac{1}{\sqrt{2}}\colvector{-1\\0\\1}\right)+
    \frac{8}{\sqrt{3}}\left(\frac{1}{\sqrt{3}}\colvector{1\\-1\\1}\right)
  \]
  which you should be able to check easily, even if you do not have much patience.
\end{example}

Not only do the columns of a unitary matrix form an orthonormal basis,
but there is a deeper connection between orthonormal bases and unitary
matrices.  Informally, the next theorem says that if we transform each
vector of an orthonormal basis by multiplying it by a unitary matrix,
then the resulting set will be another orthonormal basis.  And more
remarkably, any matrix with this property must be unitary!  As an
equivalence, we could take this as our defining property of a unitary
matrix, though it might not have the same utility as
\ref{definition:UM}.

\begin{theorem}[Unitary Matrices Convert Orthonormal Bases]
  \label{theorem:UMCOB}

  Let $A$ be an $n\times n$ matrix and $B=\set{\vectorlist{x}{n}}$ be an orthonormal basis of $\complex{n}$.  Define
  \begin{align*}
    C&=\set{A\vect{x}_1,\,A\vect{x}_2,\,A\vect{x}_3,\,\dots,\,A\vect{x}_n}
  \end{align*}

  Then $A$ is a unitary matrix if and only if $C$ is an orthonormal basis of $\complex{n}$.

  \begin{proof}
    ($\Rightarrow$)
    Assume $A$ is a unitary matrix and establish several facts about $C$.  First we check that $C$ is an orthonormal set (\ref{definition:ONS}).  By \ref{theorem:UMPIP}, for $i\neq j$,
    \begin{align*}
      \innerproduct{A\vect{x}_i}{A\vect{x}_j}&
                                               =\innerproduct{\vect{x}_i}{\vect{x}_j}=0
    \end{align*}
    
    Similarly, \ref{theorem:UMPIP} also gives, for $1\leq i\leq n$,
    \begin{align*}
      \norm{A\vect{x}_i}=\norm{\vect{x}_i}=1
    \end{align*}

    As $C$ is an orthogonal set (\ref{definition:OSV}),
    \ref{theorem:OSLI} yields the linear independence of $C$.  Having
    established that the column vectors on $C$ form a linearly
    independent set, a matrix whose columns are the vectors of $C$ is
    nonsingular (\ref{theorem:NMLIC}), and hence these vectors form a
    basis of $\complex{n}$ by \ref{theorem:CNMB}.

    ($\Leftarrow$)
    Now assume that $C$ is an orthonormal set.  Let $\vect{y}$ be an arbitrary vector from $\complex{n}$.  Since $B$ spans $\complex{n}$, there are scalars, $\scalarlist{a}{n}$, such that
    \begin{align*}
      \vect{y}&=a_1\vect{x}_1+a_2\vect{x}_2+a_3\vect{x}_3+\cdots+a_n\vect{x}_n
    \end{align*}

    Now
    \begin{align*}
      \adjoint{A}A\vect{y}
      &=\sum_{i=1}^{n}\innerproduct{\vect{x}_i}{\adjoint{A}A\vect{y}}\vect{x}_i
      &&\ref{theorem:COB}\\
      &=\sum_{i=1}^{n}\innerproduct{\vect{x}_i}{\adjoint{A}A\sum_{j=1}^{n}a_j\vect{x}_j}\vect{x}_i
      &&\ref{definition:SSVS}\\
      &=\sum_{i=1}^{n}\innerproduct{\vect{x}_i}{\sum_{j=1}^{n}\adjoint{A}Aa_j\vect{x}_j}\vect{x}_i
      &&\ref{theorem:MMDAA}\\
      &=\sum_{i=1}^{n}\innerproduct{\vect{x}_i}{\sum_{j=1}^{n}a_j\adjoint{A}A\vect{x}_j}\vect{x}_i
      &&\ref{theorem:MMSMM}\\
      &=\sum_{i=1}^{n}\sum_{j=1}^{n}\innerproduct{\vect{x}_i}{a_j\adjoint{A}A\vect{x}_j}\vect{x}_i
      &&\ref{theorem:IPVA}\\
      &=\sum_{i=1}^{n}\sum_{j=1}^{n}a_j\innerproduct{\vect{x}_i}{\adjoint{A}A\vect{x}_j}\vect{x}_i
      &&\ref{theorem:IPSM}\\
      &=\sum_{i=1}^{n}\sum_{j=1}^{n}a_j\innerproduct{A\vect{x}_i}{A\vect{x}_j}\vect{x}_i
      &&\ref{theorem:AIP}\\
      &=
        \sum_{i=1}^{n}\sum_{\substack{j=1\\j\neq i}}^{n}a_j\innerproduct{A\vect{x}_i}{A\vect{x}_j}\vect{x}_i
      +
      \sum_{\ell=1}^{n}a_\ell\innerproduct{A\vect{x}_\ell}{A\vect{x}_\ell}\vect{x}_\ell
      &&\ref{property:C}\\
      &=
        \sum_{i=1}^{n}\sum_{\substack{j=1\\j\neq i}}^{n}a_j(0)\vect{x}_i
      +
      \sum_{\ell=1}^{n}a_\ell(1)\vect{x}_\ell
      &&\ref{definition:ONS}\\
      &=
        \sum_{i=1}^{n}\sum_{\substack{j=1\\j\neq i}}^{n}\zerovector
      +
      \sum_{\ell=1}^{n}a_\ell\vect{x}_\ell
      &&\ref{theorem:ZSSM}\\
      &=\sum_{\ell=1}^{n}a_\ell\vect{x}_\ell
      &&\ref{property:Z}\\
      &=\vect{y}\\
      &=I_n\vect{y}
      &&\ref{theorem:MMIM}
    \end{align*}
    
    Since the choice of $\vect{y}$ was arbitrary, \ref{theorem:EMMVP} tells us that $\adjoint{A}A=I_n$, so $A$ is unitary (\ref{definition:UM}).
  \end{proof}
\end{theorem}

\end{document}
