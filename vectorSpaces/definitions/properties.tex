\documentclass{ximera}

% These macros are automatically generated from the "macros"
% XML element.  Make permanent edits there.
%
% History
%   2004/01/01  Initiated for FCLA, evolved from there
%   2006/09/17  Converted  _, ^  to \sb, \sp for TeX4ht
%   2014/02/01  Updated for MathBook XML projects
%               Obsolete in FCLA: \codeindent, \computerfont, \define
%               Change: MathJax wants \lt, so replaced by \lteval
%   2014/02/22  New: \orderof, \reals, \per
%   2015/08/16  Incorporated into MathBook XML version of FCLA
%
%%%%%%%%%%%%%%%%%%%%%
%
%     Conveniences
%
%%%%%%%%%%%%%%%%%%%%%
%
%  Order of (asymptotically limit of fraction is 1)
%  Usage: \orderof{some function}
%
\newcommand{\orderof}[1]{\sim #1}
%
%  Integers
%  Usage:  \Z
\newcommand{\Z}{\mathbb{Z}}
%
%  Real numbers, as set of scalars
%  Usage:  \reals
\newcommand{\reals}{\mathbb{R}}
%
%  n-space over real field
%  Usage: \complex{integer-dimension}
\newcommand{\real}[1]{\mathbb{R}^{#1}}
%
%  Complex numbers, as set of scalars
%  Usage:  \complexes
\newcommand{\complexes}{\mathbb{C}}
%
%  n-space over complex field
%  Usage: \complex{integer-dimension}
\newcommand{\complex}[1]{\mathbb{C}^{#1}}
%
%  Complex conjugation (scalar, vector, matrix)
%  Usage: \conjugate{object}
\newcommand{\conjugate}[1]{\overline{#1}}
%
%  Complex number modulus
%  Usage: \modulus{a+bi}
%  Presumes math mode
\newcommand{\modulus}[1]{\left\lvert#1\right\rvert}
%
%  Zero vector
%  Usage: \zerovector
\newcommand{\zerovector}{\vect{0}}
%
%  Zero matrix
%  Usage: \zeromatrix, use a subscript when size is important
\newcommand{\zeromatrix}{\mathcal{O}}
%
%  Inner product (brackets, not quadratic form)
%  Usage: \innerproduct{a-vector}{a-vector}
\newcommand{\innerproduct}[2]{\left\langle#1,\,#2\right\rangle}
%
%  Norm of a vector
%  Usage: \norm{a-vector}
\newcommand{\norm}[1]{\left\lVert#1\right\rVert}
%
%  Dimension
%  Usage: \dimension{vector-space-letter}
\newcommand{\dimension}[1]{\dim\left(#1\right)}
%
%  Nullity
%  Usage: \nullity{matrix-or-lintrans-letter}
\newcommand{\nullity}[1]{n\left(#1\right)}
%
%  Rank
%  Usage: \rank{matrix-or-lintrans-letter}
\newcommand{\rank}[1]{r\left(#1\right)}
%
%  Direct sum
%  Usage: \ds between a couple of subspaces
%
\newcommand{\ds}{\oplus}
%
%  Determinant of a matrix (functional)
%  Usage: \detname{A}
\newcommand{\detname}[1]{\det\left(#1\right)}
%
%  Determinant of a matrix (vertical bars)
%  Usage: \detbars{A}
\newcommand{\detbars}[1]{\left\lvert#1\right\rvert}
%
%  Trace of a Matrix
%  Usage: \trace{matrix name}
\newcommand{\trace}[1]{t\left(#1\right)}
%
%  Square Root of a Matrix
%  Usage: \sr{a-matrix}
\newcommand{\sr}[1]{#1^{1/2}}
%
%%%%%%%%%%%%%%%%%%%%%
%
%     Subspace Constructions
%
%%%%%%%%%%%%%%%%%%%%%
%
%  Span of a set of vectors
%  \span and \sp are used by TeX for other things
%  Usage: \spn{set-of-vectors}
\newcommand{\spn}[1]{\left\langle#1\right\rangle}
%
%  Null space of a matrix
%  Usage:  \nsp{A}
\newcommand{\nsp}[1]{\mathcal{N}\!\left(#1\right)}
%
%  Column space of a matrix
%  Usage:  \csp{A}
\newcommand{\csp}[1]{\mathcal{C}\!\left(#1\right)}
%
%  Row space of a matrix
%  Usage:  \rsp{A}
\newcommand{\rsp}[1]{\mathcal{R}\!\left(#1\right)}
%
%  Left null space of a matrix
%  Usage:  \lns{A}
\newcommand{\lns}[1]{\mathcal{L}\!\left(#1\right)}
%
%  Orthogonal complement of a vector space
%  Avoiding TeX's \perp
%  Usage:  \per{A}
\newcommand{\per}[1]{#1^\perp}
%
%%%%%%%%%%%%%%%%%%%%%
%
%     Systems of Equations
%
%%%%%%%%%%%%%%%%%%%%%
%
%  In-line form of an augmented matrix for a system of equations
%  Usage: \augmented{coefficient-matrix}{constant-vector}
\newcommand{\augmented}[2]{\left\lbrack\left.#1\,\right\rvert\,#2\right\rbrack}
%
%  Notation for a linear system before introducing matrix multiplication
%  Usage: \linearsystem{coefficient-matrix}{constant-vector}
\newcommand{\linearsystem}[2]{\mathcal{LS}\!\left(#1,\,#2\right)}
%
%  Notation for a homogenous system before introducing matrix multiplication
%  Usage: \homosystem{coefficient-matrix}
\newcommand{\homosystem}[1]{\linearsystem{#1}{\zerovector}}
%
%%%%%%%%%%%%%%%%%%%%%
%
%     Row Operations, Echelon Form
%
%%%%%%%%%%%%%%%%%%%%%
%
% Row operations on matrices
%
% Three commands to shorten up descriptions of gaussian elimination
%
% Usage: \rowopswap{row-i}{row-j}
% Usage: \rowopmult{scalar}{row-i}
% Usage: \rowopadd{scalar}{row-multiplied}{row-added-to}
\newcommand{\rowopswap}[2]{R_{#1}\leftrightarrow R_{#2}}
\newcommand{\rowopmult}[2]{#1R_{#2}}
\newcommand{\rowopadd}[3]{#1R_{#2}+R_{#3}}
%
% Mark leading 1's in echelon form with fbox
% Usage: \leading{a-1-usually}
\newcommand{\leading}[1]{\fbox{#1}}
%
%  Row-reduce arrow
%  Usage:  \rref inbetween a matrix and its reduced row-echelon form
\newcommand{\rref}{\xrightarrow{\text{RREF}}}
%
%  Elementary Matrices
%  Usage: \elemswap{subscript}{subscript}
%  Usage: \elemmult{scalar}{subscript}
%  Usage: \elemadd{scalar}{subscript-mult}{subscript-target}
%
\newcommand{\elemswap}[2]{E_{#1,#2}}
\newcommand{\elemmult}[2]{E_{#2}\left(#1\right)}
\newcommand{\elemadd}[3]{E_{#2,#3}\left(#1\right)}
%
%%%%%%%%%%%%%%%%%%%%%
%
%     2-D Constructions (Lists, Vectors, Matrices)
%
%%%%%%%%%%%%%%%%%%%%%
%
%  A list of scalars of generic length
%  Usage:  \scalarlist{scalar letter}{terminal subscript}
\newcommand{\scalarlist}[2]{{#1}_{1},\,{#1}_{2},\,{#1}_{3},\,\ldots,\,{#1}_{#2}}
%
%  Vector styling, bold (or use wiggles, arrows, whatever)
%  Subscripts go outside this construction
%  Usage: \vect{a symbol to use as a vector}
%  Have to already be in math mode
%
\newcommand{\vect}[1]{\mathbf{#1}}
%
%  A column vector
%  Usage: \colvector{list-delimited-by-\\}
%
\newcommand{\colvector}[1]{\begin{bmatrix}#1\end{bmatrix}}
%
%  A generic vector with components
%  Usage: \vectorcomponents{component-letter}{final-subscript}
\newcommand{\vectorcomponents}[2]{\colvector{#1_{1}\\#1_{2}\\#1_{3}\\\vdots\\#1_{#2}}}
%
%  A list of vectors of generic length
%  Usage:  \vectorlist{vector letter}{terminal subscript}
\newcommand{\vectorlist}[2]{\vect{#1}_{1},\,\vect{#1}_{2},\,\vect{#1}_{3},\,\ldots,\,\vect{#1}_{#2}}
%
%  Vector entries, entry i of vector v
%  (vector-expession still needs \vect, etc.)
%  Usage:  \vectorentry{vector-expression}{single-subscript}
\newcommand{\vectorentry}[2]{\left\lbrack#1\right\rbrack_{#2}}
%
%  Matrix entries, entry i,j of matrix A
%  Usage:  \matrixentry{matrix-expression}{paired-subscripts}
%
\newcommand{\matrixentry}[2]{\left\lbrack#1\right\rbrack_{#2}}
%
%  A generic linear combination
%  Usage:  \lincombo{scalar letter}{vector letter}{terminal subscript}
\newcommand{\lincombo}[3]{#1_{1}\vect{#2}_{1}+#1_{2}\vect{#2}_{2}+#1_{3}\vect{#2}_{3}+\cdots +#1_{#3}\vect{#2}_{#3}}
%
%  Matrix, column by column, as vectors
%  Usage:  \matrixcolumns{matrix letter}{terminal subscript}
\newcommand{\matrixcolumns}[2]{\left\lbrack\vect{#1}_{1}|\vect{#1}_{2}|\vect{#1}_{3}|\ldots|\vect{#1}_{#2}\right\rbrack}
%
%%%%%%%%%%%%%%%%%%%%%
%
%     Special Matrices
%
%%%%%%%%%%%%%%%%%%%%%
%
%  Transpose of a matrix
%  Usage:  \transpose{A}
\newcommand{\transpose}[1]{#1^{t}}
%
%  Inverse of a matrix
%  Usage:  \inverse{A}
\newcommand{\inverse}[1]{#1^{-1}}
%
%  Submatrix (for minors, determinants)
%  Usage: \submatrix{matrix-name}{delete-row}{delete-col}
\newcommand{\submatrix}[3]{#1\left(#2|#3\right)}
%
%  Adjoint of a matrix (twice)
%  This macro is a convenience to call \transpose and \conjugate properly
%  It shouldn't need to be modified (or mathematical meanings will change)
%  Usage:  \adj{A}
\newcommand{\adj}[1]{\transpose{\left(\conjugate{#1}\right)}}
%
%  This macro controls the symbol used for the adjoint
%  It can be edited to taste
%  Usage:  \adjoint{A}
\newcommand{\adjoint}[1]{#1^\ast}
%
%%%%%%%%%%%%%%%%%%%%%
%
%     Sets
%
%%%%%%%%%%%%%%%%%%%%%
%
%  A convenience for simple sets
%  Usage:  \set{list of element}
\newcommand{\set}[1]{\left\{#1\right\}}
%
%  Sets with vertical bar, "such that", sized for objects, not condition
%  Usage:  \setparts{objects}{condition}
%
%%\newcommand{\setparts}[2]{\left\{ #1\mid#2\right\}}
%%\newcommand{\setparts}[2]{\left\{\left. #1\right\rvert#2\right\}}
\newcommand{\setparts}[2]{\left\lbrace#1\,\middle|\,#2\right\rbrace}
%
%  Set Cardinality
%  Usage:  \card{a-set-letter}
\newcommand{\card}[1]{\left\lvert#1\right\rvert}
%
%  Set Union
%  Use \cup
%
%  Set Intersection
%  Use \cap
%
%  Set Complement
%  Usage:  \setcomplement{a-set-letter}
\newcommand{\setcomplement}[1]{\overline{#1}}
%
%%%%%%%%%%%%%%%%%%%%%
%
%     Eigenvalues and Eigenspaces
%
%%%%%%%%%%%%%%%%%%%%%
%
%  Characteristic polynomial
%  Usage: \charpoly{matrix-letter}{variable-letter}
\newcommand{\charpoly}[2]{p_{#1}\left(#2\right)}
%
%  Eigenspace
%  Usage: \eigenspace{matrix-letter}{eigenvalue-letter}
\newcommand{\eigenspace}[2]{\mathcal{E}_{#1}\left(#2\right)}
%
%  2013/10/03 Including ampersands is problematic here, 
%  think about fixes later
%  2014/02/22 Limited testing, seems &amp; is fine for HTML and LaTeX
%  2016-07-20 only employed in Archetypes, MBX has gather/align override
%  Eigensystem (presumes wrapped in an mrow within md)
%  Usage: \eigensystem{matrixletter}{eigenvalue}{list of basis vectors}
\newcommand{\eigensystem}[3]{\lambda&amp;=#2&amp;\eigenspace{#1}{#2}&amp;=\spn{\set{#3}}} 
%
%  Generalized Eigenspace
%  Usage: \geneigenspace{lin-trans-letter}{eigenvalue-letter}
\newcommand{\geneigenspace}[2]{\mathcal{G}_{#1}\left(#2\right)}
%
%  Algebraic multiplicty
%  Usage: \algmult{matrix-letter}{eigenvalue-letter}
\newcommand{\algmult}[2]{\alpha_{#1}\left(#2\right)}
%
%  Geometric multiplicty
%  Usage: \geomult{matrix-letter}{eigenvalue-letter}
\newcommand{\geomult}[2]{\gamma_{#1}\left(#2\right)}
%
%  Index (of eigenvalue)
%  Usage: \indx{matrix-letter}{eigenvalue-letter}
\newcommand{\indx}[2]{\iota_{#1}\left(#2\right)}
%
%%%%%%%%%%%%%%%%%%%%%
%
%     Linear Transformations
%
%%%%%%%%%%%%%%%%%%%%%
%
%  MathJax defines \lt to ease XML confusion
%
%  Linear transformation definition
%  Usage: \ltdefn{name-letter}{domain}{range}
\newcommand{\ltdefn}[3]{#1\colon #2\rightarrow#3}
%
%  Linear transformation evaluation
%  Usage: \lteval{name-letter}{input}
%  Replaces old \lt desired by MathJax
\newcommand{\lteval}[2]{#1\left(#2\right)}
%
% Linear transformation inverse
%  Usage: \ltinverse{name-letter}
\newcommand{\ltinverse}[1]{#1^{-1}}
%
%  Linear transformation restriction
%  Usage: \restrict{name-letter}{subspace-letter}
\newcommand{\restrict}[2]{{#1}|_{#2}}
%
%  Linear transformation preimage
%  Usage: \preimage{name-letter}{codomain-element}
\newcommand{\preimage}[2]{#1^{-1}\left(#2\right)}
%
%  Range of a linear transformation
%  TeX uses \range for something else
%  Usage:  \rng{T}
\newcommand{\rng}[1]{\mathcal{R}\!\left(#1\right)}
%
%  Kernel of a linear transformation
%  TeX uses \ker to do something different
%  Usage:  \krn{T}
\newcommand{\krn}[1]{\mathcal{K}\!\left(#1\right)}
%
%  Linear transformation composition
%  Usage: \compose{function-name}{function-name}
\newcommand{\compose}[2]{{#1}\circ{#2}}
%
%  Vector space of linear transformations
%  Usage: \vslt{domains}{codomains}
%  Presumes math mode
\newcommand{\vslt}[2]{\mathcal{LT}\left(#1,\,#2\right)}
%
%%%%%%%%%%%%%%%%%%%%%
%
%     Vector and Matrix Representations
%
%%%%%%%%%%%%%%%%%%%%%
%
%  Isomorphism symbol
%  Usage: \isomorphic
\newcommand{\isomorphic}{\cong}
%
%  Similarity
%  Usage: \similar{inner-matrix}{outer-invertible-matrix}
%  Rearranging this will not "fix" all desired changes throughout
%
\newcommand{\similar}[2]{\inverse{#2}#1#2}
%
%  Vector representation function name
%  Usage: \vectrepname{basis-letter}
\newcommand{\vectrepname}[1]{\rho_{#1}}
%
%  Vector representation output
%  Usage: \vectrep{basis-letter}{input}
\newcommand{\vectrep}[2]{\lteval{\vectrepname{#1}}{#2}}
%
%  Vector representation inverse function name
%  (Added later, not used consistently in FCLA)
%  Usage: \vectrepinvname{basis-letter}
\newcommand{\vectrepinvname}[1]{\ltinverse{\vectrepname{#1}}}
%
%  Vector representation inverse output
%  Usage: \vectrepinv{basis-letter}{input}
\newcommand{\vectrepinv}[2]{\lteval{\ltinverse{\vectrepname{#1}}}{#2}}
%
%  Matrix representation
%  Usage: \matrixrep{transformation-letter}{domain-basis-letter}{codomain-basis-letter}
\newcommand{\matrixrep}[3]{M^{#1}_{#2,#3}}
%
%  Matrix representation column-by-colum
%  2016-07-20 only employed once?
%  Usage: \matrixrepcolumns{transformation-letter}{codomain-basis-letter}{codomain-basis-vector-letter}{final-index}
\newcommand{\matrixrepcolumns}[4]{\left\lbrack \left.\vectrep{#2}{\lteval{#1}{\vect{#3}_{1}}}\right|\left.\vectrep{#2}{\lteval{#1}{\vect{#3}_{2}}}\right|\left.\vectrep{#2}{\lteval{#1}{\vect{#3}_{3}}}\right|\ldots\left|\vectrep{#2}{\lteval{#1}{\vect{#3}_{#4}}}\right.\right\rbrack}
%
%  Change of basis matrix
%  Usage: \cbm{domain-basis-letter}{codomain-basis-letter}
\newcommand{\cbm}[2]{C_{#1,#2}}
%
%%%%%%%%%%%%%%%%%%%%%
%
%     Canonical Forms
%
%%%%%%%%%%%%%%%%%%%%%
%
%  Jordan blocks
%  Usage: \jordan{size}{diagonal-element}
\newcommand{\jordan}[2]{J_{#1}\left(#2\right)}
%
%%%%%%%%%%%%%%%%%%%%%
%
%     Hadamard Matrices
%     Contributed by Elizabeth Million
%
%%%%%%%%%%%%%%%%%%%%%
%
%  Hadamard Product
%  Usage: \hadamard{a-matrix}{a-matrix}
\newcommand{\hadamard}[2]{#1\circ #2}
%
%  Hadamard identity matrix
%  Usage: \hadamardidentity{paired-subscripts-size-of-matrix}
\newcommand{\hadamardidentity}[1]{J_{#1}}
%
%  Hadamard inverse matrix
%  Usage: \hadamardinverse{matrix-expression}
\newcommand{\hadamardinverse}[1]{\widehat{#1}}


\title{Vector Space Properties}

\begin{document}
\begin{abstract}
  We prove some general properties of vector spaces.  Some of these results will again seem obvious, but it is important to understand why it is necessary to state and prove them.
\end{abstract}
\maketitle

As we prove general properties, a typical hypothesis will be ``Let $V$
be a vector space.''  From this we may assume the ten properties of
\ref{definition:VS}, \textit{and nothing more}.  It is like starting
over, as we learn about what can happen in this new algebra we are
learning.  But the power of this careful approach is that we can apply
these theorems to any vector space we encounter---those in the
previous examples, or new ones we have not yet contemplated.  Or
perhaps new ones that nobody has ever contemplated.  We will
illustrate some of these results with examples from the crazy vector
space (\ref{example:CVS}), but mostly we are stating theorems and
doing proofs.  These proofs do not get too involved, but are not
trivial either, so these are good theorems to try proving yourself
before you study the proof given here.

First we show that there is just one zero vector.  Notice that the
properties only require there to be <em>at least</em> one, and say
nothing about there possibly being more.  That is because we can use
the ten properties of a vector space (\ref{definition:VS}) to learn
that there can \textit{never} be more than one.  To require that this
extra condition be stated as an eleventh property would make the
definition of a vector space more complicated than it needs to be.

\begin{theorem}[Zero Vector is Unique]
  \label{theorem:ZVU}
  
  Suppose that $V$ is a vector space.  The zero vector, $\zerovector$,
  is unique.
  
  \begin{proof}
    To prove uniqueness, a standard technique is to suppose the
    existence of two objects.  So let $\zerovector_1$ and
    $\zerovector_2$ be two zero vectors in $V$.  Then
    \begin{align*}
      \zerovector_1
      &=\zerovector_1+\zerovector_2
      &&\ref{property:Z}\text{ for }\zerovector_2\\
      &=\zerovector_2+\zerovector_1
      &&\ref{property:C}\\
      &=\zerovector_2
      &&\ref{property:Z}\text{ for }\zerovector_1
    \end{align*}
    
    This proves the uniqueness since the two zero vectors are really the same.
  \end{proof}
\end{theorem}

\begin{theorem}[Additive Inverses are Unique]
  \label{theorem:AIU}
  
  Suppose that $V$ is a vector space.   For each $\vect{u}\in V$, the additive inverse, $\vect{-u}$, is unique.
  
  \begin{proof}
    To prove uniqueness, a standard technique is to suppose the existence of two objects (\ref{technique:U}).  So let $\vect{-u}_1$ and $\vect{-u}_2$ be two additive inverses for $\vect{u}$.  Then
    \begin{align*}
      \vect{-u}_1&=\vect{-u}_1+\zerovector&&\ref{property:Z}\\
                 &=\vect{-u}_1+(\vect{u}+\vect{-u}_2)&&\ref{property:AI}\\
                 &=(\vect{-u}_1+\vect{u})+\vect{-u}_2&&\ref{property:AA}\\
                 &=\zerovector+\vect{-u}_2&&\ref{property:AI}\\
                 &=\vect{-u}_2&&\ref{property:Z}
    \end{align*}
    
    So the two additive inverses are really the same.
    
  \end{proof}
\end{theorem}

As obvious as the next three theorems appear, nowhere have we
guaranteed that the zero scalar, scalar multiplication and the zero
vector all interact this way.  Until we have proved it, anyway.

\begin{theorem}[Zero Scalar in Scalar Multiplication]
  \label{theorem:ZSSM}

  Suppose that $V$ is a vector space and $\vect{u}\in V$.  Then $0\vect{u}=\zerovector$.

  \begin{proof}
    Notice that $0$ is a scalar, $\vect{u}$ is a vector, so \ref{property:SC} says $0\vect{u}$ is again a vector.  As such, $0\vect{u}$ has an additive inverse, $-(0\vect{u})$ by \ref{property:AI}.
    \begin{align*}
      0\vect{u}
      &=\zerovector+0\vect{u}&&\ref{property:Z}\\
      &=\left(-(0\vect{u}) + 0\vect{u}\right)+0\vect{u}&&\ref{property:AI}\\
      &=-(0\vect{u}) + \left(0\vect{u}+0\vect{u}\right)&&\ref{property:AA}\\
      &=-(0\vect{u}) + (0+0)\vect{u}&&\ref{property:DSA}\\
      &=-(0\vect{u}) + 0\vect{u}&&\ref{property:ZCN}\\
      &=\zerovector&&\ref{property:AI}
    \end{align*}

  \end{proof}
\end{theorem}

Here is another theorem that \textit{looks} like it should be obvious, but is still in need of a proof.

\begin{theorem}
  \label{theorem:ZVSM}
  [Zero Vector in Scalar Multiplication]
  
  Suppose that $V$ is a vector space and $\alpha\in\complexes$.   Then $\alpha\zerovector=\zerovector$.

  \begin{proof}
    Notice that $\alpha$ is a scalar, $\zerovector$ is a vector, so \ref{property:SC} means $\alpha\zerovector$ is again a vector.  As such, $\alpha\zerovector$ has an additive inverse, $-(\alpha\zerovector)$ by \ref{property:AI}.
    \begin{align*}
      \alpha\zerovector
      &=\zerovector+\alpha\zerovector&&\ref{property:Z}\\
      &=\left(-(\alpha\zerovector)+\alpha\zerovector\right)+\alpha\zerovector&&\ref{property:AI}\\
      &=-(\alpha\zerovector)+\left(\alpha\zerovector+\alpha\zerovector\right)&&\ref{property:AA}\\
      &=-(\alpha\zerovector)+\alpha\left(\zerovector+\zerovector\right)&&\ref{property:DVA}\\
      &=-(\alpha\zerovector)+\alpha\zerovector&&\ref{property:Z}\\
      &=\zerovector&&\ref{property:AI}
    \end{align*}
  \end{proof}
\end{theorem}

Here is another one that sure looks obvious.  But understand that we
have chosen to use certain notation because it makes the theorem's
conclusion look so nice.  The theorem is not true because the notation
looks so good; it still needs a proof.  If we had really wanted to
make this point, we might have used notation like $\vect{u}^\sharp$
for the additive inverse of $\vect{u}$.  Then we would have written
the defining property, \ref{property:AI}, as
$\vect{u}+\vect{u}^\sharp=\zerovector$.  This theorem would become
$\vect{u}^\sharp=(-1)\vect{u}$.  Not really quite as pretty, is it?

\begin{theorem}[Additive Inverses from Scalar Multiplication]
  \label{theorem:AISM}

  Suppose that $V$ is a vector space and $\vect{u}\in V$.  Then $\vect{-u}=(-1)\vect{u}$.

  \begin{proof}
    \begin{align*}
      \vect{-u}
      &=\vect{-u}+\zerovector&&\ref{property:Z}\\
      &=\vect{-u}+0\vect{u}&&\ref{theorem:ZSSM}\\
      &=\vect{-u}+\left(1+(-1)\right)\vect{u}&&\ref{property:AICN}\\
&=\vect{-u}+\left(1\vect{u}+(-1)\vect{u}\right)&&\ref{property:DSA}\\
      &=\vect{-u}+\left(\vect{u}+(-1)\vect{u}\right)&&\ref{property:O}\\
      &=\left(\vect{-u}+\vect{u}\right)+(-1)\vect{u}&&\ref{property:AA}\\
      &=\zerovector+(-1)\vect{u}&&\ref{property:AI}\\
      &=(-1)\vect{u}&&\ref{property:Z}
    \end{align*}
  \end{proof}
\end{theorem}

Because of this theorem, we can now write linear combinations like
$6\vect{u}_1+(-4)\vect{u}_2$ as $6\vect{u}_1-4\vect{u}_2$, even though
we have not formally defined an operation called \dfn{vector
  subtraction}.

Our next theorem is a bit different from several of the others in the
list.  Rather than making a declaration (``the zero vector is
unique'') it is an implication (``\ldots, then\ldots'') and so can be
used in proofs to convert a vector equality into two possibilities,
one a scalar equality and the other a vector equality.  It should
remind you of the situation for complex numbers.  If
$\alpha,\,\beta\in\complexes$ and $\alpha\beta=0$, then $\alpha=0$ or
$\beta=0$.  This critical property is the driving force behind using a
factorization to solve a polynomial equation.

\begin{theorem}[Scalar Multiplication Equals the Zero Vector]
  \label{theorem:SMEZV}

  Suppose that $V$ is a vector space and $\alpha\in\complexes$.  If
  $\alpha\vect{u}=\zerovector$, then either $\alpha=0$ or
  $\vect{u}=\zerovector$.

  \begin{proof}
    We prove this theorem by breaking up the analysis into two cases.  The first seems too trivial, and it is, but the logic of the argument is still legitimate.

    \textbf{Case 1.}  Suppose $\alpha=0$.  In this case our conclusion
    is true (the first part of the either/or is true) and we are done.
    That was easy.
    
    \textbf{Case 2.}  Suppose $\alpha\neq 0$.
    \begin{align*}
      \vect{u}
      &=1\vect{u}
      &&\ref{property:O}\\
      &=\left(\frac{1}{\alpha}\alpha\right)\vect{u}
      &&\alpha\neq 0, \ref{property:MICN}\\
      &=\frac{1}{\alpha}\left(\alpha\vect{u}\right)
      &&\ref{property:SMA}\\
      &=\frac{1}{\alpha}\left(\zerovector\right)
      &&\text{Hypothesis}\\
      &=\zerovector&&\ref{theorem:ZVSM}
    \end{align*}

    So in this case, the conclusion is true (the second part of the either/or is true) and we are done since the conclusion was true in each of the two cases.
  \end{proof}
\end{theorem}

\begin{exercise}
  Suppose that $V$ is a vector space, and $\vect{u},\,\vect{v},\,\vect{w}\in V$.

  If $\vect{w}+\vect{u}=\vect{w}+\vect{v}$, then it is necessarily the case that
  \begin{multipleChoice}
    \choice[correct]{$\vect{u}=\vect{v}$}
    \choice{$\vect{w}=\zerovector$}
  \end{multipleChoice}
  
  \begin{feedback}[correct]
    We can prove the statement ``if
    $\vect{w}+\vect{u}=\vect{w}+\vect{v}$ then $\vect{u}=\vect{v}$''
    with an argument such as this:
    \begin{align*}
      \vect{u}
      &=\zerovector+\vect{u}&&\ref{property:Z}\\
      &=\left(\vect{-w}+\vect{w}\right)+\vect{u}&&\ref{property:AI}\\
      &=\vect{-w}+\left(\vect{w}+\vect{u}\right)&&\ref{property:AA}\\
      &=\vect{-w}+\left(\vect{w}+\vect{v}\right)&&\text{Hypothesis}\\
      &=\left(\vect{-w}+\vect{w}\right)+\vect{v}&&\ref{property:AA}\\
      &=\zerovector+\vect{v}&&\ref{property:AI}\\
      &=\vect{v}&&\ref{property:Z}
    \end{align*}
  \end{feedback}
\end{exercise}

\begin{example}[Properties for the Crazy Vector Space]
  Several of the above theorems have interesting demonstrations when
  applied to the crazy vector space, $C$ (\ref{example:CVS}).  We are
  not proving anything new here, or learning anything we did not know
  already about $C$.  It is just plain fun to see how these general
  theorems apply in a specific instance.  For most of our examples,
  the applications are obvious or trivial, but not with $C$.

  Suppose $\vect{u}\in C$.  Then, as given by \ref{theorem:ZSSM},
  \[
    0\vect{u}=0(x_1,\,x_2)=(0x_1+0-1,\,0x_2+0-1)=(-1,-1)=\zerovector
  \]
  And as given by \ref{theorem:ZVSM},
  \begin{align*}
    \alpha\zerovector
    &=\alpha(-1,\,-1)=(\alpha(-1)+\alpha-1,\,\alpha(-1)+\alpha-1)\\
    &=(-\alpha+\alpha-1,\,-\alpha+\alpha-1)=(-1,\,-1)=\zerovector
  \end{align*}
  Finally, as given by \ref{theorem:AISM},
  \begin{align*}
    (-1)\vect{u}
    &=(-1)(x_1,\,x_2)=((-1)x_1+(-1)-1,\,(-1)x_2+(-1)-1)\\
    &=(-x_1-2,\,-x_2-2)=-\vect{u}
  \end{align*}
\end{example}

\end{document}
