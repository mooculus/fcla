\documentclass{ximera}

% These macros are automatically generated from the "macros"
% XML element.  Make permanent edits there.
%
% History
%   2004/01/01  Initiated for FCLA, evolved from there
%   2006/09/17  Converted  _, ^  to \sb, \sp for TeX4ht
%   2014/02/01  Updated for MathBook XML projects
%               Obsolete in FCLA: \codeindent, \computerfont, \define
%               Change: MathJax wants \lt, so replaced by \lteval
%   2014/02/22  New: \orderof, \reals, \per
%   2015/08/16  Incorporated into MathBook XML version of FCLA
%
%%%%%%%%%%%%%%%%%%%%%
%
%     Conveniences
%
%%%%%%%%%%%%%%%%%%%%%
%
%  Order of (asymptotically limit of fraction is 1)
%  Usage: \orderof{some function}
%
\newcommand{\orderof}[1]{\sim #1}
%
%  Integers
%  Usage:  \Z
\newcommand{\Z}{\mathbb{Z}}
%
%  Real numbers, as set of scalars
%  Usage:  \reals
\newcommand{\reals}{\mathbb{R}}
%
%  n-space over real field
%  Usage: \complex{integer-dimension}
\newcommand{\real}[1]{\mathbb{R}^{#1}}
%
%  Complex numbers, as set of scalars
%  Usage:  \complexes
\newcommand{\complexes}{\mathbb{C}}
%
%  n-space over complex field
%  Usage: \complex{integer-dimension}
\newcommand{\complex}[1]{\mathbb{C}^{#1}}
%
%  Complex conjugation (scalar, vector, matrix)
%  Usage: \conjugate{object}
\newcommand{\conjugate}[1]{\overline{#1}}
%
%  Complex number modulus
%  Usage: \modulus{a+bi}
%  Presumes math mode
\newcommand{\modulus}[1]{\left\lvert#1\right\rvert}
%
%  Zero vector
%  Usage: \zerovector
\newcommand{\zerovector}{\vect{0}}
%
%  Zero matrix
%  Usage: \zeromatrix, use a subscript when size is important
\newcommand{\zeromatrix}{\mathcal{O}}
%
%  Inner product (brackets, not quadratic form)
%  Usage: \innerproduct{a-vector}{a-vector}
\newcommand{\innerproduct}[2]{\left\langle#1,\,#2\right\rangle}
%
%  Norm of a vector
%  Usage: \norm{a-vector}
\newcommand{\norm}[1]{\left\lVert#1\right\rVert}
%
%  Dimension
%  Usage: \dimension{vector-space-letter}
\newcommand{\dimension}[1]{\dim\left(#1\right)}
%
%  Nullity
%  Usage: \nullity{matrix-or-lintrans-letter}
\newcommand{\nullity}[1]{n\left(#1\right)}
%
%  Rank
%  Usage: \rank{matrix-or-lintrans-letter}
\newcommand{\rank}[1]{r\left(#1\right)}
%
%  Direct sum
%  Usage: \ds between a couple of subspaces
%
\newcommand{\ds}{\oplus}
%
%  Determinant of a matrix (functional)
%  Usage: \detname{A}
\newcommand{\detname}[1]{\det\left(#1\right)}
%
%  Determinant of a matrix (vertical bars)
%  Usage: \detbars{A}
\newcommand{\detbars}[1]{\left\lvert#1\right\rvert}
%
%  Trace of a Matrix
%  Usage: \trace{matrix name}
\newcommand{\trace}[1]{t\left(#1\right)}
%
%  Square Root of a Matrix
%  Usage: \sr{a-matrix}
\newcommand{\sr}[1]{#1^{1/2}}
%
%%%%%%%%%%%%%%%%%%%%%
%
%     Subspace Constructions
%
%%%%%%%%%%%%%%%%%%%%%
%
%  Span of a set of vectors
%  \span and \sp are used by TeX for other things
%  Usage: \spn{set-of-vectors}
\newcommand{\spn}[1]{\left\langle#1\right\rangle}
%
%  Null space of a matrix
%  Usage:  \nsp{A}
\newcommand{\nsp}[1]{\mathcal{N}\!\left(#1\right)}
%
%  Column space of a matrix
%  Usage:  \csp{A}
\newcommand{\csp}[1]{\mathcal{C}\!\left(#1\right)}
%
%  Row space of a matrix
%  Usage:  \rsp{A}
\newcommand{\rsp}[1]{\mathcal{R}\!\left(#1\right)}
%
%  Left null space of a matrix
%  Usage:  \lns{A}
\newcommand{\lns}[1]{\mathcal{L}\!\left(#1\right)}
%
%  Orthogonal complement of a vector space
%  Avoiding TeX's \perp
%  Usage:  \per{A}
\newcommand{\per}[1]{#1^\perp}
%
%%%%%%%%%%%%%%%%%%%%%
%
%     Systems of Equations
%
%%%%%%%%%%%%%%%%%%%%%
%
%  In-line form of an augmented matrix for a system of equations
%  Usage: \augmented{coefficient-matrix}{constant-vector}
\newcommand{\augmented}[2]{\left\lbrack\left.#1\,\right\rvert\,#2\right\rbrack}
%
%  Notation for a linear system before introducing matrix multiplication
%  Usage: \linearsystem{coefficient-matrix}{constant-vector}
\newcommand{\linearsystem}[2]{\mathcal{LS}\!\left(#1,\,#2\right)}
%
%  Notation for a homogenous system before introducing matrix multiplication
%  Usage: \homosystem{coefficient-matrix}
\newcommand{\homosystem}[1]{\linearsystem{#1}{\zerovector}}
%
%%%%%%%%%%%%%%%%%%%%%
%
%     Row Operations, Echelon Form
%
%%%%%%%%%%%%%%%%%%%%%
%
% Row operations on matrices
%
% Three commands to shorten up descriptions of gaussian elimination
%
% Usage: \rowopswap{row-i}{row-j}
% Usage: \rowopmult{scalar}{row-i}
% Usage: \rowopadd{scalar}{row-multiplied}{row-added-to}
\newcommand{\rowopswap}[2]{R_{#1}\leftrightarrow R_{#2}}
\newcommand{\rowopmult}[2]{#1R_{#2}}
\newcommand{\rowopadd}[3]{#1R_{#2}+R_{#3}}
%
% Mark leading 1's in echelon form with fbox
% Usage: \leading{a-1-usually}
\newcommand{\leading}[1]{\fbox{#1}}
%
%  Row-reduce arrow
%  Usage:  \rref inbetween a matrix and its reduced row-echelon form
\newcommand{\rref}{\xrightarrow{\text{RREF}}}
%
%  Elementary Matrices
%  Usage: \elemswap{subscript}{subscript}
%  Usage: \elemmult{scalar}{subscript}
%  Usage: \elemadd{scalar}{subscript-mult}{subscript-target}
%
\newcommand{\elemswap}[2]{E_{#1,#2}}
\newcommand{\elemmult}[2]{E_{#2}\left(#1\right)}
\newcommand{\elemadd}[3]{E_{#2,#3}\left(#1\right)}
%
%%%%%%%%%%%%%%%%%%%%%
%
%     2-D Constructions (Lists, Vectors, Matrices)
%
%%%%%%%%%%%%%%%%%%%%%
%
%  A list of scalars of generic length
%  Usage:  \scalarlist{scalar letter}{terminal subscript}
\newcommand{\scalarlist}[2]{{#1}_{1},\,{#1}_{2},\,{#1}_{3},\,\ldots,\,{#1}_{#2}}
%
%  Vector styling, bold (or use wiggles, arrows, whatever)
%  Subscripts go outside this construction
%  Usage: \vect{a symbol to use as a vector}
%  Have to already be in math mode
%
\newcommand{\vect}[1]{\mathbf{#1}}
%
%  A column vector
%  Usage: \colvector{list-delimited-by-\\}
%
\newcommand{\colvector}[1]{\begin{bmatrix}#1\end{bmatrix}}
%
%  A generic vector with components
%  Usage: \vectorcomponents{component-letter}{final-subscript}
\newcommand{\vectorcomponents}[2]{\colvector{#1_{1}\\#1_{2}\\#1_{3}\\\vdots\\#1_{#2}}}
%
%  A list of vectors of generic length
%  Usage:  \vectorlist{vector letter}{terminal subscript}
\newcommand{\vectorlist}[2]{\vect{#1}_{1},\,\vect{#1}_{2},\,\vect{#1}_{3},\,\ldots,\,\vect{#1}_{#2}}
%
%  Vector entries, entry i of vector v
%  (vector-expession still needs \vect, etc.)
%  Usage:  \vectorentry{vector-expression}{single-subscript}
\newcommand{\vectorentry}[2]{\left\lbrack#1\right\rbrack_{#2}}
%
%  Matrix entries, entry i,j of matrix A
%  Usage:  \matrixentry{matrix-expression}{paired-subscripts}
%
\newcommand{\matrixentry}[2]{\left\lbrack#1\right\rbrack_{#2}}
%
%  A generic linear combination
%  Usage:  \lincombo{scalar letter}{vector letter}{terminal subscript}
\newcommand{\lincombo}[3]{#1_{1}\vect{#2}_{1}+#1_{2}\vect{#2}_{2}+#1_{3}\vect{#2}_{3}+\cdots +#1_{#3}\vect{#2}_{#3}}
%
%  Matrix, column by column, as vectors
%  Usage:  \matrixcolumns{matrix letter}{terminal subscript}
\newcommand{\matrixcolumns}[2]{\left\lbrack\vect{#1}_{1}|\vect{#1}_{2}|\vect{#1}_{3}|\ldots|\vect{#1}_{#2}\right\rbrack}
%
%%%%%%%%%%%%%%%%%%%%%
%
%     Special Matrices
%
%%%%%%%%%%%%%%%%%%%%%
%
%  Transpose of a matrix
%  Usage:  \transpose{A}
\newcommand{\transpose}[1]{#1^{t}}
%
%  Inverse of a matrix
%  Usage:  \inverse{A}
\newcommand{\inverse}[1]{#1^{-1}}
%
%  Submatrix (for minors, determinants)
%  Usage: \submatrix{matrix-name}{delete-row}{delete-col}
\newcommand{\submatrix}[3]{#1\left(#2|#3\right)}
%
%  Adjoint of a matrix (twice)
%  This macro is a convenience to call \transpose and \conjugate properly
%  It shouldn't need to be modified (or mathematical meanings will change)
%  Usage:  \adj{A}
\newcommand{\adj}[1]{\transpose{\left(\conjugate{#1}\right)}}
%
%  This macro controls the symbol used for the adjoint
%  It can be edited to taste
%  Usage:  \adjoint{A}
\newcommand{\adjoint}[1]{#1^\ast}
%
%%%%%%%%%%%%%%%%%%%%%
%
%     Sets
%
%%%%%%%%%%%%%%%%%%%%%
%
%  A convenience for simple sets
%  Usage:  \set{list of element}
\newcommand{\set}[1]{\left\{#1\right\}}
%
%  Sets with vertical bar, "such that", sized for objects, not condition
%  Usage:  \setparts{objects}{condition}
%
%%\newcommand{\setparts}[2]{\left\{ #1\mid#2\right\}}
%%\newcommand{\setparts}[2]{\left\{\left. #1\right\rvert#2\right\}}
\newcommand{\setparts}[2]{\left\lbrace#1\,\middle|\,#2\right\rbrace}
%
%  Set Cardinality
%  Usage:  \card{a-set-letter}
\newcommand{\card}[1]{\left\lvert#1\right\rvert}
%
%  Set Union
%  Use \cup
%
%  Set Intersection
%  Use \cap
%
%  Set Complement
%  Usage:  \setcomplement{a-set-letter}
\newcommand{\setcomplement}[1]{\overline{#1}}
%
%%%%%%%%%%%%%%%%%%%%%
%
%     Eigenvalues and Eigenspaces
%
%%%%%%%%%%%%%%%%%%%%%
%
%  Characteristic polynomial
%  Usage: \charpoly{matrix-letter}{variable-letter}
\newcommand{\charpoly}[2]{p_{#1}\left(#2\right)}
%
%  Eigenspace
%  Usage: \eigenspace{matrix-letter}{eigenvalue-letter}
\newcommand{\eigenspace}[2]{\mathcal{E}_{#1}\left(#2\right)}
%
%  2013/10/03 Including ampersands is problematic here, 
%  think about fixes later
%  2014/02/22 Limited testing, seems &amp; is fine for HTML and LaTeX
%  2016-07-20 only employed in Archetypes, MBX has gather/align override
%  Eigensystem (presumes wrapped in an mrow within md)
%  Usage: \eigensystem{matrixletter}{eigenvalue}{list of basis vectors}
\newcommand{\eigensystem}[3]{\lambda&amp;=#2&amp;\eigenspace{#1}{#2}&amp;=\spn{\set{#3}}} 
%
%  Generalized Eigenspace
%  Usage: \geneigenspace{lin-trans-letter}{eigenvalue-letter}
\newcommand{\geneigenspace}[2]{\mathcal{G}_{#1}\left(#2\right)}
%
%  Algebraic multiplicty
%  Usage: \algmult{matrix-letter}{eigenvalue-letter}
\newcommand{\algmult}[2]{\alpha_{#1}\left(#2\right)}
%
%  Geometric multiplicty
%  Usage: \geomult{matrix-letter}{eigenvalue-letter}
\newcommand{\geomult}[2]{\gamma_{#1}\left(#2\right)}
%
%  Index (of eigenvalue)
%  Usage: \indx{matrix-letter}{eigenvalue-letter}
\newcommand{\indx}[2]{\iota_{#1}\left(#2\right)}
%
%%%%%%%%%%%%%%%%%%%%%
%
%     Linear Transformations
%
%%%%%%%%%%%%%%%%%%%%%
%
%  MathJax defines \lt to ease XML confusion
%
%  Linear transformation definition
%  Usage: \ltdefn{name-letter}{domain}{range}
\newcommand{\ltdefn}[3]{#1\colon #2\rightarrow#3}
%
%  Linear transformation evaluation
%  Usage: \lteval{name-letter}{input}
%  Replaces old \lt desired by MathJax
\newcommand{\lteval}[2]{#1\left(#2\right)}
%
% Linear transformation inverse
%  Usage: \ltinverse{name-letter}
\newcommand{\ltinverse}[1]{#1^{-1}}
%
%  Linear transformation restriction
%  Usage: \restrict{name-letter}{subspace-letter}
\newcommand{\restrict}[2]{{#1}|_{#2}}
%
%  Linear transformation preimage
%  Usage: \preimage{name-letter}{codomain-element}
\newcommand{\preimage}[2]{#1^{-1}\left(#2\right)}
%
%  Range of a linear transformation
%  TeX uses \range for something else
%  Usage:  \rng{T}
\newcommand{\rng}[1]{\mathcal{R}\!\left(#1\right)}
%
%  Kernel of a linear transformation
%  TeX uses \ker to do something different
%  Usage:  \krn{T}
\newcommand{\krn}[1]{\mathcal{K}\!\left(#1\right)}
%
%  Linear transformation composition
%  Usage: \compose{function-name}{function-name}
\newcommand{\compose}[2]{{#1}\circ{#2}}
%
%  Vector space of linear transformations
%  Usage: \vslt{domains}{codomains}
%  Presumes math mode
\newcommand{\vslt}[2]{\mathcal{LT}\left(#1,\,#2\right)}
%
%%%%%%%%%%%%%%%%%%%%%
%
%     Vector and Matrix Representations
%
%%%%%%%%%%%%%%%%%%%%%
%
%  Isomorphism symbol
%  Usage: \isomorphic
\newcommand{\isomorphic}{\cong}
%
%  Similarity
%  Usage: \similar{inner-matrix}{outer-invertible-matrix}
%  Rearranging this will not "fix" all desired changes throughout
%
\newcommand{\similar}[2]{\inverse{#2}#1#2}
%
%  Vector representation function name
%  Usage: \vectrepname{basis-letter}
\newcommand{\vectrepname}[1]{\rho_{#1}}
%
%  Vector representation output
%  Usage: \vectrep{basis-letter}{input}
\newcommand{\vectrep}[2]{\lteval{\vectrepname{#1}}{#2}}
%
%  Vector representation inverse function name
%  (Added later, not used consistently in FCLA)
%  Usage: \vectrepinvname{basis-letter}
\newcommand{\vectrepinvname}[1]{\ltinverse{\vectrepname{#1}}}
%
%  Vector representation inverse output
%  Usage: \vectrepinv{basis-letter}{input}
\newcommand{\vectrepinv}[2]{\lteval{\ltinverse{\vectrepname{#1}}}{#2}}
%
%  Matrix representation
%  Usage: \matrixrep{transformation-letter}{domain-basis-letter}{codomain-basis-letter}
\newcommand{\matrixrep}[3]{M^{#1}_{#2,#3}}
%
%  Matrix representation column-by-colum
%  2016-07-20 only employed once?
%  Usage: \matrixrepcolumns{transformation-letter}{codomain-basis-letter}{codomain-basis-vector-letter}{final-index}
\newcommand{\matrixrepcolumns}[4]{\left\lbrack \left.\vectrep{#2}{\lteval{#1}{\vect{#3}_{1}}}\right|\left.\vectrep{#2}{\lteval{#1}{\vect{#3}_{2}}}\right|\left.\vectrep{#2}{\lteval{#1}{\vect{#3}_{3}}}\right|\ldots\left|\vectrep{#2}{\lteval{#1}{\vect{#3}_{#4}}}\right.\right\rbrack}
%
%  Change of basis matrix
%  Usage: \cbm{domain-basis-letter}{codomain-basis-letter}
\newcommand{\cbm}[2]{C_{#1,#2}}
%
%%%%%%%%%%%%%%%%%%%%%
%
%     Canonical Forms
%
%%%%%%%%%%%%%%%%%%%%%
%
%  Jordan blocks
%  Usage: \jordan{size}{diagonal-element}
\newcommand{\jordan}[2]{J_{#1}\left(#2\right)}
%
%%%%%%%%%%%%%%%%%%%%%
%
%     Hadamard Matrices
%     Contributed by Elizabeth Million
%
%%%%%%%%%%%%%%%%%%%%%
%
%  Hadamard Product
%  Usage: \hadamard{a-matrix}{a-matrix}
\newcommand{\hadamard}[2]{#1\circ #2}
%
%  Hadamard identity matrix
%  Usage: \hadamardidentity{paired-subscripts-size-of-matrix}
\newcommand{\hadamardidentity}[1]{J_{#1}}
%
%  Hadamard inverse matrix
%  Usage: \hadamardinverse{matrix-expression}
\newcommand{\hadamardinverse}[1]{\widehat{#1}}


\title{Null Spaces, Spans, Linear Independence}

\begin{document}
\begin{abstract}
  Linearly dependent sets carry redundant vectors with them when used
  in building a set as a span, so when describing a null space as a
  span of vectors, it would be best if those vectors were linearly
  independent.
\end{abstract}
\maketitle

Earlier, we proved \ref{theorem:SSNS} which provided $n-r$ vectors
that could be used with the span construction to build the entire null
space of a matrix.

Linearly dependent sets carry redundant vectors with them when used in
building a set as a span.  Our aim now is to show that the vectors
provided by \ref{theorem:SSNS} form a linearly independent set, so in
one sense they are as efficient as possible a way to describe the null
space.

Notice that the vectors $\vect{z}_j$, $1\leq j\leq n-r$ first appear
in the vector form of solutions to arbitrary linear systems
(\ref{theorem:VFSLS}).  The exact same vectors appear again in the
span construction in the conclusion of \ref{theorem:SSNS}.  Since this
second theorem specializes to homogeneous systems the only real
difference is that the vector $\vect{c}$ in \ref{theorem:VFSLS} is the
zero vector for a homogeneous system.  Finally, \ref{theorem:BNS} will
now show that these same vectors are a linearly independent set.  We
will set the stage for the proof of this theorem with a moderately
large example.  Study the example carefully, as it will make it easier
to understand the proof.

\begin{example}%[Linear independence of null space basis]
  Suppose that we are interested in the null space of a $3\times 7$
  matrix, $A$, which row-reduces to
  \[
    B=
    \begin{bmatrix}
      \leading{1} & 0 & -2 & 4 & 0 & 3 & 9\\
      0 & \leading{1} & 5 & 6 & 0 & 7 & 1\\
      0 & 0 & 0 & 0 & \leading{1} & 8 & -5
    \end{bmatrix}
  \]

  The set $F=\set{3,\,4,\,6,\,7}$ is the set of indices for our four
  free variables that would be used in a description of the solution
  set for the homogeneous system $\homosystem{A}$.  Applying
  \ref{theorem:SSNS} we can begin to construct a set of four vectors
  whose span is the null space of $A$, a set of vectors we will
  reference as $T$.
  \[
    \nsp{A}=\spn{T}=\spn{\set{\vect{z}_1,\,\vect{z}_2,\,\vect{z}_3,\,\vect{z}_4}}=\spn{\set{
        \colvector{ \\ \\1\\0\\ \\0\\0},\,
        \colvector{ \\ \\0\\1\\ \\0\\0},\,
        \colvector{ \\ \\0\\0\\ \\1\\0},\,
        \colvector{ \\ \\0\\0\\ \\0\\1}
      }}
  \]
  So far, we have constructed as much of these individual vectors as
  we can, based just on the knowledge of the contents of the set $F$.
  This has allowed us to determine the entries in slots 3, 4, 6 and 7,
  while we have left slots 1, 2 and 5 blank.  Without doing any more,
  let us ask if $T$ is linearly independent?

  Begin with a relation of linear dependence on $T$, and see what we
  can learn about the scalars,
  \begin{align*}
    \zerovector&=\alpha_1\vect{z}_1+\alpha_2\vect{z}_2+\alpha_3\vect{z}_3+\alpha_4\vect{z}_4\\
    \colvector{0\\0\\0\\0\\0\\0\\0}
               &=
                 \alpha_1\colvector{ \\ \\1\\0\\ \\0\\0}+
                 \alpha_2\colvector{ \\ \\0\\1\\ \\0\\0}+
                 \alpha_3\colvector{ \\ \\0\\0\\ \\1\\0}+
                 \alpha_4\colvector{ \\ \\0\\0\\ \\0\\1}\\
               &=
                 \colvector{ \\ \\\alpha_1\\0\\ \\0\\0}+
                 \colvector{ \\ \\0\\\alpha_2\\ \\0\\0}+
                 \colvector{ \\ \\0\\0\\ \\\alpha_3\\0}+
                 \colvector{ \\ \\0\\0\\ \\0\\\alpha_4}
                =
                 \colvector{ \\ \\\alpha_1\\\alpha_2\\ \\\alpha_3\\\alpha_4}
  \end{align*}
  Applying \ref{definition:CVE} to the two ends of this chain of
  equalities, we see that $\alpha_1=\alpha_2=\alpha_3=\alpha_4=0$.  
  Therefore, the only relation of linear dependence on the set $T$ is
  \begin{multipleChoice}
    \choice[correct]{the trivial one}
    \choice{a nontrivial one}
  \end{multipleChoice}

  \begin{feedback}[correct]
    Exactly!  Therefore the set $T$ is linearly independent.
    The important feature of this example is how the ``pattern of zeros
    and ones'' in the four vectors led to the conclusion of linear
    independence.
  \end{feedback}
\end{example}

The proof of \ref{theorem:BNS} relies on the ``pattern of zeros and
ones'' that arise in the vectors $\vect{z}_i$, $1\leq i\leq n-r$ in
the entries that arise with the locations of the non-pivot columns.
Play along with the above example as you study the proof.  This proof
is also a good first example of how to prove a conclusion that states
a set is linearly independent.

\begin{theorem}[Basis for Null Spaces]
  \label{theorem:BNS}

  Suppose that $A$ is an $m\times n$ matrix, and $B$ is a
  row-equivalent matrix in reduced row-echelon form with $r$ pivot
  columns.  Let $D=\{d_1,\,d_2,\,d_3,\,\ldots,\,d_r\}$ and
  $F=\{f_1,\,f_2,\,f_3,\,\ldots,\,f_{n-r}\}$ be the sets of column
  indices where $B$ does and does not (respectively) have pivot
  columns.  Construct the $n-r$ vectors $\vect{z}_j$,
  $1\leq j\leq n-r$ of size $n$ as
  \[
    \vectorentry{\vect{z}_j}{i}=
    \begin{cases}
      1&\text{if $i\in F$, $i=f_j$}\\
      0&\text{if $i\in F$, $i\neq f_j$}\\
      -\matrixentry{B}{k,f_j}&\text{if $i\in D$, $i=d_k$}
    \end{cases}
  \]

  Define the set $S=\set{\vectorlist{z}{n-r}}$.Then
  \begin{enumerate}
  \item $\nsp{A}=\spn{S}$.
  \item $S$ is a linearly independent set.
  \end{enumerate}

\begin{proof}
  Notice first that the vectors $\vect{z}_j$, $1\leq j\leq n-r$ are
  exactly the same as the $n-r$ vectors defined in \ref{theorem:SSNS}.
  Also, the hypotheses of \ref{theorem:SSNS} are the same as the
  hypotheses of the theorem we are currently proving.  So it is then
  simply the conclusion of \ref{theorem:SSNS} that tells us that
  $\nsp{A}=\spn{S}$.  That was the easy half, but the second part is
  not much harder.  What is new here is the claim that $S$ is a
  linearly
  \wordChoice{\choice{dependent}\choice[correct]{independent}} set.

  To prove the linear independence of a set, we need to start with a
  relation of linear dependence and somehow conclude that the scalars
  involved \wordChoice{\choice[correct]{must all be zero}\choice{must
      all be nonzero}}, i.e., that the relation of linear dependence
  only happens in the trivial fashion.  So to establish the linear
  \wordChoice{\choice{dependence}\choice[correct]{independence}} of
  $S$, we start with
  \[
    \lincombo{\alpha}{z}{n-r}=\zerovector.
  \]

  For each $j$, $1\leq j\leq n-r$, consider the equality of the
  individual entries of the vectors on both sides of this equality in
  position $f_j$,
  \begin{align*}
    0
    &=\vectorentry{\zerovector}{f_j}\\
    &=\vectorentry{\lincombo{\alpha}{z}{n-r}}{f_j} \\ %&&\ref{definition:CVE}\\
    &=
      \vectorentry{\alpha_1\vect{z}_1}{f_j}+
      \vectorentry{\alpha_2\vect{z}_2}{f_j}+
      \vectorentry{\alpha_3\vect{z}_3}{f_j}+
      \cdots+
      \vectorentry{\alpha_{n-r}\vect{z}_{n-r}}{f_j} \\ %&&\ref{definition:CVA}\\
    &=
      \alpha_1\vectorentry{\vect{z}_1}{f_j}+
      \alpha_2\vectorentry{\vect{z}_2}{f_j}+
      \alpha_3\vectorentry{\vect{z}_3}{f_j}+
      \cdots+\\
    &\quad\quad
      \alpha_{j-1}\vectorentry{\vect{z}_{j-1}}{f_j}+
      \alpha_{j}\vectorentry{\vect{z}_j}{f_j}+
      \alpha_{j+1}\vectorentry{\vect{z}_{j+1}}{f_j}+
      \cdots+\\
    &\quad\quad
      \alpha_{n-r}\vectorentry{\vect{z}_{n-r}}{f_j} \\ %&&\ref{definition:CVSM}\\
    &=\alpha_1(0)+
      \alpha_2(0)+
      \alpha_3(0)+
      \cdots+\\
    &\quad\quad
      \alpha_{j-1}(0)+
      \alpha_{j}(1)+
      \alpha_{j+1}(0)+
      \cdots+\alpha_{n-r}(0) \\ %&&\text{Definition of $\vect{z}_j$}\\
    &=\alpha_{j}
  \end{align*}
  So for all $j$, $1\leq j\leq n-r$, we have $\alpha_j=\answer{0}$,
  which is the conclusion that tells us that the \textit{only}
  relation of linear dependence on $S=\set{\vectorlist{z}{n-r}}$ is
  the trivial one.  Hence the set is linearly independent, as desired.
\end{proof}
\end{theorem}

\begin{example}%[Null space spanned by linearly independent set,  Archetype L]
  The matrix
  \[
    L = \begin{bmatrix}
      -2 & -1 & -2 & -4 & 4 \\
      -6 & -5 &  -4 &  -4 & 6 \\
      10 & 7 & 7 &  10 & -13 \\
      -7 & -5 &  -6 & -9 & 10 \\
      -4 &  -3 & -4 & -6 & 6 \\
    \end{bmatrix}
  \]
  row reduces to
  \[
    \begin{bmatrix}
      \leading{1} & 0 & 0 & 1 & -2 \\
      0 & \leading{1} & 0 &  -2 & 2 \\
      0 & 0 & \leading{1} &  2 &  -1 \\
      0 & 0 & 0 & 0 &  0 \\
      0 & 0 & 0 & 0 &  0
    \end{bmatrix}
  \]
  and the null space can be written as
  \[
    \nsp{L} = \spn{\set{\colvector{-1\\2\\-2\\1\\\answer{0}},\,\colvector{2\\-2\\1\\0\\\answer{1}}}}
  \]
  Solving the homogeneous system $\homosystem{L}$ resulted in
  recognizing $x_4$ and $x_5$ as the free variables.  So look in
  entries 4 and 5 of the two vectors above and notice the pattern of
  zeros and ones that provides the linear independence of the set.
\end{example}

\end{document}


