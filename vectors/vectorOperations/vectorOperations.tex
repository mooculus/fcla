\documentclass{ximera}

\input{../../preamble.tex}

\title{Vector Operations}

\begin{document}
\begin{abstract}
  We define some new operations involving vectors, and collect some basic properties of these operations.
\end{abstract}
\maketitle

We start our study of this set by first defining what it means for two vectors to be the same.

\begin{definition}[Column Vector Equality]
Suppose that $\vect{u},\,\vect{v}\in\complex{m}$.  Then $\vect{u}$ and $\vect{v}$ are \dfn{equal}, written $\vect{u}=\vect{v}$ if
\begin{align*}
\vectorentry{\vect{u}}{i}&=\vectorentry{\vect{v}}{i}
&&1\leq i\leq m
\end{align*}
\end{definition}

Now this may seem like a silly thing to say so carefully.  Of course
two vectors are equal if they are equal for each corresponding entry!
Well, this is not as silly as it appears.  We will see a few occasions
later where the obvious definition is \textit{not} the right one.  And
besides, in doing mathematics we need to be very careful about making
all the necessary definitions and making them unambiguous.  And we
have done that here.

Notice now that the symbol ``='' is now doing triple-duty.  We know
from our earlier education what it means for two numbers (real or
complex) to be equal, and we take this for granted.  We already
defined what it meant for two sets to be equal.  Now we have defined
what it means for two vectors to be equal, and that definition builds
on our definition for when two numbers are equal when we use the
condition $u_i=v_i$ for all $1\leq i\leq m$.  So think carefully about
your objects when you see an equal sign and think about just which
notion of equality you have encountered.  This will be especially
important when you are asked to construct proofs whose conclusion
states that two objects are equal.

Let us do an example of vector equality that begins to hint at the
utility of this definition.

% BADBAD
\begin{example}[Vector equality for a system of equations]

<indexlocation index="Archetype B!vector equality" />
Consider the system of linear equations in \ref{archetype:B},
<archetypepart acro="B" part="systemdefn" />


Note the use of three equals signs---each indicates an equality of numbers (the linear expressions are numbers when we evaluate them with fixed values of the variable quantities).  Now write the vector equality,
\[
\colvector{-7x_1 -6 x_2 - 12x_3\\ 5x_1  + 5x_2 + 7x_3\\ x_1 +4x_3}
=
\colvector{-33\\24\\5}.
\]
This \textit{single} equality (of two column vectors) translates into
\textit{three} simultaneous equalities of numbers that form the system
of equations.  So with this new notion of vector equality we can
become less reliant on referring to \textit{systems} of
\textit{simultaneous} equations.  There is more to vector equality
than just this, but this is a good example for starters and we will
develop it further.
\end{example}

We will now define two operations on the set $\complex{m}$.  By this
we mean well-defined procedures that somehow convert vectors into
other vectors.  Here are two of the most basic definitions of the
entire course.

\begin{definition}[Column Vector Addition]
  Suppose that $\vect{u},\,\vect{v}\in\complex{m}$. The \dfn{sum} of
  $\vect{u}$ and $\vect{v}$ is the vector $\vect{u}+\vect{v}$ defined
  by
  \begin{align*}
    \vectorentry{\vect{u}+\vect{v}}{i}
    &=\vectorentry{\vect{u}}{i}+\vectorentry{\vect{v}}{i}
    &&1\leq i\leq m
  \end{align*}
\end{definition}

So vector addition takes two vectors of the same size and combines
them (in a natural way!) to create a new vector of the same size.
Notice that this definition is required, even if we agree that this is
the obvious, right, natural or correct way to do it.  Notice too that
the symbol `+' is being recycled.  We all know how to add
\textit{numbers}, but now we have the same symbol extended to
double-duty and we use it to indicate how to add two new objects,
vectors.  And this definition of our new meaning is built on our
previous meaning of addition via the expressions $u_i+v_i$.  Think
about your objects, especially when doing proofs.  Vector addition is
easy, here is an example from $\complex{4}$.

\begin{example}[Addition of two vectors in $\complex{4}$]
  If
  \begin{align*}
    \vect{u}=\colvector{2\\-3\\4\\2}&&\vect{v}=\colvector{-1\\5\\2\\-7}
  \end{align*}
  then
  \[\vect{u}+\vect{v}=
    \colvector{\answer{2}\\-3\\4\\2}+\colvector{-1\\5\\2\\-7}=
    \colvector{2+(-1)\\-3+5\\4+2\\2+(-7)}=
    \colvector{\answer{1}\\\answer{2}\\6\\-5}\]
\end{example}

Our second operation takes two objects of different types,
specifically a number and a vector, and combines them to create
another vector.  In this context we call a number a \dfn{scalar} in
order to emphasize that it is not a vector.

\begin{definition}[Column Vector Scalar Multiplication]
  Suppose $\vect{u}\in\complex{m}$ and $\alpha\in\complexes$, then the
  \dfn{scalar multiple} of $\vect{u}$ by $\alpha$ is the vector
  $\alpha\vect{u}$ defined by
  \begin{align*}
    \vectorentry{\alpha\vect{u}}{i}
    &=\alpha\vectorentry{\vect{u}}{i}
    &&1\leq i\leq m
  \end{align*}
\end{definition}

Notice that we are doing a kind of multiplication here, but we are
\textit{defining} a new type, perhaps in what appears to be a natural
way.  We use juxtaposition (smashing two symbols together
side-by-side) to denote this operation rather than using a symbol like
we did with vector addition.  So this can be another source of
confusion.  When two symbols are next to each other, are we doing
regular old multiplication, the kind we have done for years, or are we
doing scalar vector multiplication, the operation we just defined?
Think about your objects---if the first object is a scalar, and the
second is a vector, then it \textit{must} be that we are doing our new
operation, and the \textit{result} of this operation will be another
vector.

Notice how consistency in notation can be an aid here.  If we write
scalars as lower case Greek letters from the start of the alphabet
(such as $\alpha$, $\beta$, ldots ) and write vectors in bold Latin
letters from the end of the alphabet ($\vect{u}$, $\vect{v}$, ldots ),
then we have some hints about what type of objects we are working
with.  This can be a blessing \textit{and} a curse, since when we go
read another book about linear algebra, or read an application in
another discipline (physics, economics, ldots ) the types of notation
employed may be very different and hence unfamiliar.

Again, computationally, vector scalar multiplication is very easy.

\begin{example}[Scalar multiplication in $\complex{5}$]
  If
  \[
    \vect{u}=\colvector{3\\1\\-2\\4\\-1}
  \]
  and $\alpha=6$, then
  \[
    \alpha\vect{u}=
    6\colvector{3\\1\\-2\\4\\-1}=
    \colvector{6(3)\\6(1)\\6(-2)\\6(4)\\6(-1)}=
    \colvector{18\\6\\-12\\24\\-6}.
  \]
\end{example}

\end{document}
