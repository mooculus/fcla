\documentclass{ximera}

% These macros are automatically generated from the "macros"
% XML element.  Make permanent edits there.
%
% History
%   2004/01/01  Initiated for FCLA, evolved from there
%   2006/09/17  Converted  _, ^  to \sb, \sp for TeX4ht
%   2014/02/01  Updated for MathBook XML projects
%               Obsolete in FCLA: \codeindent, \computerfont, \define
%               Change: MathJax wants \lt, so replaced by \lteval
%   2014/02/22  New: \orderof, \reals, \per
%   2015/08/16  Incorporated into MathBook XML version of FCLA
%
%%%%%%%%%%%%%%%%%%%%%
%
%     Conveniences
%
%%%%%%%%%%%%%%%%%%%%%
%
%  Order of (asymptotically limit of fraction is 1)
%  Usage: \orderof{some function}
%
\newcommand{\orderof}[1]{\sim #1}
%
%  Integers
%  Usage:  \Z
\newcommand{\Z}{\mathbb{Z}}
%
%  Real numbers, as set of scalars
%  Usage:  \reals
\newcommand{\reals}{\mathbb{R}}
%
%  n-space over real field
%  Usage: \complex{integer-dimension}
\newcommand{\real}[1]{\mathbb{R}^{#1}}
%
%  Complex numbers, as set of scalars
%  Usage:  \complexes
\newcommand{\complexes}{\mathbb{C}}
%
%  n-space over complex field
%  Usage: \complex{integer-dimension}
\newcommand{\complex}[1]{\mathbb{C}^{#1}}
%
%  Complex conjugation (scalar, vector, matrix)
%  Usage: \conjugate{object}
\newcommand{\conjugate}[1]{\overline{#1}}
%
%  Complex number modulus
%  Usage: \modulus{a+bi}
%  Presumes math mode
\newcommand{\modulus}[1]{\left\lvert#1\right\rvert}
%
%  Zero vector
%  Usage: \zerovector
\newcommand{\zerovector}{\vect{0}}
%
%  Zero matrix
%  Usage: \zeromatrix, use a subscript when size is important
\newcommand{\zeromatrix}{\mathcal{O}}
%
%  Inner product (brackets, not quadratic form)
%  Usage: \innerproduct{a-vector}{a-vector}
\newcommand{\innerproduct}[2]{\left\langle#1,\,#2\right\rangle}
%
%  Norm of a vector
%  Usage: \norm{a-vector}
\newcommand{\norm}[1]{\left\lVert#1\right\rVert}
%
%  Dimension
%  Usage: \dimension{vector-space-letter}
\newcommand{\dimension}[1]{\dim\left(#1\right)}
%
%  Nullity
%  Usage: \nullity{matrix-or-lintrans-letter}
\newcommand{\nullity}[1]{n\left(#1\right)}
%
%  Rank
%  Usage: \rank{matrix-or-lintrans-letter}
\newcommand{\rank}[1]{r\left(#1\right)}
%
%  Direct sum
%  Usage: \ds between a couple of subspaces
%
\newcommand{\ds}{\oplus}
%
%  Determinant of a matrix (functional)
%  Usage: \detname{A}
\newcommand{\detname}[1]{\det\left(#1\right)}
%
%  Determinant of a matrix (vertical bars)
%  Usage: \detbars{A}
\newcommand{\detbars}[1]{\left\lvert#1\right\rvert}
%
%  Trace of a Matrix
%  Usage: \trace{matrix name}
\newcommand{\trace}[1]{t\left(#1\right)}
%
%  Square Root of a Matrix
%  Usage: \sr{a-matrix}
\newcommand{\sr}[1]{#1^{1/2}}
%
%%%%%%%%%%%%%%%%%%%%%
%
%     Subspace Constructions
%
%%%%%%%%%%%%%%%%%%%%%
%
%  Span of a set of vectors
%  \span and \sp are used by TeX for other things
%  Usage: \spn{set-of-vectors}
\newcommand{\spn}[1]{\left\langle#1\right\rangle}
%
%  Null space of a matrix
%  Usage:  \nsp{A}
\newcommand{\nsp}[1]{\mathcal{N}\!\left(#1\right)}
%
%  Column space of a matrix
%  Usage:  \csp{A}
\newcommand{\csp}[1]{\mathcal{C}\!\left(#1\right)}
%
%  Row space of a matrix
%  Usage:  \rsp{A}
\newcommand{\rsp}[1]{\mathcal{R}\!\left(#1\right)}
%
%  Left null space of a matrix
%  Usage:  \lns{A}
\newcommand{\lns}[1]{\mathcal{L}\!\left(#1\right)}
%
%  Orthogonal complement of a vector space
%  Avoiding TeX's \perp
%  Usage:  \per{A}
\newcommand{\per}[1]{#1^\perp}
%
%%%%%%%%%%%%%%%%%%%%%
%
%     Systems of Equations
%
%%%%%%%%%%%%%%%%%%%%%
%
%  In-line form of an augmented matrix for a system of equations
%  Usage: \augmented{coefficient-matrix}{constant-vector}
\newcommand{\augmented}[2]{\left\lbrack\left.#1\,\right\rvert\,#2\right\rbrack}
%
%  Notation for a linear system before introducing matrix multiplication
%  Usage: \linearsystem{coefficient-matrix}{constant-vector}
\newcommand{\linearsystem}[2]{\mathcal{LS}\!\left(#1,\,#2\right)}
%
%  Notation for a homogenous system before introducing matrix multiplication
%  Usage: \homosystem{coefficient-matrix}
\newcommand{\homosystem}[1]{\linearsystem{#1}{\zerovector}}
%
%%%%%%%%%%%%%%%%%%%%%
%
%     Row Operations, Echelon Form
%
%%%%%%%%%%%%%%%%%%%%%
%
% Row operations on matrices
%
% Three commands to shorten up descriptions of gaussian elimination
%
% Usage: \rowopswap{row-i}{row-j}
% Usage: \rowopmult{scalar}{row-i}
% Usage: \rowopadd{scalar}{row-multiplied}{row-added-to}
\newcommand{\rowopswap}[2]{R_{#1}\leftrightarrow R_{#2}}
\newcommand{\rowopmult}[2]{#1R_{#2}}
\newcommand{\rowopadd}[3]{#1R_{#2}+R_{#3}}
%
% Mark leading 1's in echelon form with fbox
% Usage: \leading{a-1-usually}
\newcommand{\leading}[1]{\fbox{#1}}
%
%  Row-reduce arrow
%  Usage:  \rref inbetween a matrix and its reduced row-echelon form
\newcommand{\rref}{\xrightarrow{\text{RREF}}}
%
%  Elementary Matrices
%  Usage: \elemswap{subscript}{subscript}
%  Usage: \elemmult{scalar}{subscript}
%  Usage: \elemadd{scalar}{subscript-mult}{subscript-target}
%
\newcommand{\elemswap}[2]{E_{#1,#2}}
\newcommand{\elemmult}[2]{E_{#2}\left(#1\right)}
\newcommand{\elemadd}[3]{E_{#2,#3}\left(#1\right)}
%
%%%%%%%%%%%%%%%%%%%%%
%
%     2-D Constructions (Lists, Vectors, Matrices)
%
%%%%%%%%%%%%%%%%%%%%%
%
%  A list of scalars of generic length
%  Usage:  \scalarlist{scalar letter}{terminal subscript}
\newcommand{\scalarlist}[2]{{#1}_{1},\,{#1}_{2},\,{#1}_{3},\,\ldots,\,{#1}_{#2}}
%
%  Vector styling, bold (or use wiggles, arrows, whatever)
%  Subscripts go outside this construction
%  Usage: \vect{a symbol to use as a vector}
%  Have to already be in math mode
%
\newcommand{\vect}[1]{\mathbf{#1}}
%
%  A column vector
%  Usage: \colvector{list-delimited-by-\\}
%
\newcommand{\colvector}[1]{\begin{bmatrix}#1\end{bmatrix}}
%
%  A generic vector with components
%  Usage: \vectorcomponents{component-letter}{final-subscript}
\newcommand{\vectorcomponents}[2]{\colvector{#1_{1}\\#1_{2}\\#1_{3}\\\vdots\\#1_{#2}}}
%
%  A list of vectors of generic length
%  Usage:  \vectorlist{vector letter}{terminal subscript}
\newcommand{\vectorlist}[2]{\vect{#1}_{1},\,\vect{#1}_{2},\,\vect{#1}_{3},\,\ldots,\,\vect{#1}_{#2}}
%
%  Vector entries, entry i of vector v
%  (vector-expession still needs \vect, etc.)
%  Usage:  \vectorentry{vector-expression}{single-subscript}
\newcommand{\vectorentry}[2]{\left\lbrack#1\right\rbrack_{#2}}
%
%  Matrix entries, entry i,j of matrix A
%  Usage:  \matrixentry{matrix-expression}{paired-subscripts}
%
\newcommand{\matrixentry}[2]{\left\lbrack#1\right\rbrack_{#2}}
%
%  A generic linear combination
%  Usage:  \lincombo{scalar letter}{vector letter}{terminal subscript}
\newcommand{\lincombo}[3]{#1_{1}\vect{#2}_{1}+#1_{2}\vect{#2}_{2}+#1_{3}\vect{#2}_{3}+\cdots +#1_{#3}\vect{#2}_{#3}}
%
%  Matrix, column by column, as vectors
%  Usage:  \matrixcolumns{matrix letter}{terminal subscript}
\newcommand{\matrixcolumns}[2]{\left\lbrack\vect{#1}_{1}|\vect{#1}_{2}|\vect{#1}_{3}|\ldots|\vect{#1}_{#2}\right\rbrack}
%
%%%%%%%%%%%%%%%%%%%%%
%
%     Special Matrices
%
%%%%%%%%%%%%%%%%%%%%%
%
%  Transpose of a matrix
%  Usage:  \transpose{A}
\newcommand{\transpose}[1]{#1^{t}}
%
%  Inverse of a matrix
%  Usage:  \inverse{A}
\newcommand{\inverse}[1]{#1^{-1}}
%
%  Submatrix (for minors, determinants)
%  Usage: \submatrix{matrix-name}{delete-row}{delete-col}
\newcommand{\submatrix}[3]{#1\left(#2|#3\right)}
%
%  Adjoint of a matrix (twice)
%  This macro is a convenience to call \transpose and \conjugate properly
%  It shouldn't need to be modified (or mathematical meanings will change)
%  Usage:  \adj{A}
\newcommand{\adj}[1]{\transpose{\left(\conjugate{#1}\right)}}
%
%  This macro controls the symbol used for the adjoint
%  It can be edited to taste
%  Usage:  \adjoint{A}
\newcommand{\adjoint}[1]{#1^\ast}
%
%%%%%%%%%%%%%%%%%%%%%
%
%     Sets
%
%%%%%%%%%%%%%%%%%%%%%
%
%  A convenience for simple sets
%  Usage:  \set{list of element}
\newcommand{\set}[1]{\left\{#1\right\}}
%
%  Sets with vertical bar, "such that", sized for objects, not condition
%  Usage:  \setparts{objects}{condition}
%
%%\newcommand{\setparts}[2]{\left\{ #1\mid#2\right\}}
%%\newcommand{\setparts}[2]{\left\{\left. #1\right\rvert#2\right\}}
\newcommand{\setparts}[2]{\left\lbrace#1\,\middle|\,#2\right\rbrace}
%
%  Set Cardinality
%  Usage:  \card{a-set-letter}
\newcommand{\card}[1]{\left\lvert#1\right\rvert}
%
%  Set Union
%  Use \cup
%
%  Set Intersection
%  Use \cap
%
%  Set Complement
%  Usage:  \setcomplement{a-set-letter}
\newcommand{\setcomplement}[1]{\overline{#1}}
%
%%%%%%%%%%%%%%%%%%%%%
%
%     Eigenvalues and Eigenspaces
%
%%%%%%%%%%%%%%%%%%%%%
%
%  Characteristic polynomial
%  Usage: \charpoly{matrix-letter}{variable-letter}
\newcommand{\charpoly}[2]{p_{#1}\left(#2\right)}
%
%  Eigenspace
%  Usage: \eigenspace{matrix-letter}{eigenvalue-letter}
\newcommand{\eigenspace}[2]{\mathcal{E}_{#1}\left(#2\right)}
%
%  2013/10/03 Including ampersands is problematic here, 
%  think about fixes later
%  2014/02/22 Limited testing, seems &amp; is fine for HTML and LaTeX
%  2016-07-20 only employed in Archetypes, MBX has gather/align override
%  Eigensystem (presumes wrapped in an mrow within md)
%  Usage: \eigensystem{matrixletter}{eigenvalue}{list of basis vectors}
\newcommand{\eigensystem}[3]{\lambda&amp;=#2&amp;\eigenspace{#1}{#2}&amp;=\spn{\set{#3}}} 
%
%  Generalized Eigenspace
%  Usage: \geneigenspace{lin-trans-letter}{eigenvalue-letter}
\newcommand{\geneigenspace}[2]{\mathcal{G}_{#1}\left(#2\right)}
%
%  Algebraic multiplicty
%  Usage: \algmult{matrix-letter}{eigenvalue-letter}
\newcommand{\algmult}[2]{\alpha_{#1}\left(#2\right)}
%
%  Geometric multiplicty
%  Usage: \geomult{matrix-letter}{eigenvalue-letter}
\newcommand{\geomult}[2]{\gamma_{#1}\left(#2\right)}
%
%  Index (of eigenvalue)
%  Usage: \indx{matrix-letter}{eigenvalue-letter}
\newcommand{\indx}[2]{\iota_{#1}\left(#2\right)}
%
%%%%%%%%%%%%%%%%%%%%%
%
%     Linear Transformations
%
%%%%%%%%%%%%%%%%%%%%%
%
%  MathJax defines \lt to ease XML confusion
%
%  Linear transformation definition
%  Usage: \ltdefn{name-letter}{domain}{range}
\newcommand{\ltdefn}[3]{#1\colon #2\rightarrow#3}
%
%  Linear transformation evaluation
%  Usage: \lteval{name-letter}{input}
%  Replaces old \lt desired by MathJax
\newcommand{\lteval}[2]{#1\left(#2\right)}
%
% Linear transformation inverse
%  Usage: \ltinverse{name-letter}
\newcommand{\ltinverse}[1]{#1^{-1}}
%
%  Linear transformation restriction
%  Usage: \restrict{name-letter}{subspace-letter}
\newcommand{\restrict}[2]{{#1}|_{#2}}
%
%  Linear transformation preimage
%  Usage: \preimage{name-letter}{codomain-element}
\newcommand{\preimage}[2]{#1^{-1}\left(#2\right)}
%
%  Range of a linear transformation
%  TeX uses \range for something else
%  Usage:  \rng{T}
\newcommand{\rng}[1]{\mathcal{R}\!\left(#1\right)}
%
%  Kernel of a linear transformation
%  TeX uses \ker to do something different
%  Usage:  \krn{T}
\newcommand{\krn}[1]{\mathcal{K}\!\left(#1\right)}
%
%  Linear transformation composition
%  Usage: \compose{function-name}{function-name}
\newcommand{\compose}[2]{{#1}\circ{#2}}
%
%  Vector space of linear transformations
%  Usage: \vslt{domains}{codomains}
%  Presumes math mode
\newcommand{\vslt}[2]{\mathcal{LT}\left(#1,\,#2\right)}
%
%%%%%%%%%%%%%%%%%%%%%
%
%     Vector and Matrix Representations
%
%%%%%%%%%%%%%%%%%%%%%
%
%  Isomorphism symbol
%  Usage: \isomorphic
\newcommand{\isomorphic}{\cong}
%
%  Similarity
%  Usage: \similar{inner-matrix}{outer-invertible-matrix}
%  Rearranging this will not "fix" all desired changes throughout
%
\newcommand{\similar}[2]{\inverse{#2}#1#2}
%
%  Vector representation function name
%  Usage: \vectrepname{basis-letter}
\newcommand{\vectrepname}[1]{\rho_{#1}}
%
%  Vector representation output
%  Usage: \vectrep{basis-letter}{input}
\newcommand{\vectrep}[2]{\lteval{\vectrepname{#1}}{#2}}
%
%  Vector representation inverse function name
%  (Added later, not used consistently in FCLA)
%  Usage: \vectrepinvname{basis-letter}
\newcommand{\vectrepinvname}[1]{\ltinverse{\vectrepname{#1}}}
%
%  Vector representation inverse output
%  Usage: \vectrepinv{basis-letter}{input}
\newcommand{\vectrepinv}[2]{\lteval{\ltinverse{\vectrepname{#1}}}{#2}}
%
%  Matrix representation
%  Usage: \matrixrep{transformation-letter}{domain-basis-letter}{codomain-basis-letter}
\newcommand{\matrixrep}[3]{M^{#1}_{#2,#3}}
%
%  Matrix representation column-by-colum
%  2016-07-20 only employed once?
%  Usage: \matrixrepcolumns{transformation-letter}{codomain-basis-letter}{codomain-basis-vector-letter}{final-index}
\newcommand{\matrixrepcolumns}[4]{\left\lbrack \left.\vectrep{#2}{\lteval{#1}{\vect{#3}_{1}}}\right|\left.\vectrep{#2}{\lteval{#1}{\vect{#3}_{2}}}\right|\left.\vectrep{#2}{\lteval{#1}{\vect{#3}_{3}}}\right|\ldots\left|\vectrep{#2}{\lteval{#1}{\vect{#3}_{#4}}}\right.\right\rbrack}
%
%  Change of basis matrix
%  Usage: \cbm{domain-basis-letter}{codomain-basis-letter}
\newcommand{\cbm}[2]{C_{#1,#2}}
%
%%%%%%%%%%%%%%%%%%%%%
%
%     Canonical Forms
%
%%%%%%%%%%%%%%%%%%%%%
%
%  Jordan blocks
%  Usage: \jordan{size}{diagonal-element}
\newcommand{\jordan}[2]{J_{#1}\left(#2\right)}
%
%%%%%%%%%%%%%%%%%%%%%
%
%     Hadamard Matrices
%     Contributed by Elizabeth Million
%
%%%%%%%%%%%%%%%%%%%%%
%
%  Hadamard Product
%  Usage: \hadamard{a-matrix}{a-matrix}
\newcommand{\hadamard}[2]{#1\circ #2}
%
%  Hadamard identity matrix
%  Usage: \hadamardidentity{paired-subscripts-size-of-matrix}
\newcommand{\hadamardidentity}[1]{J_{#1}}
%
%  Hadamard inverse matrix
%  Usage: \hadamardinverse{matrix-expression}
\newcommand{\hadamardinverse}[1]{\widehat{#1}}


\title{Spanning Sets}

\begin{document}
\begin{abstract}
  ``Spanning sets'' provide a quick way to describe an infinite set of
  vectors, making use of linear combinations.
\end{abstract}
\maketitle

In this section we will provide a quick way to describe an infinite
set of vectors, making use of linear combinations.  This will give us
a convenient way to describe the solution set of a linear system, the
null space of a matrix, and many other sets of vectors.

In \ref{example:VFSAL} we saw the solution set of a homogeneous system
described as all possible linear combinations of two particular
vectors.  This is a useful way to construct or describe infinite sets
of vectors, so we encapsulate the idea in a definition.

\begin{definition}[Span of a Set of Column Vectors]
  Given a set of vectors $S=\{\vectorlist{u}{p}\}$, their \dfn{span},
  $\spn{S}$, is the set of all possible linear combinations of
  $\vectorlist{u}{p}$.  Symbolically,
  \begin{align*}
    \spn{S}&=\setparts{\lincombo{\alpha}{u}{p}}{\alpha_i\in\complexes,\,1\leq i\leq p}\\
           &=\setparts{\sum_{i=1}^{p}\alpha_i\vect{u}_i}{\alpha_i\in\complexes,\,1\leq i\leq p}
  \end{align*}
\end{definition}

The span is just a set of vectors, though in all but one situation it
is an infinite set.  (Just when is it not infinite?)  So we start with
a finite collection of vectors $S$ ($p$ of them to be precise), and
use this finite set to describe an infinite set of vectors, $\spn{S}$.
Confusing the \textit{finite} set $S$ with the \textit{infinite} set
$\spn{S}$ is one of the most persistent problems in understanding
introductory linear algebra.  We will see this construction
repeatedly, so let us work through some examples to get comfortable
with it.  The most obvious question about a set is if a particular
item of the correct type is in the set, or not in the set.

\begin{example}[A basic span]
  Consider the set of 5 vectors, $S$, from $\complex{4}$
  \[
    S=\set{
      \colvector{1 \\ 1 \\ 3 \\ 1},\,
      \colvector{2 \\ 1 \\ 2 \\ -1},\,
      \colvector{7 \\ 3 \\ 5 \\ -5},\,
      \colvector{1 \\ 1 \\ -1 \\ 2},\,
      \colvector{-1 \\ 0 \\ 9 \\ 0}
    }
  \]
  and consider the infinite set of vectors $\spn{S}$ formed from all
  possible linear combinations of the elements of $S$.  Here are four
  vectors we definitely know are elements of $\spn{S}$, since we will
  construct them in accordance with \ref{definition:SSCV},
  \[
    \vect{w}=
    (2)\colvector{1 \\ 1 \\ 3 \\ 1}+
    (1)\colvector{2 \\ 1 \\ 2 \\ -1}+
    (-1)\colvector{7 \\ 3 \\ 5 \\ -5}+
    (2)\colvector{1 \\ 1 \\ -1 \\ 2}+
    (3)\colvector{-1 \\ 0 \\ 9 \\ 0}
    =
    \colvector{-4\\2\\28\\10}
  \]
  \[
    \vect{x}=
    (5)\colvector{1 \\ 1 \\ 3 \\ 1}+
    (-6)\colvector{2 \\ 1 \\ 2 \\ -1}+
    (-3)\colvector{7 \\ 3 \\ 5 \\ -5}+
    (4)\colvector{1 \\ 1 \\ -1 \\ 2}+
    (2)\colvector{-1 \\ 0 \\ 9 \\ 0}
    =
    \colvector{-26\\-6\\2\\34}
  \]
  \[
    \vect{y}=
    (1)\colvector{1 \\ 1 \\ 3 \\ 1}+
    (0)\colvector{2 \\ 1 \\ 2 \\ -1}+
    (1)\colvector{7 \\ 3 \\ 5 \\ -5}+
    (0)\colvector{1 \\ 1 \\ -1 \\ 2}+
    (1)\colvector{-1 \\ 0 \\ 9 \\ 0}
    =
    \colvector{\answer{7}\\4\\17\\-4}
  \]
  \[
    \vect{z}=
    (0)\colvector{1 \\ 1 \\ 3 \\ 1}+
    (0)\colvector{2 \\ 1 \\ 2 \\ -1}+
    (0)\colvector{7 \\ 3 \\ 5 \\ -5}+
    (0)\colvector{1 \\ 1 \\ -1 \\ 2}+
    (0)\colvector{-1 \\ 0 \\ 9 \\ 0}
    =
    \colvector{\answer{0}\\0\\0\\0}
  \]
  The purpose of a set is to collect objects with some common
  property, and to exclude objects without that property.  So the most
  fundamental question about a set is if a given object is an element
  of the set or not.  Let us learn more about $\spn{S}$ by
  investigating which vectors are elements of the set, and which are
  not.

  \begin{question}
    First, is $\vect{u}=\colvector{-15\\-6\\19\\5}$ an element of
    $\spn{S}$?
    
    \begin{multipleChoice}
      \choice[correct]{Yes.}
      \choice{No.}
    \end{multipleChoice}
    
    \begin{feedback}[correct]
      We are asking if there are scalars
      $\alpha_1,\,\alpha_2,\,\alpha_3,\,\alpha_4,\,\alpha_5$ such that
      \[
        \alpha_1\colvector{1 \\ 1 \\ 3 \\ 1}+
        \alpha_2\colvector{2 \\ 1 \\ 2 \\ -1}+
        \alpha_3\colvector{7 \\ 3 \\ 5 \\ -5}+
        \alpha_4\colvector{1 \\ 1 \\ -1 \\ 2}+
        \alpha_5\colvector{-1 \\ 0 \\ 9 \\ 0}
        =\vect{u}
        =\colvector{-15\\-6\\19\\5}
      \]
      
      Applying \ref{theorem:SLSLC} we recognize the search for these scalars
      as a solution to a linear system of equations with augmented matrix
      \[
        \begin{bmatrix}
          1 & 2 & 7 & 1 & -1 & -15 \\
          1 & 1 & 3 & 1 & 0 & -6 \\
          3 & 2 & 5 & -1 & 9 & 19 \\
          1 & -1 & -5 & 2 & 0 & 5
        \end{bmatrix}
      \]
      which row-reduces to
      \[
        \begin{bmatrix}
          \leading{1} & 0 & -1 & 0 & 3 & 10 \\
          0 & \leading{1} & 4 & 0 & -1 & -9 \\
          0 & 0 & 0 & \leading{1} & -2 & -7 \\
          0 & 0 & 0 & 0 & 0 & 0
        \end{bmatrix}
      \]
      
      At this point, we see that the system is consistent
      (\ref{theorem:RCLS}), so we know there \textit{is} a solution for
      the five scalars
      $\alpha_1,\,\alpha_2,\,\alpha_3,\,\alpha_4,\,\alpha_5$.  This is
      enough evidence for us to say that $\vect{u}\in\spn{S}$.  If we
      wished further evidence, we could compute an actual solution, say
      \begin{align*}
        \alpha_1&=2
        &
          \alpha_2&=1
        &
          \alpha_3&=-2
        &
          \alpha_4&=-3
        &
          \alpha_5&=2
      \end{align*}
      
      This particular solution allows us to write
      \[
        (2)\colvector{1 \\ 1 \\ 3 \\ 1}+
        (1)\colvector{2 \\ 1 \\ 2 \\ -1}+
        (-2)\colvector{7 \\ 3 \\ 5 \\ -5}+
        (-3)\colvector{1 \\ 1 \\ -1 \\ 2}+
        (2)\colvector{-1 \\ 0 \\ 9 \\ 0}
        =\vect{u}
        =\colvector{-15\\-6\\19\\5}
      \]
      making it even more obvious that $\vect{u}\in\spn{S}$.
    \end{feedback}
  \end{question}
  
  \begin{question}
    Is $\vect{v}=\colvector{3\\1\\2\\-1}$ an element of $\spn{S}$?

    \begin{hint}
      We are asking if there are scalars $\alpha_1,\,\alpha_2,\,\alpha_3,\,\alpha_4,\,\alpha_5$ such that
      \[
        \alpha_1\colvector{1 \\ 1 \\ 3 \\ 1}+
        \alpha_2\colvector{2 \\ 1 \\ 2 \\ -1}+
        \alpha_3\colvector{7 \\ 3 \\ 5 \\ -5}+
        \alpha_4\colvector{1 \\ 1 \\ -1 \\ 2}+
        \alpha_5\colvector{-1 \\ 0 \\ 9 \\ 0}
        =\vect{v}
        =\colvector{3\\1\\2\\-1}
      \]
    \end{hint}
    
    \begin{multipleChoice}
      \choice{Yes.}
      \choice[correct]{No.}
    \end{multipleChoice}

    \begin{feedback}[correct]
      Applying \ref{theorem:SLSLC} we recognize the search for these scalars as a solution to a linear system of equations with augmented matrix
      \[
        \begin{bmatrix}
          1 & 2 & 7 & 1 & -1 & 3 \\
          1 & 1 & 3 & 1 & 0 & 1 \\
          3 & 2 & 5 & -1 & 9 & 2 \\
          1 & -1 & -5 & 2 & 0 & -1
        \end{bmatrix}
      \]
      which row-reduces to
      \[
        \begin{bmatrix}
          \leading{1} & 0 & -1 & 0 & 3 & 0 \\
          0 & \leading{1} & 4 & 0 & -1 & 0 \\
          0 & 0 & 0 & \leading{1} & -2 & 0 \\
          0 & 0 & 0 & 0 & 0 & \leading{1}
        \end{bmatrix}
      \]

      At this point, we see that the system is inconsistent by
      \ref{theorem:RCLS}, so we know there <em>is not</em> a solution
      for the five scalars
      $\alpha_1,\,\alpha_2,\,\alpha_3,\,\alpha_4,\,\alpha_5$.  This is
      enough evidence for us to say that $\vect{v}\not\in\spn{S}$.
      End of story.
    \end{feedback}
  \end{question}
\end{example}

\begin{example}[Span of columns]
  Begin with the finite set of three vectors of size $3$
  \[
    S=\{\vect{u}_1,\,\vect{u}_2,\,\vect{u}_3\}
    =\left\{
      \colvector{1\\2\\1},\,\colvector{-1\\1\\1},\,\colvector{2\\1\\0}
    \right\}
  \]
  and consider the infinite set $\spn{S}$.  The vectors of $S$ could
  have been chosen to be anything, but for reasons that will become
  clear later, we have chosen the three columns of the matrix
  \[
    \begin{bmatrix}
      1 & -1 & 2 \\
      2 & 1 & 1 \\
      1 & 1 & 0 
    \end{bmatrix}
  \]

  \begin{question}
    Is the vector
    \[
      \vect{v}=(5)\colvector{1\\2\\1}+(-3)\colvector{-1\\1\\1}+(7)\colvector{2\\1\\0}=
      \colvector{22\\14\\2}
    \]
    in $\spn{S}$?

    \begin{multipleChoice}
      \choice[correct]{Yes.}
      \choice{No.}
    \end{multipleChoice}

    \begin{feedback}[correct]
      Yes, $\vect{v} \in \spn{S}$ because $\vect{v}$ is a linear
      combination of $\vect{u}_1,\,\vect{u}_2,\,\vect{u}_3$.  There is
      nothing magical about the scalars
      $\alpha_1=5,\,\alpha_2=-3,\,\alpha_3=7$, they could have been
      chosen to be anything.  So repeat this part of the example
      yourself, using different values of
      $\alpha_1,\,\alpha_2,\,\alpha_3$.

      What happens if you choose all three scalars to be zero?  You
      may conclude that
      \begin{multipleChoice}
        \choice[correct]{$\vect{0} \in \spn{S}$.}
        \choice{$\vect{0} \not\in \spn{S}$.}
      \end{multipleChoice} 
    \end{feedback}
  \end{question}

  \begin{question}
    So we know how to quickly construct sample elements of the set
    $\spn{S}$.  A slightly different question arises when you are
    handed a vector of the correct size and asked if it is an element
    of $\spn{S}$.  For example, is $\vect{w}=\colvector{1\\8\\5}$ in
    $\spn{S}$?  More succinctly, $\vect{w}\in\spn{S}$?

    To answer this question, we will look for scalars
    $\alpha_1,\,\alpha_2,\,\alpha_3$ so that
    \begin{align*}
      \alpha_1\vect{u}_1+\alpha_2\vect{u}_2+\alpha_3\vect{u}_3&=\vect{w}
    \end{align*}
    By \ref{theorem:SLSLC}, solutions to this vector equation are
    solutions to the system of equations
    \begin{align*}
      \alpha_1-\alpha_2+2\alpha_3&=\answer{1}\\
      2\alpha_1+\alpha_2+\alpha_3&=answer{8}\\
      \alpha_1+\alpha_2&=\answer{5}
    \end{align*}
    Building the augmented matrix for this linear system, and row-reducing, gives
    \[
      \begin{bmatrix}
        \leading{1} & 0 & 1 & \answer{3}\\
        0 & \leading{1} & -1 & 2\\
        0 & 0 & 0 & 0
      \end{bmatrix}
    \]
    This system has \wordChoice{\choice[correct]{infinitely many
        solutions}\choice{finitely many solutions}} (there is a free
    variable in $x_3$), but all we need is one solution vector.  The
    solution
    \begin{align*}
      \alpha_1 &= 2
      &
        \alpha_2 &= 3
      &
        \alpha_3 &= 1
    \end{align*}
    tells us that
    \[
      (2)\vect{u}_1+(3)\vect{u}_2+(1)\vect{u}_3=\vect{w}
    \]
    so we may conclude
    \begin{multipleChoice}
      \choice[correct]{that $\vect{w} \in \spn{S}$.}
      \choice{that $\vect{w} \not\in \spn{S}$.}
    \end{multipleChoice} 
    
    Notice that there are an infinite number of ways to answer this
    question affirmatively.  We could choose a different solution,
    this time choosing the free variable to be zero,
    \begin{align*}
      \alpha_1 &= 3
      &
        \alpha_2 &= 2
      &
        \alpha_3 &= 0
    \end{align*}
    shows us that
    \[
      (3)\vect{u}_1+(2)\vect{u}_2+(0)\vect{u}_3=\vect{w}
    \]
    Verifying the arithmetic in this second solution will make it
    obvious that $\vect{w}$ is in this span.  And of course, we now
    realize that there are an infinite number of ways to realize
    $\vect{w}$ as element of $\spn{S}$.
  \end{question}

    
  \begin{question}
    Let us ask the same type of question again, but this time with
    $\vect{y}=\colvector{2\\4\\3}$, i.e., is $\vect{y}\in\spn{S}$?

    \begin{multipleChoice}
      \choice{Yes.}
      \choice[correct]{No.}
    \end{multipleChoice}

    \begin{feedback}[correct]
      So we will look for scalars $\alpha_1,\,\alpha_2,\,\alpha_3$ so that
      \begin{align*}
        \alpha_1\vect{u}_1+\alpha_2\vect{u}_2+\alpha_3\vect{u}_3&=\vect{y}\\
      \end{align*}
      By \ref{theorem:SLSLC} solutions to this vector equation are the solutions to the system of equations,
      \begin{align*}
        \alpha_1-\alpha_2+2\alpha_3&=2\\
        2\alpha_1+\alpha_2+\alpha_3&=4\\
        \alpha_1+\alpha_2&=3
      \end{align*}
      Building the augmented matrix for this linear system, and row-reducing, gives,
      \[
        \begin{bmatrix}
          \leading{1} & 0 & 1 & 0\\
          0 & \leading{1} & -1 & 0\\
          0 & 0 & 0 & \leading{1}
        \end{bmatrix}
      \]
      
      This system is inconsistent (there is a pivot column in the last
      column, \ref{theorem:RCLS}), so there are no scalars
      $\alpha_1,\,\alpha_2,\,\alpha_3$ that will create a linear
      combination of $\vect{u}_1,\,\vect{u}_2,\,\vect{u}_3$ that equals
      $\vect{y}$.  More precisely, $\vect{y}\not\in\spn{S}$.
    \end{feedback}
  \end{question}

  There are three things to observe in this example.  First, it is
  easy to construct vectors in $\spn{S}$.  Second, it is possible that
  some vectors are in $\spn{S}$ (e.g., $\vect{w}$), while others are
  not (e.g., $\vect{y}$).  Deciding if a given vector is in $\spn{S}$
  leads to solving a linear system of equations and asking if the
  system is consistent.

\end{example}

\begin{example}[Span of the columns, again]
  Begin with the finite set of three vectors of size $3$
  \[
    R=\{\vect{v}_1,\,\vect{v}_2,\,\vect{v}_3\}
    =\left\{
      \colvector{-7\\5\\1},\,\colvector{-6\\5\\0},\,\colvector{-12\\7\\4}
    \right\}
  \]
  that are the columns of the coefficient matrix
  \[
    \begin{bmatrix}
      -7&-6&- 12\\
      5&5&7\\
      1&0&4
    \end{bmatrix}.
  \]
  Let's consider the infinite set $\spn{R}$.

  \begin{question}
    Is the vector
    \[
      \vect{x}=(2)\colvector{-7\\5\\1}+(4)\colvector{-6\\5\\0}+(-3)\colvector{-12\\7\\4}=
      \colvector{-2\\9\\-10}
    \]
    in $\spn{R}$?

    \begin{multipleChoice}
      \choice[correct]{Yes, $\vect{x} \in \spn{R}$.}
      \choice{No, $\vect{x} \not\in \spn{R}$.}
    \end{multipleChoice}
    
    \begin{feedback}[correct]
      Note that $\vect{x}$ is a linear combination of
      $\vect{v}_1,\,\vect{v}_2,\,\vect{v}_3$ and therefore
      $\vect{x}\in\spn{R}$.  Try some different values of
      $\alpha_1,\,\alpha_2,\,\alpha_3$ yourself, and see what vectors
      you can create as elements of $\spn{R}$.

      \begin{question}
        Now ask if a given vector is an element of $\spn{R}$.  For
        example, is $\vect{z}=\colvector{-33\\24\\5}$ in $\spn{R}$?  Is
        $\vect{z}\in\spn{R}$?
        \begin{multipleChoice}
          \choice[correct]{Yes, $\vect{z} \in \spn{R}$.}
          \choice{No, $\vect{z} \not\in \spn{R}$.}
        \end{multipleChoice}

        \begin{feedback}[correct]
          We looked for scalars $\alpha_1,\,\alpha_2,\,\alpha_3$ so that
          \begin{align*}
            \alpha_1\vect{v}_1+\alpha_2\vect{v}_2+\alpha_3\vect{v}_3&=\vect{z}\\
          \end{align*}
          By \ref{theorem:SLSLC} solutions to this vector equation are the solutions to the system of equations,
          \begin{align*}
            -7\alpha_1-6\alpha_2-12\alpha_3&=-33\\
            5\alpha_1+5\alpha_2+7\alpha_3&=24\\
            \alpha_1+4\alpha_3&=5
          \end{align*}
          Building the augmented matrix for this linear system, and row-reducing, gives
          \[
            \begin{bmatrix}
              \leading{1}&0&0&-3\\
              0&\leading{1}&0&5\\
              0&0&\leading{1}&2
            \end{bmatrix}
          \]
          This system has a unique solution,
          \begin{align*}
            \alpha_1 = -3&&\alpha_2 = 5&&\alpha_3 = 2
          \end{align*}
          telling us that
          \[
            (-3)\vect{v}_1+(5)\vect{v}_2+(2)\vect{v}_3=\vect{z}
          \]
          so we are convinced that $\vect{z}$ really is in $\spn{R}$.
          Notice that in this case we have only one way to answer the
          question affirmatively since the solution is unique.
        \end{feedback}
      \end{question}
    \end{feedback}
  \end{question}

  \begin{question}
    Let us ask about another vector, say is
    $\vect{x}=\colvector{-7\\8\\-3}$ in $\spn{R}$?  Is
    $\vect{x}\in\spn{R}$?

    \begin{multipleChoice}
      \choice[correct]{Yes, $\vect{x} \in \spn{R}$.}
      \choice{No, $\vect{x} \not\in \spn{R}$.}
    \end{multipleChoice}

    \begin{feedback}[correct]
      We desire scalars $\alpha_1,\,\alpha_2,\,\alpha_3$ so that
      \begin{align*}
        \alpha_1\vect{v}_1+\alpha_2\vect{v}_2+\alpha_3\vect{v}_3&=\vect{x}\\
      \end{align*}
      By \ref{theorem:SLSLC} solutions to this vector equation are the solutions to the system of equations
      \begin{align*}
        -7\alpha_1-6\alpha_2-12\alpha_3&=-7\\
        5\alpha_1+5\alpha_2+7\alpha_3&=8\\
        \alpha_1+4\alpha_3&=-3
      \end{align*}
      Building the augmented matrix for this linear system, and row-reducing, gives
      \[
        \begin{bmatrix}
          \leading{1} & 0 & 0 & 1 \\
          0 & \leading{1} & 0 & 2 \\
          0 & 0 & \leading{1} & -1
        \end{bmatrix}
      \]
      This system has a unique solution,
      \begin{align*}
        \alpha_1 = 1&&\alpha_2 = 2&&\alpha_3 = -1
      \end{align*}
      telling us that
      \[
        (1)\vect{v}_1+(2)\vect{v}_2+(-1)\vect{v}_3=\vect{x}
      \]
      so we are convinced that $\vect{x}$ really is in $\spn{R}$.
      Notice that in this case we again have only one way to answer the
      question affirmatively since the solution is again unique.
    \end{feedback}
  \end{question}

  \begin{question}
    We could continue to test other vectors for membership in $\spn{R}$,
    but there is no point.

    A question about membership in $\spn{R}$ inevitably leads to a
    system of \wordChoice{\choice{one}\choice{two}\choice[correct]{three}} equations in \wordChoice{\choice{one}\choice{two}\choice[correct]{three}} variables
    $\alpha_1,\,\alpha_2,\,\alpha_3$ with a coefficient matrix whose
    columns are the vectors $\vect{v}_1,\,\vect{v}_2,\,\vect{v}_3$.

    This particular coefficient matrix is
    \wordChoice{\choice{singular}\choice[correct]{nonsingular}}, so by
    \ref{theorem:NMUS}, the system is guaranteed to have a solution.
    (This solution is unique, but that is not critical here.)  So
    \textit{no matter} which vector we might have chosen for
    $\vect{z}$, we would have been \textit{certain} to discover that
    it was an element of $\spn{R}$.  Stated differently, every vector
    of size 3 is in $\spn{R}$, or $\spn{R}=\complex{3}$.
  \end{question}
\end{example}

\end{document}
