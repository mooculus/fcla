\documentclass{ximera}

% These macros are automatically generated from the "macros"
% XML element.  Make permanent edits there.
%
% History
%   2004/01/01  Initiated for FCLA, evolved from there
%   2006/09/17  Converted  _, ^  to \sb, \sp for TeX4ht
%   2014/02/01  Updated for MathBook XML projects
%               Obsolete in FCLA: \codeindent, \computerfont, \define
%               Change: MathJax wants \lt, so replaced by \lteval
%   2014/02/22  New: \orderof, \reals, \per
%   2015/08/16  Incorporated into MathBook XML version of FCLA
%
%%%%%%%%%%%%%%%%%%%%%
%
%     Conveniences
%
%%%%%%%%%%%%%%%%%%%%%
%
%  Order of (asymptotically limit of fraction is 1)
%  Usage: \orderof{some function}
%
\newcommand{\orderof}[1]{\sim #1}
%
%  Integers
%  Usage:  \Z
\newcommand{\Z}{\mathbb{Z}}
%
%  Real numbers, as set of scalars
%  Usage:  \reals
\newcommand{\reals}{\mathbb{R}}
%
%  n-space over real field
%  Usage: \complex{integer-dimension}
\newcommand{\real}[1]{\mathbb{R}^{#1}}
%
%  Complex numbers, as set of scalars
%  Usage:  \complexes
\newcommand{\complexes}{\mathbb{C}}
%
%  n-space over complex field
%  Usage: \complex{integer-dimension}
\newcommand{\complex}[1]{\mathbb{C}^{#1}}
%
%  Complex conjugation (scalar, vector, matrix)
%  Usage: \conjugate{object}
\newcommand{\conjugate}[1]{\overline{#1}}
%
%  Complex number modulus
%  Usage: \modulus{a+bi}
%  Presumes math mode
\newcommand{\modulus}[1]{\left\lvert#1\right\rvert}
%
%  Zero vector
%  Usage: \zerovector
\newcommand{\zerovector}{\vect{0}}
%
%  Zero matrix
%  Usage: \zeromatrix, use a subscript when size is important
\newcommand{\zeromatrix}{\mathcal{O}}
%
%  Inner product (brackets, not quadratic form)
%  Usage: \innerproduct{a-vector}{a-vector}
\newcommand{\innerproduct}[2]{\left\langle#1,\,#2\right\rangle}
%
%  Norm of a vector
%  Usage: \norm{a-vector}
\newcommand{\norm}[1]{\left\lVert#1\right\rVert}
%
%  Dimension
%  Usage: \dimension{vector-space-letter}
\newcommand{\dimension}[1]{\dim\left(#1\right)}
%
%  Nullity
%  Usage: \nullity{matrix-or-lintrans-letter}
\newcommand{\nullity}[1]{n\left(#1\right)}
%
%  Rank
%  Usage: \rank{matrix-or-lintrans-letter}
\newcommand{\rank}[1]{r\left(#1\right)}
%
%  Direct sum
%  Usage: \ds between a couple of subspaces
%
\newcommand{\ds}{\oplus}
%
%  Determinant of a matrix (functional)
%  Usage: \detname{A}
\newcommand{\detname}[1]{\det\left(#1\right)}
%
%  Determinant of a matrix (vertical bars)
%  Usage: \detbars{A}
\newcommand{\detbars}[1]{\left\lvert#1\right\rvert}
%
%  Trace of a Matrix
%  Usage: \trace{matrix name}
\newcommand{\trace}[1]{t\left(#1\right)}
%
%  Square Root of a Matrix
%  Usage: \sr{a-matrix}
\newcommand{\sr}[1]{#1^{1/2}}
%
%%%%%%%%%%%%%%%%%%%%%
%
%     Subspace Constructions
%
%%%%%%%%%%%%%%%%%%%%%
%
%  Span of a set of vectors
%  \span and \sp are used by TeX for other things
%  Usage: \spn{set-of-vectors}
\newcommand{\spn}[1]{\left\langle#1\right\rangle}
%
%  Null space of a matrix
%  Usage:  \nsp{A}
\newcommand{\nsp}[1]{\mathcal{N}\!\left(#1\right)}
%
%  Column space of a matrix
%  Usage:  \csp{A}
\newcommand{\csp}[1]{\mathcal{C}\!\left(#1\right)}
%
%  Row space of a matrix
%  Usage:  \rsp{A}
\newcommand{\rsp}[1]{\mathcal{R}\!\left(#1\right)}
%
%  Left null space of a matrix
%  Usage:  \lns{A}
\newcommand{\lns}[1]{\mathcal{L}\!\left(#1\right)}
%
%  Orthogonal complement of a vector space
%  Avoiding TeX's \perp
%  Usage:  \per{A}
\newcommand{\per}[1]{#1^\perp}
%
%%%%%%%%%%%%%%%%%%%%%
%
%     Systems of Equations
%
%%%%%%%%%%%%%%%%%%%%%
%
%  In-line form of an augmented matrix for a system of equations
%  Usage: \augmented{coefficient-matrix}{constant-vector}
\newcommand{\augmented}[2]{\left\lbrack\left.#1\,\right\rvert\,#2\right\rbrack}
%
%  Notation for a linear system before introducing matrix multiplication
%  Usage: \linearsystem{coefficient-matrix}{constant-vector}
\newcommand{\linearsystem}[2]{\mathcal{LS}\!\left(#1,\,#2\right)}
%
%  Notation for a homogenous system before introducing matrix multiplication
%  Usage: \homosystem{coefficient-matrix}
\newcommand{\homosystem}[1]{\linearsystem{#1}{\zerovector}}
%
%%%%%%%%%%%%%%%%%%%%%
%
%     Row Operations, Echelon Form
%
%%%%%%%%%%%%%%%%%%%%%
%
% Row operations on matrices
%
% Three commands to shorten up descriptions of gaussian elimination
%
% Usage: \rowopswap{row-i}{row-j}
% Usage: \rowopmult{scalar}{row-i}
% Usage: \rowopadd{scalar}{row-multiplied}{row-added-to}
\newcommand{\rowopswap}[2]{R_{#1}\leftrightarrow R_{#2}}
\newcommand{\rowopmult}[2]{#1R_{#2}}
\newcommand{\rowopadd}[3]{#1R_{#2}+R_{#3}}
%
% Mark leading 1's in echelon form with fbox
% Usage: \leading{a-1-usually}
\newcommand{\leading}[1]{\fbox{#1}}
%
%  Row-reduce arrow
%  Usage:  \rref inbetween a matrix and its reduced row-echelon form
\newcommand{\rref}{\xrightarrow{\text{RREF}}}
%
%  Elementary Matrices
%  Usage: \elemswap{subscript}{subscript}
%  Usage: \elemmult{scalar}{subscript}
%  Usage: \elemadd{scalar}{subscript-mult}{subscript-target}
%
\newcommand{\elemswap}[2]{E_{#1,#2}}
\newcommand{\elemmult}[2]{E_{#2}\left(#1\right)}
\newcommand{\elemadd}[3]{E_{#2,#3}\left(#1\right)}
%
%%%%%%%%%%%%%%%%%%%%%
%
%     2-D Constructions (Lists, Vectors, Matrices)
%
%%%%%%%%%%%%%%%%%%%%%
%
%  A list of scalars of generic length
%  Usage:  \scalarlist{scalar letter}{terminal subscript}
\newcommand{\scalarlist}[2]{{#1}_{1},\,{#1}_{2},\,{#1}_{3},\,\ldots,\,{#1}_{#2}}
%
%  Vector styling, bold (or use wiggles, arrows, whatever)
%  Subscripts go outside this construction
%  Usage: \vect{a symbol to use as a vector}
%  Have to already be in math mode
%
\newcommand{\vect}[1]{\mathbf{#1}}
%
%  A column vector
%  Usage: \colvector{list-delimited-by-\\}
%
\newcommand{\colvector}[1]{\begin{bmatrix}#1\end{bmatrix}}
%
%  A generic vector with components
%  Usage: \vectorcomponents{component-letter}{final-subscript}
\newcommand{\vectorcomponents}[2]{\colvector{#1_{1}\\#1_{2}\\#1_{3}\\\vdots\\#1_{#2}}}
%
%  A list of vectors of generic length
%  Usage:  \vectorlist{vector letter}{terminal subscript}
\newcommand{\vectorlist}[2]{\vect{#1}_{1},\,\vect{#1}_{2},\,\vect{#1}_{3},\,\ldots,\,\vect{#1}_{#2}}
%
%  Vector entries, entry i of vector v
%  (vector-expession still needs \vect, etc.)
%  Usage:  \vectorentry{vector-expression}{single-subscript}
\newcommand{\vectorentry}[2]{\left\lbrack#1\right\rbrack_{#2}}
%
%  Matrix entries, entry i,j of matrix A
%  Usage:  \matrixentry{matrix-expression}{paired-subscripts}
%
\newcommand{\matrixentry}[2]{\left\lbrack#1\right\rbrack_{#2}}
%
%  A generic linear combination
%  Usage:  \lincombo{scalar letter}{vector letter}{terminal subscript}
\newcommand{\lincombo}[3]{#1_{1}\vect{#2}_{1}+#1_{2}\vect{#2}_{2}+#1_{3}\vect{#2}_{3}+\cdots +#1_{#3}\vect{#2}_{#3}}
%
%  Matrix, column by column, as vectors
%  Usage:  \matrixcolumns{matrix letter}{terminal subscript}
\newcommand{\matrixcolumns}[2]{\left\lbrack\vect{#1}_{1}|\vect{#1}_{2}|\vect{#1}_{3}|\ldots|\vect{#1}_{#2}\right\rbrack}
%
%%%%%%%%%%%%%%%%%%%%%
%
%     Special Matrices
%
%%%%%%%%%%%%%%%%%%%%%
%
%  Transpose of a matrix
%  Usage:  \transpose{A}
\newcommand{\transpose}[1]{#1^{t}}
%
%  Inverse of a matrix
%  Usage:  \inverse{A}
\newcommand{\inverse}[1]{#1^{-1}}
%
%  Submatrix (for minors, determinants)
%  Usage: \submatrix{matrix-name}{delete-row}{delete-col}
\newcommand{\submatrix}[3]{#1\left(#2|#3\right)}
%
%  Adjoint of a matrix (twice)
%  This macro is a convenience to call \transpose and \conjugate properly
%  It shouldn't need to be modified (or mathematical meanings will change)
%  Usage:  \adj{A}
\newcommand{\adj}[1]{\transpose{\left(\conjugate{#1}\right)}}
%
%  This macro controls the symbol used for the adjoint
%  It can be edited to taste
%  Usage:  \adjoint{A}
\newcommand{\adjoint}[1]{#1^\ast}
%
%%%%%%%%%%%%%%%%%%%%%
%
%     Sets
%
%%%%%%%%%%%%%%%%%%%%%
%
%  A convenience for simple sets
%  Usage:  \set{list of element}
\newcommand{\set}[1]{\left\{#1\right\}}
%
%  Sets with vertical bar, "such that", sized for objects, not condition
%  Usage:  \setparts{objects}{condition}
%
%%\newcommand{\setparts}[2]{\left\{ #1\mid#2\right\}}
%%\newcommand{\setparts}[2]{\left\{\left. #1\right\rvert#2\right\}}
\newcommand{\setparts}[2]{\left\lbrace#1\,\middle|\,#2\right\rbrace}
%
%  Set Cardinality
%  Usage:  \card{a-set-letter}
\newcommand{\card}[1]{\left\lvert#1\right\rvert}
%
%  Set Union
%  Use \cup
%
%  Set Intersection
%  Use \cap
%
%  Set Complement
%  Usage:  \setcomplement{a-set-letter}
\newcommand{\setcomplement}[1]{\overline{#1}}
%
%%%%%%%%%%%%%%%%%%%%%
%
%     Eigenvalues and Eigenspaces
%
%%%%%%%%%%%%%%%%%%%%%
%
%  Characteristic polynomial
%  Usage: \charpoly{matrix-letter}{variable-letter}
\newcommand{\charpoly}[2]{p_{#1}\left(#2\right)}
%
%  Eigenspace
%  Usage: \eigenspace{matrix-letter}{eigenvalue-letter}
\newcommand{\eigenspace}[2]{\mathcal{E}_{#1}\left(#2\right)}
%
%  2013/10/03 Including ampersands is problematic here, 
%  think about fixes later
%  2014/02/22 Limited testing, seems &amp; is fine for HTML and LaTeX
%  2016-07-20 only employed in Archetypes, MBX has gather/align override
%  Eigensystem (presumes wrapped in an mrow within md)
%  Usage: \eigensystem{matrixletter}{eigenvalue}{list of basis vectors}
\newcommand{\eigensystem}[3]{\lambda&amp;=#2&amp;\eigenspace{#1}{#2}&amp;=\spn{\set{#3}}} 
%
%  Generalized Eigenspace
%  Usage: \geneigenspace{lin-trans-letter}{eigenvalue-letter}
\newcommand{\geneigenspace}[2]{\mathcal{G}_{#1}\left(#2\right)}
%
%  Algebraic multiplicty
%  Usage: \algmult{matrix-letter}{eigenvalue-letter}
\newcommand{\algmult}[2]{\alpha_{#1}\left(#2\right)}
%
%  Geometric multiplicty
%  Usage: \geomult{matrix-letter}{eigenvalue-letter}
\newcommand{\geomult}[2]{\gamma_{#1}\left(#2\right)}
%
%  Index (of eigenvalue)
%  Usage: \indx{matrix-letter}{eigenvalue-letter}
\newcommand{\indx}[2]{\iota_{#1}\left(#2\right)}
%
%%%%%%%%%%%%%%%%%%%%%
%
%     Linear Transformations
%
%%%%%%%%%%%%%%%%%%%%%
%
%  MathJax defines \lt to ease XML confusion
%
%  Linear transformation definition
%  Usage: \ltdefn{name-letter}{domain}{range}
\newcommand{\ltdefn}[3]{#1\colon #2\rightarrow#3}
%
%  Linear transformation evaluation
%  Usage: \lteval{name-letter}{input}
%  Replaces old \lt desired by MathJax
\newcommand{\lteval}[2]{#1\left(#2\right)}
%
% Linear transformation inverse
%  Usage: \ltinverse{name-letter}
\newcommand{\ltinverse}[1]{#1^{-1}}
%
%  Linear transformation restriction
%  Usage: \restrict{name-letter}{subspace-letter}
\newcommand{\restrict}[2]{{#1}|_{#2}}
%
%  Linear transformation preimage
%  Usage: \preimage{name-letter}{codomain-element}
\newcommand{\preimage}[2]{#1^{-1}\left(#2\right)}
%
%  Range of a linear transformation
%  TeX uses \range for something else
%  Usage:  \rng{T}
\newcommand{\rng}[1]{\mathcal{R}\!\left(#1\right)}
%
%  Kernel of a linear transformation
%  TeX uses \ker to do something different
%  Usage:  \krn{T}
\newcommand{\krn}[1]{\mathcal{K}\!\left(#1\right)}
%
%  Linear transformation composition
%  Usage: \compose{function-name}{function-name}
\newcommand{\compose}[2]{{#1}\circ{#2}}
%
%  Vector space of linear transformations
%  Usage: \vslt{domains}{codomains}
%  Presumes math mode
\newcommand{\vslt}[2]{\mathcal{LT}\left(#1,\,#2\right)}
%
%%%%%%%%%%%%%%%%%%%%%
%
%     Vector and Matrix Representations
%
%%%%%%%%%%%%%%%%%%%%%
%
%  Isomorphism symbol
%  Usage: \isomorphic
\newcommand{\isomorphic}{\cong}
%
%  Similarity
%  Usage: \similar{inner-matrix}{outer-invertible-matrix}
%  Rearranging this will not "fix" all desired changes throughout
%
\newcommand{\similar}[2]{\inverse{#2}#1#2}
%
%  Vector representation function name
%  Usage: \vectrepname{basis-letter}
\newcommand{\vectrepname}[1]{\rho_{#1}}
%
%  Vector representation output
%  Usage: \vectrep{basis-letter}{input}
\newcommand{\vectrep}[2]{\lteval{\vectrepname{#1}}{#2}}
%
%  Vector representation inverse function name
%  (Added later, not used consistently in FCLA)
%  Usage: \vectrepinvname{basis-letter}
\newcommand{\vectrepinvname}[1]{\ltinverse{\vectrepname{#1}}}
%
%  Vector representation inverse output
%  Usage: \vectrepinv{basis-letter}{input}
\newcommand{\vectrepinv}[2]{\lteval{\ltinverse{\vectrepname{#1}}}{#2}}
%
%  Matrix representation
%  Usage: \matrixrep{transformation-letter}{domain-basis-letter}{codomain-basis-letter}
\newcommand{\matrixrep}[3]{M^{#1}_{#2,#3}}
%
%  Matrix representation column-by-colum
%  2016-07-20 only employed once?
%  Usage: \matrixrepcolumns{transformation-letter}{codomain-basis-letter}{codomain-basis-vector-letter}{final-index}
\newcommand{\matrixrepcolumns}[4]{\left\lbrack \left.\vectrep{#2}{\lteval{#1}{\vect{#3}_{1}}}\right|\left.\vectrep{#2}{\lteval{#1}{\vect{#3}_{2}}}\right|\left.\vectrep{#2}{\lteval{#1}{\vect{#3}_{3}}}\right|\ldots\left|\vectrep{#2}{\lteval{#1}{\vect{#3}_{#4}}}\right.\right\rbrack}
%
%  Change of basis matrix
%  Usage: \cbm{domain-basis-letter}{codomain-basis-letter}
\newcommand{\cbm}[2]{C_{#1,#2}}
%
%%%%%%%%%%%%%%%%%%%%%
%
%     Canonical Forms
%
%%%%%%%%%%%%%%%%%%%%%
%
%  Jordan blocks
%  Usage: \jordan{size}{diagonal-element}
\newcommand{\jordan}[2]{J_{#1}\left(#2\right)}
%
%%%%%%%%%%%%%%%%%%%%%
%
%     Hadamard Matrices
%     Contributed by Elizabeth Million
%
%%%%%%%%%%%%%%%%%%%%%
%
%  Hadamard Product
%  Usage: \hadamard{a-matrix}{a-matrix}
\newcommand{\hadamard}[2]{#1\circ #2}
%
%  Hadamard identity matrix
%  Usage: \hadamardidentity{paired-subscripts-size-of-matrix}
\newcommand{\hadamardidentity}[1]{J_{#1}}
%
%  Hadamard inverse matrix
%  Usage: \hadamardinverse{matrix-expression}
\newcommand{\hadamardinverse}[1]{\widehat{#1}}


\title{Spanning Sets of Null Spaces}

\begin{document}
\begin{abstract}
  The null space of a matrix can be expressed as a linear combination,
  where the scalars are the free variables.
\end{abstract}
\maketitle

When a system of equations is homogeneous the solution set can be
expressed in the form described by \ref{theorem:VFSLS} where the
vector $\vect{c}$ is the zero vector.  We can essentially ignore this
vector, so that the remainder of the typical expression for a solution
looks like an arbitrary linear combination, where the scalars are the
free variables and the vectors are $\vectorlist{u}{n-r}$.  Which
sounds a lot like a span.  This is the substance of the next theorem.

\begin{theorem}[Spanning Sets for Null Spaces]
  \label{theorem:SSNS}Suppose that $A$ is an $m\times n$ matrix, and $B$ is a row-equivalent
  matrix in reduced row-echelon form.  Suppose that $B$ has $r$ pivot
  columns, with indices given by
  $D=\set{d_1,\,d_2,\,d_3,\,\ldots,\,d_r}$, while the $n-r$ non-pivot
  columns have indices
  $F=\set{f_1,\,f_2,\,f_3,\,\ldots,\,f_{n-r},\,n+1}$.  Construct the
  $n-r$ vectors $\vect{z}_j$, $1\leq j\leq n-r$ of size $n$,
  \[
    \vectorentry{\vect{z}_j}{i}=
    \begin{cases}
      1&\text{if $i\in F$, $i=f_j$}\\
      0&\text{if $i\in F$, $i\neq f_j$}\\
      -\matrixentry{B}{k,f_j}&\text{if $i\in D$, $i=d_k$}
    \end{cases}
  \]
  Then the null space of $A$ is given by
  \[
    \nsp{A}=\spn{\left\{\vectorlist{z}{n-r}\right\}}
  \]

  \begin{proof}
    Consider the homogeneous system with $A$ as a coefficient matrix,
    $\homosystem{A}$.  Its set of solutions, $S$, is by
    \ref{definition:NSM}, the null space of $A$, $\nsp{A}$.  Let
    $B^{\prime}$ denote the result of row-reducing the augmented
    matrix of this homogeneous system.  Since the system is
    homogeneous, the final column of the augmented matrix will be all
    zeros, and after any number of row operations
    (\ref{definition:RO}), the column will still be all zeros.  So
    $B^{\prime}$ has a final column that is totally zeros.

    Now apply \ref{theorem:VFSLS} to $B^{\prime}$, after noting that
    our homogeneous system must be consistent (\ref{theorem:HSC}).
    The vector $\vect{c}$ has zeros for each entry that has an index
    in $F$.  For entries with their index in $D$, the value is
    $-\matrixentry{B^{\prime}}{k,{n+1}}$, but for $B^{\prime}$ any
    entry in the final column (index $n+1$) is zero.  So
    $\vect{c}=\zerovector$.  The vectors $\vect{z}_j$,
    $1\leq j\leq n-r$ are identical to the vectors $\vect{u}_j$,
    $1\leq j\leq n-r$ described in \ref{theorem:VFSLS}.  Putting it
    all together and applying \ref{definition:SSCV} in the final step,
    \begin{align*}
      \nsp{A}
      &=S\\
      &=\setparts{
        \vect{c}+\alpha_1\vect{u}_1+\alpha_2\vect{u}_2+\alpha_3\vect{u}_3+\cdots+\alpha_{n-r}\vect{u}_{n-r}
        }{
        \alpha_1,\,\alpha_2,\,\alpha_3,\,\ldots,\,\alpha_{n-r}\in\complexes
        }\\
      &=\setparts{
        \alpha_1\vect{u}_1+\alpha_2\vect{u}_2+\alpha_3\vect{u}_3+\cdots+\alpha_{n-r}\vect{u}_{n-r}
        }{
\alpha_1,\,\alpha_2,\,\alpha_3,\,\ldots,\,\alpha_{n-r}\in\complexes
        }\\
      &=\spn{\set{\vectorlist{z}{n-r}}}
    \end{align*}
  \end{proof}
\end{theorem}

Notice that the hypotheses of \ref{theorem:VFSLS} and
\ref{theorem:SSNS} are slightly different.  In the former, $B$ is the
row-reduced version of an augmented matrix of a linear system, while
in the latter, $B$ is the row-reduced version of an arbitrary matrix.
Understanding this subtlety now will avoid confusion later.

\begin{example}[Spanning set of a null space]
  Find a set of vectors, $S$, so that the null space of the matrix $A$
  below is the span of $S$, that is, $\spn{S}=\nsp{A}$.
  \[
    A=
    \begin{bmatrix}
      1 & 3 & 3 & -1 & -5\\
      2 & 5 & 7 & 1 & 1\\
      1 & 1 & 5 & 1 & 5\\
      -1 & -4 & -2 & 0 & 4
    \end{bmatrix}
  \]

  The null space of $A$ is the set of all solutions to the homogeneous
  system $\homosystem{A}$.  If we find the vector form of the
  solutions to this homogeneous system (\ref{theorem:VFSLS}) then the
  vectors $\vect{u}_j$, $1\leq j\leq n-r$ in the linear combination
  are exactly the vectors $\vect{z}_j$, $1\leq j\leq n-r$ described in
  \ref{theorem:SSNS}.  So we can mimic \ref{example:VFSAL} to arrive
  at these vectors (rather than just referring to the formulas in the
  statement of the theorem).

  Begin by row-reducing $A$.  The result is
  \[
    \begin{bmatrix}
      \leading{1} & 0 & 6 & 0 & 4\\
      0 & \leading{1} & -1 & 0 & -2\\
      0 & 0 & 0 & \leading{1} & 3\\
      0 & 0 & 0 & 0 & 0
    \end{bmatrix}
  \]

  With $D=\set{1,\,2,\,4}$ and $F=\set{3,\,5}$ we recognize that $x_3$
  and $x_5$ are free variables and we can interpret each nonzero row
  as an expression for the \wordChoice{\choice[correct]{dependent}\choice{free}} variables $x_1$, $x_2$, $x_4$
  (respectively) in the \wordChoice{\choice{dependent}\choice[correct]{free}} variables $x_3$ and $x_5$.  With this we
  can write the vector form of a solution vector as
  \[
    \colvector{x_1\\x_2\\x_3\\x_4\\x_5}=
    \colvector{-6x_3-4x_5\\x_3+2x_5\\x_3\\-3x_5\\x_5}=
    x_3\colvector{-6\\1\\1\\0\\0}+
    x_5\colvector{-4\\2\\0\\-3\\1}
  \]

  Then in the notation of \ref{theorem:SSNS},
  \begin{align*}
    \vect{z}_1&=\colvector{-6\\1\\1\\0\\0}&
    \vect{z}_2&=\colvector{-4\\\answer{2}\\0\\-3\\1}
  \end{align*}
  and
  \[
    \nsp{A}
    =\spn{\set{\vect{z}_1,\,\vect{z}_2}}
    =\spn{\set{\colvector{-6\\1\\1\\0\\0},\,\colvector{-4\\2\\0\\-3\\1}}}
  \]
\end{example}

\begin{example}[Null space directly as a span]
  Let us express the null space of $A$ as the span of a set of
  vectors, applying \ref{theorem:SSNS} as economically as possible,
  without reference to the underlying homogeneous system of equations
  (in contrast to \ref{example:SSNS}).
  \[
    A=
    \begin{bmatrix}
      2 & 1 & 5 & 1 & 5 & 1 \\
      1 & 1 & 3 & 1 & 6 & -1 \\
      -1 & 1 & -1 & 0 & 4 & -3 \\
      -3 & 2 & -4 & -4 & -7 & 0 \\
      3 & -1 & 5 & 2 & 2 & 3
    \end{bmatrix}
  \]
  Then \ref{theorem:SSNS} creates vectors for the span by first
  row-reducing the matrix in question.  The row-reduced version of $A$
  is
  \[
    B=
    \begin{bmatrix}
      \leading{1} & 0 & 2 & 0 & -1 & 2 \\
      0 & \leading{1} & 1 & 0 & 3 & -1 \\
      0 & 0 & 0 & \leading{1} & 4 & -2 \\
      0 & 0 & 0 & 0 & 0 & 0 \\
      0 & 0 & 0 & 0 & 0 & 0
    \end{bmatrix}
  \]
  We will mechanically follow the prescription of \ref{theorem:SSNS}.
  Here we go, in two big steps.

  First, the non-pivot columns have indices $F=\set{3,\,5,\,6}$, so we
  will construct the $n-r=6-3=3$ vectors with a pattern of zeros and
  ones dictated by the indices in $F$.  This is the realization of the
  first two lines of the three-case definition of the vectors
  $\vect{z}_j$, $1\leq j\leq n-r$.
  \begin{align*}
    \vect{z}_1&=\colvector{\\ \\ 1\\ \\0 \\ 0}
              &
                \vect{z}_2&=\colvector{\\ \\ 0\\ \\ 1\\ 0}
              &
                \vect{z}_3&=\colvector{\\ \\ 0\\ \\ 0\\ 1}
  \end{align*}

  Each of these vectors arises due to the presence of a column that is
  not a pivot column.  The remaining entries of each vector are the
  entries of the non-pivot column, negated, and distributed into the
  empty slots in order (these slots have indices in the set $D$, so
  also refer to pivot columns).  This is the realization of the third
  line of the three-case definition of the vectors $\vect{z}_j$,
  $1\leq j\leq n-r$.
  \begin{align*}
    \vect{z}_1&=\colvector{-2\\ -1\\ 1\\  0 \\0 \\ 0}
              &
                \vect{z}_2&=\colvector{1\\ -3\\ 0\\ -4\\ 1\\ 0}
              &
                \vect{z}_3&=\colvector{-2\\ 1\\ 0\\ 2\\ 0\\ 1}
  \end{align*}

  So, by \ref{theorem:SSNS}, we have
  \[
    \nsp{A}
    =
    \spn{\set{\vect{z}_1,\,\vect{z}_2,\,\vect{z}_3}}
    =
    \spn{\set{
        \colvector{-2\\ -1\\ 1\\  0 \\0 \\ 0},\,
        \colvector{1\\ -3\\ 0\\ -4\\ 1\\ 0},\,
        \colvector{-2\\ 1\\ 0\\ 2\\ 0\\ 1}
      }}
  \]

  We know that the null space of $A$ is the solution set of the
  homogeneous system $\homosystem{A}$, but nowhere in this application
  of \ref{theorem:SSNS} have we found occasion to reference the
  variables or equations of this system.  These details are all buried
  in the proof of \ref{theorem:SSNS}.
\end{example}

Here is an example that will simultaneously exercise the span
construction and \ref{theorem:SSNS}, while also pointing the way to
the next section.

\begin{example}[Span of the columns]
  Begin with the set of four vectors of size $3$
  \[
    T=\set{\vect{w}_1,\,\vect{w}_2,\,\vect{w}_3,\,\vect{w}_4}
    =\set{
      \colvector{2\\-3\\1},\,
      \colvector{1\\4\\1},\,
      \colvector{7\\-5\\4},\,
      \colvector{-7\\-6\\-5}
    }
  \]
  and consider the infinite set $W=\spn{T}$.  The vectors of $T$ have
  been chosen as the four columns of the matrix
  \[
    D = \begin{bmatrix}
      2 & 1 & 7 & -7 \\
      -3 & 4 &  -5 & -6 \\
      1 & 1 & 4 &  -5 
    \end{bmatrix}.
  \]
  Check that the vector
  \[
    \vect{z}_2=\colvector{2\\3\\0\\1}
  \]
  is a solution to the homogeneous system $\homosystem{D}$ (it is the
  vector $\vect{z}_2$ provided by the description of the null space of
  the coefficient matrix $D$ from \ref{theorem:SSNS}).

  Applying \ref{theorem:SLSLC}, we can write the linear combination,
  \[
    2\vect{w}_1+3\vect{w}_2+0\vect{w}_3+1\vect{w}_4=\zerovector
  \]
  which we can solve for $\vect{w}_4$,
  \[
    \vect{w}_4=(-2)\vect{w}_1+(-3)\vect{w}_2.
  \]
  This equation says that whenever we encounter the vector
  $\vect{w}_4$, we can replace it with a specific linear combination
  of the vectors $\vect{w}_1$ and $\vect{w}_2$.  So using $\vect{w}_4$
  in the set $T$, along with $\vect{w}_1$ and $\vect{w}_2$, is
  excessive.  An example of what we mean here can be illustrated by
  the computation,
  \begin{align*}
    5\vect{w}_1&+(-4)\vect{w}_2+6\vect{w}_3+(-3)\vect{w}_4\\
               &=5\vect{w}_1+(-4)\vect{w}_2+6\vect{w}_3+(-3)\left((-2)\vect{w}_1+(-3)\vect{w}_2\right)\\
               &=5\vect{w}_1+(-4)\vect{w}_2+6\vect{w}_3+\left(6\vect{w}_1+9\vect{w}_2\right)\\
               &=11\vect{w}_1+5\vect{w}_2+6\vect{w}_3
  \end{align*}
  
  So what began as a linear combination of the vectors
  $\vect{w}_1,\,\vect{w}_2,\,\vect{w}_3,\,\vect{w}_4$ has been reduced
  to a linear combination of the vectors
  $\vect{w}_1,\,\vect{w}_2,\,\vect{w}_3$.  A careful proof using our
  definition of set equality (\ref{definition:SE}) would now allow us
  to conclude that this reduction is possible for any vector in $W$,
  so
  \[
    W=\spn{\left\{\vect{w}_1,\,\vect{w}_2,\,\vect{w}_3\right\}}
  \]
  So the span of our set of vectors, $W$, has not changed, but we have
  \textit{described} it by the span of a set of \textit{three}
  vectors, rather than \textit{four}.  Furthermore, we can achieve yet
  another, similar, reduction.

  Check that the vector
  \[
    \vect{z}_1=\colvector{-3\\-1\\1\\0}
  \]
  is a solution to the homogeneous system
  $\linearsystem{D}{\zerovector}$ (it is the vector $\vect{z}_1$
  provided by the description of the null space of the coefficient
  matrix $D$ from \ref{theorem:SSNS}).  Applying \ref{theorem:SLSLC},
  we can write the linear combination,
  \[
    (-3)\vect{w}_1+(-1)\vect{w}_2+1\vect{w}_3=\zerovector
  \]
  which we can solve for $\vect{w}_3$,
  \[
    \vect{w}_3=3\vect{w}_1+1\vect{w}_2
  \]
  This equation says that whenever we encounter the vector
  $\vect{w}_3$, we can replace it with a specific linear combination
  of the vectors $\vect{w}_1$ and $\vect{w}_2$.  So, as before, the
  vector $\vect{w}_3$ is not needed in the description of $W$,
  provided we have $\vect{w}_1$ and $\vect{w}_2$ available.  In
  particular, a careful proof would show that
  \[
    W=\spn{\left\{\vect{w}_1,\,\vect{w}_2\right\}}
  \]
  So $W$ began life as the span of a set of four vectors, and we have
  now shown (utilizing solutions to a homogeneous system) that $W$ can
  also be described as the span of a set of just two vectors.
  Convince yourself that we cannot go any further.  In other words, it
  is not possible to dismiss either $\vect{w}_1$ or $\vect{w}_2$ in a
  similar fashion and winnow the set down to just one vector.

  What was it about the original set of four vectors that allowed us
  to declare certain vectors as surplus?  And just which vectors were
  we able to dismiss?  And why did we have to stop once we had two
  vectors remaining?  The answers to these questions motivate ``linear
  independence,'' our next big topic, and so are worth considering
  carefully \textit{now}.
\end{example}

\end{document}
