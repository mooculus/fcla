\documentclass{ximera}

% These macros are automatically generated from the "macros"
% XML element.  Make permanent edits there.
%
% History
%   2004/01/01  Initiated for FCLA, evolved from there
%   2006/09/17  Converted  _, ^  to \sb, \sp for TeX4ht
%   2014/02/01  Updated for MathBook XML projects
%               Obsolete in FCLA: \codeindent, \computerfont, \define
%               Change: MathJax wants \lt, so replaced by \lteval
%   2014/02/22  New: \orderof, \reals, \per
%   2015/08/16  Incorporated into MathBook XML version of FCLA
%
%%%%%%%%%%%%%%%%%%%%%
%
%     Conveniences
%
%%%%%%%%%%%%%%%%%%%%%
%
%  Order of (asymptotically limit of fraction is 1)
%  Usage: \orderof{some function}
%
\newcommand{\orderof}[1]{\sim #1}
%
%  Integers
%  Usage:  \Z
\newcommand{\Z}{\mathbb{Z}}
%
%  Real numbers, as set of scalars
%  Usage:  \reals
\newcommand{\reals}{\mathbb{R}}
%
%  n-space over real field
%  Usage: \complex{integer-dimension}
\newcommand{\real}[1]{\mathbb{R}^{#1}}
%
%  Complex numbers, as set of scalars
%  Usage:  \complexes
\newcommand{\complexes}{\mathbb{C}}
%
%  n-space over complex field
%  Usage: \complex{integer-dimension}
\newcommand{\complex}[1]{\mathbb{C}^{#1}}
%
%  Complex conjugation (scalar, vector, matrix)
%  Usage: \conjugate{object}
\newcommand{\conjugate}[1]{\overline{#1}}
%
%  Complex number modulus
%  Usage: \modulus{a+bi}
%  Presumes math mode
\newcommand{\modulus}[1]{\left\lvert#1\right\rvert}
%
%  Zero vector
%  Usage: \zerovector
\newcommand{\zerovector}{\vect{0}}
%
%  Zero matrix
%  Usage: \zeromatrix, use a subscript when size is important
\newcommand{\zeromatrix}{\mathcal{O}}
%
%  Inner product (brackets, not quadratic form)
%  Usage: \innerproduct{a-vector}{a-vector}
\newcommand{\innerproduct}[2]{\left\langle#1,\,#2\right\rangle}
%
%  Norm of a vector
%  Usage: \norm{a-vector}
\newcommand{\norm}[1]{\left\lVert#1\right\rVert}
%
%  Dimension
%  Usage: \dimension{vector-space-letter}
\newcommand{\dimension}[1]{\dim\left(#1\right)}
%
%  Nullity
%  Usage: \nullity{matrix-or-lintrans-letter}
\newcommand{\nullity}[1]{n\left(#1\right)}
%
%  Rank
%  Usage: \rank{matrix-or-lintrans-letter}
\newcommand{\rank}[1]{r\left(#1\right)}
%
%  Direct sum
%  Usage: \ds between a couple of subspaces
%
\newcommand{\ds}{\oplus}
%
%  Determinant of a matrix (functional)
%  Usage: \detname{A}
\newcommand{\detname}[1]{\det\left(#1\right)}
%
%  Determinant of a matrix (vertical bars)
%  Usage: \detbars{A}
\newcommand{\detbars}[1]{\left\lvert#1\right\rvert}
%
%  Trace of a Matrix
%  Usage: \trace{matrix name}
\newcommand{\trace}[1]{t\left(#1\right)}
%
%  Square Root of a Matrix
%  Usage: \sr{a-matrix}
\newcommand{\sr}[1]{#1^{1/2}}
%
%%%%%%%%%%%%%%%%%%%%%
%
%     Subspace Constructions
%
%%%%%%%%%%%%%%%%%%%%%
%
%  Span of a set of vectors
%  \span and \sp are used by TeX for other things
%  Usage: \spn{set-of-vectors}
\newcommand{\spn}[1]{\left\langle#1\right\rangle}
%
%  Null space of a matrix
%  Usage:  \nsp{A}
\newcommand{\nsp}[1]{\mathcal{N}\!\left(#1\right)}
%
%  Column space of a matrix
%  Usage:  \csp{A}
\newcommand{\csp}[1]{\mathcal{C}\!\left(#1\right)}
%
%  Row space of a matrix
%  Usage:  \rsp{A}
\newcommand{\rsp}[1]{\mathcal{R}\!\left(#1\right)}
%
%  Left null space of a matrix
%  Usage:  \lns{A}
\newcommand{\lns}[1]{\mathcal{L}\!\left(#1\right)}
%
%  Orthogonal complement of a vector space
%  Avoiding TeX's \perp
%  Usage:  \per{A}
\newcommand{\per}[1]{#1^\perp}
%
%%%%%%%%%%%%%%%%%%%%%
%
%     Systems of Equations
%
%%%%%%%%%%%%%%%%%%%%%
%
%  In-line form of an augmented matrix for a system of equations
%  Usage: \augmented{coefficient-matrix}{constant-vector}
\newcommand{\augmented}[2]{\left\lbrack\left.#1\,\right\rvert\,#2\right\rbrack}
%
%  Notation for a linear system before introducing matrix multiplication
%  Usage: \linearsystem{coefficient-matrix}{constant-vector}
\newcommand{\linearsystem}[2]{\mathcal{LS}\!\left(#1,\,#2\right)}
%
%  Notation for a homogenous system before introducing matrix multiplication
%  Usage: \homosystem{coefficient-matrix}
\newcommand{\homosystem}[1]{\linearsystem{#1}{\zerovector}}
%
%%%%%%%%%%%%%%%%%%%%%
%
%     Row Operations, Echelon Form
%
%%%%%%%%%%%%%%%%%%%%%
%
% Row operations on matrices
%
% Three commands to shorten up descriptions of gaussian elimination
%
% Usage: \rowopswap{row-i}{row-j}
% Usage: \rowopmult{scalar}{row-i}
% Usage: \rowopadd{scalar}{row-multiplied}{row-added-to}
\newcommand{\rowopswap}[2]{R_{#1}\leftrightarrow R_{#2}}
\newcommand{\rowopmult}[2]{#1R_{#2}}
\newcommand{\rowopadd}[3]{#1R_{#2}+R_{#3}}
%
% Mark leading 1's in echelon form with fbox
% Usage: \leading{a-1-usually}
\newcommand{\leading}[1]{\fbox{#1}}
%
%  Row-reduce arrow
%  Usage:  \rref inbetween a matrix and its reduced row-echelon form
\newcommand{\rref}{\xrightarrow{\text{RREF}}}
%
%  Elementary Matrices
%  Usage: \elemswap{subscript}{subscript}
%  Usage: \elemmult{scalar}{subscript}
%  Usage: \elemadd{scalar}{subscript-mult}{subscript-target}
%
\newcommand{\elemswap}[2]{E_{#1,#2}}
\newcommand{\elemmult}[2]{E_{#2}\left(#1\right)}
\newcommand{\elemadd}[3]{E_{#2,#3}\left(#1\right)}
%
%%%%%%%%%%%%%%%%%%%%%
%
%     2-D Constructions (Lists, Vectors, Matrices)
%
%%%%%%%%%%%%%%%%%%%%%
%
%  A list of scalars of generic length
%  Usage:  \scalarlist{scalar letter}{terminal subscript}
\newcommand{\scalarlist}[2]{{#1}_{1},\,{#1}_{2},\,{#1}_{3},\,\ldots,\,{#1}_{#2}}
%
%  Vector styling, bold (or use wiggles, arrows, whatever)
%  Subscripts go outside this construction
%  Usage: \vect{a symbol to use as a vector}
%  Have to already be in math mode
%
\newcommand{\vect}[1]{\mathbf{#1}}
%
%  A column vector
%  Usage: \colvector{list-delimited-by-\\}
%
\newcommand{\colvector}[1]{\begin{bmatrix}#1\end{bmatrix}}
%
%  A generic vector with components
%  Usage: \vectorcomponents{component-letter}{final-subscript}
\newcommand{\vectorcomponents}[2]{\colvector{#1_{1}\\#1_{2}\\#1_{3}\\\vdots\\#1_{#2}}}
%
%  A list of vectors of generic length
%  Usage:  \vectorlist{vector letter}{terminal subscript}
\newcommand{\vectorlist}[2]{\vect{#1}_{1},\,\vect{#1}_{2},\,\vect{#1}_{3},\,\ldots,\,\vect{#1}_{#2}}
%
%  Vector entries, entry i of vector v
%  (vector-expession still needs \vect, etc.)
%  Usage:  \vectorentry{vector-expression}{single-subscript}
\newcommand{\vectorentry}[2]{\left\lbrack#1\right\rbrack_{#2}}
%
%  Matrix entries, entry i,j of matrix A
%  Usage:  \matrixentry{matrix-expression}{paired-subscripts}
%
\newcommand{\matrixentry}[2]{\left\lbrack#1\right\rbrack_{#2}}
%
%  A generic linear combination
%  Usage:  \lincombo{scalar letter}{vector letter}{terminal subscript}
\newcommand{\lincombo}[3]{#1_{1}\vect{#2}_{1}+#1_{2}\vect{#2}_{2}+#1_{3}\vect{#2}_{3}+\cdots +#1_{#3}\vect{#2}_{#3}}
%
%  Matrix, column by column, as vectors
%  Usage:  \matrixcolumns{matrix letter}{terminal subscript}
\newcommand{\matrixcolumns}[2]{\left\lbrack\vect{#1}_{1}|\vect{#1}_{2}|\vect{#1}_{3}|\ldots|\vect{#1}_{#2}\right\rbrack}
%
%%%%%%%%%%%%%%%%%%%%%
%
%     Special Matrices
%
%%%%%%%%%%%%%%%%%%%%%
%
%  Transpose of a matrix
%  Usage:  \transpose{A}
\newcommand{\transpose}[1]{#1^{t}}
%
%  Inverse of a matrix
%  Usage:  \inverse{A}
\newcommand{\inverse}[1]{#1^{-1}}
%
%  Submatrix (for minors, determinants)
%  Usage: \submatrix{matrix-name}{delete-row}{delete-col}
\newcommand{\submatrix}[3]{#1\left(#2|#3\right)}
%
%  Adjoint of a matrix (twice)
%  This macro is a convenience to call \transpose and \conjugate properly
%  It shouldn't need to be modified (or mathematical meanings will change)
%  Usage:  \adj{A}
\newcommand{\adj}[1]{\transpose{\left(\conjugate{#1}\right)}}
%
%  This macro controls the symbol used for the adjoint
%  It can be edited to taste
%  Usage:  \adjoint{A}
\newcommand{\adjoint}[1]{#1^\ast}
%
%%%%%%%%%%%%%%%%%%%%%
%
%     Sets
%
%%%%%%%%%%%%%%%%%%%%%
%
%  A convenience for simple sets
%  Usage:  \set{list of element}
\newcommand{\set}[1]{\left\{#1\right\}}
%
%  Sets with vertical bar, "such that", sized for objects, not condition
%  Usage:  \setparts{objects}{condition}
%
%%\newcommand{\setparts}[2]{\left\{ #1\mid#2\right\}}
%%\newcommand{\setparts}[2]{\left\{\left. #1\right\rvert#2\right\}}
\newcommand{\setparts}[2]{\left\lbrace#1\,\middle|\,#2\right\rbrace}
%
%  Set Cardinality
%  Usage:  \card{a-set-letter}
\newcommand{\card}[1]{\left\lvert#1\right\rvert}
%
%  Set Union
%  Use \cup
%
%  Set Intersection
%  Use \cap
%
%  Set Complement
%  Usage:  \setcomplement{a-set-letter}
\newcommand{\setcomplement}[1]{\overline{#1}}
%
%%%%%%%%%%%%%%%%%%%%%
%
%     Eigenvalues and Eigenspaces
%
%%%%%%%%%%%%%%%%%%%%%
%
%  Characteristic polynomial
%  Usage: \charpoly{matrix-letter}{variable-letter}
\newcommand{\charpoly}[2]{p_{#1}\left(#2\right)}
%
%  Eigenspace
%  Usage: \eigenspace{matrix-letter}{eigenvalue-letter}
\newcommand{\eigenspace}[2]{\mathcal{E}_{#1}\left(#2\right)}
%
%  2013/10/03 Including ampersands is problematic here, 
%  think about fixes later
%  2014/02/22 Limited testing, seems &amp; is fine for HTML and LaTeX
%  2016-07-20 only employed in Archetypes, MBX has gather/align override
%  Eigensystem (presumes wrapped in an mrow within md)
%  Usage: \eigensystem{matrixletter}{eigenvalue}{list of basis vectors}
\newcommand{\eigensystem}[3]{\lambda&amp;=#2&amp;\eigenspace{#1}{#2}&amp;=\spn{\set{#3}}} 
%
%  Generalized Eigenspace
%  Usage: \geneigenspace{lin-trans-letter}{eigenvalue-letter}
\newcommand{\geneigenspace}[2]{\mathcal{G}_{#1}\left(#2\right)}
%
%  Algebraic multiplicty
%  Usage: \algmult{matrix-letter}{eigenvalue-letter}
\newcommand{\algmult}[2]{\alpha_{#1}\left(#2\right)}
%
%  Geometric multiplicty
%  Usage: \geomult{matrix-letter}{eigenvalue-letter}
\newcommand{\geomult}[2]{\gamma_{#1}\left(#2\right)}
%
%  Index (of eigenvalue)
%  Usage: \indx{matrix-letter}{eigenvalue-letter}
\newcommand{\indx}[2]{\iota_{#1}\left(#2\right)}
%
%%%%%%%%%%%%%%%%%%%%%
%
%     Linear Transformations
%
%%%%%%%%%%%%%%%%%%%%%
%
%  MathJax defines \lt to ease XML confusion
%
%  Linear transformation definition
%  Usage: \ltdefn{name-letter}{domain}{range}
\newcommand{\ltdefn}[3]{#1\colon #2\rightarrow#3}
%
%  Linear transformation evaluation
%  Usage: \lteval{name-letter}{input}
%  Replaces old \lt desired by MathJax
\newcommand{\lteval}[2]{#1\left(#2\right)}
%
% Linear transformation inverse
%  Usage: \ltinverse{name-letter}
\newcommand{\ltinverse}[1]{#1^{-1}}
%
%  Linear transformation restriction
%  Usage: \restrict{name-letter}{subspace-letter}
\newcommand{\restrict}[2]{{#1}|_{#2}}
%
%  Linear transformation preimage
%  Usage: \preimage{name-letter}{codomain-element}
\newcommand{\preimage}[2]{#1^{-1}\left(#2\right)}
%
%  Range of a linear transformation
%  TeX uses \range for something else
%  Usage:  \rng{T}
\newcommand{\rng}[1]{\mathcal{R}\!\left(#1\right)}
%
%  Kernel of a linear transformation
%  TeX uses \ker to do something different
%  Usage:  \krn{T}
\newcommand{\krn}[1]{\mathcal{K}\!\left(#1\right)}
%
%  Linear transformation composition
%  Usage: \compose{function-name}{function-name}
\newcommand{\compose}[2]{{#1}\circ{#2}}
%
%  Vector space of linear transformations
%  Usage: \vslt{domains}{codomains}
%  Presumes math mode
\newcommand{\vslt}[2]{\mathcal{LT}\left(#1,\,#2\right)}
%
%%%%%%%%%%%%%%%%%%%%%
%
%     Vector and Matrix Representations
%
%%%%%%%%%%%%%%%%%%%%%
%
%  Isomorphism symbol
%  Usage: \isomorphic
\newcommand{\isomorphic}{\cong}
%
%  Similarity
%  Usage: \similar{inner-matrix}{outer-invertible-matrix}
%  Rearranging this will not "fix" all desired changes throughout
%
\newcommand{\similar}[2]{\inverse{#2}#1#2}
%
%  Vector representation function name
%  Usage: \vectrepname{basis-letter}
\newcommand{\vectrepname}[1]{\rho_{#1}}
%
%  Vector representation output
%  Usage: \vectrep{basis-letter}{input}
\newcommand{\vectrep}[2]{\lteval{\vectrepname{#1}}{#2}}
%
%  Vector representation inverse function name
%  (Added later, not used consistently in FCLA)
%  Usage: \vectrepinvname{basis-letter}
\newcommand{\vectrepinvname}[1]{\ltinverse{\vectrepname{#1}}}
%
%  Vector representation inverse output
%  Usage: \vectrepinv{basis-letter}{input}
\newcommand{\vectrepinv}[2]{\lteval{\ltinverse{\vectrepname{#1}}}{#2}}
%
%  Matrix representation
%  Usage: \matrixrep{transformation-letter}{domain-basis-letter}{codomain-basis-letter}
\newcommand{\matrixrep}[3]{M^{#1}_{#2,#3}}
%
%  Matrix representation column-by-colum
%  2016-07-20 only employed once?
%  Usage: \matrixrepcolumns{transformation-letter}{codomain-basis-letter}{codomain-basis-vector-letter}{final-index}
\newcommand{\matrixrepcolumns}[4]{\left\lbrack \left.\vectrep{#2}{\lteval{#1}{\vect{#3}_{1}}}\right|\left.\vectrep{#2}{\lteval{#1}{\vect{#3}_{2}}}\right|\left.\vectrep{#2}{\lteval{#1}{\vect{#3}_{3}}}\right|\ldots\left|\vectrep{#2}{\lteval{#1}{\vect{#3}_{#4}}}\right.\right\rbrack}
%
%  Change of basis matrix
%  Usage: \cbm{domain-basis-letter}{codomain-basis-letter}
\newcommand{\cbm}[2]{C_{#1,#2}}
%
%%%%%%%%%%%%%%%%%%%%%
%
%     Canonical Forms
%
%%%%%%%%%%%%%%%%%%%%%
%
%  Jordan blocks
%  Usage: \jordan{size}{diagonal-element}
\newcommand{\jordan}[2]{J_{#1}\left(#2\right)}
%
%%%%%%%%%%%%%%%%%%%%%
%
%     Hadamard Matrices
%     Contributed by Elizabeth Million
%
%%%%%%%%%%%%%%%%%%%%%
%
%  Hadamard Product
%  Usage: \hadamard{a-matrix}{a-matrix}
\newcommand{\hadamard}[2]{#1\circ #2}
%
%  Hadamard identity matrix
%  Usage: \hadamardidentity{paired-subscripts-size-of-matrix}
\newcommand{\hadamardidentity}[1]{J_{#1}}
%
%  Hadamard inverse matrix
%  Usage: \hadamardinverse{matrix-expression}
\newcommand{\hadamardinverse}[1]{\widehat{#1}}


\title{Orthogonal Vectors}

\begin{document}
\begin{abstract}
``Orthogonal'' is a generalization of ``perpendicular.''
\end{abstract}
\maketitle

``Orthogonal'' is a generalization of ``perpendicular.''  You may have
used mutually perpendicular vectors in a physics class, or you may
recall from a calculus class that perpendicular vectors have a zero
dot product.  We will now extend these ideas into the realm of higher
dimensions and complex scalars.

\begin{definition}[Orthogonal Vectors]
  A pair of vectors, $\vect{u}$ and $\vect{v}$, from $\complex{m}$ are
  \dfn{orthogonal} if their inner product is zero, that is,
  $\innerproduct{\vect{u}}{\vect{v}}=0$.
\end{definition}

\begin{example}[Orthogonal vectors]
  Consider the vectors
  \begin{align*}
    \vect{u}&=\colvector{2 + 3i\\4 - 2i\\1 + i\\1 + i} \\
    \vect{a}&=\colvector{1 + i\\2 + 3i\\4 - 6i\\1} \\
    \vect{b}&=\colvector{1 - i\\2 + 3i\\4 - 6i\\1}.
  \end{align*}
  Then
  \begin{multipleChoice}
    \choice{$\vect{u}$ and $\vect{a}$ are orthogonal.}
    \choice[correct]{$\vect{u}$ and $\vect{b}$ are orthogonal.}
  \end{multipleChoice}
  
  \begin{hint}
    \begin{align*}
      \innerproduct{\vect{u}}{\vect{v}}
      &=(2-3i)(1-i)+(4+2i)(2+3i)+(1-i)(4-6i)+(1-i)(1)\\
      &=(-1-5i)+(2+16i)+(-2-10i)+(1-i)\\
      &=0+0i.
    \end{align*}
  \end{hint}
\end{example}

We extend this definition to whole sets by requiring vectors to be
pairwise orthogonal.  Despite using the same word, careful thought
about what objects you are using will eliminate any source of
confusion.

\begin{definition}[Orthogonal Set of Vectors]

  Suppose that $S=\set{\vectorlist{u}{n}}$ is a set of vectors from
  $\complex{m}$.  Then $S$ is an \dfn{orthogonal set} if every pair of
  different vectors from $S$ is orthogonal, that is
  $\innerproduct{\vect{u}_i}{\vect{u}_j}=0$ whenever $i\neq j$.

\end{definition}

We now define the prototypical orthogonal set, which we will reference
repeatedly.

\begin{definition}[Standard Unit Vectors]

  Let $\vect{e}_j\in\complex{m}$, $1\leq j\leq m$ denote the column
  vectors defined by
  \begin{align*}
    \vectorentry{\vect{e}_j}{i}
    &=
      \begin{cases}
        0&\text{if $i\neq j$}\\
        1&\text{if $i=j$}
      \end{cases}
  \end{align*}

  Then the set
  \begin{align*}
    \set{\vectorlist{e}{m}}&=\setparts{\vect{e}_j}{1\leq j\leq m}
  \end{align*}
  is the set of \dfn{standard unit vectors} in $\complex{m}$.
  
\end{definition}

Notice that $\vect{e}_j$ is identical to column $j$ of the $m\times m$
identity matrix $I_m$, and is a pivot column for $I_m$, since the
identity matrix is in reduced row-echelon form.  These observations
will often be useful.  We will reserve the notation $\vect{e}_i$ for
these vectors.  It is not hard to see that the set of standard unit
vectors is an orthogonal set.

\begin{example}[Standard Unit Vectors are an Orthogonal Set]

Compute the inner product of two distinct vectors from the set of standard unit vectors (\ref{definition:SUV}), say $\vect{e}_i$, $\vect{e}_j$, where $i\neq j$, to discover that in this case
\[
\innerproduct{\vect{e}_i}{\vect{e}_j}&=\answer{0}
\]
\begin{hint}
\begin{align*}
  \innerproduct{\vect{e}_i}{\vect{e}_j}&=
                                         \conjugate{0}0+
                                         \conjugate{0}0+\cdots+
                                         \conjugate{1}0+\cdots+
                                         \conjugate{0}0+\cdots+
                                         \conjugate{0}1+\cdots+
                                         \conjugate{0}0+
                                         \conjugate{0}0\\
                                       &=0(0)+0(0)+\cdots+1(0)+\cdots+0(1)+\cdots+0(0)+0(0)\\
                                       &=0
\end{align*}
\end{hint}

So the set $\set{\vectorlist{e}{m}}$ is an orthogonal set.
\end{example}

\begin{example}[An orthogonal set]

The set
\[
\set{\vect{x}_1,\,\vect{x}_2,\,\vect{x}_3,\,\vect{x}_4}=
\set{
\colvector{1+i\\1\\1-i\\i},\,
\colvector{1+5i\\6+5i\\-7-i\\1-6i},\,
\colvector{-7+34i\\-8-23i\\-10+22i\\30+13i},\,
\colvector{-2-4i\\6+i\\4+3i\\6-i}
}
\]
is an orthogonal set.


Since the inner product is anti-commutative (\ref{theorem:IPAC}) we can test pairs of different vectors in any order.  If the result is zero, then it will also be zero if the inner product is computed in the opposite order.  This means there are six different pairs of vectors to use in an inner product computation.
\begin{align*}
\innerproduct{\vect{x}_1}{\vect{x}_3}&=
(1-i)(-7+34i)+(1)(-8-23i)+(1+i)(-10+22i)+(-i)(30+13i)\\
&=(\answer{27+41i})+(-8-23i)+(-32+12i)+(13-30i)\\
&=0+0i
<intertext>and</intertext>
\innerproduct{\vect{x}_2}{\vect{x}_4}&=
(1-5i)(-2-4i)+(6-5i)(6+i)+(-7+i)(4+3i)+(1+6i)(6-i)\\
&=(-22+6i)+(41-24i)+(-31-17i)+(12+35i)\\
&=0+0i
\end{align*}
You can practice your inner products on the other four.

\end{example}

So far, this section has seen lots of definitions, and lots of
theorems establishing un-surprising consequences of those definitions.
But here is our first theorem that suggests that inner products and
orthogonal vectors have some utility.  It is also one of our first
illustrations of how to arrive at linear independence as the
conclusion of a theorem.

\begin{theorem}[Orthogonal Sets are Linearly Independent]
\label{theorem:OSLI}

Suppose that $S$ is an orthogonal set of nonzero vectors.  Then $S$ is linearly independent.

\begin{proof}
  Let $S=\set{\vectorlist{u}{n}}$ be an orthogonal set of nonzero
  vectors.  To prove the linear independence of $S$, we can appeal to
  the definition (\ref{definition:LICV}) and begin with an arbitrary
  relation of linear dependence (\ref{definition:RLDCV}),
  \[
    \lincombo{\alpha}{u}{n}=\zerovector.
  \]
  
  Then, for every $1\leq i\leq n$, we have
  \begin{align*}
    &\alpha_i\innerproduct{\vect{u}_i}{\vect{u}_i}\\
    &\quad\quad=\alpha_1(0)+\alpha_2(0)+\cdots+\alpha_i\innerproduct{\vect{u}_i}{\vect{u}_i}+\cdots+\alpha_n(0)
    \\ %&&\ref{property:ZCN}\\
    &\quad\quad=
      \alpha_1\innerproduct{\vect{u}_i}{\vect{u}_1}+
      \cdots+
      \alpha_i\innerproduct{\vect{u}_i}{\vect{u}_i}+
      \cdots+
      \alpha_n\innerproduct{\vect{u}_i}{\vect{u}_n}
    \\ %&&\ref{definition:OSV}\\
    &\quad\quad=
      \innerproduct{\vect{u}_i}{\alpha_1\vect{u}_1}+
      \innerproduct{\vect{u}_i}{\alpha_2\vect{u}_2}+
      \cdots+
      \innerproduct{\vect{u}_i}{\alpha_n\vect{u}_n}
    \\ %&&\ref{theorem:IPSM}\\
    &\quad\quad=
      \innerproduct{\vect{u}_i}{\lincombo{\alpha}{u}{n}}
    \\ %&&\ref{theorem:IPVA}\\
    &\quad\quad=
      \innerproduct{\vect{u}_i}{\zerovector}
    \\ %&&\ref{definition:RLDCV}\\
    &\quad\quad=0
    \\ %&&\ref{definition:IP}
  \end{align*}
  
  Because $\vect{u}_i$ was assumed to be nonzero, \ref{theorem:PIP}
  says $\innerproduct{\vect{u}_i}{\vect{u}_i}$ is nonzero and thus
  $\alpha_i$ must be zero.  So we conclude that $\alpha_i=0$ for all
  $1\leq i\leq n$ in any relation of linear dependence on $S$.  But
  this says that $S$ is a linearly independent set since the only way
  to form a relation of linear dependence is the trivial way
  (\ref{definition:LICV}).
\end{proof}
\end{theorem}

\end{document}
