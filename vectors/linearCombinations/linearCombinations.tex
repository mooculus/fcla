\documentclass{ximera}

% These macros are automatically generated from the "macros"
% XML element.  Make permanent edits there.
%
% History
%   2004/01/01  Initiated for FCLA, evolved from there
%   2006/09/17  Converted  _, ^  to \sb, \sp for TeX4ht
%   2014/02/01  Updated for MathBook XML projects
%               Obsolete in FCLA: \codeindent, \computerfont, \define
%               Change: MathJax wants \lt, so replaced by \lteval
%   2014/02/22  New: \orderof, \reals, \per
%   2015/08/16  Incorporated into MathBook XML version of FCLA
%
%%%%%%%%%%%%%%%%%%%%%
%
%     Conveniences
%
%%%%%%%%%%%%%%%%%%%%%
%
%  Order of (asymptotically limit of fraction is 1)
%  Usage: \orderof{some function}
%
\newcommand{\orderof}[1]{\sim #1}
%
%  Integers
%  Usage:  \Z
\newcommand{\Z}{\mathbb{Z}}
%
%  Real numbers, as set of scalars
%  Usage:  \reals
\newcommand{\reals}{\mathbb{R}}
%
%  n-space over real field
%  Usage: \complex{integer-dimension}
\newcommand{\real}[1]{\mathbb{R}^{#1}}
%
%  Complex numbers, as set of scalars
%  Usage:  \complexes
\newcommand{\complexes}{\mathbb{C}}
%
%  n-space over complex field
%  Usage: \complex{integer-dimension}
\newcommand{\complex}[1]{\mathbb{C}^{#1}}
%
%  Complex conjugation (scalar, vector, matrix)
%  Usage: \conjugate{object}
\newcommand{\conjugate}[1]{\overline{#1}}
%
%  Complex number modulus
%  Usage: \modulus{a+bi}
%  Presumes math mode
\newcommand{\modulus}[1]{\left\lvert#1\right\rvert}
%
%  Zero vector
%  Usage: \zerovector
\newcommand{\zerovector}{\vect{0}}
%
%  Zero matrix
%  Usage: \zeromatrix, use a subscript when size is important
\newcommand{\zeromatrix}{\mathcal{O}}
%
%  Inner product (brackets, not quadratic form)
%  Usage: \innerproduct{a-vector}{a-vector}
\newcommand{\innerproduct}[2]{\left\langle#1,\,#2\right\rangle}
%
%  Norm of a vector
%  Usage: \norm{a-vector}
\newcommand{\norm}[1]{\left\lVert#1\right\rVert}
%
%  Dimension
%  Usage: \dimension{vector-space-letter}
\newcommand{\dimension}[1]{\dim\left(#1\right)}
%
%  Nullity
%  Usage: \nullity{matrix-or-lintrans-letter}
\newcommand{\nullity}[1]{n\left(#1\right)}
%
%  Rank
%  Usage: \rank{matrix-or-lintrans-letter}
\newcommand{\rank}[1]{r\left(#1\right)}
%
%  Direct sum
%  Usage: \ds between a couple of subspaces
%
\newcommand{\ds}{\oplus}
%
%  Determinant of a matrix (functional)
%  Usage: \detname{A}
\newcommand{\detname}[1]{\det\left(#1\right)}
%
%  Determinant of a matrix (vertical bars)
%  Usage: \detbars{A}
\newcommand{\detbars}[1]{\left\lvert#1\right\rvert}
%
%  Trace of a Matrix
%  Usage: \trace{matrix name}
\newcommand{\trace}[1]{t\left(#1\right)}
%
%  Square Root of a Matrix
%  Usage: \sr{a-matrix}
\newcommand{\sr}[1]{#1^{1/2}}
%
%%%%%%%%%%%%%%%%%%%%%
%
%     Subspace Constructions
%
%%%%%%%%%%%%%%%%%%%%%
%
%  Span of a set of vectors
%  \span and \sp are used by TeX for other things
%  Usage: \spn{set-of-vectors}
\newcommand{\spn}[1]{\left\langle#1\right\rangle}
%
%  Null space of a matrix
%  Usage:  \nsp{A}
\newcommand{\nsp}[1]{\mathcal{N}\!\left(#1\right)}
%
%  Column space of a matrix
%  Usage:  \csp{A}
\newcommand{\csp}[1]{\mathcal{C}\!\left(#1\right)}
%
%  Row space of a matrix
%  Usage:  \rsp{A}
\newcommand{\rsp}[1]{\mathcal{R}\!\left(#1\right)}
%
%  Left null space of a matrix
%  Usage:  \lns{A}
\newcommand{\lns}[1]{\mathcal{L}\!\left(#1\right)}
%
%  Orthogonal complement of a vector space
%  Avoiding TeX's \perp
%  Usage:  \per{A}
\newcommand{\per}[1]{#1^\perp}
%
%%%%%%%%%%%%%%%%%%%%%
%
%     Systems of Equations
%
%%%%%%%%%%%%%%%%%%%%%
%
%  In-line form of an augmented matrix for a system of equations
%  Usage: \augmented{coefficient-matrix}{constant-vector}
\newcommand{\augmented}[2]{\left\lbrack\left.#1\,\right\rvert\,#2\right\rbrack}
%
%  Notation for a linear system before introducing matrix multiplication
%  Usage: \linearsystem{coefficient-matrix}{constant-vector}
\newcommand{\linearsystem}[2]{\mathcal{LS}\!\left(#1,\,#2\right)}
%
%  Notation for a homogenous system before introducing matrix multiplication
%  Usage: \homosystem{coefficient-matrix}
\newcommand{\homosystem}[1]{\linearsystem{#1}{\zerovector}}
%
%%%%%%%%%%%%%%%%%%%%%
%
%     Row Operations, Echelon Form
%
%%%%%%%%%%%%%%%%%%%%%
%
% Row operations on matrices
%
% Three commands to shorten up descriptions of gaussian elimination
%
% Usage: \rowopswap{row-i}{row-j}
% Usage: \rowopmult{scalar}{row-i}
% Usage: \rowopadd{scalar}{row-multiplied}{row-added-to}
\newcommand{\rowopswap}[2]{R_{#1}\leftrightarrow R_{#2}}
\newcommand{\rowopmult}[2]{#1R_{#2}}
\newcommand{\rowopadd}[3]{#1R_{#2}+R_{#3}}
%
% Mark leading 1's in echelon form with fbox
% Usage: \leading{a-1-usually}
\newcommand{\leading}[1]{\fbox{#1}}
%
%  Row-reduce arrow
%  Usage:  \rref inbetween a matrix and its reduced row-echelon form
\newcommand{\rref}{\xrightarrow{\text{RREF}}}
%
%  Elementary Matrices
%  Usage: \elemswap{subscript}{subscript}
%  Usage: \elemmult{scalar}{subscript}
%  Usage: \elemadd{scalar}{subscript-mult}{subscript-target}
%
\newcommand{\elemswap}[2]{E_{#1,#2}}
\newcommand{\elemmult}[2]{E_{#2}\left(#1\right)}
\newcommand{\elemadd}[3]{E_{#2,#3}\left(#1\right)}
%
%%%%%%%%%%%%%%%%%%%%%
%
%     2-D Constructions (Lists, Vectors, Matrices)
%
%%%%%%%%%%%%%%%%%%%%%
%
%  A list of scalars of generic length
%  Usage:  \scalarlist{scalar letter}{terminal subscript}
\newcommand{\scalarlist}[2]{{#1}_{1},\,{#1}_{2},\,{#1}_{3},\,\ldots,\,{#1}_{#2}}
%
%  Vector styling, bold (or use wiggles, arrows, whatever)
%  Subscripts go outside this construction
%  Usage: \vect{a symbol to use as a vector}
%  Have to already be in math mode
%
\newcommand{\vect}[1]{\mathbf{#1}}
%
%  A column vector
%  Usage: \colvector{list-delimited-by-\\}
%
\newcommand{\colvector}[1]{\begin{bmatrix}#1\end{bmatrix}}
%
%  A generic vector with components
%  Usage: \vectorcomponents{component-letter}{final-subscript}
\newcommand{\vectorcomponents}[2]{\colvector{#1_{1}\\#1_{2}\\#1_{3}\\\vdots\\#1_{#2}}}
%
%  A list of vectors of generic length
%  Usage:  \vectorlist{vector letter}{terminal subscript}
\newcommand{\vectorlist}[2]{\vect{#1}_{1},\,\vect{#1}_{2},\,\vect{#1}_{3},\,\ldots,\,\vect{#1}_{#2}}
%
%  Vector entries, entry i of vector v
%  (vector-expession still needs \vect, etc.)
%  Usage:  \vectorentry{vector-expression}{single-subscript}
\newcommand{\vectorentry}[2]{\left\lbrack#1\right\rbrack_{#2}}
%
%  Matrix entries, entry i,j of matrix A
%  Usage:  \matrixentry{matrix-expression}{paired-subscripts}
%
\newcommand{\matrixentry}[2]{\left\lbrack#1\right\rbrack_{#2}}
%
%  A generic linear combination
%  Usage:  \lincombo{scalar letter}{vector letter}{terminal subscript}
\newcommand{\lincombo}[3]{#1_{1}\vect{#2}_{1}+#1_{2}\vect{#2}_{2}+#1_{3}\vect{#2}_{3}+\cdots +#1_{#3}\vect{#2}_{#3}}
%
%  Matrix, column by column, as vectors
%  Usage:  \matrixcolumns{matrix letter}{terminal subscript}
\newcommand{\matrixcolumns}[2]{\left\lbrack\vect{#1}_{1}|\vect{#1}_{2}|\vect{#1}_{3}|\ldots|\vect{#1}_{#2}\right\rbrack}
%
%%%%%%%%%%%%%%%%%%%%%
%
%     Special Matrices
%
%%%%%%%%%%%%%%%%%%%%%
%
%  Transpose of a matrix
%  Usage:  \transpose{A}
\newcommand{\transpose}[1]{#1^{t}}
%
%  Inverse of a matrix
%  Usage:  \inverse{A}
\newcommand{\inverse}[1]{#1^{-1}}
%
%  Submatrix (for minors, determinants)
%  Usage: \submatrix{matrix-name}{delete-row}{delete-col}
\newcommand{\submatrix}[3]{#1\left(#2|#3\right)}
%
%  Adjoint of a matrix (twice)
%  This macro is a convenience to call \transpose and \conjugate properly
%  It shouldn't need to be modified (or mathematical meanings will change)
%  Usage:  \adj{A}
\newcommand{\adj}[1]{\transpose{\left(\conjugate{#1}\right)}}
%
%  This macro controls the symbol used for the adjoint
%  It can be edited to taste
%  Usage:  \adjoint{A}
\newcommand{\adjoint}[1]{#1^\ast}
%
%%%%%%%%%%%%%%%%%%%%%
%
%     Sets
%
%%%%%%%%%%%%%%%%%%%%%
%
%  A convenience for simple sets
%  Usage:  \set{list of element}
\newcommand{\set}[1]{\left\{#1\right\}}
%
%  Sets with vertical bar, "such that", sized for objects, not condition
%  Usage:  \setparts{objects}{condition}
%
%%\newcommand{\setparts}[2]{\left\{ #1\mid#2\right\}}
%%\newcommand{\setparts}[2]{\left\{\left. #1\right\rvert#2\right\}}
\newcommand{\setparts}[2]{\left\lbrace#1\,\middle|\,#2\right\rbrace}
%
%  Set Cardinality
%  Usage:  \card{a-set-letter}
\newcommand{\card}[1]{\left\lvert#1\right\rvert}
%
%  Set Union
%  Use \cup
%
%  Set Intersection
%  Use \cap
%
%  Set Complement
%  Usage:  \setcomplement{a-set-letter}
\newcommand{\setcomplement}[1]{\overline{#1}}
%
%%%%%%%%%%%%%%%%%%%%%
%
%     Eigenvalues and Eigenspaces
%
%%%%%%%%%%%%%%%%%%%%%
%
%  Characteristic polynomial
%  Usage: \charpoly{matrix-letter}{variable-letter}
\newcommand{\charpoly}[2]{p_{#1}\left(#2\right)}
%
%  Eigenspace
%  Usage: \eigenspace{matrix-letter}{eigenvalue-letter}
\newcommand{\eigenspace}[2]{\mathcal{E}_{#1}\left(#2\right)}
%
%  2013/10/03 Including ampersands is problematic here, 
%  think about fixes later
%  2014/02/22 Limited testing, seems &amp; is fine for HTML and LaTeX
%  2016-07-20 only employed in Archetypes, MBX has gather/align override
%  Eigensystem (presumes wrapped in an mrow within md)
%  Usage: \eigensystem{matrixletter}{eigenvalue}{list of basis vectors}
\newcommand{\eigensystem}[3]{\lambda&amp;=#2&amp;\eigenspace{#1}{#2}&amp;=\spn{\set{#3}}} 
%
%  Generalized Eigenspace
%  Usage: \geneigenspace{lin-trans-letter}{eigenvalue-letter}
\newcommand{\geneigenspace}[2]{\mathcal{G}_{#1}\left(#2\right)}
%
%  Algebraic multiplicty
%  Usage: \algmult{matrix-letter}{eigenvalue-letter}
\newcommand{\algmult}[2]{\alpha_{#1}\left(#2\right)}
%
%  Geometric multiplicty
%  Usage: \geomult{matrix-letter}{eigenvalue-letter}
\newcommand{\geomult}[2]{\gamma_{#1}\left(#2\right)}
%
%  Index (of eigenvalue)
%  Usage: \indx{matrix-letter}{eigenvalue-letter}
\newcommand{\indx}[2]{\iota_{#1}\left(#2\right)}
%
%%%%%%%%%%%%%%%%%%%%%
%
%     Linear Transformations
%
%%%%%%%%%%%%%%%%%%%%%
%
%  MathJax defines \lt to ease XML confusion
%
%  Linear transformation definition
%  Usage: \ltdefn{name-letter}{domain}{range}
\newcommand{\ltdefn}[3]{#1\colon #2\rightarrow#3}
%
%  Linear transformation evaluation
%  Usage: \lteval{name-letter}{input}
%  Replaces old \lt desired by MathJax
\newcommand{\lteval}[2]{#1\left(#2\right)}
%
% Linear transformation inverse
%  Usage: \ltinverse{name-letter}
\newcommand{\ltinverse}[1]{#1^{-1}}
%
%  Linear transformation restriction
%  Usage: \restrict{name-letter}{subspace-letter}
\newcommand{\restrict}[2]{{#1}|_{#2}}
%
%  Linear transformation preimage
%  Usage: \preimage{name-letter}{codomain-element}
\newcommand{\preimage}[2]{#1^{-1}\left(#2\right)}
%
%  Range of a linear transformation
%  TeX uses \range for something else
%  Usage:  \rng{T}
\newcommand{\rng}[1]{\mathcal{R}\!\left(#1\right)}
%
%  Kernel of a linear transformation
%  TeX uses \ker to do something different
%  Usage:  \krn{T}
\newcommand{\krn}[1]{\mathcal{K}\!\left(#1\right)}
%
%  Linear transformation composition
%  Usage: \compose{function-name}{function-name}
\newcommand{\compose}[2]{{#1}\circ{#2}}
%
%  Vector space of linear transformations
%  Usage: \vslt{domains}{codomains}
%  Presumes math mode
\newcommand{\vslt}[2]{\mathcal{LT}\left(#1,\,#2\right)}
%
%%%%%%%%%%%%%%%%%%%%%
%
%     Vector and Matrix Representations
%
%%%%%%%%%%%%%%%%%%%%%
%
%  Isomorphism symbol
%  Usage: \isomorphic
\newcommand{\isomorphic}{\cong}
%
%  Similarity
%  Usage: \similar{inner-matrix}{outer-invertible-matrix}
%  Rearranging this will not "fix" all desired changes throughout
%
\newcommand{\similar}[2]{\inverse{#2}#1#2}
%
%  Vector representation function name
%  Usage: \vectrepname{basis-letter}
\newcommand{\vectrepname}[1]{\rho_{#1}}
%
%  Vector representation output
%  Usage: \vectrep{basis-letter}{input}
\newcommand{\vectrep}[2]{\lteval{\vectrepname{#1}}{#2}}
%
%  Vector representation inverse function name
%  (Added later, not used consistently in FCLA)
%  Usage: \vectrepinvname{basis-letter}
\newcommand{\vectrepinvname}[1]{\ltinverse{\vectrepname{#1}}}
%
%  Vector representation inverse output
%  Usage: \vectrepinv{basis-letter}{input}
\newcommand{\vectrepinv}[2]{\lteval{\ltinverse{\vectrepname{#1}}}{#2}}
%
%  Matrix representation
%  Usage: \matrixrep{transformation-letter}{domain-basis-letter}{codomain-basis-letter}
\newcommand{\matrixrep}[3]{M^{#1}_{#2,#3}}
%
%  Matrix representation column-by-colum
%  2016-07-20 only employed once?
%  Usage: \matrixrepcolumns{transformation-letter}{codomain-basis-letter}{codomain-basis-vector-letter}{final-index}
\newcommand{\matrixrepcolumns}[4]{\left\lbrack \left.\vectrep{#2}{\lteval{#1}{\vect{#3}_{1}}}\right|\left.\vectrep{#2}{\lteval{#1}{\vect{#3}_{2}}}\right|\left.\vectrep{#2}{\lteval{#1}{\vect{#3}_{3}}}\right|\ldots\left|\vectrep{#2}{\lteval{#1}{\vect{#3}_{#4}}}\right.\right\rbrack}
%
%  Change of basis matrix
%  Usage: \cbm{domain-basis-letter}{codomain-basis-letter}
\newcommand{\cbm}[2]{C_{#1,#2}}
%
%%%%%%%%%%%%%%%%%%%%%
%
%     Canonical Forms
%
%%%%%%%%%%%%%%%%%%%%%
%
%  Jordan blocks
%  Usage: \jordan{size}{diagonal-element}
\newcommand{\jordan}[2]{J_{#1}\left(#2\right)}
%
%%%%%%%%%%%%%%%%%%%%%
%
%     Hadamard Matrices
%     Contributed by Elizabeth Million
%
%%%%%%%%%%%%%%%%%%%%%
%
%  Hadamard Product
%  Usage: \hadamard{a-matrix}{a-matrix}
\newcommand{\hadamard}[2]{#1\circ #2}
%
%  Hadamard identity matrix
%  Usage: \hadamardidentity{paired-subscripts-size-of-matrix}
\newcommand{\hadamardidentity}[1]{J_{#1}}
%
%  Hadamard inverse matrix
%  Usage: \hadamardinverse{matrix-expression}
\newcommand{\hadamardinverse}[1]{\widehat{#1}}


\title{Linear Combinations}

\begin{document}
\begin{abstract}
  A linear combination takes scalars and vectors and combines them
  using scalar multiplication and vector addition to creates a single
  brand-new vector.
\end{abstract}
\maketitle

We defined vector addition and scalar multiplication.  These two
operations combine nicely to give us a construction known as a
\dfn{linear combination}, a construct that we will work with
throughout this course.

\begin{definition}[Linear Combination of Column Vectors]
  Given $n$ vectors $\vectorlist{u}{n}$ from $\complex{m}$ and $n$
  scalars $\alpha_1,\,\alpha_2,\,\alpha_3,\,\ldots,\,\alpha_n$, their
  \dfn{linear combination} is the vector
  \[
    \lincombo{\alpha}{u}{n}
  \]
\end{definition}

So this definition takes an equal number of scalars and vectors,
combines them using our two new operations (scalar multiplication and
vector addition) and creates a single brand-new vector, of the same
size as the original vectors.  When a definition or theorem employs a
linear combination, think about the nature of the objects that go into
its creation (lists of scalars and vectors), and the type of object
that results (a single vector).

\begin{example}[Two linear combinations in $\complex{6}$]
  Suppose that
  \begin{align*}
    \alpha_1=1&&\alpha_2=-4&&\alpha_3=2&&\alpha_4=-1
  \end{align*}
  and
  \begin{align*}
    \vect{u}_1&=\colvector{2\\4\\-3\\1\\2\\9}&
    \vect{u}_2&=\colvector{6\\3\\0\\-2\\1\\4}&
    \vect{u}_3&=\colvector{-5\\2\\1\\1\\-3\\0}&
    \vect{u}_4&=\colvector{3\\2\\-5\\7\\1\\3}
  \end{align*}
  then their linear combination is
  \begin{align*}
    \alpha_1\vect{u_1}+ \alpha_2\vect{u_2}+ \alpha_3\vect{u_3}+ \alpha_4\vect{u_4}&=
    (1)\colvector{2\\4\\-3\\1\\2\\9}+
    (-4)\colvector{6\\3\\0\\-2\\1\\4}+
    (2)\colvector{-5\\2\\1\\1\\-3\\0}+
    (-1)\colvector{3\\2\\-5\\7\\1\\3}\\
    &=
    \colvector{2\\4\\-3\\1\\2\\9}+
    \colvector{-24\\-12\\0\\8\\-4\\-16}+
    \colvector{-10\\\answer{4}\\2\\2\\-6\\0}+
    \colvector{-3\\\answer{-2}\\5\\-7\\-1\\-3} \\
    &=\colvector{-35\\\answer{-6}\\4\\4\\-9\\-10}
  \end{align*}
  
  A different linear combination, of the same set of vectors, can be
  formed with different scalars. Take
  \begin{align*}
    \beta_1=3&&\beta_2=0&&\beta_3=5&&\beta_4=-1
  \end{align*}
  and form the linear combination
  \begin{align*}
    \beta_1\vect{u_1}+ \beta_2\vect{u_2}+ \beta_3\vect{u_3}+ \beta_4\vect{u_4}
    &=
      (3)\colvector{2\\4\\-3\\1\\2\\9}+
      (0)\colvector{6\\3\\0\\-2\\1\\4}+
      (5)\colvector{-5\\2\\1\\1\\-3\\0}+
      (-1)\colvector{3\\2\\-5\\7\\1\\3}\\
    &=
      \colvector{6\\12\\-9\\3\\6\\27}+
      \colvector{0\\0\\0\\0\\0\\0}+
      \colvector{-25\\\answer{10}\\5\\5\\-15\\0}+
      \colvector{-3\\\answer{-2}\\5\\-7\\-1\\-3} \\
    &=
      \colvector{-22\\\answer{20}\\1\\1\\-10\\24}
  \end{align*}
  Notice how we could keep our set of vectors fixed, and use different
  sets of scalars to construct different vectors.  You might build a
  few new linear combinations of
  $\vect{u}_1,\,\vect{u}_2,\,\vect{u}_3,\,\vect{u}_4$ right now.  We
  will be right here when you get back.  What vectors were you able to
  create?  Do you think you could create the vector 
  \[\vect{w}=\colvector{13\\15\\5\\-17\\2\\25}\]
  ``suitable'' choice of four scalars?
  
  Do you think you could create \textit{any} possible vector from
  $\complex{6}$ by choosing the proper scalars?  These questions are
  very fundamental, and time spent considering them \textit{now} will
  prove beneficial later.
\end{example}

\begin{example}
  The system of $m=\answer{3}$ linear equations
  \begin{align*}
    -7x_1 -6 x_2 - 12x_3 &=-33\\
    5x_1  + 5x_2 + 7x_3 &=24\\
    x_1 +4x_3 &=5
  \end{align*}
  can be rewritten as the vector equality
  \[
    \colvector{\answer{-7}x_1 -6 x_2 - 12x_3\\ 5x_1  + 5x_2 + 7x_3\\ x_1 +4x_3}
    =
    \colvector{-33\\\answer{24}\\5}
  \]
  Now we will break up the linear expressions on the left, first using vector addition,
  \[
    \colvector{-7x_1\\ 5x_1\\x_1}+
    \colvector{-6 x_2\\5x_2\\0x_2}+
    \colvector{-12x_3\\7x_3\\4x_3}
    =
    \colvector{-33\\24\\5}
  \]
  Now we can rewrite each of these vectors as a scalar multiple of a
  fixed vector, where the scalar is one of the unknown variables,
  converting the left-hand side into a linear combination
  \[
    x_1\colvector{\answer{-7}\\5\\1}+
    x_2\colvector{-6\\5\\0}+
    x_3\colvector{-12\\7\\4}
    =
    \colvector{-33\\24\\5}
  \]
  
  \begin{question}
    What is the solution to this system?

    We can now interpret the problem of solving the system of
    equations as determining values for the scalar multiples that make
    the vector equation true.  We can determine that this system has
    only one solution: a quick way to see this is to row-reduce the
    coefficient matrix to the $3\times 3$ identity matrix and apply
    \ref{theorem:NMRRI} to determine that the coefficient matrix is
    nonsingular.  Then \ref{theorem:NMUS} tells us that the system of
    equations has a unique solution.  This solution is
    \begin{align*}
      x_1 = -3&&x_2 = 5&&x_3 = \answer{2}
    \end{align*}

    \begin{feedback}[correct]
      So, in the context of this example, we can express the fact that
      these values of the variables are a solution by writing the
      linear combination,
      \[
        (-3)\colvector{-7\\5\\1}+
        (5)\colvector{-6\\5\\0}+
        (2)\colvector{-12\\7\\4}
        =
        \colvector{-33\\24\\5}
      \]
      Furthermore, these are the \textit{only} three scalars that will
      accomplish this equality, since they come from a unique
      solution.

      Notice how the three vectors in this example are the columns of
      the coefficient matrix of the system of equations.  This is our
      first hint of the important interplay between the vectors that
      form the columns of a matrix, and the matrix itself.
    \end{feedback}
  \end{question}

\end{example}

\begin{example}
  As a vector equality, the system of linear equations
  \begin{align*}
    x_1 -x_2 +2x_3 & =1\\
    2x_1+ x_2 + x_3 & =8\\
    x_1 + x_2 & =5
  \end{align*}            
  can be written as
  \[
    \colvector{x_1 -x_2 +2x_3\\2x_1+ x_2 + x_3\\ x_1 + x_2}
    =
    \colvector{1\\8\\5}
  \]
  Now break up the linear expressions on the left, first using vector addition,
  \[
    \colvector{x_1\\2x_1\\x_1}+
    \colvector{-x_2\\x_2\\x_2}+
    \colvector{2x_3\\x_3\\0x_3}
    =
    \colvector{1\\\answer{8}\\5}
  \]
  Rewrite each of these vectors as a scalar multiple of a fixed
  vector, where the scalar is one of the unknown variables, converting
  the left-hand side into a linear combination
  \[
    x_1\colvector{1\\2\\1}+
    x_2\colvector{-1\\1\\1}+
    x_3\colvector{\answer{2}\\1\\0}
    =
    \colvector{1\\8\\5}
  \]
  Row-reducing the corresponding augmented matrix leads to the
  conclusion that the system is consistent and has free variables,
  hence infinitely many solutions.  So for example, the two solutions
  \begin{align*}
    x_1 = 2&&x_2 = 3&&x_3 = 1\\
    x_1 = 3&&x_2 = 2&&x_3 = 0
  \end{align*}
  can be used together to say that,
  \[
    (2)\colvector{1\\2\\1}+
    (3)\colvector{-1\\1\\1}+
    (1)\colvector{2\\1\\0}
    =
    \colvector{1\\8\\5}
    =
    (3)\colvector{1\\2\\1}+
    (2)\colvector{-1\\1\\1}+
    (0)\colvector{2\\1\\0}
  \]
  Ignoring the middle of this equation and moving all the terms to the left-hand side yields
  \[
    (2)\colvector{1\\2\\1}+
    (3)\colvector{-1\\1\\1}+
    (1)\colvector{2\\1\\0}+
    (-3)\colvector{1\\2\\1}+
    (-2)\colvector{-1\\1\\1}+
    (-0)\colvector{2\\1\\0}
    =
    \colvector{0\\0\\0}
  \]
  Regrouping gives
  \[
    (-1)\colvector{1\\2\\1}+
    (1)\colvector{-1\\1\\1}+
    (1)\colvector{2\\1\\0}
    =
    \colvector{0\\0\\0}
  \]
  Notice that these three vectors are the columns of the coefficient
  matrix for the original system of equations.  This equality says
  there is a linear combination of those columns that equals the
  vector of all zeros.  Give it some thought, but this says that
  \begin{align*}
    x_1=-1&&x_2=1&&x_3=1
  \end{align*}
  is a \wordChoice{\choice{trivial}\choice[correct]{nontrivial}}
  solution to the homogeneous system of equations with the coefficient
  matrix for the original system.  In particular, this demonstrates
  that this coefficient matrix is
  \wordChoice{\choice{nonsingular}\choice[correct]{singular}}.
\end{example}

There is a lot going on in the last two examples.  Come back to them
in a while and make some connections with the intervening material.
For now, we will summarize and explain some of this behavior with a
theorem.

\begin{theorem}[Solutions to Linear Systems are Linear Combinations]
  \label{theorem:SLSLC}
  Denote the columns of the $m\times n$ matrix $A$ as the vectors
  $\vectorlist{A}{n}$.  Then $\vect{x}\in\complex{n}$ is a solution to
  the linear system of equations $\linearsystem{A}{\vect{b}}$ if and
  only if $\vect{b}$ equals the linear combination of the columns of
  $A$ formed with the entries of $\vect{x}$,
  \[
    \vectorentry{\vect{x}}{1}\vect{A}_1+
    \vectorentry{\vect{x}}{2}\vect{A}_2+
    \vectorentry{\vect{x}}{3}\vect{A}_3+
    \cdots+
    \vectorentry{\vect{x}}{n}\vect{A}_n
    =
    \vect{b}
  \]

  \begin{proof}
    The proof of this theorem is as much about a change in notation as
    it is about making logical deductions.  Write the system of
    equations $\linearsystem{A}{\vect{b}}$ as
    \begin{align*}
      a_{11}x_1+a_{12}x_2+a_{13}x_3+\dots+a_{1n}x_n&=b_1\\
      a_{21}x_1+a_{22}x_2+a_{23}x_3+\dots+a_{2n}x_n&=b_2\\
      a_{31}x_1+a_{32}x_2+a_{33}x_3+\dots+a_{3n}x_n&=b_3\\
      \vdots&\\
      a_{m1}x_1+a_{m2}x_2+a_{m3}x_3+\dots+a_{mn}x_n&=b_m
    \end{align*}
    
    Notice then that the entry of the coefficient matrix $A$ in row $i$
    and column $j$ has two names: $a_{ij}$ as the coefficient of $x_j$
    in equation $i$ of the system and $\vectorentry{\vect{A}_j}{i}$ as
    the $i$-th entry of the column vector in column $j$ of the
    coefficient matrix $A$.  Likewise, entry $i$ of $\vect{b}$ has two
    names: $b_i$ from the linear system and $\vectorentry{\vect{b}}{i}$
    as an entry of a vector.  Our theorem is an equivalence so we need
    to prove both directions.
    
    ($\Leftarrow$) Suppose we have the vector equality between
    $\vect{b}$ and the linear combination of the columns of
    $\answer{A}$.  Then for $1\leq i\leq m$,
    \begin{align*}
      b_i
      &=\vectorentry{\vect{b}}{i}&&\ref{definition:CV}\\
      &=\vectorentry{
        \vectorentry{\vect{x}}{1}\vect{A}_1+
        \vectorentry{\vect{x}}{2}\vect{A}_2+
        \vectorentry{\vect{x}}{3}\vect{A}_3+
        \cdots+
        \vectorentry{\vect{x}}{n}\vect{A}_n
        }{i}&&\text{Hypothesis}\\
      &=
        \vectorentry{\vectorentry{\vect{x}}{1}\vect{A}_1}{i}+
        \vectorentry{\vectorentry{\vect{x}}{2}\vect{A}_2}{i}+
        \vectorentry{\vectorentry{\vect{x}}{3}\vect{A}_3}{i}+
        \cdots+
        \vectorentry{\vectorentry{\vect{x}}{n}\vect{A}_n}{i}
                                 &&\ref{definition:CVA}\\
      &=
        \vectorentry{\vect{x}}{1}\vectorentry{\vect{A}_1}{i}+
        \vectorentry{\vect{x}}{2}\vectorentry{\vect{A}_2}{i}+
        \vectorentry{\vect{x}}{3}\vectorentry{\vect{A}_3}{i}+
        \cdots+
        \vectorentry{\vect{x}}{n}\vectorentry{\vect{A}_n}{i}
                                 &&\ref{definition:CVSM}\\
      &=
        \vectorentry{\vect{x}}{1}a_{i1}+
        \vectorentry{\vect{x}}{2}a_{i2}+
        \vectorentry{\vect{x}}{3}a_{i3}+
        \cdots+
        \vectorentry{\vect{x}}{n}a_{in}
                                 &&\ref{definition:CV}\\
      &=
        a_{i1}\vectorentry{\vect{x}}{1}+
        a_{i2}\vectorentry{\vect{x}}{2}+
        a_{i3}\vectorentry{\vect{x}}{3}+
        \cdots+
        a_{in}\vectorentry{\vect{x}}{n}
                                 &&\ref{property:CMCN}
    \end{align*}
    This says that the entries of $\vect{x}$ form a solution to
    equation $\answer{i}$ of $\linearsystem{A}{\vect{b}}$ for all
    $1\leq i\leq m$, in other words, $\vect{x}$ is a solution to
    $\linearsystem{A}{\vect{b}}$.

    ($\Rightarrow$) Suppose now that $\vect{x}$ is a solution to the
    linear system $\linearsystem{A}{\vect{b}}$.  Then for all
    $1\leq i\leq m$,
    \begin{align*}
      \vectorentry{\vect{b}}{i}
      &=b_i&&\ref{definition:CV}\\
      &=
        a_{i1}\vectorentry{\vect{x}}{1}+
        a_{i2}\vectorentry{\vect{x}}{2}+
        a_{i3}\vectorentry{\vect{x}}{3}+
        \cdots+
        a_{in}\vectorentry{\vect{x}}{n}
           &&\text{Hypothesis}\\
      &=
        \vectorentry{\vect{x}}{1}a_{i1}+
        \vectorentry{\vect{x}}{2}a_{i2}+
        \vectorentry{\vect{x}}{3}a_{i3}+
        \cdots+
        \vectorentry{\vect{x}}{n}a_{in}
           &&\ref{property:CMCN}\\
      &=
        \vectorentry{\vect{x}}{1}\vectorentry{\vect{A}_1}{i}+
        \vectorentry{\vect{x}}{2}\vectorentry{\vect{A}_2}{i}+
        \vectorentry{\vect{x}}{3}\vectorentry{\vect{A}_3}{i}+
        \cdots+
        \vectorentry{\vect{x}}{n}\vectorentry{\vect{A}_n}{i}
           &&\ref{definition:CV}\\
      &=
        \vectorentry{\vectorentry{\vect{x}}{1}\vect{A}_1}{i}+
        \vectorentry{\vectorentry{\vect{x}}{2}\vect{A}_2}{i}+
        \vectorentry{\vectorentry{\vect{x}}{3}\vect{A}_3}{i}+
        \cdots+
        \vectorentry{\vectorentry{\vect{x}}{n}\vect{A}_n}{i}
           &&\ref{definition:CVSM}\\
      &=\vectorentry{
        \vectorentry{\vect{x}}{1}\vect{A}_1+
        \vectorentry{\vect{x}}{2}\vect{A}_2+
        \vectorentry{\vect{x}}{3}\vect{A}_3+
        \cdots+
        \vectorentry{\vect{x}}{n}\vect{A}_n
        }{i}&&\ref{definition:CVA}
    \end{align*}
    So the entries of the vector $\vect{b}$, and the entries of the
    vector that is the linear combination of the columns of
    $\answer{A}$, agree for all $1\leq i\leq m$.  By
    \ref{definition:CVE} we see that the two vectors are equal, as
    desired.
\end{proof}
\end{theorem}

\begin{example}
  In other words, this theorem tells us that solutions to systems of
  equations are linear combinations of the $n$ column vectors of the
  coefficient matrix ($\vect{A}_j$) which yield the constant vector
  $\vect{b}$.  Or said another way, a solution to a system of
  equations $\linearsystem{A}{\vect{b}}$ is an answer to the question
  ``How can I form the vector $\vect{b}$ as a linear combination of
  the \wordChoice{\choice{rows}\choice[correct]{columns}} of $A$?''
\end{example}

\end{document}

