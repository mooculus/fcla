\documentclass{ximera}

% These macros are automatically generated from the "macros"
% XML element.  Make permanent edits there.
%
% History
%   2004/01/01  Initiated for FCLA, evolved from there
%   2006/09/17  Converted  _, ^  to \sb, \sp for TeX4ht
%   2014/02/01  Updated for MathBook XML projects
%               Obsolete in FCLA: \codeindent, \computerfont, \define
%               Change: MathJax wants \lt, so replaced by \lteval
%   2014/02/22  New: \orderof, \reals, \per
%   2015/08/16  Incorporated into MathBook XML version of FCLA
%
%%%%%%%%%%%%%%%%%%%%%
%
%     Conveniences
%
%%%%%%%%%%%%%%%%%%%%%
%
%  Order of (asymptotically limit of fraction is 1)
%  Usage: \orderof{some function}
%
\newcommand{\orderof}[1]{\sim #1}
%
%  Integers
%  Usage:  \Z
\newcommand{\Z}{\mathbb{Z}}
%
%  Real numbers, as set of scalars
%  Usage:  \reals
\newcommand{\reals}{\mathbb{R}}
%
%  n-space over real field
%  Usage: \complex{integer-dimension}
\newcommand{\real}[1]{\mathbb{R}^{#1}}
%
%  Complex numbers, as set of scalars
%  Usage:  \complexes
\newcommand{\complexes}{\mathbb{C}}
%
%  n-space over complex field
%  Usage: \complex{integer-dimension}
\newcommand{\complex}[1]{\mathbb{C}^{#1}}
%
%  Complex conjugation (scalar, vector, matrix)
%  Usage: \conjugate{object}
\newcommand{\conjugate}[1]{\overline{#1}}
%
%  Complex number modulus
%  Usage: \modulus{a+bi}
%  Presumes math mode
\newcommand{\modulus}[1]{\left\lvert#1\right\rvert}
%
%  Zero vector
%  Usage: \zerovector
\newcommand{\zerovector}{\vect{0}}
%
%  Zero matrix
%  Usage: \zeromatrix, use a subscript when size is important
\newcommand{\zeromatrix}{\mathcal{O}}
%
%  Inner product (brackets, not quadratic form)
%  Usage: \innerproduct{a-vector}{a-vector}
\newcommand{\innerproduct}[2]{\left\langle#1,\,#2\right\rangle}
%
%  Norm of a vector
%  Usage: \norm{a-vector}
\newcommand{\norm}[1]{\left\lVert#1\right\rVert}
%
%  Dimension
%  Usage: \dimension{vector-space-letter}
\newcommand{\dimension}[1]{\dim\left(#1\right)}
%
%  Nullity
%  Usage: \nullity{matrix-or-lintrans-letter}
\newcommand{\nullity}[1]{n\left(#1\right)}
%
%  Rank
%  Usage: \rank{matrix-or-lintrans-letter}
\newcommand{\rank}[1]{r\left(#1\right)}
%
%  Direct sum
%  Usage: \ds between a couple of subspaces
%
\newcommand{\ds}{\oplus}
%
%  Determinant of a matrix (functional)
%  Usage: \detname{A}
\newcommand{\detname}[1]{\det\left(#1\right)}
%
%  Determinant of a matrix (vertical bars)
%  Usage: \detbars{A}
\newcommand{\detbars}[1]{\left\lvert#1\right\rvert}
%
%  Trace of a Matrix
%  Usage: \trace{matrix name}
\newcommand{\trace}[1]{t\left(#1\right)}
%
%  Square Root of a Matrix
%  Usage: \sr{a-matrix}
\newcommand{\sr}[1]{#1^{1/2}}
%
%%%%%%%%%%%%%%%%%%%%%
%
%     Subspace Constructions
%
%%%%%%%%%%%%%%%%%%%%%
%
%  Span of a set of vectors
%  \span and \sp are used by TeX for other things
%  Usage: \spn{set-of-vectors}
\newcommand{\spn}[1]{\left\langle#1\right\rangle}
%
%  Null space of a matrix
%  Usage:  \nsp{A}
\newcommand{\nsp}[1]{\mathcal{N}\!\left(#1\right)}
%
%  Column space of a matrix
%  Usage:  \csp{A}
\newcommand{\csp}[1]{\mathcal{C}\!\left(#1\right)}
%
%  Row space of a matrix
%  Usage:  \rsp{A}
\newcommand{\rsp}[1]{\mathcal{R}\!\left(#1\right)}
%
%  Left null space of a matrix
%  Usage:  \lns{A}
\newcommand{\lns}[1]{\mathcal{L}\!\left(#1\right)}
%
%  Orthogonal complement of a vector space
%  Avoiding TeX's \perp
%  Usage:  \per{A}
\newcommand{\per}[1]{#1^\perp}
%
%%%%%%%%%%%%%%%%%%%%%
%
%     Systems of Equations
%
%%%%%%%%%%%%%%%%%%%%%
%
%  In-line form of an augmented matrix for a system of equations
%  Usage: \augmented{coefficient-matrix}{constant-vector}
\newcommand{\augmented}[2]{\left\lbrack\left.#1\,\right\rvert\,#2\right\rbrack}
%
%  Notation for a linear system before introducing matrix multiplication
%  Usage: \linearsystem{coefficient-matrix}{constant-vector}
\newcommand{\linearsystem}[2]{\mathcal{LS}\!\left(#1,\,#2\right)}
%
%  Notation for a homogenous system before introducing matrix multiplication
%  Usage: \homosystem{coefficient-matrix}
\newcommand{\homosystem}[1]{\linearsystem{#1}{\zerovector}}
%
%%%%%%%%%%%%%%%%%%%%%
%
%     Row Operations, Echelon Form
%
%%%%%%%%%%%%%%%%%%%%%
%
% Row operations on matrices
%
% Three commands to shorten up descriptions of gaussian elimination
%
% Usage: \rowopswap{row-i}{row-j}
% Usage: \rowopmult{scalar}{row-i}
% Usage: \rowopadd{scalar}{row-multiplied}{row-added-to}
\newcommand{\rowopswap}[2]{R_{#1}\leftrightarrow R_{#2}}
\newcommand{\rowopmult}[2]{#1R_{#2}}
\newcommand{\rowopadd}[3]{#1R_{#2}+R_{#3}}
%
% Mark leading 1's in echelon form with fbox
% Usage: \leading{a-1-usually}
\newcommand{\leading}[1]{\fbox{#1}}
%
%  Row-reduce arrow
%  Usage:  \rref inbetween a matrix and its reduced row-echelon form
\newcommand{\rref}{\xrightarrow{\text{RREF}}}
%
%  Elementary Matrices
%  Usage: \elemswap{subscript}{subscript}
%  Usage: \elemmult{scalar}{subscript}
%  Usage: \elemadd{scalar}{subscript-mult}{subscript-target}
%
\newcommand{\elemswap}[2]{E_{#1,#2}}
\newcommand{\elemmult}[2]{E_{#2}\left(#1\right)}
\newcommand{\elemadd}[3]{E_{#2,#3}\left(#1\right)}
%
%%%%%%%%%%%%%%%%%%%%%
%
%     2-D Constructions (Lists, Vectors, Matrices)
%
%%%%%%%%%%%%%%%%%%%%%
%
%  A list of scalars of generic length
%  Usage:  \scalarlist{scalar letter}{terminal subscript}
\newcommand{\scalarlist}[2]{{#1}_{1},\,{#1}_{2},\,{#1}_{3},\,\ldots,\,{#1}_{#2}}
%
%  Vector styling, bold (or use wiggles, arrows, whatever)
%  Subscripts go outside this construction
%  Usage: \vect{a symbol to use as a vector}
%  Have to already be in math mode
%
\newcommand{\vect}[1]{\mathbf{#1}}
%
%  A column vector
%  Usage: \colvector{list-delimited-by-\\}
%
\newcommand{\colvector}[1]{\begin{bmatrix}#1\end{bmatrix}}
%
%  A generic vector with components
%  Usage: \vectorcomponents{component-letter}{final-subscript}
\newcommand{\vectorcomponents}[2]{\colvector{#1_{1}\\#1_{2}\\#1_{3}\\\vdots\\#1_{#2}}}
%
%  A list of vectors of generic length
%  Usage:  \vectorlist{vector letter}{terminal subscript}
\newcommand{\vectorlist}[2]{\vect{#1}_{1},\,\vect{#1}_{2},\,\vect{#1}_{3},\,\ldots,\,\vect{#1}_{#2}}
%
%  Vector entries, entry i of vector v
%  (vector-expession still needs \vect, etc.)
%  Usage:  \vectorentry{vector-expression}{single-subscript}
\newcommand{\vectorentry}[2]{\left\lbrack#1\right\rbrack_{#2}}
%
%  Matrix entries, entry i,j of matrix A
%  Usage:  \matrixentry{matrix-expression}{paired-subscripts}
%
\newcommand{\matrixentry}[2]{\left\lbrack#1\right\rbrack_{#2}}
%
%  A generic linear combination
%  Usage:  \lincombo{scalar letter}{vector letter}{terminal subscript}
\newcommand{\lincombo}[3]{#1_{1}\vect{#2}_{1}+#1_{2}\vect{#2}_{2}+#1_{3}\vect{#2}_{3}+\cdots +#1_{#3}\vect{#2}_{#3}}
%
%  Matrix, column by column, as vectors
%  Usage:  \matrixcolumns{matrix letter}{terminal subscript}
\newcommand{\matrixcolumns}[2]{\left\lbrack\vect{#1}_{1}|\vect{#1}_{2}|\vect{#1}_{3}|\ldots|\vect{#1}_{#2}\right\rbrack}
%
%%%%%%%%%%%%%%%%%%%%%
%
%     Special Matrices
%
%%%%%%%%%%%%%%%%%%%%%
%
%  Transpose of a matrix
%  Usage:  \transpose{A}
\newcommand{\transpose}[1]{#1^{t}}
%
%  Inverse of a matrix
%  Usage:  \inverse{A}
\newcommand{\inverse}[1]{#1^{-1}}
%
%  Submatrix (for minors, determinants)
%  Usage: \submatrix{matrix-name}{delete-row}{delete-col}
\newcommand{\submatrix}[3]{#1\left(#2|#3\right)}
%
%  Adjoint of a matrix (twice)
%  This macro is a convenience to call \transpose and \conjugate properly
%  It shouldn't need to be modified (or mathematical meanings will change)
%  Usage:  \adj{A}
\newcommand{\adj}[1]{\transpose{\left(\conjugate{#1}\right)}}
%
%  This macro controls the symbol used for the adjoint
%  It can be edited to taste
%  Usage:  \adjoint{A}
\newcommand{\adjoint}[1]{#1^\ast}
%
%%%%%%%%%%%%%%%%%%%%%
%
%     Sets
%
%%%%%%%%%%%%%%%%%%%%%
%
%  A convenience for simple sets
%  Usage:  \set{list of element}
\newcommand{\set}[1]{\left\{#1\right\}}
%
%  Sets with vertical bar, "such that", sized for objects, not condition
%  Usage:  \setparts{objects}{condition}
%
%%\newcommand{\setparts}[2]{\left\{ #1\mid#2\right\}}
%%\newcommand{\setparts}[2]{\left\{\left. #1\right\rvert#2\right\}}
\newcommand{\setparts}[2]{\left\lbrace#1\,\middle|\,#2\right\rbrace}
%
%  Set Cardinality
%  Usage:  \card{a-set-letter}
\newcommand{\card}[1]{\left\lvert#1\right\rvert}
%
%  Set Union
%  Use \cup
%
%  Set Intersection
%  Use \cap
%
%  Set Complement
%  Usage:  \setcomplement{a-set-letter}
\newcommand{\setcomplement}[1]{\overline{#1}}
%
%%%%%%%%%%%%%%%%%%%%%
%
%     Eigenvalues and Eigenspaces
%
%%%%%%%%%%%%%%%%%%%%%
%
%  Characteristic polynomial
%  Usage: \charpoly{matrix-letter}{variable-letter}
\newcommand{\charpoly}[2]{p_{#1}\left(#2\right)}
%
%  Eigenspace
%  Usage: \eigenspace{matrix-letter}{eigenvalue-letter}
\newcommand{\eigenspace}[2]{\mathcal{E}_{#1}\left(#2\right)}
%
%  2013/10/03 Including ampersands is problematic here, 
%  think about fixes later
%  2014/02/22 Limited testing, seems &amp; is fine for HTML and LaTeX
%  2016-07-20 only employed in Archetypes, MBX has gather/align override
%  Eigensystem (presumes wrapped in an mrow within md)
%  Usage: \eigensystem{matrixletter}{eigenvalue}{list of basis vectors}
\newcommand{\eigensystem}[3]{\lambda&amp;=#2&amp;\eigenspace{#1}{#2}&amp;=\spn{\set{#3}}} 
%
%  Generalized Eigenspace
%  Usage: \geneigenspace{lin-trans-letter}{eigenvalue-letter}
\newcommand{\geneigenspace}[2]{\mathcal{G}_{#1}\left(#2\right)}
%
%  Algebraic multiplicty
%  Usage: \algmult{matrix-letter}{eigenvalue-letter}
\newcommand{\algmult}[2]{\alpha_{#1}\left(#2\right)}
%
%  Geometric multiplicty
%  Usage: \geomult{matrix-letter}{eigenvalue-letter}
\newcommand{\geomult}[2]{\gamma_{#1}\left(#2\right)}
%
%  Index (of eigenvalue)
%  Usage: \indx{matrix-letter}{eigenvalue-letter}
\newcommand{\indx}[2]{\iota_{#1}\left(#2\right)}
%
%%%%%%%%%%%%%%%%%%%%%
%
%     Linear Transformations
%
%%%%%%%%%%%%%%%%%%%%%
%
%  MathJax defines \lt to ease XML confusion
%
%  Linear transformation definition
%  Usage: \ltdefn{name-letter}{domain}{range}
\newcommand{\ltdefn}[3]{#1\colon #2\rightarrow#3}
%
%  Linear transformation evaluation
%  Usage: \lteval{name-letter}{input}
%  Replaces old \lt desired by MathJax
\newcommand{\lteval}[2]{#1\left(#2\right)}
%
% Linear transformation inverse
%  Usage: \ltinverse{name-letter}
\newcommand{\ltinverse}[1]{#1^{-1}}
%
%  Linear transformation restriction
%  Usage: \restrict{name-letter}{subspace-letter}
\newcommand{\restrict}[2]{{#1}|_{#2}}
%
%  Linear transformation preimage
%  Usage: \preimage{name-letter}{codomain-element}
\newcommand{\preimage}[2]{#1^{-1}\left(#2\right)}
%
%  Range of a linear transformation
%  TeX uses \range for something else
%  Usage:  \rng{T}
\newcommand{\rng}[1]{\mathcal{R}\!\left(#1\right)}
%
%  Kernel of a linear transformation
%  TeX uses \ker to do something different
%  Usage:  \krn{T}
\newcommand{\krn}[1]{\mathcal{K}\!\left(#1\right)}
%
%  Linear transformation composition
%  Usage: \compose{function-name}{function-name}
\newcommand{\compose}[2]{{#1}\circ{#2}}
%
%  Vector space of linear transformations
%  Usage: \vslt{domains}{codomains}
%  Presumes math mode
\newcommand{\vslt}[2]{\mathcal{LT}\left(#1,\,#2\right)}
%
%%%%%%%%%%%%%%%%%%%%%
%
%     Vector and Matrix Representations
%
%%%%%%%%%%%%%%%%%%%%%
%
%  Isomorphism symbol
%  Usage: \isomorphic
\newcommand{\isomorphic}{\cong}
%
%  Similarity
%  Usage: \similar{inner-matrix}{outer-invertible-matrix}
%  Rearranging this will not "fix" all desired changes throughout
%
\newcommand{\similar}[2]{\inverse{#2}#1#2}
%
%  Vector representation function name
%  Usage: \vectrepname{basis-letter}
\newcommand{\vectrepname}[1]{\rho_{#1}}
%
%  Vector representation output
%  Usage: \vectrep{basis-letter}{input}
\newcommand{\vectrep}[2]{\lteval{\vectrepname{#1}}{#2}}
%
%  Vector representation inverse function name
%  (Added later, not used consistently in FCLA)
%  Usage: \vectrepinvname{basis-letter}
\newcommand{\vectrepinvname}[1]{\ltinverse{\vectrepname{#1}}}
%
%  Vector representation inverse output
%  Usage: \vectrepinv{basis-letter}{input}
\newcommand{\vectrepinv}[2]{\lteval{\ltinverse{\vectrepname{#1}}}{#2}}
%
%  Matrix representation
%  Usage: \matrixrep{transformation-letter}{domain-basis-letter}{codomain-basis-letter}
\newcommand{\matrixrep}[3]{M^{#1}_{#2,#3}}
%
%  Matrix representation column-by-colum
%  2016-07-20 only employed once?
%  Usage: \matrixrepcolumns{transformation-letter}{codomain-basis-letter}{codomain-basis-vector-letter}{final-index}
\newcommand{\matrixrepcolumns}[4]{\left\lbrack \left.\vectrep{#2}{\lteval{#1}{\vect{#3}_{1}}}\right|\left.\vectrep{#2}{\lteval{#1}{\vect{#3}_{2}}}\right|\left.\vectrep{#2}{\lteval{#1}{\vect{#3}_{3}}}\right|\ldots\left|\vectrep{#2}{\lteval{#1}{\vect{#3}_{#4}}}\right.\right\rbrack}
%
%  Change of basis matrix
%  Usage: \cbm{domain-basis-letter}{codomain-basis-letter}
\newcommand{\cbm}[2]{C_{#1,#2}}
%
%%%%%%%%%%%%%%%%%%%%%
%
%     Canonical Forms
%
%%%%%%%%%%%%%%%%%%%%%
%
%  Jordan blocks
%  Usage: \jordan{size}{diagonal-element}
\newcommand{\jordan}[2]{J_{#1}\left(#2\right)}
%
%%%%%%%%%%%%%%%%%%%%%
%
%     Hadamard Matrices
%     Contributed by Elizabeth Million
%
%%%%%%%%%%%%%%%%%%%%%
%
%  Hadamard Product
%  Usage: \hadamard{a-matrix}{a-matrix}
\newcommand{\hadamard}[2]{#1\circ #2}
%
%  Hadamard identity matrix
%  Usage: \hadamardidentity{paired-subscripts-size-of-matrix}
\newcommand{\hadamardidentity}[1]{J_{#1}}
%
%  Hadamard inverse matrix
%  Usage: \hadamardinverse{matrix-expression}
\newcommand{\hadamardinverse}[1]{\widehat{#1}}


\title{Casting out vectors}

\begin{document}
\begin{abstract}
  With a nontrivial relation of linear dependence in hand, we are able
  to ``toss out'' one of these vectors and span the same set with
  fewer vectors.
\end{abstract}
\maketitle

We consider the span of a handful of vectors.  With a nontrivial
relation of linear dependence in hand, we are able to ``toss out'' one
of these vectors and span the same set with fewer vectors!  We do have
to take some care as to just which vector we toss out.  In the next
example, we will be more methodical about just how we choose to
eliminate vectors from a linearly dependent set while preserving a
span.

\begin{example}[Casting out vectors]
  We begin with a set $S$ containing seven vectors from $\complex{4}$,
  \[
    S=\set{
      \colvector{1\\2\\0\\-1},\,
      \colvector{4\\8\\0\\-4},\,
      \colvector{0\\-1\\2\\2},\,
      \colvector{-1\\3\\-3\\4},\,
      \colvector{0\\9\\-4\\8},\,
      \colvector{7\\-13\\12\\-31},\,
      \colvector{-9\\7\\-8\\37}
    }
  \]
  and define $W=\spn{S}$.
  
  The set $S$ is obviously linearly dependent by \ref{theorem:MVSLD},
  since we have $n=7$ vectors from $\complex{4}$.  So we can slim down
  $S$ some, and still create $W$ as the span of a smaller set of
  vectors.

  As a device for identifying relations of linear dependence among the
  vectors of $S$, we place the seven column vectors of $S$ into a
  matrix as columns,
  \[
    A=\matrixcolumns{A}{7}=
    \begin{bmatrix}
      1 & 4 & 0 & -1 & 0 & 7 &  -9 \\
      2 & 8 &  -1 & 3 & 9 &  -13 & 7\\
      0 & 0 &  2 & -3 & -4 & 12 &  -8\\
      -1 &  -4 & 2 & 4 & 8 &  -31 & 37
    \end{bmatrix}
  \]
  By \ref{theorem:SLSLC} a nontrivial solution to $\homosystem{A}$
  will give us a nontrivial relation of linear dependence
  (\ref{definition:RLDCV}) on the \wordChoice{\choice[correct]{columns}\choice{rows}} of $A$ (which are the
  elements of the set $S$).  The row-reduced form for $A$ is the
  matrix
  \[
    B=\begin{bmatrix}
      \leading{1} & 4 & 0 & 0 & 2 & 1 & -3\\
      0 & 0 & \leading{1} &  0 & 1 &  -3 & 5\\
      0 & 0 &  0 & \leading{1} & 2 &  -6 & 6\\
      0 & 0 &  0 & 0 & 0 & 0 & 0
    \end{bmatrix}
  \]
  so we can easily create solutions to the homogeneous system
  $\homosystem{A}$ using the free variables $x_2,\,x_5,\,x_6,\,x_7$.
  Any such solution will provide a relation of linear dependence on
  the \wordChoice{\choice[correct]{columns}\choice{rows}} of $B$.
  These solutions will allow us to solve for one column vector as a
  linear combination of some others, in the spirit of
  \ref{theorem:DLDS}, and remove that vector from the set.  We will
  set about forming these linear combinations methodically.

  Set the free variable $x_2=1$, and set the other free variables to
  zero.  Then a solution to $\linearsystem{A}{\zerovector}$ is
  \[
    \vect{x}=\colvector{-4\\1\\0\\0\\0\\0\\0}
  \]
  which can be used to create the linear combination
  \[
    (-4)\vect{A}_1+
    \answer{1}\vect{A}_2+
    0\vect{A}_3+
    0\vect{A}_4+
    0\vect{A}_5+
    0\vect{A}_6+
    0\vect{A}_7
    =\zerovector
  \]
  This can then be arranged and solved for $\vect{A}_2$, resulting in
  $\vect{A}_2$ expressed as a linear combination of
  $\set{\vect{A}_1,\,\vect{A}_3,\,\vect{A}_4}$,
  \[
    \vect{A}_2=
    \answer{4}\vect{A}_1+
    0\vect{A}_3+
    0\vect{A}_4
  \]
  This means that $\vect{A}_2$ is surplus, and we can create $W$ just
  as well with a smaller set with this vector removed,
  \[
    W=\spn{\set{\vect{A}_1,\,\vect{A}_3,\,\vect{A}_4,\,\vect{A}_5,\,\vect{A}_6,\,\vect{A}_7}}
  \]

  Technically, this set equality for $W$ requires a proof, but we will
  bypass this requirement here, and in the next few paragraphs.

  Now, set the free variable $x_5=1$, and set the other free variables
  to zero.  Then a solution to $\linearsystem{B}{\zerovector}$ is
  \[
    \vect{x}=\colvector{-2\\0\\-1\\-2\\1\\0\\0}
  \]
  which can be used to create the linear combination
  \[
    (-2)\vect{A}_1+
    0\vect{A}_2+
    (-1)\vect{A}_3+
    (-2)\vect{A}_4+
    1\vect{A}_5+
    0\vect{A}_6+
    0\vect{A}_7
    =\zerovector
  \]

  This can then be arranged and solved for $\vect{A}_5$, resulting in
  $\vect{A}_5$ expressed as a linear combination of
  $\set{\vect{A}_1,\,\vect{A}_3,\,\vect{A}_4}$,
  \[
    \vect{A}_5=
    2\vect{A}_1+
    \answer{1}\vect{A}_3+
    2\vect{A}_4
  \]

  This means that $\vect{A}_5$ is surplus, and we can create $W$ just
  as well with a smaller set with this vector removed,
  \[
    W=\spn{\left\{\vect{A}_1,\,\vect{A}_3,\,\vect{A}_4,\,\vect{A}_6,\,\vect{A}_7\right\}}
  \]

  Do it again, set the free variable $x_6=1$, and set the other free
  variables to zero.  Then a solution to
  $\linearsystem{B}{\zerovector}$ is
  \[
    \vect{x}=\colvector{-1\\0\\3\\6\\0\\1\\0}
  \]
  which can be used to create the linear combination
  \[
    (-1)\vect{A}_1+
    0\vect{A}_2+
    3\vect{A}_3+
    6\vect{A}_4+
    0\vect{A}_5+
    1\vect{A}_6+
    0\vect{A}_7
    =\zerovector.
  \]

  This can then be arranged and solved for $\vect{A}_6$, resulting in
  $\vect{A}_6$ expressed as a linear combination of
  $\set{\vect{A}_1,\,\vect{A}_3,\,\vect{A}_4}$,
  \[
    \vect{A}_6=
    1\vect{A}_1+
    (-3)\vect{A}_3+
    (-6)\vect{A}_4
  \]
  This means that $\vect{A}_6$ is surplus, and we can create $W$ just
  as well with a smaller set with this vector removed,
  \[
    W=\spn{\set{\vect{A}_1,\,\vect{A}_3,\,\vect{A}_4,\,\vect{A}_7}}
  \]

  Set the free variable $x_7=1$, and set the other free variables to
  zero.  Then a solution to $\linearsystem{B}{\zerovector}$ is
  \[
    \vect{x}=\colvector{3\\0\\-5\\-6\\0\\0\\1}
  \]
  which can be used to create the linear combination
  \[
    3\vect{A}_1+
    0\vect{A}_2+
    (-5)\vect{A}_3+
    (-6)\vect{A}_4+
    0\vect{A}_5+
    0\vect{A}_6+
    1\vect{A}_7
    =\zerovector
  \]

  This can then be arranged and solved for $\vect{A}_7$, resulting in
  $\vect{A}_7$ expressed as a linear combination of
  $\set{\vect{A}_1,\,\vect{A}_3,\,\vect{A}_4}$,
  \[
    \vect{A}_7=
    (-3)\vect{A}_1+
    5\vect{A}_3+
    6\vect{A}_4
  \]

  This means that $\vect{A}_7$ is surplus, and we can create $W$ just
  as well with a smaller set with this vector removed,
  \[
    W=\spn{\set{\vect{A}_1,\,\vect{A}_3,\,\vect{A}_4}}
  \]

  \begin{question}
    Can we do this again?
    
    \begin{multipleChoice}
      \choice{Yes.}
      \choice[correct]{No.}
    \end{multipleChoice}

    \begin{feedback}[correct]
      You might have thought we could keep this up, but we have run out
      of free variables!  And not coincidentally, the set
      $\set{\vect{A}_1,\,\vect{A}_3,\,\vect{A}_4}$ is linearly
      independent (check this!).  It should be clear how each free
      variable was used to eliminate the a column from the set used to
      span the column space, as this will be the essence of the proof of
      the next theorem.
    \end{feedback}
  \end{question}
\end{example}

That example deserves your careful attention, since this important
example motivates the following very fundamental theorem.

\begin{theorem}[Basis of a Span]
  \label{theorem:BS}Suppose that $S=\set{\vectorlist{v}{n}}$ is a set
  of column vectors.  Define $W=\spn{S}$ and let $A$ be the matrix
  whose columns are the vectors from $S$.  Let $B$ be the reduced
  row-echelon form of $A$, with $D=\set{\scalarlist{d}{r}}$ the set of
  indices for the pivot columns of $B$.  Then
  \begin{enumerate}
  \item $T=\set{\vect{v}_{d_1},\,\vect{v}_{d_2},\,\vect{v}_{d_3},\,\ldots\,\vect{v}_{d_r}}$ is a linearly independent set.
  \item $W=\spn{T}$.
  \end{enumerate}

  \begin{proof}
    To prove that $T$ is linearly independent, begin with a relation of linear dependence on $T$,
    \[
      \zerovector=
      \alpha_1\vect{v}_{d_1}+\alpha_2\vect{v}_{d_2}+\alpha_3\vect{v}_{d_3}+\ldots+\alpha_r\vect{v}_{d_r}
    \]
    and we will try to conclude that the only possibility for the scalars $\alpha_i$ is that they are all zero.
    Denote the non-pivot columns of $B$ by $F=\set{\scalarlist{f}{n-r}}$.  Then we can preserve the equality by adding a big fat zero to the linear combination,
    \[
      \zerovector=
      \alpha_1\vect{v}_{d_1}+\alpha_2\vect{v}_{d_2}+\alpha_3\vect{v}_{d_3}+\ldots+\alpha_r\vect{v}_{d_r}+
      0\vect{v}_{f_1}+0\vect{v}_{f_2}+0\vect{v}_{f_3}+\ldots+0\vect{v}_{f_{n-r}}
    \]
    
    By \ref{theorem:SLSLC}, the scalars in this linear combination
    (suitably reordered) are a solution to the homogeneous system
    $\homosystem{A}$.  But notice that this is the solution obtained
    by setting each free variable to zero.  If we consider the
    description of a solution vector in the conclusion of
    \ref{theorem:VFSLS}, in the case of a homogeneous system, then we
    see that if all the free variables are set to zero the resulting
    solution vector is trivial (all zeros).  So it must be that
    $\alpha_i=0$, $1\leq i\leq r$.  This implies by
    \ref{definition:LICV} that $T$ is a linearly independent set.

    The second conclusion of this theorem is an equality of sets
    (\ref{definition:SE}).  Since $T$ is a subset of $S$, any linear
    combination of elements of the set $T$ can also be viewed as a
    linear combination of elements of the set $S$.  So
    $\spn{T}\subseteq\spn{S}=W$.  It remains to prove that
    $W=\spn{S}\subseteq\spn{T}$.

    For each $k$, $1\leq k\leq n-r$, form a solution $\vect{x}$ to
    $\homosystem{A}$ by setting the free variables as follows:
    \begin{align*}
      x_{f_1}&=0
      &
        x_{f_2}&=0
      &
        x_{f_3}&=0
      &
        \ldots&
      &
        x_{f_k}&=1
      &
        \ldots&
      &
        x_{f_{n-r}}&=0
    \end{align*}
    
    By \ref{theorem:VFSLS}, the remainder of this solution vector is given by,
    \begin{align*}
      x_{d_1}&=-\matrixentry{B}{1,f_k}
      &
        x_{d_2}&=-\matrixentry{B}{2,f_k}
      &
        x_{d_3}&=-\matrixentry{B}{3,f_k}
      &
        \dots&
      &
        x_{d_r}&=-\matrixentry{B}{r,f_k}
    \end{align*}
    
    From this solution, we obtain a relation of linear dependence on
    the columns of $A$,
    \[
      -\matrixentry{B}{1,f_k}\vect{v}_{d_1}
      -\matrixentry{B}{2,f_k}\vect{v}_{d_2}
      -\matrixentry{B}{3,f_k}\vect{v}_{d_3}
      -\ldots
      -\matrixentry{B}{r,f_k}\vect{v}_{d_r}
      +1\vect{v}_{f_k}
      =\zerovector
    \]
    which can be arranged as the equality
    \[
      \vect{v}_{f_k}=
      \matrixentry{B}{1,f_k}\vect{v}_{d_1}+
      \matrixentry{B}{2,f_k}\vect{v}_{d_2}+
      \matrixentry{B}{3,f_k}\vect{v}_{d_3}+
      \ldots+
      \matrixentry{B}{r,f_k}\vect{v}_{d_r}
    \]
    
    Now, suppose we take an arbitrary element, $\vect{w}$, of
    $W=\spn{S}$ and write it as a linear combination of the elements
    of $S$, but with the terms organized according to the indices in
    $D$ and $F$,
    \begin{align*}
      \vect{w}&=
                \alpha_1\vect{v}_{d_1}+
                \alpha_2\vect{v}_{d_2}+
                \ldots+
                \alpha_r\vect{v}_{d_r}+
                \beta_1\vect{v}_{f_1}+
                \beta_2\vect{v}_{f_2}+
                \ldots+
                \beta_{n-r}\vect{v}_{f_{n-r}}
    \end{align*}
    
    From the above, we can replace each $\vect{v}_{f_j}$ by a linear
    combination of the $\vect{v}_{d_i}$,
    \begin{align*}
      \vect{w}&=
                \alpha_1\vect{v}_{d_1}+
                \alpha_2\vect{v}_{d_2}+
                \ldots+
                \alpha_r\vect{v}_{d_r}+\\
              &\beta_1\left(
                \matrixentry{B}{1,f_1}\vect{v}_{d_1}+
                \matrixentry{B}{2,f_1}\vect{v}_{d_2}+
                \matrixentry{B}{3,f_1}\vect{v}_{d_3}+
                \ldots+
                \matrixentry{B}{r,f_1}\vect{v}_{d_r}
                \right)+\\
              &\beta_2\left(
                \matrixentry{B}{1,f_2}\vect{v}_{d_1}+
                \matrixentry{B}{2,f_2}\vect{v}_{d_2}+
                \matrixentry{B}{3,f_2}\vect{v}_{d_3}+
                \ldots+
                \matrixentry{B}{r,f_2}\vect{v}_{d_r}
                \right)+\\
              &\quad\quad\vdots\\
              &\beta_{n-r}\left(
                \matrixentry{B}{1,f_{n-r}}\vect{v}_{d_1}+
                \matrixentry{B}{2,f_{n-r}}\vect{v}_{d_2}+
                \matrixentry{B}{3,f_{n-r}}\vect{v}_{d_3}+
                \ldots+
                \matrixentry{B}{r,f_{n-r}}\vect{v}_{d_r}
                \right)
    \end{align*}
    With repeated applications of several of the properties of \ref{theorem:VSPCV} we can rearrange this expression as,
    \begin{align*}
      =&\ \left(
         \alpha_1+
         \beta_1\matrixentry{B}{1,f_1}+
         \beta_2\matrixentry{B}{1,f_2}+
         \beta_3\matrixentry{B}{1,f_3}+
         \ldots+
         \beta_{n-r}\matrixentry{B}{1,f_{n-r}}
         \right)\vect{v}_{d_1}+\\
       &\left(\alpha_2+
         \beta_1\matrixentry{B}{2,f_1}+
         \beta_2\matrixentry{B}{2,f_2}+
         \beta_3\matrixentry{B}{2,f_3}+
         \ldots+
         \beta_{n-r}\matrixentry{B}{2,f_{n-r}}
         \right)\vect{v}_{d_2}+\\
       &\quad\quad\vdots\\
       &\left(\alpha_r+
         \beta_1\matrixentry{B}{r,f_1}+
         \beta_2\matrixentry{B}{r,f_2}+
         \beta_3\matrixentry{B}{r,f_3}+
         \ldots+\beta_{n-r}\matrixentry{B}{r,f_{n-r}}
         \right)\vect{v}_{d_r}
    \end{align*}
    This mess expresses the vector $\vect{w}$ as a linear combination of the vectors in
    \[
      T=\set{\vect{v}_{d_1},\,\vect{v}_{d_2},\,\vect{v}_{d_3},\,\ldots\,\vect{v}_{d_r}}
    \]
    thus saying that $\vect{w}\in\spn{T}$.  Therefore, $W=\spn{S}\subseteq\spn{T}$.
  \end{proof}
\end{theorem}

In \ref{example:COV}, we tossed-out vectors one at a time.  But in
each instance, we rewrote the offending vector as a linear combination
of those vectors with the column indices of the pivot columns of the
reduced row-echelon form of the matrix of columns.  In the proof of
\ref{theorem:BS}, we accomplish this reduction in one big step.  In
\ref{example:COV} we arrived at a linearly independent set at exactly
the same moment that we ran out of free variables to exploit.  This
was not a coincidence, it is the substance of our conclusion of linear
independence in \ref{theorem:BS}.

Here is a straightforward application of \ref{theorem:BS}.

\begin{example}[Reducing a span in $\real{4}$]
  Begin with a set of five vectors from $\real{4}$,
  \[
    S=\set{
      \colvector{ 1 \\ 1 \\ 2 \\ 1},\,
      \colvector{ 2 \\ 2 \\ 4 \\ 2},\,
      \colvector{ 2 \\ 0 \\ -1 \\ 1},\,
      \colvector{ 7 \\ 1 \\ -1 \\ 4},\,
      \colvector{ 0 \\ 2 \\ 5 \\ 1}
    }
  \]
  and let $W=\spn{S}$.  To arrive at a (smaller) linearly independent
  set, follow the procedure described in \ref{theorem:BS}.  Place the
  vectors from $S$ into a matrix as columns, and row-reduce,
  \[
    \begin{bmatrix}
      1 & 2 & 2 & 7 & 0 \\
      1 & 2 & 0 & 1 & 2 \\
      2 & 4 & -1 & -1 & 5 \\
      1 & 2 & 1 & 4 & 1
    \end{bmatrix}
    \rref
    \begin{bmatrix}
      \leading{1} & 2 & 0 & 1 & 2 \\
      0 & 0 & \leading{1} & 3 & -1 \\
      0 & 0 & 0 & 0 & 0 \\
      0 & 0 & 0 & 0 & 0
    \end{bmatrix}
  \]

  Columns 1 and 3 are the pivot columns ($D=\set{1,\,3}$) so the set
  \[
    T=\set{
      \colvector{ 1 \\ 1 \\ 2 \\ 1},\,
      \colvector{ 2 \\ 0 \\ -1 \\ 1}
    }
  \]
  is linearly independent and $\spn{T}=\spn{S}=W$.  Boom!
  
  Since the reduced row-echelon form of a matrix is unique
  (\ref{theorem:RREFU}), the procedure of \ref{theorem:BS} leads us to
  a unique set $T$.  However, there is a wide variety of possibilities
  for sets $T$ that are linearly independent and which can be employed
  in a span to create $W$.  Without proof, we list two other
  possibilities:
  \begin{align*}
    T^{\prime}&=\set{
                \colvector{ 2 \\ 2 \\ 4 \\ 2},\,
    \colvector{ 2 \\ 0 \\ -1 \\ 1}
    }\\
    T^{*}&=\set{
           \colvector{3 \\ 1 \\ 1 \\ 2},\,
    \colvector{-1 \\ 1 \\ 3 \\ 0}
    }
  \end{align*}

  Can you prove that $T^{\prime}$ and $T^{*}$ are linearly independent
  sets and $W=\spn{S}=\spn{T^{\prime}}=\spn{T^{*}}$?
\end{example}


\begin{example}[Reworking elements of a span]
  Begin with a set of five vectors from $\real{4}$,
  \[
    R=\set{
      \colvector{ 2 \\ 1 \\ 3 \\ 2 },\,
      \colvector{ -1 \\ 1 \\ 0 \\ 1 },\,
      \colvector{ -8 \\ -1 \\ -9 \\ -4 },\,
      \colvector{ 3 \\ 1 \\ -1 \\ -2 },\,
      \colvector{ -10 \\ -1 \\ -1 \\ 4}
    }
  \]
  
  It is easy to create elements of $X=\spn{R}$.  We will create one at random,
  \[
    \vect{y}=
    6\colvector{ 2 \\ 1 \\ 3 \\ 2 }+
    (-7)\colvector{ -1 \\ 1 \\ 0 \\ 1 }+
    1\colvector{ -8 \\ -1 \\ -9 \\ -4 }+
    6\colvector{ 3 \\ 1 \\ -1 \\ -2 }+
    2\colvector{ -10 \\ -1 \\ -1 \\ 4}
    =
    \colvector{9\\2\\1\\-3}
  \]
  
  We know we can replace $R$ by a smaller set (since it is obviously
  linearly dependent by \ref{theorem:MVSLD}) that will create the same
  span.  Here goes,
  \[
    \begin{bmatrix}
      2 & -1 & -8 & 3 & -10 \\
      1 & 1 & -1 & 1 & -1 \\
      3 & 0 & -9 & -1 & -1 \\
      2 & 1 & -4 & -2 & 4
    \end{bmatrix}
    \rref
    \begin{bmatrix}
      \leading{1} & 0 & -3 & 0 & -1 \\
      0 & \leading{1} & 2 & 0 & 2 \\
      0 & 0 & 0 & \leading{1} & -2 \\
      0 & 0 & 0 & 0 & 0
    \end{bmatrix}
  \]
  
  So, if we collect the first, second and fourth vectors from $R$,
  \[
    P=\set{
      \colvector{ 2 \\ 1 \\ 3 \\ 2 },\,
      \colvector{ -1 \\ 1 \\ 0 \\ 1 },\,
      \colvector{ \answer{3} \\ 1 \\ -1 \\ -2 }
    }
  \]
  then $P$ is linearly independent and $\spn{P}=\spn{R}=X$ by
  \ref{theorem:BS}.  Since we built $\vect{y}$ as an element of
  $\spn{R}$ it must also be an element of $\spn{P}$.  

  \begin{question}
    Can we write
    $\vect{y}$ as a linear combination of just the three vectors in $P$?
    
    \begin{multipleChoice}
      \choice[correct]{Yes.}
      \choice{No.}
    \end{multipleChoice}

    \begin{feedback}[correct]
      Let us compute an explicit linear combination just for fun.  By
      \ref{theorem:SLSLC} we can get such a linear combination by
      solving a system of equations with the column vectors of $R$ as
      the columns of a coefficient matrix, and $\vect{y}$ as the
      vector of constants.

      Employing an augmented matrix to solve this system,
      \[
        \begin{bmatrix}
          2 & -1 & 3 & 9 \\
          1 & 1 & 1 & 2 \\
          3 & 0 & -1 & 1 \\
          2 & 1 & -2 & -3
        \end{bmatrix}
        \rref
        \begin{bmatrix}
          \leading{1} & 0  & 0 & 1 \\
          0 & \leading{1} & 0 & -1 \\
          0 & 0 & \leading{1} & 2 \\
          0 & 0 & 0 & 0
        \end{bmatrix}
      \]
      
      So we see, as expected, that
      \[
        1\colvector{ 2 \\ 1 \\ 3 \\ 2 }+
        (-1)\colvector{ -1 \\ 1 \\ 0 \\ 1 }+
        2\colvector{ 3 \\ 1 \\ -1 \\ -2 }
        =\colvector{9 \\ 2 \\ 1 \\ -3}
        =\vect{y}
      \]
      
      A key feature of this example is that the linear combination that
      expresses $\vect{y}$ as a linear combination of the vectors in $P$
      is unique.  This is a consequence of the linear independence of $P$.
      The linearly independent set $P$ is smaller than $R$, but still just
      (barely) big enough to create elements of the set $X=\spn{R}$.
      There are many, many ways to write $\vect{y}$ as a linear
      combination of the five vectors in $R$ (the appropriate system of
      equations to verify this claim yields two free variables in the
      description of the solution set), yet there is precisely one way to
      write $\vect{y}$ as a linear combination of the three vectors in
      $P$.
    \end{feedback}
  \end{question}
\end{example}

\end{document}

%%% Local Variables:
%%% mode: latex
%%% TeX-master: t
%%% End:
