\documentclass{ximera}

% These macros are automatically generated from the "macros"
% XML element.  Make permanent edits there.
%
% History
%   2004/01/01  Initiated for FCLA, evolved from there
%   2006/09/17  Converted  _, ^  to \sb, \sp for TeX4ht
%   2014/02/01  Updated for MathBook XML projects
%               Obsolete in FCLA: \codeindent, \computerfont, \define
%               Change: MathJax wants \lt, so replaced by \lteval
%   2014/02/22  New: \orderof, \reals, \per
%   2015/08/16  Incorporated into MathBook XML version of FCLA
%
%%%%%%%%%%%%%%%%%%%%%
%
%     Conveniences
%
%%%%%%%%%%%%%%%%%%%%%
%
%  Order of (asymptotically limit of fraction is 1)
%  Usage: \orderof{some function}
%
\newcommand{\orderof}[1]{\sim #1}
%
%  Integers
%  Usage:  \Z
\newcommand{\Z}{\mathbb{Z}}
%
%  Real numbers, as set of scalars
%  Usage:  \reals
\newcommand{\reals}{\mathbb{R}}
%
%  n-space over real field
%  Usage: \complex{integer-dimension}
\newcommand{\real}[1]{\mathbb{R}^{#1}}
%
%  Complex numbers, as set of scalars
%  Usage:  \complexes
\newcommand{\complexes}{\mathbb{C}}
%
%  n-space over complex field
%  Usage: \complex{integer-dimension}
\newcommand{\complex}[1]{\mathbb{C}^{#1}}
%
%  Complex conjugation (scalar, vector, matrix)
%  Usage: \conjugate{object}
\newcommand{\conjugate}[1]{\overline{#1}}
%
%  Complex number modulus
%  Usage: \modulus{a+bi}
%  Presumes math mode
\newcommand{\modulus}[1]{\left\lvert#1\right\rvert}
%
%  Zero vector
%  Usage: \zerovector
\newcommand{\zerovector}{\vect{0}}
%
%  Zero matrix
%  Usage: \zeromatrix, use a subscript when size is important
\newcommand{\zeromatrix}{\mathcal{O}}
%
%  Inner product (brackets, not quadratic form)
%  Usage: \innerproduct{a-vector}{a-vector}
\newcommand{\innerproduct}[2]{\left\langle#1,\,#2\right\rangle}
%
%  Norm of a vector
%  Usage: \norm{a-vector}
\newcommand{\norm}[1]{\left\lVert#1\right\rVert}
%
%  Dimension
%  Usage: \dimension{vector-space-letter}
\newcommand{\dimension}[1]{\dim\left(#1\right)}
%
%  Nullity
%  Usage: \nullity{matrix-or-lintrans-letter}
\newcommand{\nullity}[1]{n\left(#1\right)}
%
%  Rank
%  Usage: \rank{matrix-or-lintrans-letter}
\newcommand{\rank}[1]{r\left(#1\right)}
%
%  Direct sum
%  Usage: \ds between a couple of subspaces
%
\newcommand{\ds}{\oplus}
%
%  Determinant of a matrix (functional)
%  Usage: \detname{A}
\newcommand{\detname}[1]{\det\left(#1\right)}
%
%  Determinant of a matrix (vertical bars)
%  Usage: \detbars{A}
\newcommand{\detbars}[1]{\left\lvert#1\right\rvert}
%
%  Trace of a Matrix
%  Usage: \trace{matrix name}
\newcommand{\trace}[1]{t\left(#1\right)}
%
%  Square Root of a Matrix
%  Usage: \sr{a-matrix}
\newcommand{\sr}[1]{#1^{1/2}}
%
%%%%%%%%%%%%%%%%%%%%%
%
%     Subspace Constructions
%
%%%%%%%%%%%%%%%%%%%%%
%
%  Span of a set of vectors
%  \span and \sp are used by TeX for other things
%  Usage: \spn{set-of-vectors}
\newcommand{\spn}[1]{\left\langle#1\right\rangle}
%
%  Null space of a matrix
%  Usage:  \nsp{A}
\newcommand{\nsp}[1]{\mathcal{N}\!\left(#1\right)}
%
%  Column space of a matrix
%  Usage:  \csp{A}
\newcommand{\csp}[1]{\mathcal{C}\!\left(#1\right)}
%
%  Row space of a matrix
%  Usage:  \rsp{A}
\newcommand{\rsp}[1]{\mathcal{R}\!\left(#1\right)}
%
%  Left null space of a matrix
%  Usage:  \lns{A}
\newcommand{\lns}[1]{\mathcal{L}\!\left(#1\right)}
%
%  Orthogonal complement of a vector space
%  Avoiding TeX's \perp
%  Usage:  \per{A}
\newcommand{\per}[1]{#1^\perp}
%
%%%%%%%%%%%%%%%%%%%%%
%
%     Systems of Equations
%
%%%%%%%%%%%%%%%%%%%%%
%
%  In-line form of an augmented matrix for a system of equations
%  Usage: \augmented{coefficient-matrix}{constant-vector}
\newcommand{\augmented}[2]{\left\lbrack\left.#1\,\right\rvert\,#2\right\rbrack}
%
%  Notation for a linear system before introducing matrix multiplication
%  Usage: \linearsystem{coefficient-matrix}{constant-vector}
\newcommand{\linearsystem}[2]{\mathcal{LS}\!\left(#1,\,#2\right)}
%
%  Notation for a homogenous system before introducing matrix multiplication
%  Usage: \homosystem{coefficient-matrix}
\newcommand{\homosystem}[1]{\linearsystem{#1}{\zerovector}}
%
%%%%%%%%%%%%%%%%%%%%%
%
%     Row Operations, Echelon Form
%
%%%%%%%%%%%%%%%%%%%%%
%
% Row operations on matrices
%
% Three commands to shorten up descriptions of gaussian elimination
%
% Usage: \rowopswap{row-i}{row-j}
% Usage: \rowopmult{scalar}{row-i}
% Usage: \rowopadd{scalar}{row-multiplied}{row-added-to}
\newcommand{\rowopswap}[2]{R_{#1}\leftrightarrow R_{#2}}
\newcommand{\rowopmult}[2]{#1R_{#2}}
\newcommand{\rowopadd}[3]{#1R_{#2}+R_{#3}}
%
% Mark leading 1's in echelon form with fbox
% Usage: \leading{a-1-usually}
\newcommand{\leading}[1]{\fbox{#1}}
%
%  Row-reduce arrow
%  Usage:  \rref inbetween a matrix and its reduced row-echelon form
\newcommand{\rref}{\xrightarrow{\text{RREF}}}
%
%  Elementary Matrices
%  Usage: \elemswap{subscript}{subscript}
%  Usage: \elemmult{scalar}{subscript}
%  Usage: \elemadd{scalar}{subscript-mult}{subscript-target}
%
\newcommand{\elemswap}[2]{E_{#1,#2}}
\newcommand{\elemmult}[2]{E_{#2}\left(#1\right)}
\newcommand{\elemadd}[3]{E_{#2,#3}\left(#1\right)}
%
%%%%%%%%%%%%%%%%%%%%%
%
%     2-D Constructions (Lists, Vectors, Matrices)
%
%%%%%%%%%%%%%%%%%%%%%
%
%  A list of scalars of generic length
%  Usage:  \scalarlist{scalar letter}{terminal subscript}
\newcommand{\scalarlist}[2]{{#1}_{1},\,{#1}_{2},\,{#1}_{3},\,\ldots,\,{#1}_{#2}}
%
%  Vector styling, bold (or use wiggles, arrows, whatever)
%  Subscripts go outside this construction
%  Usage: \vect{a symbol to use as a vector}
%  Have to already be in math mode
%
\newcommand{\vect}[1]{\mathbf{#1}}
%
%  A column vector
%  Usage: \colvector{list-delimited-by-\\}
%
\newcommand{\colvector}[1]{\begin{bmatrix}#1\end{bmatrix}}
%
%  A generic vector with components
%  Usage: \vectorcomponents{component-letter}{final-subscript}
\newcommand{\vectorcomponents}[2]{\colvector{#1_{1}\\#1_{2}\\#1_{3}\\\vdots\\#1_{#2}}}
%
%  A list of vectors of generic length
%  Usage:  \vectorlist{vector letter}{terminal subscript}
\newcommand{\vectorlist}[2]{\vect{#1}_{1},\,\vect{#1}_{2},\,\vect{#1}_{3},\,\ldots,\,\vect{#1}_{#2}}
%
%  Vector entries, entry i of vector v
%  (vector-expession still needs \vect, etc.)
%  Usage:  \vectorentry{vector-expression}{single-subscript}
\newcommand{\vectorentry}[2]{\left\lbrack#1\right\rbrack_{#2}}
%
%  Matrix entries, entry i,j of matrix A
%  Usage:  \matrixentry{matrix-expression}{paired-subscripts}
%
\newcommand{\matrixentry}[2]{\left\lbrack#1\right\rbrack_{#2}}
%
%  A generic linear combination
%  Usage:  \lincombo{scalar letter}{vector letter}{terminal subscript}
\newcommand{\lincombo}[3]{#1_{1}\vect{#2}_{1}+#1_{2}\vect{#2}_{2}+#1_{3}\vect{#2}_{3}+\cdots +#1_{#3}\vect{#2}_{#3}}
%
%  Matrix, column by column, as vectors
%  Usage:  \matrixcolumns{matrix letter}{terminal subscript}
\newcommand{\matrixcolumns}[2]{\left\lbrack\vect{#1}_{1}|\vect{#1}_{2}|\vect{#1}_{3}|\ldots|\vect{#1}_{#2}\right\rbrack}
%
%%%%%%%%%%%%%%%%%%%%%
%
%     Special Matrices
%
%%%%%%%%%%%%%%%%%%%%%
%
%  Transpose of a matrix
%  Usage:  \transpose{A}
\newcommand{\transpose}[1]{#1^{t}}
%
%  Inverse of a matrix
%  Usage:  \inverse{A}
\newcommand{\inverse}[1]{#1^{-1}}
%
%  Submatrix (for minors, determinants)
%  Usage: \submatrix{matrix-name}{delete-row}{delete-col}
\newcommand{\submatrix}[3]{#1\left(#2|#3\right)}
%
%  Adjoint of a matrix (twice)
%  This macro is a convenience to call \transpose and \conjugate properly
%  It shouldn't need to be modified (or mathematical meanings will change)
%  Usage:  \adj{A}
\newcommand{\adj}[1]{\transpose{\left(\conjugate{#1}\right)}}
%
%  This macro controls the symbol used for the adjoint
%  It can be edited to taste
%  Usage:  \adjoint{A}
\newcommand{\adjoint}[1]{#1^\ast}
%
%%%%%%%%%%%%%%%%%%%%%
%
%     Sets
%
%%%%%%%%%%%%%%%%%%%%%
%
%  A convenience for simple sets
%  Usage:  \set{list of element}
\newcommand{\set}[1]{\left\{#1\right\}}
%
%  Sets with vertical bar, "such that", sized for objects, not condition
%  Usage:  \setparts{objects}{condition}
%
%%\newcommand{\setparts}[2]{\left\{ #1\mid#2\right\}}
%%\newcommand{\setparts}[2]{\left\{\left. #1\right\rvert#2\right\}}
\newcommand{\setparts}[2]{\left\lbrace#1\,\middle|\,#2\right\rbrace}
%
%  Set Cardinality
%  Usage:  \card{a-set-letter}
\newcommand{\card}[1]{\left\lvert#1\right\rvert}
%
%  Set Union
%  Use \cup
%
%  Set Intersection
%  Use \cap
%
%  Set Complement
%  Usage:  \setcomplement{a-set-letter}
\newcommand{\setcomplement}[1]{\overline{#1}}
%
%%%%%%%%%%%%%%%%%%%%%
%
%     Eigenvalues and Eigenspaces
%
%%%%%%%%%%%%%%%%%%%%%
%
%  Characteristic polynomial
%  Usage: \charpoly{matrix-letter}{variable-letter}
\newcommand{\charpoly}[2]{p_{#1}\left(#2\right)}
%
%  Eigenspace
%  Usage: \eigenspace{matrix-letter}{eigenvalue-letter}
\newcommand{\eigenspace}[2]{\mathcal{E}_{#1}\left(#2\right)}
%
%  2013/10/03 Including ampersands is problematic here, 
%  think about fixes later
%  2014/02/22 Limited testing, seems &amp; is fine for HTML and LaTeX
%  2016-07-20 only employed in Archetypes, MBX has gather/align override
%  Eigensystem (presumes wrapped in an mrow within md)
%  Usage: \eigensystem{matrixletter}{eigenvalue}{list of basis vectors}
\newcommand{\eigensystem}[3]{\lambda&amp;=#2&amp;\eigenspace{#1}{#2}&amp;=\spn{\set{#3}}} 
%
%  Generalized Eigenspace
%  Usage: \geneigenspace{lin-trans-letter}{eigenvalue-letter}
\newcommand{\geneigenspace}[2]{\mathcal{G}_{#1}\left(#2\right)}
%
%  Algebraic multiplicty
%  Usage: \algmult{matrix-letter}{eigenvalue-letter}
\newcommand{\algmult}[2]{\alpha_{#1}\left(#2\right)}
%
%  Geometric multiplicty
%  Usage: \geomult{matrix-letter}{eigenvalue-letter}
\newcommand{\geomult}[2]{\gamma_{#1}\left(#2\right)}
%
%  Index (of eigenvalue)
%  Usage: \indx{matrix-letter}{eigenvalue-letter}
\newcommand{\indx}[2]{\iota_{#1}\left(#2\right)}
%
%%%%%%%%%%%%%%%%%%%%%
%
%     Linear Transformations
%
%%%%%%%%%%%%%%%%%%%%%
%
%  MathJax defines \lt to ease XML confusion
%
%  Linear transformation definition
%  Usage: \ltdefn{name-letter}{domain}{range}
\newcommand{\ltdefn}[3]{#1\colon #2\rightarrow#3}
%
%  Linear transformation evaluation
%  Usage: \lteval{name-letter}{input}
%  Replaces old \lt desired by MathJax
\newcommand{\lteval}[2]{#1\left(#2\right)}
%
% Linear transformation inverse
%  Usage: \ltinverse{name-letter}
\newcommand{\ltinverse}[1]{#1^{-1}}
%
%  Linear transformation restriction
%  Usage: \restrict{name-letter}{subspace-letter}
\newcommand{\restrict}[2]{{#1}|_{#2}}
%
%  Linear transformation preimage
%  Usage: \preimage{name-letter}{codomain-element}
\newcommand{\preimage}[2]{#1^{-1}\left(#2\right)}
%
%  Range of a linear transformation
%  TeX uses \range for something else
%  Usage:  \rng{T}
\newcommand{\rng}[1]{\mathcal{R}\!\left(#1\right)}
%
%  Kernel of a linear transformation
%  TeX uses \ker to do something different
%  Usage:  \krn{T}
\newcommand{\krn}[1]{\mathcal{K}\!\left(#1\right)}
%
%  Linear transformation composition
%  Usage: \compose{function-name}{function-name}
\newcommand{\compose}[2]{{#1}\circ{#2}}
%
%  Vector space of linear transformations
%  Usage: \vslt{domains}{codomains}
%  Presumes math mode
\newcommand{\vslt}[2]{\mathcal{LT}\left(#1,\,#2\right)}
%
%%%%%%%%%%%%%%%%%%%%%
%
%     Vector and Matrix Representations
%
%%%%%%%%%%%%%%%%%%%%%
%
%  Isomorphism symbol
%  Usage: \isomorphic
\newcommand{\isomorphic}{\cong}
%
%  Similarity
%  Usage: \similar{inner-matrix}{outer-invertible-matrix}
%  Rearranging this will not "fix" all desired changes throughout
%
\newcommand{\similar}[2]{\inverse{#2}#1#2}
%
%  Vector representation function name
%  Usage: \vectrepname{basis-letter}
\newcommand{\vectrepname}[1]{\rho_{#1}}
%
%  Vector representation output
%  Usage: \vectrep{basis-letter}{input}
\newcommand{\vectrep}[2]{\lteval{\vectrepname{#1}}{#2}}
%
%  Vector representation inverse function name
%  (Added later, not used consistently in FCLA)
%  Usage: \vectrepinvname{basis-letter}
\newcommand{\vectrepinvname}[1]{\ltinverse{\vectrepname{#1}}}
%
%  Vector representation inverse output
%  Usage: \vectrepinv{basis-letter}{input}
\newcommand{\vectrepinv}[2]{\lteval{\ltinverse{\vectrepname{#1}}}{#2}}
%
%  Matrix representation
%  Usage: \matrixrep{transformation-letter}{domain-basis-letter}{codomain-basis-letter}
\newcommand{\matrixrep}[3]{M^{#1}_{#2,#3}}
%
%  Matrix representation column-by-colum
%  2016-07-20 only employed once?
%  Usage: \matrixrepcolumns{transformation-letter}{codomain-basis-letter}{codomain-basis-vector-letter}{final-index}
\newcommand{\matrixrepcolumns}[4]{\left\lbrack \left.\vectrep{#2}{\lteval{#1}{\vect{#3}_{1}}}\right|\left.\vectrep{#2}{\lteval{#1}{\vect{#3}_{2}}}\right|\left.\vectrep{#2}{\lteval{#1}{\vect{#3}_{3}}}\right|\ldots\left|\vectrep{#2}{\lteval{#1}{\vect{#3}_{#4}}}\right.\right\rbrack}
%
%  Change of basis matrix
%  Usage: \cbm{domain-basis-letter}{codomain-basis-letter}
\newcommand{\cbm}[2]{C_{#1,#2}}
%
%%%%%%%%%%%%%%%%%%%%%
%
%     Canonical Forms
%
%%%%%%%%%%%%%%%%%%%%%
%
%  Jordan blocks
%  Usage: \jordan{size}{diagonal-element}
\newcommand{\jordan}[2]{J_{#1}\left(#2\right)}
%
%%%%%%%%%%%%%%%%%%%%%
%
%     Hadamard Matrices
%     Contributed by Elizabeth Million
%
%%%%%%%%%%%%%%%%%%%%%
%
%  Hadamard Product
%  Usage: \hadamard{a-matrix}{a-matrix}
\newcommand{\hadamard}[2]{#1\circ #2}
%
%  Hadamard identity matrix
%  Usage: \hadamardidentity{paired-subscripts-size-of-matrix}
\newcommand{\hadamardidentity}[1]{J_{#1}}
%
%  Hadamard inverse matrix
%  Usage: \hadamardinverse{matrix-expression}
\newcommand{\hadamardinverse}[1]{\widehat{#1}}


\title{Equivalent Systems and Equation Operations}

\begin{document}
\begin{abstract}
  Sometimes different looking systems of equations have the same solution sets.
\end{abstract}
\maketitle

With all this talk about finding solution sets for systems of linear
equations, you might be ready to begin learning how to find these
solution sets yourself.  We begin with our first definition that takes
a common word and gives it a very precise meaning in the context of
systems of linear equations.

\begin{definition}
  Two systems of linear equations are \definedTerm{equivalent} if
  their solution sets are equal.
\end{definition}

Notice here that the two systems of equations could \textit{look} very
different (i.e., not be the same systems on the nose), but still have
equal solution sets, and we would then call the systems equivalent.
Two linear equations in two variables might be plotted as two lines
that intersect in a single point.  A different system, with three
equations in two variables might have a plot that is three lines, all
intersecting at a common point, with this common point identical to
the intersection point for the first system---even though the second
system has more equations!  By our definition, we could then say these
two very different looking systems of equations are equivalent, since
they have identical solution sets.  It is really like a weaker form of
equality, where we allow the systems to be different in some respects,
but we use the term equivalent to highlight the situation when their
solution sets are equal.

With this definition, we can begin to describe our strategy for
solving linear systems.  Given a system of linear equations that looks
difficult to solve, we would like to have an \textit{equivalent}
system that is easy to solve.  Since the systems will have equal
solution sets, we can solve the ``easy'' system and get the solution
set to the ``difficult'' system.  Here come the tools for making this
strategy viable.

\begin{definition}[Equation Operations]
  Given a system of linear equations, the following three operations will transform the system into a different one, and each operation is known as an \definedTerm{equation operation}.
  \begin{enumerate}
  \item Swap the locations of two equations in the list of equations.
  \item Multiply each term of an equation by a nonzero quantity.
  \item Multiply each term of one equation by some quantity, and add these terms to a second equation, on both sides of the equality.  Leave the first equation the same after this operation, but replace the second equation by the new one.
  \end{enumerate}
\end{definition}

These descriptions might seem a bit vague, but the proof or the
examples that follow should make it clear what is meant by each.  We
will shortly prove a key theorem about equation operations and
solutions to linear systems of equations.

\begin{theorem}[Equation Operations Preserve Solution Sets]
  If we apply one of the three equation operations to a system of
  linear equations, then the original system and the transformed
  system are equivalent.
\end{theorem}

This theorem is the necessary tool to complete our strategy for
solving systems of equations.  We will use equation operations to move
from one system to another, all the while keeping the solution set the
same.  With the right sequence of operations, we will arrive at a
simpler equation to solve.  The next two examples illustrate this
idea, while saving some of the details for later.

\begin{example}
  Let's solve the following system by a sequence of equation operations:
  \begin{align*}
    x_1+2x_2+2x_3&=4,\\
    x_1+3x_2+3x_3&=5,\\
    2x_1+6x_2+5x_3&=6.
  \end{align*}

  \begin{exercise}
    Multiply Equation~1 by $-1$ and add the result to Equation~2 to get
    \begin{align*}
      x_1+2x_2+2x_3&=4\\
      \answer{0}x_1+\answer{1}x_2+ \answer{1} x_3&=1\\
      2x_1+6x_2+5x_3&=6
    \end{align*}
    Multiply Equation~1 by $-2$ and add the result to Equation~3:
    \begin{align*}
      x_1+2x_2+2x_3&=4\\
      0x_1+1x_2+ 1x_3&=1\\
      0x_1+2x_2+1x_3&=-2
    \end{align*}
    Multiply Equation~2 by $-2$ and add the result to Equation~3:
    \begin{align*}
      x_1+2x_2+2x_3&=4\\
      0x_1+1x_2+ 1x_3&=1\\
      \answer{0}x_1+\answer{0}x_2-1x_3&=\answer{-4}
    \end{align*}
    Multiply Equation~3 by $-1$:
    \begin{align*}
      x_1+2x_2+2x_3&=4\\
      0x_1+1x_2+ 1x_3&=1\\
      0x_1+0x_2+1x_3&=\answer{4},
    \end{align*}
    which can be written more clearly as
    \begin{align*}
      x_1+2x_2+2x_3&=4\\
      x_2+ x_3&=1\\
      x_3&=4.
    \end{align*}
  \end{exercise}

  \begin{question}
    This is now a very easy system of equations to solve.  The third
    equation requires that $x_3=\answer{4}$ to be true.  Making this substitution
    into Equation~2 we arrive at $x_2=\answer{-3}$, and finally, substituting
    these values of $x_2$ and $x_3$ into Equation~1, we find that
    $x_1=\answer{2}$.
    
    Note too that this is the only solution to this final system of
    equations, since we were forced to choose these values to make the
    equations true.  Since we performed equation operations on each
    system to obtain the next one in the list, all of the systems listed
    here are all equivalent to each other by the above theorem.
  \end{question}

  \begin{question}
    Therefore \[
      (x_1,\,x_2,\,x_3)=(2,-3,4)
    \]
    is \wordChoice{\choice{a}\choice[correct]{the unique}} solution to
    the \textit{original} system of equations---and all of the other
    intermediate systems of equations listed as we transformed one
    into another.
  \end{question}
\end{example}

\begin{example}
  The following system of equations made an appearance earlier when we
  listed \textit{one} of its solutions.  Let's find all of the
  solutions to this system.  Do not concern yourself too much about
  why we choose this particular sequence of equation operations.
  
  \begin{exercise}
    \begin{align*}
      x_1+2x_2 +0x_3+ x_4&= 7\\
      x_1+x_2+x_3-x_4&=3\\
      3x_1+x_2+5x_3-7x_4&=1
    \end{align*}
    Multiply Equation~1 by $\answer{-1}$ and add the result to Equation~2:
    \begin{align*}
      x_1+2x_2 +0x_3+ x_4&= 7\\
      0x_1-x_2+x_3-2x_4&=-4\\
      3x_1+x_2+5x_3-7x_4&=1
    \end{align*}
    Multiply Equation~1 by $-3$ and add the result to Equation~3:
    \begin{align*}
      x_1+2x_2 +0x_3+ x_4&= 7\\
      0x_1-x_2+x_3-2x_4&=-4\\
      0x_1-5x_2+5x_3-10x_4&=\answer{-20}
    \end{align*}
    Multiply Equation~2 by $-5$ and add the result to Equation~3:
    \begin{align*}
      x_1+2x_2 +0x_3+ x_4&= 7\\
      0x_1-x_2+x_3-2x_4&=-4\\
      0x_1+0x_2+0x_3+0x_4&=0
    \end{align*}
    Replace Equation~2 by $\answer{-1}$ times Equation~2:
    \begin{align*}
      x_1+2x_2 +0x_3+ x_4&= 7\\
      0x_1+x_2-x_3+2x_4&=4\\
      0x_1+0x_2+0x_3+0x_4&=0
    \end{align*}
    Multiply Equation~2 by $-2$ and add the result to Equation~1:
    \begin{align*}
      x_1+0x_2 +2x_3-3x_4&= -1\\
      0x_1+x_2-x_3+2x_4&=4\\
      0x_1+0x_2+0x_3+0x_4&=0,
    \end{align*}
    which can be written more clearly as
    \begin{align*}
      x_1+2x_3 - 3x_4&= -1\\
      x_2-x_3+2x_4&=4\\
      0&=\answer{0}
    \end{align*}
  \end{exercise}

  \begin{question}
    What does that last equation, $0=0$, even mean?  It means that we
    can choose \textit{any} values for $x_1$, $x_2$, $x_3$, $x_4$ and
    this equation $0=0$ will be true, so we only need to consider
    further the first two equations, since the third is true no matter
    what!  We can analyze the second equation without consideration of
    the variable $x_1$.  It would appear that there is considerable
    latitude in how we can choose $x_2$, $x_3$, $x_4$ and make this
    equation true.  Let us choose $x_3$ and $x_4$ to be
    \textit{anything} we please, say $x_3=a$ and $x_4=b$.

    Now we can take these arbitrary values for $x_3$ and $x_4$, substitute them in Equation~1, to obtain
    \begin{align*}
      x_1+2a - 3b&= -1\\
      x_1&=-1-2a+3b
    \end{align*}
    Similarly, Equation~2 becomes
    \begin{align*}
      x_2-a+2b&=4\\
      x_2&=\answer{4 +a -2 b}
    \end{align*}
  \end{question}
  
  \begin{question}
    So our arbitrary choices of values for $x_3$ and $x_4$ ($a$ and
    $b$) translate into specific values of $x_1$ and $x_2$.  For
    instance, suppose we choose $a=5$ and $b=-2$, then we compute
    \begin{align*}
      x_1&=-1-2(5)+3(-2)=-17\\
      x_2&=4+5-2(-2)=13
    \end{align*}
    and you can verify that $(x_1,\,x_2,\,x_3,\,x_4)=(-17,\,13,\,\answer{5},\,\answer{-2})$ makes all three equations true.  The entire solution set is written as
    \[
      S=\setparts{(-1-2a+3b,\,4 +a-2b,\,a,\,b)}{ a\in\CC,\,b\in\CC}.
    \]
  \end{question}
\end{example}


\end{document}